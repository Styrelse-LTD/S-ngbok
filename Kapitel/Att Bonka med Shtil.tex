\fancypagestyle{Att Bonka med Shtil}{
    \fancyhead{} % clear all header fields
    \fancyhead[LE,RO]{\textbf{Att Bonka med Shtil}}
}
\pagestyle{Att Bonka med Shtil}

\section*{\textbf{Att Bonka med Shtil}}

På de flesta tillställningar av studentikos natur är det lämpligt att bekläda sig med overall. Overallen är det oslagbara instrumentet när det gäller raggning, allmän ösning, bonkning  mm. Det kan dock, då så påbjudes, vara nödvändigt att klä sig enligt andra seder och bruk än rent djuriska. Här nedan pekar vi på det viktigaste för att bli accepterad bland det klientel vi en gång i framtiden (hemska tanke) kanske kommer att vilja tillhöra.

\subsection*{\textbf{Klädselbeteckningar}}
Högtidsdräkt innebär frackklädsel eller paraduniform samt å spinnsidan galaklänning. Aftondräkt är i första hand smoking eller aftonklänning. Men i andra hand kan likväl begagnas lunchcoat med svarta byxor utan revärer och mellanklänning. Frack och galaklänning godtages såsom aftondräkt. Då för vissa lokaler föreskrivits aftondräkt för besökande, tillåtas i allmänhet även blå kavajkostym med vit skjorta och krage samt promenadklänning. Kavaj är blå eller mörk kavajkostym med vit skjorta och krage samt vardagsklänning. Vid bjudningar angives stundom: "klädsel: kavaj". Detta skall tolkas såsom nyss angiven klädsel med mörk kavaj respektive mellanklänning.

\subsection*{\textbf{Klädsel för Herrar}}
\subsubsection*{\textbf{Frack}}
\begin{itemize}

    \item[]\textbf{Frackrock:} svart rock med "svalstjärt", långa slag med eller utan sidenbeläggning, 2 knappar i ryggen, med eller utan bröstficka, framsidans undre kanter i höjd med byxlinningens övre framkant; rocken skall sitta spänt och sluta åt strikt utan rynkor och skall framtill kunna käppas precis. Bäres aldrig knäppt annars än möjligheten med bandögla, vilken förenar rocksidorna, vit bandögla för vit väst, svart för svart väst. Stickerier i spegel på slaget anger akademisk doktorsgrad. På ärmarna lämnas manchetterna synliga, västen synes framtill under sidostyckerna. Då man sätter sig vikas skärten undan.

    \item[]\textbf{Frackbyxor:} Svarta långbyxor med svarta revärer, byxorna nedtill snett skurna, så att de baktill når klackens överkant och framtill utan rynkor ligga över foten men aldrig över hela foten.

    \item[]\textbf{Frackväst:} vit urringad väst med eller utan slag. Västen synes framtill nedanför rocksidorna.

    \item[]\textbf{Skor:} svarta lackskor eller möjligheten prydliga inneskor.

    \item[]\textbf{Skjorta:} vit stärkskjorta med styvt veck.

    \item[]\textbf{Krage:} enkel stärkkrage med snibbar.

    \item[]\textbf{Halsduk:} vit rosett.

    \item[]\textbf{Näsduk:} vit silkesnäsduk, bäres eventuellt i bröstfickan och då synlig som ett vitt band ovan fickans kant.

    \item[]\textbf{Strumpor:} svarta silkenstrumpor.

    \item[]\textbf{Smycken:} skjortknappar i vitt eller guld, manschettknappar i samma färg, klockkedja bäres hängande från hängselstropp till vänster byxficka.

    \item[]\textbf{Överplagg:} mörk paletå (överrock utan skärp eller ryggleif) och vit kragskyddare. Hög hatt eller chapeau-claque, vilket senare hopfälld medföres inomhus.

    \item[]\textbf{Handskat:} vita glacéhandskar. Inomhus bäres handskar ofta i handen. Vid bal och uppvaktning bäres de pådragna.

\end{itemize}

Det finns tre saken en riktig man \textbf{aldrig} gör:

\textit{Han ger aldrig upp}

\textit{Han ljuger inte}

\textit{Han ber aldrig om ursäkt}

\textit{\textbf{-Zeb Macahan}}

\subsubsection*{\textbf{Smoking}}

\begin{itemize}

    \item[]\textbf{Smokingrock:} svart rock med eller utan sidenbeläggning, rocken med eller utan bröstficka, sidofickor med eller utan lock. Rocken bären öppen eller knäppt. Stundom dubbelknäppt smoking.

    \item[]\textbf{Smokingbyxor:} samma som frackbyxor.

    \item[]\textbf{Smokingväst:} svart urringad väst med eller utan slag, i det senare fallet stundom med stärkt linnebård, västens och rockens urringningar lika djupa. Vit väst bäres icke.

    \item[]\textbf{Skor:} samma som till frack.

    \item[]\textbf{Skjorta:} vit stärkskjorta med styvt veck, eventuellt lättare stärkt eller mjuk för bruk vid danstillställning.

    \item[]\textbf{Krage:} enkel stärkkrage med snibbar.

    \item[]\textbf{Halsduk:} svart rosett.

    \item[]\textbf{Näsduk:} vit silkesnäsduk i bröstfickan som ett smalt band synlig ovan dess kant.

    \item[]\textbf{Strumpor:} svarta silkesstrumpor.

    \item[]\textbf{Smycken:} knappar såsom till frack, klockkedja över västen. Vid begravning svarta knappar. Ordnar får inte bäras till smoking.

    \item[]\textbf{Överplagg:} Mörk paletå med vit kragskyddare.

    \item[]\textbf{Handskar:} Vita glacéhandskar eller vita gants de suéde. Handskar bäres inte inomhus.

\end{itemize}
Kuriosa om Smokingen

I Amerika kallas smokingen för "Tuxedo". Enligt kufiska källor tillkom företeelsen tuxedo när ett gäng amerikaner hade en riktig blöt tillställning på "The Tuxedo Country Club". Allt eftersom natten led tyckte deltagarna att deras frackar blev något otympliga med svalstjärten hängandes efter benen. Styrkta av det myckna drickandet, klippte de av svalstjärtarna. Så lätt skapar man ett nytt plagg!

\subsubsection*{\textbf{Kavajkostym}}

Färg, snitt och utförande beroende av egen smak, uppfinningsrikedom och personlig karaktär. Mörk kavajkostym är en för alla vardagstillfällen lämpad kostym. Såsom arbetskostym användes den med vit krage och vit skjorta. Alltid svarta skodon. Sommartid kommer denna kostym till användning i stället för smoking, där den icke direkt påbjudes. Kulört kavajkostym är utpräglad vardagsdräkt men tillåten vid mindre vardagsbjudning. Till kostym hör kulört skjorta och krage samt ofta bruna skodon.

\textit{"Spotta inte på eller över bordet, det kan tolkas som om sällskapet inte faller en i smaken"} - \textbf{Ur en bok om bodsskick från 1300-talet}

\subsection*{\textbf{Klädsel för Damer}}
\subsubsection*{\textbf{Galaklänning}}
Galaklänning innebär ljus hellång klänning med släp, vanligen urringad och helt utan ärmar. Medaljer och ordnar bäres. Hovdräkten, antigen helsvart eller helvit, är urringad och försedd med puffärmar och släp.

\subsubsection*{\textbf{Aftonklänning}}
Aftonklänning är mörk eller ljus, numera ofta kort, klänning urringad och ärmlös. Ofta hör en jacka till klänningen. Medaljer och ordnar bäres.

\subsubsection*{\textbf{Mellanklänning}}
Mellanklänning är en mörk eller ljus klänning av lättare tyg, vanligen obetydligt urringad och med långa ärmar.

\subsubsection*{\textbf{Vardagsklänning}}
Vardagsklänning är den fullkomligt valfria klänningen eller kjol och blus kombinerad klädsel.

\subsection*{\textbf{Ovverallen}}
Studentoverallen, kännetecket för den studentikosa kulturen dokumenterades först på ett omslag på vinylskivan från AB Kruthornen "Dancing With AB Kruthornen" 1966   
där ses de bärandes deras rödråsa overaller. 

Overallen bytte ut den tidigare B-fracken som användas som en "slit och släng" frack så att sin finare frack skulle hålla sig fin till de tillfällen
den krävdes.

Traditionen att besmycka sina studentkläder sträcker sig längre än ovverallens historia och har varit integral till studentlivet.

Regler om när, var, hur man bär sin ovverall är olika beroende på tid och plats. Här under defineras ett antal som är relevanta 
för året av bokens tryckning och för LTD på MDU. Skulle det vara så att ditt år eller förening divergeras från dessa ta kontakt med 
någon gammal räv som kan mer än du.

\begin{itemize}
    \setlength{\itemindent}{0em}
    \item Ovven bärs oftast "nercabbad" (som ett par byxor med överdelen hängandes).
    \item Ovven får inte bäras nercabbad innan den är invigd.
    \item Ditt ovve-namn (vid brist av ovve-namn, införskaffa ett sådant eller använd ditt egna.) bör stå skrivet på vänster framsida ben.
    \item Din NØlleslips (eller varierande NØllnings symbol) bör sys fast på vänster baksida ben.
    \item Bärs Ovven nercabbad bör ärmarnas ändar vara knutna så att de fungerar som fickor.
    \item kapsyl och/eller vinöppnare bör alltid finnas i/på ovven.
    \item Börjar hål skapas någonstans på ovven uppmuntras märken att användas för lapplagning.
    \item Är hålen stora, använd större märken.
    \item Är hålen på tok för stora, gör om ovven till en rock.
\end{itemize}



\newpage