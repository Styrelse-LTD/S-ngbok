\fancypagestyle{Att Bonka med Shtil}{
    \fancyhead{} % clear all header fields
    \fancyhead[LE,RO]{\textbf{Att Bonka med Shtil}}
}
\pagestyle{Att Bonka med Shtil}

\section*{\textbf{Att Bonka med Shtil}}

På de flesta tillställningar av studentikos natur är det lämpligt att bekläda sig med overall. Overallen är det oslagbara instrumentet när det gäller raggning, allmän ösning, bonkning  mm. Det kan dock, då så påbjudes, vara nödvändigt att klä sig enligt andra seder och bruk än rent djuriska. Här nedan pekar vi på det viktigaste för att bli accepterad bland det klientel vi en gång i framtiden (hemska tanke) kanske kommer att vilja tillhöra.

\subsection*{\textbf{Klädselbeteckningar}}
Högtidsdräkt innebär frackklädsel eller paraduniform samt å spinnsidan galaklänning. Aftondräkt är i första hand smoking eller aftonklänning. Men i andra hand kan likväl begagnas lunchcoat med svarta byxor utan revärer och mellanklänning. Frack och galaklänning godtages såsom aftondräkt. Då för vissa lokaler föreskrivits aftondräkt för besökande, tillåtas i allmänhet även blå kavajkostym med vit skjorta och krage samt promenadklänning. Kavaj är blå eller mörk kavajkostym med vit skjorta och krage samt\\* vardagsklänning. Vid bjudningar angives stundom: "klädsel: kavaj". Detta skall tolkas såsom nyss angiven klädsel med mörk kavaj respektive mellanklänning.

\subsection*{\textbf{Klädsel för Herrar}}
\subsubsection*{\textbf{Frack}}
\begin{itemize}

    \item[]\textbf{Frackrock:} svart rock med "svalstjärt", långa slag med eller utan sidenbeläggning, 2 knappar i ryggen, med eller utan bröstficka, framsidans undre kanter i höjd med byxlinningens övre framkant; rocken skall sitta spänt och sluta åt strikt utan rynkor och skall framtill kunna knäppas precis. Bäres aldrig knäppt annars än möjligheten med bandögla, vilken förenar rocksidorna, vit bandögla för vit väst, svart för svart väst. Stickerier i spegel på slaget anger akademisk doktorsgrad. På ärmarna lämnas manchetterna synliga, västen synes framtill under sidostyckerna. Då man sätter sig vikas skärten undan.

    \item[]\textbf{Frackbyxor:} Svarta långbyxor med svarta revärer, byxorna nedtill snett skurna, så att de baktill når klackens överkant och framtill utan rynkor ligga över foten men aldrig över hela foten.

    \item[]\textbf{Frackväst:} vit urringad väst med eller utan slag. Västen synes framtill nedanför rocksidorna.

    \item[]\textbf{Skor:} svarta lackskor eller möjligheten prydliga inneskor.

    \item[]\textbf{Skjorta:} vit stärkskjorta med styvt veck.

    \item[]\textbf{Krage:} enkel stärkkrage med snibbar.

    \item[]\textbf{Halsduk:} vit rosett.

    \item[]\textbf{Näsduk:} vit silkesnäsduk, bäres eventuellt i bröstfickan och då synlig som ett vitt band ovan fickans kant.

    \item[]\textbf{Strumpor:} svarta silkenstrumpor.

    \item[]\textbf{Smycken:} skjortknappar i vitt eller guld, manschettknappar i samma färg, klockkedja bäres hängande från hängselstropp till vänster byxficka.

    \item[]\textbf{Överplagg:} mörk paletå (överrock utan skärp eller ryggleif) och vit kragskyddare. Hög hatt eller chapeau-claque, vilket senare hopfälld medföres inomhus.

    \item[]\textbf{Handskar:} vita glacéhandskar. Inomhus bäres handskar ofta i handen. Vid bal och uppvaktning bäres de pådragna.

\end{itemize}

\newpage

Det finns tre saken en riktig man \textbf{aldrig} gör:

\textit{Han ger aldrig upp}

\textit{Han ljuger inte}

\textit{Han ber aldrig om ursäkt}

\textit{\textbf{-Zeb Macahan}}

\subsubsection*{\textbf{Smoking}}

\begin{itemize}

    \item[]\textbf{Smokingrock:} svart rock med\\* sidenbeläggning, rocken med eller utan\\* bröstficka, sidofickor med eller utan lock. Rocken knäppes endast upp i sittande ställning, dubbelknäppt rock knäpps aldrig upp.

    \item[]\textbf{Smokingbyxor:} samma som frackbyxor.

    \item[]\textbf{Smokingväst:} svart urringad väst med eller utan slag, i det senare fallet stundom med stärkt linnebård, västens och rockens urringningar lika djupa. Vit väst bäres icke.

    \item[]\textbf{Skor:} samma som till frack.

    \item[]\textbf{Skjorta:} vit stärkskjorta med styvt veck, eventuellt lättare stärkt eller mjuk för bruk vid danstillställning.

    \item[]\textbf{Krage:} enkel stärkkrage med snibbar.

    \item[]\textbf{Halsduk:} svart rosett.

    \item[]\textbf{Näsduk:} vit silkesnäsduk i bröstfickan som ett smalt band synlig ovan dess kant.

    \item[]\textbf{Strumpor:} svarta silkesstrumpor.

    \item[]\textbf{Smycken:} knappar såsom till frack, klockkedja över västen. Vid begravning svarta knappar. Ordnar får inte bäras till smoking.

    \item[]\textbf{Överplagg:} Mörk paletå med vit kragskyddare.

    \item[]\textbf{Handskar:} Vita glacéhandskar eller vita gants de suéde. Handskar bäres inte inomhus.

\end{itemize}
Kuriosa om Smokingen

I Amerika kallas smokingen för "Tuxedo". Enligt kufiska källor tillkom företeelsen tuxedo när ett gäng amerikaner hade en riktig blöt tillställning på "The Tuxedo Country Club". Allt eftersom natten led tyckte deltagarna att deras frackar blev något otympliga med svalstjärten hängandes efter benen. Styrkta av det myckna drickandet, klippte de av svalstjärtarna. Så lätt skapar man ett nytt plagg!

\subsubsection*{\textbf{Kavajkostym}}

Färg, snitt och utförande beroende av egen smak, uppfinningsrikedom och personlig karaktär. Mörk kavajkostym är en för alla vardagstillfällen lämpad kostym. Såsom arbetskostym användes den med vit krage och vit skjorta. Alltid svarta skodon. Sommartid kommer denna kostym till användning i stället för smoking, där den icke direkt påbjudes. Kulört kavajkostym är utpräglad vardagsdräkt men tillåten vid mindre vardagsbjudning. Till kostym hör kulört skjorta och krage samt ofta bruna skodon.

\textit{"Spotta inte på eller över bordet, det kan tolkas som om sällskapet inte faller en i smaken"}\\* - \textbf{Ur en bok om bordsskick från 1300-talet}

\subsection*{\textbf{Klädsel för Damer}}
\subsubsection*{\textbf{Galaklänning}}
Galaklänning innebär ljus hellång klänning med släp, vanligen urringad och helt utan ärmar. Medaljer och ordnar bäres. Hovdräkten, antigen helsvart eller helvit, är urringad och försedd med puffärmar och släp.

\subsubsection*{\textbf{Aftonklänning}}
Aftonklänning är mörk eller ljus, numera ofta kort, klänning urringad och ärmlös. Ofta hör en jacka till klänningen. Medaljer och ordnar bäres.

\subsubsection*{\textbf{Mellanklänning}}
Mellanklänning är en mörk eller ljus klänning av lättare tyg, vanligen obetydligt urringad och med långa ärmar.

\subsubsection*{\textbf{Vardagsklänning}}
Vardagsklänning är den fullkomligt valfria\\* klänningen eller kjol och blus kombinerad klädsel.

\subsection*{\textbf{Ovverallen}}
Studentoverallen, kännetecket för den studentikosa kulturen dokumenterades först på omslaget på vinylskivan från AB Kruthornen "Dancing With AB Kruthornen" 1966   
där ses de bärandes deras rödråsa overaller. 

Overallen bytte ut den tidigare B-fracken som användes som en "slit och släng" frack så att sin finare frack skulle hålla sig fin till de tillfällen
den krävdes.

Traditionen att besmycka sina studentkläder sträcker sig längre än ovverallens historia och har varit en vital del av studentlivet.

Regler om när, var, hur man bär sin ovverall är olika beroende på tid och plats. Här under defineras ett antal som är relevanta 
för bokens tryckningsår och för LTD på MDU. Skulle det vara så att ditt år eller förening avviker från dessa ta kontakt med 
någon gammal räv som kan mer än du.


%\begin{itemize}
%    \setlength{\itemindent}{0em}
%    \item Ovven bärs oftast "nercabbad" (som ett par byxor med överdelen hängandes).
%    \item Ovven får inte bäras nercabbad innan den är invigd.
%    \item Ditt ovve-namn (vid brist av ovve-namn, införskaffa ett sådant eller använd ditt egna.) bör stå skrivet på vänster framsida ben.
%    \item Din NØlleslips (eller varierande NØllnings symbol) bör sys fast på vänster baksida ben.
%    \item Bärs Ovven nercabbad bör ärmarnas ändar vara knutna så att de fungerar som fickor.
%    \item kapsyl och/eller vinöppnare bör alltid finnas i/på ovven.
%    \item Börjar hål skapas någonstans på ovven uppmuntras märken att användas för lapplagning.
%    \item Är hålen stora, använd större märken.
%    \item Är hålen på tok för stora, gör om ovven till en rock.
%\end{itemize}
\textbf{OVVE-REGLER}
\begin{itemize}
    \setlength\itemsep{-0.8em}
    \item Ovven är rätt plagg för alla tillfällen förutom dop, begravning och giftemål (om inget annat angivits).\\
    \item Ovven är ditt finaste plagg.\\
    \item Slipsen skall sättas på vänster bakben eller under ryggtrycket.\\
    \item Namn ska sättas på vänster framben eller över ryggtrycket.\\
    \item Ovven har en liten "flärp", detta är din ovve-oskuld. Du kan be någon riva av den med tänderna. Detta är ofantligt intimt.\\
    \item Skulle du ha en annan ovve (tidigare studier / program) kan ovven splittas på mitten\\
    \item Föreningsmärket skall sitta på vänster bröstficka (över hjärtat) och valfritt annat föreningsmärke skall sitta på högra.\\
    \item Revärer (vanligtvis av zebra tyg) sys längst benen efter 180-hp.\\
    \item Ovven får inte bäras nercabbad innan den är invigd.\\
    \item Ovven får inte tvättas om man själv inte bär den.\\
    \item Ovanstående får brytas mot om ovven innehåller tre olika typer av kroppsvätskor av tre olika personer.\\
    \item Det ses som ovårdat att använda lim för att fästa sina märken på ovven.\\
    \item Ovven skall alltid innehålla någon mängd alkohol.\\
\end{itemize}

%\begin{tabular}{p{\textwidth}}
%    \textbf{OVVE-REGLER}\\
%    $\cdot$ Ovven är rätt plagg för alla tillfällen förutom dop, begravning och giftemål (om inget annat angivits).\\
%    $\cdot$ Ovven är ditt finaste plagg.\\
%    $\cdot$ Slipsen skall sättas på vänster bakben eller under ryggtrycket.\\
%    $\cdot$ Namn ska sättas på vänster framben eller över ryggtrycket.\\
%    $\cdot$ Ovven har en liten "flärp", detta är din ovve-oskuld. Du kan be någon riva av den med tänderna. Detta är ofantligt intimt.\\
%    $\cdot$ Skulle du ha en annan ovve (tidigare studier / program) kan ovven splittas på mitten\\
%    $\cdot$ Föreningsmärket skall sitta på vänster bröstficka (över hjärtat) och valfritt annat föreningsmärke skall sitta på högra.\\
%    $\cdot$ Revärer (vanligtvis av zebra tyg) sys längst benen efter 180-hp.\\
%    $\cdot$ Ovven får inte bäras nercabbad innan den är invigd.\\
%    $\cdot$ Ovven får inte tvättas om man själv inte bär den.\\
%    $\cdot$ Ovanstående får brytas mot om ovven innehåller tre olika typer av kroppsvätskor av tre olika personer.\\
%    $\cdot$ Det ses som ovårdat att använda lim för att fästa sina märken på ovven.\\
%    $\cdot$ Ovven skall alltid innehålla någon mängd alkohol.\\
%\end{tabular}
\textbf{OVVE-BYTE AV DELAR} 
\begin{itemize}
    \setlength\itemsep{-0.8em}
    \item Benrand: Samlag\\
    \item Ärmrand: Utbyte av kroppsvätska\\
    \item Krage: Förhållande.\\
    \item Bröstficka flärp: Förlovning \\
    \item Bröstficka hel: giftemål\\
    \item Bakficka vänster: Analsex (mottaget)\\
    \item Bakficka höger: Analsex (givet)\\
\end{itemize}
%\begin{tabular}{p{\textwidth}}
%    \textbf{OVVE-BYTE AV DELAR} \\
%    $\cdot$ Benrand: Samlag\\
%    $\cdot$ Ärmrand: Utbyte av kroppsvätska\\
%    $\cdot$ Krage: Förhållande.\\
%    $\cdot$ Bröstficka flärp: Förlovning \\
%    $\cdot$ Bröstficka hel: giftemål\\
%    $\cdot$ Bakficka vänster: Analsex (mottaget)\\
%    $\cdot$ Bakficka höger: Analsex (givet)\\
%\end{tabular}

\subsection*{\textbf{Schmäcken}}
Så som historien går, sägs det att Chalmerister 1878 fått nog av att studentmössan inte  längre var något akademiskt exklusivt.
Som starka män tog de saken i egna händer, sålunda föddes\\* teknologmössan. en studentmössa med en assymetrisk mösskulle, så att
en nedhängande plös bildas på höger sida i vilkens ände en stor prydlig tofs monterades. På MDU heter teknologmössan även schmäcken. Sydd i finaste mörkblå sammet.
Den skall bäras på det mest upprätta sätt för att visa allmänheten att man är en stolt teknolog.

\subsubsection*{\textbf{Spegatter}}
En spegatt är en ändlös flätad sidenknut. Det flätade snöre schmäckens tofs hänger i skall fyllas av spegatter.
Spegatterna som är baserade på LTDs progam är naturligtvis högvördigt lila, helt för robotik, med ett silverstreck tillförlitliga system,
med en ljusare nyans av lila för teknisk matematik och ett orangt streck för de som är aktiva inom ett av föreningens utskott.\\\\ 
\begin{tabular}{p{\textwidth}}
    \textbf{SPEGATT-REGLER} \\
    $\cdot$ En spegatt per påbörjat läsår, i konologisk ordning, svart för sabbatsår.\\
    $\cdot$ Spegatterna närmast plösen: Vill ha hela inne.\\
    $\cdot$ Spegatterna på mitten: Vill ha halva inne.\\
    $\cdot$ Spegatterna närmast tofsen: Vill ha hela utanför \\
    $\cdot$ Knut på snöret: Upptagen.\\
\end{tabular}

\subsubsection*{\textbf{Tofsen}}
En teknolog med en fransig tofs är en ledsen teknolog. tofsens hundratals fransar måste förseglas på något sätt. Likt limmade patcher är det fusk att smälta
ändarna med en tändare. Första alternativet är att doppa varje ände i lim eller nagellack, författarna önskar läsaren all lycka till med att utföra detta
företag utan att klistra ihop två fransar. Ett mer rationellt sätt att försegla tofsen, är att doppa dess ände i ett glas punsch. glaset skall innehålla 
en godtycklig mängd punch á $n$ centiliter. då doppningen behöver göras om med jämna mellanrum har denna metod den fantastiska fördelen att den överblivna 
punschen kan avnjutas. Det rätta sättet är att hitta en individ, företrädesvis hållfast sällskap. I brist på hållbart sällskap, någon trevlig man har halva inne med.
Saknas båda? Åtgärda det först. När individen är funnen, sätt denna på att knyta änden av varje frans. Detta till det högst rättvisa priset av en kyss per knut.
Arbetet är mödosamt och utdraget. Precis som en teknologs tillvaro. När arbetet äntligen är färdigt återstår ett glas $n$ cl punch á $n =\int_{-1}^{1} \frac{dx}{\sqrt{1-x^2}}$
när punchen är drucken är schmäcken formellt invigd. Det är först då kan den bäras med stolthet.\\*\\*
\rotatebox[origin=c]{180}{ska inte behöva känna dig utanför. svaret är: $n = \pi$}\\*
\rotatebox[origin=c]{180}{skanderat analytiska termer efter Olof Bergvall}\\*
\rotatebox[origin=c]{180}{Du som failade envarren efter att ha}

\newpage

\subsection*{\textbf{N0lleslipsen}}
Från första dagen tills n0llningsperiodens slut och dygnet runt ska en lila tygslips vara knuten runt halsen (tyg, papper innebär ett helgerån).
På slipsen skall finnas.\\
\begin{tabular}{p{\textwidth}}
    $\cdot$ N0llans namn\\
    $\cdot$ Texten "DAT Ø"\\
    $\cdot$ Ett H eller V ifall n0llan är höger eller vänsterhänt\\
    $\cdot$ Ett schema att kryssa av n0llningsaktiviteter i, 24 rutor. Samt en specialruta märkt Ø, för avslutat n0lleuppdrag\\
    $\cdot$ En slips är 20 cm bred
\end{tabular}

\subsection*{\textbf{Tentamen}}
Under en students tid i universitetslivet så kommer de att stöta på ett antal tentamen. Dessa fasansfylda cermonier händer ca fyra gånger per år och är livsviktiga för studentens 
livnärande. Om studenten skulle falla på den lägre sidan av balansgången mot att få exakt godkänt så kommer denne student få en så kallad om-tentamen. Skulle detta hände att antal gånger 
ändras riten under följande sätt.

\begin{tabular}{p{\textwidth}}
    5 försök: Ovven skall bäras under tentamen\\
    10 försök: Balklädsel skall bäras under tentamen\\
    25 försök: En kandelabra skall tas med och helst tändas under tentamen\\
    50 försök: Riddarrustning skall bäras under tentamen
\end{tabular}


\newpage