
\section{Spel \& lekar}

\fancypagestyle{Spel & lekar}{
\fancyhead{} % clear all header fields
\fancyhead[LE,RO]{\textbf{Spel \& lekar}}
}
\pagestyle{Spel & lekar}



\subsection*{\textbf{Fuck U}}

//Från bonkboken känns som en gammel släkting till ring of fire???

Dra kort i tur från en kortlek. Följ regeln från listan under 
och möjligtvis egna ettablerade regler (se 2)

Regler:

Ess: Dela ut en klunk (Var sportslig)

2: Regelkort. Hitta på en regel eller ändra/tabort en regel. \textbf{EJ grundreglerna eller något personligt.} 

3-6: Dela ut antalet klunkar som det står på kortet.

7: Räknekort. Den som drar kortet börjar med att högt säga "1" 
och den vänster fortsätter då med "2". 
Så håller man på tills någon antigen säger något som är 
delbart med 7 eller innehåller en 7:a. t.ex. 14 eller 17. 
Missar man eller tvekar drickar man.

8: Pissekort. Sparas tills man använder det. Man lägger in kortet om man måste på toa.

9: Tumkort. Sparas tills det används. 
Den som har kortet sätter när de vill upp sin tumme på bordet 
och håller den där och resten följer efter. 
Sist dricker

10: "Fuck you"-kort.
Sparas tills det används. Den som har kortet säger han de vill "Fuck U", den som säger det sist dricker.

Kn: Frågekort. Den som drar detta börjar med att ställa fråga till någon 
och den personen ska direkt ställa en nu fråga till någon 
annan som spelar. Ställer man tillbaka en fråga, tvekar, svarar 
eller ställer en fråga so mredan ställts dricker man.

D: Temakort. Man säger ett tema och så går man runt i
ringen och alla får säga något som har med temat att göra.
Kan man inte, tvekar eller om man säger något som redan sagts dricker man.

K:Sparas till alla kungar dragits. Om en person får alla fyra kungar får alla andra svepa vad de har i glaset/flaskan.
Annars får den som drar sista kungen svepa.
\subsection*{\textbf{Caps}}

//Lägg till västerås Caps regler. Kanske nämn utlandsregler??

\subsection*{\textbf{Threeman}}

Threeman är ett spel med ett bräde, tärningar och en hattdjävel. När man blir
tilldelad klunkar av någon som kan spelet ska man dricka dessa.

Regler:

\begin{enumerate}
    \item Man får endast berätta den första regeln
    \item[...]
\end{enumerate}

\subsection*{\textbf{Fuck the Dealer}}

Tillbehör:

Folk, dryck, kortlek

En dealer kollar på det översta kortet i en kortlek och frågar personen till vänster vilken valör kortet har. Gissar dom rätt dricker dealern 10 klunkar. Gissar dom fel säger dealern om kortet är högre eller lägre än gissningen. personen gissar då igen. Gissar dom rätt dricker dealern 5 klunkar. Gissar dom fel så dricker spelaren mellan skillnaden av deras gissning och kortet. Kortet läggs sedan på bordet för alla att se. 

Lyckas dealern inte bli tilldelar klunkar av tre personen i rad flyttas kortleken vidare medsols och en den nya dealern börjar där den gammla slutade.

Variant:

Kung eller ess

Eftersom hur mycket man själv dricker är beroende på Deltat av kortet som drogs och sin gissning spelar LTD ofta under förståelsen att det bara kan finnas kungar och ess i leken. Även om detta inte visar sig vara sant när kortet vänder på sig.

\newpage