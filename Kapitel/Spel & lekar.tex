
\invisiblesection{Spel \& lekar}

\fancypagestyle{Spel & lekar}{
\fancyhead{} % clear all header fields
\fancyhead[LE,RO]{\textbf{Spel \& lekar}}
}
\pagestyle{Spel & lekar}



\invisiblesubsection{\textbf{Fuck U}}

//Från bonkboken känns som en gammel släkting till ring of fire???


\invisiblesubsection{\textbf{Caps}}

//Lägg till västerås Caps regler. Kanske nämn utlandsregler??

\invisiblesubsection{\textbf{Threeman}}

Threeman är ett spel med ett bräde, tärningar och en hattdjävel. När man blir
tilldelad klunkar av någon som kan spelet ska man dricka dessa.

Regler:

\begin{enumerate}
    \item Man får endast berätta den första regeln
    \item[...]
\end{enumerate}

\invisiblesubsection{\textbf{Fuck the Dealer}}

Tillbehör:

Folk, dryck, kortlek

En dealer kollar på det översta kortet i en kortlek och frågar personen till vänster vilken valör kortet har. Gissar dom rätt dricker dealern 10 klunkar. Gissar dom fel säger dealern om kortet är högre eller lägre än gissningen. personen gissar då igen. Gissar dom rätt dricker dealern 5 klunkar. Gissar dom fel så dricker spelaren mellan skillnaden av deras gissning och kortet. Kortet läggs sedan på bordet för alla att se. 

Lyckas dealern inte bli tilldelar klunkar av tre personen i rad flyttas kortleken vidare medsols och en den nya dealern börjar där den gammla slutade.

Variant:

Kung eller ess

Eftersom hur mycket man själv dricker är beroende på Deltat av kortet som drogs och sin gissning spelar LTD ofta under förståelsen att det bara kan finnas kungar och ess i leken. Även om detta inte visar sig vara sant när kortet vänder på sig.

\newpage