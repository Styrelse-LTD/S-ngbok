\fakesection{Vinvisor}
\fancypagestyle{Vinvisor}{
    \fancyhead{} % clear all header fields
    \fancyhead[LE,RO]{\textbf{Vinvisor}}
}
\pagestyle{Vinvisor}
\begin{SongText}[Bordeaux, Bordeaux]
    \begin{SongInfo}
        Mel: I sommarens soliga dag
    \end{SongInfo}
    \begin{SongVerse}
        Jag minns än idag hur min fader\\*%
        kom hem ifrån staden så glader\\*%
        och stälde upp flaskor i rader\\*%
        och sade nöjd som så: ”Bordeaux, Bordeaux”
    \end{SongVerse}
    \begin{SongVerse}
        Han drack ett glas\\*%
        kom i extas\\*%
        och sedan blev det stort kalas\\*%
        och vi små glin\\*%
        ja, vi drack vin\\*%
        som första klassens fyllesvin\\*%
        och vi dansade runt där på golvet\\*%
        och skrek så vi blev blå:\\*%
        Bordeaux, Bordeaux
    \end{SongVerse}
\end{SongText}
\begin{SongText}[Feta fransyskor]
    \begin{SongInfo}
        Mel: Tomtarnas julmarsch
    \end{SongInfo}
    \begin{SongVerse}
        Feta fransyskor som svettas om fötterna\\*%
        De trampar druvor som sedan skall jäsas till vin\\*%
        Transpirationen viktig e’\\*%
        Ty den ger fin boqué.\\*%
        Vårtor och svampar följer me’,\\*%
        Men vad gör väl de’?
    \end{SongVerse}
    \begin{SongVerse}
        För...\\*%
        Vi vill ha vin, vill ha vin, vill ha mera vin\\*%
        även om följderna blir att vi må lida pin.\\*%
        Flickor: Flaska och glaset gått i sin.\\*%
        Pojkar: Hit med vin, mera vin.\\*%
        Flickor: Tror ni att vi är fyllesvin?\\*%
        Pojkar: JA!, fast större!
    \end{SongVerse}
\end{SongText}
\begin{SongText}[Vin i glasen]
    \begin{SongInfo}
        Mel: Vind i seglen
    \end{SongInfo}
    \begin{SongVerse}
        Vin i glasen\\*%
        Har vi fått och det smakar gott.\\*%
        Smått i gasen\\*%
        Blir vi nu alla som ett skott.\\*%
        Ta och öppna din mun du skall tanka,\\*%
        Ja nu vankas ett glas Villa France.\\*%
        Låt det rinna\\*%
        I en bångstyrig liten ström,\\*%
        Så som en dröm, så öm. Så töm!\\*%
        Ja låt tankarna slajda\\*%
        Nu i lite saida\\*%
        Hej hopp och Saidaidaida!\\*%
        Ja det smakar bra.
    \end{SongVerse}
\end{SongText}
\begin{SongText}[Hädiska tanke]
    \begin{SongInfo}
        Mel: Det var dans bort vid vägen
    \end{SongInfo}
    \begin{SongVerse}
        Det var fan tänkte Moses vad jag är full\\*%
        När han med Guds tio budord han ramlade kull\\*%
        Men han räddade kruset med vin.\\*%
        Det var jävlar i mig det ett riktigt kalas\\*%
        Med två stentunga tavlor sim visst gick i kras,\\*%
        Tänkte Moses och drog på ett flin\\*%
        Gudskelov att jag rädda’ mitt vin.
    \end{SongVerse}
\end{SongText}
\begin{SongText}[Crassus vinsång]
    \begin{SongInfo}
        Mel: Mors lilla Olle
    \end{SongInfo}
    \begin{SongVerse}
        Kom alla vänner, drick ur ert glas.\\*%
        Ni vet väl alla, hur vinet ska tas?\\*%
        Först med din näsa Du känner Bouquén,\\*%
        Då kan du skilja det röda från rosén.
    \end{SongVerse}
    \begin{SongVerse}
        Sen med Din tunga Du läppjar ditt vin.\\*%
        Vad sägs om aromen, nog är den väl fin?\\*%
        Sedan Du munnen med saft har blött,\\*%
        Då kan Du svälja det vin som är rött.
    \end{SongVerse}
    \begin{SongVerse}
        Efter ett glas eller två mår Du bra.\\*%
        Du sjunger och skrålar, att mer vill Du ha.\\*%
        Alla kan se, att Du börjar bli sne.\\*%
        Synd att du dricker en billig Châtelet.
    \end{SongVerse}
    \begin{SongVerse}
        Så lyft nu glasen upp till skål.\\*%
        För resten i kupan nog säkert Du tål.\\*%
        Det är vad vi kallar dryckeskultur:\\*%
        Seså, krök nu armen, ja drickom, drick ur!
    \end{SongVerse}
\end{SongText}
\newpage