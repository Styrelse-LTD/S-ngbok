\fakesection{Kräftskivan}
\fancypagestyle{Kräftskivan}{
    \fancyhead{} % clear all header fields
    \fancyhead[LE,RO]{\textbf{Kräftskivan}}
}
\pagestyle{Kräftskivan}
\begin{SongText}[Kräfta, kräfta prydd med dill]
    \begin{SongInfo}
        Mel: Blinka lilla stjärna
    \end{SongInfo}
    \begin{SongVerse}
        Kräfta, kräfta prydd med dill\\*%
        och en immig sup därtill,\\*%
        bröd och smör och ost och sill,\\*%
        och så några supar till.\\*%
        Kräfta, kräfta prydd med dill,\\*%
        nu vi fått allt vad vi vill.
    \end{SongVerse}
\end{SongText}
\begin{SongText}[Kräftvisa]
    \begin{SongInfo}
        Mel: En sockerbagare
    \end{SongInfo}
    \begin{SongVerse}
        Å detta brännvin som alltid jäklas\\*%
        när framåt natten man måste kräkas,\\*%
        men kräftor har ju så hårda skal,\\*%
        de kan ju fastna i ens anal.
    \end{SongVerse}
\end{SongText}
\begin{SongText}[Kräftälskaren]
    \begin{SongInfo}
        Mel: En sockerbagare
    \end{SongInfo}
    \begin{SongVerse}
        En stor kräftälskare här bor i staden,\\*%
        han äter kräftor mest hela dagen,\\*%
        han äter stora, men inte små,\\*%
        han äter en del med skalet på.\\*%
        Och vid hans kräftbord där flödar spriten,\\*%
        man sjunger sånger om akvaviten.\\*%
        Och är du snäller så kan du få\\*%
        en liten kräftklo att suga på.
    \end{SongVerse}
\end{SongText}
\begin{SongText}[Läckert röda ]
    \begin{SongInfo}
        Mel: Kors på Idas grav
    \end{SongInfo}
    \begin{SongVerse}
        Läckert röda många kräftor ligga här,\\*%
        utan möda ett par tjog vi friskt förtär.\\*%
        Glasen imma, kräftor simma\\*%
        bäst i brännvin som vi vet,\\*%
        lyktor glimma\\*%
        känsligt mot all härlighet.
    \end{SongVerse}
\end{SongText}
\begin{SongText}[Nu samlas vi alla kring faten]
    \begin{SongInfo}
        Mel: Albertina
    \end{SongInfo}
    \begin{SongVerse}
        Nu samlas vi alla kring faten,\\*%
        omkring faten med kräftor och med dill - mycke’ dill!\\*%
        Nu vi skall i kvällen särla,\\*%
        njuta stjärtar till en pärla,\\*%
        när vi tatt den, så tar vi strax en till - ta en till!\\*%
        (Här tages under andakt en sup.)\\*%
        Nu pärlan är allredan tagen,\\*%
        den är tagen och längtar efter fler - fram med fler!\\*%
        Men till dess vi skall oss gona\\*%
        med en stjärt utav en hona,\\*%
        det är också en sak som smakar mer - smakar mer!\\*%
        (Här taga alla en stjärt i handen.)\\*%
        Ja, hell dig, du stjärt ibland stjärtar,\\*%
        utan dig vore livet trist och grått - bara grått!\\*%
        Men när nu din stjärt vi smeker\\*%
        hela livet liksom leker,\\*%
        du som tycker så mycket bra om vått - mycket vått!
    \end{SongVerse}
\end{SongText}
\begin{SongText}[Snapsflaska, dillkrona]
    \begin{SongInfo}
        Mel: Sjösala vals
    \end{SongInfo}
    \begin{SongVerse}
        Hejsan alla vänner av en hutt och en klo,\\*%
        känner ni receptet som hör till en kräftskiva?\\*%
        Jo, det gör vi alla som är här, ska ni tro:\\*%
        Kräftan ska va’ hona, en fullväxt och go’!\\*%
        Mitt på bordet står hon i jättestor karott\\*%
        bland smör och bröd och ost och en massa annat gott\\*%
        men viktigast av allt är att värden har ställt fram på\\*%
        bordet, snapsflaska, dillkrona, korkskruv och mycke’ glas!
    \end{SongVerse}
\end{SongText}
\newpage