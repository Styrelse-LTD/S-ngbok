\fakesection{Patriotiska sånger}
\fancypagestyle{Patriotiska sånger}{
    \fancyhead{} % clear all header fields
    \fancyhead[LE,RO]{\textbf{Patriotiska sånger}}
}
\pagestyle{Patriotiska sånger}

Vid sjungande av svenska nationalsången gäller följande:

1. Stå upp.

2. Inga händer i byxfickorna.

3. Huvudbonad avtages.

\begin{SongText}[Du Gamla, Du fria]
    \begin{SongVerse}
        Du gamla, Du fria, Du fjällhöga nord\\*% 
        Du tysta, Du glädjerika sköna!\\*% 
        Jag hälsar Dig, vänaste land uppå jord,\\*% 
        $\|\:$ Din sol, Din himmel, Dina ängder gröna. $\:\|$
    \end{SongVerse}
    \begin{SongVerse}
        Du tronar på minnen från fornstora dar,\\*% 
        då ärat Ditt namn flög över jorden.\\*% 
        Jag vet att Du är och förblir vad du var.\\*% 
        $\|\:$ Ja, jag vill leva jag vill dö i Norden. $\:\|$
    \end{SongVerse}
    \begin{SongVerse}
        Jag städs vill dig tjäna mitt älskade land,\\*% 
        din trohet till döden vill jag svära.\\*% 
        Din rätt, skall jag värna, med håg och med hand,\\*% 
        $\|\:$ din fana, högt den bragderika bära. $\:\|$
    \end{SongVerse}
    \begin{SongVerse}
        Med Gud skall jag kämpa, för hem och för härd,\\*% 
        för Sverige, den kära fosterjorden.\\*% 
        Jag byter Dig ej, mot allt i en värld\\*% 
        $\|\:$ Nej, jag vill leva jag vill dö i Norden. $\:\|$
    \end{SongVerse}
\end{SongText}
%\begin{SongText}[Kungssången]
%    \begin{SongInfo}
%        Text: C V A Strandberg
%    \end{SongInfo}
%    \begin{SongVerse}
%        Ur svenska hjärtans djup en gång\\*% 
%        en samfälld och en enkel sång,\\*% 
%        som går till kungen fram!\\*% 
%        Var honom trofast och hans ätt,\\*% 
%        gör kronan på hans hjässa lätt,\\*% 
%        och all din tro till honom sätt,\\*% 
%        du folk av frejdad stam!
%    \end{SongVerse}
%    \begin{SongVerse}
%        O konung, folkets majestät\\*% 
%        är även ditt. Beskärma det\\*% 
%        och värna det från fall!\\*% 
%        Stå oss all världens härar mot,\\*% 
%        vi blinka ej för deras hot,\\*% 
%        vi lägga dem inför din fot,\\*% 
%        en kunglig fotapall.
%    \end{SongVerse}
%    \begin{SongVerse}
%        Du himlens Herre, med oss var,\\*% 
%        som förr Du med oss varit har,\\*% 
%        och liva på vår strand\\*% 
%        det gamla lynnets art igen\\*% 
%        hos Sveakungen och hans män,\\*% 
%        och låt Din ande vila än\\*% 
%        utöver nordanland
%    \end{SongVerse}
%\end{SongText}
%\begin{SongText}[Sveriges Flagga]
%    \begin{SongInfo}
%        Text: K.G. Ossiannilsson\\*%
%        Musik : Hugo Alfvén
%    \end{SongInfo}
%    \begin{SongVerse}
%        Flamma stolt mot dunkla skyar\\*%
%        lik en glimt av sommarens sol!\\*%
%        Över Sveriges skogar, berg och byar,\\*%
%        över vatten och viol!\\*%
%        Du som sjunger, när Du bredes\\*%
%        som vår gamla lyckas tolk.\\*%
%        Solen lyser! Solen lyser!\\*%
%        Ingen vredes åska slog vårt tappra folk!
%    \end{SongVerse}
%    \begin{SongVerse}
%        Flamma högt vårt kärlekstecken!\\*%
%        Värm oss, när det blåser kallt!\\*%
%        Ståla ut de blåa vecken\\*%
%        kärlek mera stark än allt!\\*%
%        Sveriges flagga! Sveriges ära!\\*%
%        Fornklenod och framtidstolk!\\*%
%        Gud är med oss! Gud är med oss!\\*%
%        Han skall bära stark vårt fria svenska folk
%    \end{SongVerse}
%\end{SongText}
%\begin{SongText}[Vårt land]
%    \begin{SongInfo}
%        (Finska nationalsången)
%        Text: Johan Ludvig Runeberg
%    \end{SongInfo}
%    \begin{SongVerse}
%        Vårt land, vårt land, vårt fosterland\\*%
%        $\|\:$ljud högt, o dyra ord!$\:\|$\\*%
%        Ej lyfts en höjd mot himels rand,\\*%
%        Ej sänkes en dal, ej sköljs en strand,\\*%
%        $\|\:$Mer älskad än vår bygd i nord, Än våra fäders jord.$\:\|$
%    \end{SongVerse}
%    \begin{SongVerse}
%        Din blomning, sluten än i knopp, skall mogna ur sitt tvång.\\*%
%        Se ur vår kärlek skall gå i opp ditt ljus, din glans, din fröjd\\*%
%        ditt hopp, och högre klinga skall en gång vår fosterländska sång.
%    \end{SongVerse}
%\end{SongText}
%\begin{SongText}[Sverige]
%    \begin{SongInfo}
%        Text: Verner von Heidenstam
%    \end{SongInfo}
%    \begin{SongVerse}
%        Sverige, Sverige,\\*%
%        Sverige, fosterland,\\*%
%        vår längtans bygd, vårt hem på jorden!\\*%
%        Nu spela skällorna, där härar lysts av brand,\\*%
%        och dåd blev saga, men med hand vid hand svär\\*%
%        än ditt folk som förr de gamla trohetsorden
%    \end{SongVerse}
%    \begin{SongVerse}
%        Fall, jule-snö, och suna djupa mo!\\*%
%        Brinn, österstjärna, genom junikvällen!\\*%
%        Sverige, moder! Bliv vår strid, vår ro, du land,\\*%
%        där våra barn en gång få bo och våra fäder sova\\*%
%        under kyrkohällen
%    \end{SongVerse}
%\end{SongText}
%\begin{SongText}[Längtan till landet]
%    \begin{SongInfo}
%        Text: H.sätherberg
%    \end{SongInfo}
%    \begin{SongVerse}
%        Vintern raser ut bland våra fjällar,\\*%
%        drivans blomma smälta ner och dö.\\*%
%        Himlen ler i vårens ljusa kvällar,\\*%
%        solen kysser liv i skog och sjö.
%    \end{SongVerse}
%    \begin{SongVerse}
%        $\|\:$Snart är sommarn här i purpurvågor,\\*%
%        guldbelagda, azurskiftande\\*%
%        ligga ängarne i dagens lågor,\\*%
%        och i lunden dansa källorne$\:\|$
%    \end{SongVerse}
%    \begin{SongVerse}
%        Ja, jag kommer! Hälsan, glada vindar,\\*%
%        ut till landet, ut till fåglarne,\\*%
%        att jag älskar dem, till björke och lindar,\\*%
%        sjö och berg, jag vill dem återse.
%    \end{SongVerse}
%    \begin{SongVerse}
%        $\|\:$Se dem än som i min barndoms stunder,\\*%
%        följa bäckens dans till klarnad sjö,\\*%
%        trastens sång i furuskogens lunder,\\*%
%        vattenfågelns lek kring fjärd och ö.$\:\|$
%    \end{SongVerse}
%\end{SongText}
%\begin{SongText}[Vårvindar friska]
%    \begin{SongVerse}
%        Vårvindar friska, leka och viska\\*%
%        Lundena kring likt älskande par\\*%
%        Strömmarna ila, finna ej vila\\*%
%        Förrän i havet störtvågen far\\*%
%        Klappa mitt hjärta, klaga och hör\\*%
%        Vallhornets klang, bland klipporna dör\\*%
%        Strömkarlen spelar, sorgerna delar\\*%
%        Vaken kring berg och dal.
%    \end{SongVerse}
%\end{SongText}
%\begin{SongText}[Nu grönskar det]
%    \begin{SongVerse}
%        Nu grönskar det i dalens famn, nu fotar äng och lid.\\*%
%        Kom med, kom med på vandringsfärd i vårens glada tid!\\*%
%        Var dag är som en gyllne skål, till bredden fylld med vin.\\*%
%        Så drick, min vän, drick sol och doft ty dagen den är din.\\*%
%    \end{SongVerse}
%    \begin{SongVerse}
%        Långt bort från dagens gråa hus vi glatt vår kosa styr\\*%
%        Och följer vägens vita band mot ljusa äventyr\\*%
%        Med öppna ögon låt oss se på livets rikedom.\\*%
%        som gror och sjuder överallt där våre ngår i blom!
%    \end{SongVerse}
%\end{SongText}
%\begin{SongText}[Ja vi elsker]
%    \begin{SongInfo}
%        (Norska nationalsången)
%    \end{SongInfo}
%    \begin{SongVerse}
%        Ja, vi elsker dette landet, som det stiger frem.\\*%
%        Furet, værbitt över vannet, med de tusen hjem.\\*%
%        Elsker, elsker det og tenker, på vår far og mor.\\*%
%        $\|\:$Og den saganatt som senker drømmer på vår jord$\:\|$
%    \end{SongVerse}
%    \begin{SongVerse}
%        Norske mann i hus og hytte. Takk din store Gud!\\*%
%        Landet ville han beskytte, skønt det mørkt så ut.\\*%
%        Alt hva fedrene har kjempet, mødrene har grett,\\*%
%        $\|\:$Har den Herre stille lampet, så vi vant vår rett$\:\|$
%    \end{SongVerse}
%    \begin{SongVerse}
%        Ja, vi elsker detta landet, som det stiger frem.
%        Furet, værbitt över vannet, med de tusen hjem!
%        Og som fedres kamp har hevet, det av nød till seier.
%        $\|\:$Også vi, når det blir krevt, för dets fred slåt leir.$\:\|$
%    \end{SongVerse}
%\end{SongText}
%\begin{SongText}[Den blomstedtid nu kommer]
%    \begin{SongVerse}
%        Den blomstertid nu kommer\\*%
%        med lust och fägring stor.\\*%
%        Du nalkas ljuva sommar,\\*%
%        då gräs och gröda gror.\\*%
%        Med blid och livlig värma,\\*%
%        till allt som varit dött,\\*%
%        sig solen stålar närma och allt blir återfött.
%    \end{SongVerse}
%    \begin{SongVerse}
%        De fagra blomsterängar\\*%
%        och åkerns ädla säd,\\*%
%        de rika örtesängar\\*%
%        och lundens gröna träd.\\*%
%        De skola oss påminna\\*%
%        Guds godhets rikedom.\\*%
%        Att vi den någ besinna,\\*%
%        som räcker året om.
%    \end{SongVerse}
%\end{SongText}
\newpage