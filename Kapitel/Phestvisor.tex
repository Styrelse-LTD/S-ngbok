\fakesection{Phestvisor}
\fancypagestyle{Phestvisor}{
    \fancyhead{} % clear all header fields
    \fancyhead[LE,RO]{\textbf{Phestvisor}}
}
\pagestyle{Phestvisor}
\begin{SongText}[Bolaget]
    \begin{SongInfo}
        Mel: Snickerboa
    \end{SongInfo}
    \begin{SongVerse}
        Till Bolaget så ränner jag\\*%
        och bankar på dess port. (BANK BANK!)\\*%
        Jag vill ha nått som bränner bra\\*%
        och gör mig sket full fort.\\*%
        Expediten sade goda.\\*%
        "Hur gammal kan min herre va'?\\*%
        Har du nått leg ditt fula dregg?\\*%
        Kom hit igen när du fått skägg!"
    \end{SongVerse}
    \begin{SongVerse}
        Men detta var ju inte bra\\*%
        jag vill bli full ikväll. (Igen!)\\*%
        Då plötsligt en idé jag fick\\*%
        de har ju sprit på Shell. (OK!)\\*%
        Många flaskor stod där på rad.\\*%
        Så nu kan jag bli full och glad.\\*%
        Den röda drycken åkte ner.\\*%
        Nu kan jag inte se nå mer!
    \end{SongVerse}
    \begin{SongVerse}
        Men gjort är gjort och skett är skett\\*%
        Och jag blev full som fan\\*%
        Så nu kan jag leva ett\\*% 
        nattliv även på dan\\*%
        Med vit käpp och ledarhund\\*%
        så har jag fest var vaken stund\\*%
        Så lyd mitt råd och fyll din kyl\\*%
        med alla sorters kementyl!
    \end{SongVerse}
\end{SongText}
\begin{SongText}[Det var i vår ungdoms fagraste vår]
    \begin{SongInfo}
        Sjungs vanligtvis efter ett bra spex eller annat kul som hänt. Dom som spexat noteras i textes som N.N.
    \end{SongInfo}
    \begin{SongVerse}
        Det där det gjorde dom fan så bra, HEJ!\\*%
        En skål nu till botten för dom nu vi tar.\\*%
        Hugg i och dra, HEJ!\\*%
        Hugg i och dra, HEJ!\\*%
        En skål nu till botten för dom vi nu tar.
    \end{SongVerse}
    \begin{SongVerse}
        Och alla så dricka vi nu N.N. till\\*%
        (Solo:) Och N.N. säger inte nej där till.
    \end{SongVerse}
    \begin{SongVerse}
        För det var i vår ungdoms fagraste vår\\*%
        Vi drack varandra till och vi sade: GUTÅR!
    \end{SongVerse}
\end{SongText}
\begin{SongText}[Portos visa]
    \begin{SongInfo}
        Mel: You can’t get a man with a gun
        (Dans skall förekomma under sången)
    \end{SongInfo}
    \begin{SongVerse}
        Jag vill börja gasqua, var fan är min flaska,\\*%
        vem i helvete stal min butelj?\\*%
        Skall törsten mig tvinga, en TT börja svinga,\\*%
        Nej, för fan, bara blunda och svälj.
    \end{SongVerse}
    \begin{SongVerse}
        Vilken smörja(ClapClap),\\*%
        får jag spörja(ClapClap)\\*%
        vem för fan tror att jag är en älg?
    \end{SongVerse}
    \begin{SongVerse}
        Till England vi rider och sedan vad det lider,\\*%
        träffar vi välan på någon pub.\\*%
        Och där skall vi festa, blott dricka av det bästa\\*%
        utav whisky och portvin,jag tänker gå hårt in\\*%
        för att smaka på rubb och stubb.\\*%
        rubb och stubb\\*%
        rubb och stubb\\*%
        rubb och stubb\\*%
        rubb och stubb\\*%
        rubbo stubb! (Sista stubb får endast sjungas av
        examinerade)
    \end{SongVerse}
\end{SongText}
\begin{SongText}[Vit vecka]
    \begin{SongInfo}
        Mel: White christmas och *Flottarkärlek
    \end{SongInfo}
    \begin{SongVerse}
        Jag drömmer om en vit vecka.\\*%
        Inte en droppe alkohol.\\*%
        Ingen punsch till kaffet,\\*%
        det ska bli straffet,\\*%
        för jag har fått vad jag tål\\*%
        *Men det var ganska länge sen.\\*%
        Nu är det dags igen.\\*%
        Jag har varit alkoholfri länge nog.\\*%
        Jag ska dricka mycket pilsner,\\*%
        Jag ska dricka mycket bål.\\*%
        Jag ska kolla hur mycket levern tål.\\*%
        Hej, skål!
    \end{SongVerse}
\end{SongText}
\begin{SongText}[Druckna sällskap]
    \begin{SongInfo}
        Mel: Öppna landskap
    \end{SongInfo}
    \begin{SongVerse}
        Jag trivs bäst i druckna sällskap,\\*%
        nära baren vill jag bo.\\*%
        Några kvällar varje vecka,\\*%
        så själen kan få ro.\\*%
        Jag trivs bäst i druckna sällskap,\\*%
        där tungorna får fart.\\*%
        Där klackarna slår högt i tak,\\*%
        och allt känns underbart\\*%
        Där dricker jag mitt brännvin själv,\\*%
        och tafsar på Johannas bröst.\\*%
        Jag dricker det med välbehag.\\*%
        tills jag får annan tröst.\\*%
        Jag trivs bäst i druckna sällskap,\\*%
        nära baren vill jag bo.
    \end{SongVerse}
\end{SongText}
\begin{SongText}[Jag ska festa]
    \begin{SongInfo}
        Mel: Bamse
    \end{SongInfo}
    \begin{SongVerse}
        Jag skall festa, ta det lugnt med spriten.\\*%
        Ha det roligt utan att va’ full.\\*%
        Inte krypa runt med festeliten.\\*%
        Ta det sansat för min egen skull
    \end{SongVerse}
    \begin{SongVerse}
        Ref:\\*%
        Först en öl i torra strupen,\\*%
        efter det så kommer supen,\\*%
        i med vinet, ner med punschen,\\*%
        sist en groggbuffé.
    \end{SongVerse}
    \begin{SongVerse}
        Jag är skitfull däckar först av alla,\\*%
        missar festen, men vad gör väl det.\\*%
        Blandar hejdlöst öl och gammal filmjölk\\*%
        Kastar upp på bordsdamen breve’
    \end{SongVerse}
    \begin{SongVerse}
        Ref...
    \end{SongVerse}
    \begin{SongVerse}
        Spyan rinner ner för n0lleslipsen,\\*%
        Raviolin torkar i mitt hår,\\*%
        Vem har lagt mig under matsalsbordet,\\*%
        Vems är gaffeln i mitt högra lår?
    \end{SongVerse}
\end{SongText}
%\begin{SongText}[Drinking song]
%    \begin{SongInfo}
%        Mel: When I'm 64
%    \end{SongInfo}
%    \begin{SongVerse}
%        When we get drunker\\*%
%        losing our minds,\\*%
%        many beers from now.\\*%
%        We will still be having us\\*%
%        a reel good time.\\*%
%        Whisky, gin and bottles of wine.\\*%
%        So fill up your glass now,\\*%
%        get drunk as a skunk.\\*%
%        Don’t say you want no more.
%    \end{SongVerse}
%    \begin{SongVerse}
%        We are the swingers,\\*%
%        we are the singers.
%    \end{SongVerse}
%\end{SongText}
\begin{SongText}[Studenternas Fyllenatt]
    \begin{SongInfo}
        Mel: Midnatt råder
    \end{SongInfo}
    \begin{SongVerse}
        Midnatt råder livat är på kåren,\\*%
        Är på Kåren\\*%
        Små studenter söka läka såren,\\*%
        Läka såren\\*%
        Ticktack ticktack ticketicketicktack\\*%
        Ta en öl!
    \end{SongVerse}
    \begin{SongVerse}
        Rök och fylla alla går i dimma,\\*%
        Går i dimma.\\*%
        Allt fler drypa efter midnattstimma,\\*%
        Midnattstimma.\\*%
        Ticktack…\\*%
        Klockan går
    \end{SongVerse}
    \begin{SongVerse}
        Fram till baren krypa sista metern,\\*%
        Sista metern\\*%
        Jävla radio, allt går ut i etern,\\*%
        Ut i etern\\*%
        Ticktack…\\*%
        Klockan två.
    \end{SongVerse}
    \begin{SongVerse}
        Jacka stövlar, fan vad det var trångt här,\\*%
        Det är trångt här.\\*%
        Huvudet snurrar ser skära elefanter,\\*%
        Elefanter.\\*%
        Ticktack…\\*%
        Ut vi gå.
    \end{SongVerse}
    \begin{SongVerse}
        Sedan åter till sin trygga lya,\\*%
        Trygga lya.\\*%
        Kräla, krypa genom annans spya,\\*%
        Annans spya.\\*%
        Ticktack…\\*%
        5, GOD NATT!
    \end{SongVerse}
\end{SongText}
%\begin{SongText}[Störthärligt full]
%    \begin{SongInfo}
%        Mel: Fat mummy brown
%    \end{SongInfo}
%    \begin{SongVerse}
%        Nu har alla lämnat festen\\*%
%        och jag siter ensam kvar,\\*%
%        Ibland groggar, pilsnerflaskor i en sönderslagen bar.\\*%
%        Första pilsnerflaskan tog jag\\*%
%        Vid min frukost klockan fem\\*%
%        Och nu sitter jag och väntar\\*%
%        På at få bli buren hem.
%    \end{SongVerse}
%    \begin{SongVerse}
%        För jag är störthärligt ful!\\*%
%        Och jag ramlar mest omkull.\\*%
%        Jag ser skära elefanter\\*%
%        Som har jättekonstig ull!\\*%
%        Ja, jag ramlar mest omkull.\\*%
%        Ja det är präktigt härligt\\*%
%        Supa och va’ full!
%    \end{SongVerse}
%    \begin{SongVerse}
%        Ifrån festen minns jag inget\\*%
%        -jo- mitt öga blev visst blått!\\*%
%        Det måste jag ha fått\\*%
%        När någon kasta en karott,\\*%
%        Fylld med vispgrädde och\\*%
%        Fimpar och en okammad peruk,\\*%
%        Eller också när jag stod i\\*%
%        Moraklockan och var sjuk!\\*%
%        Ja, jag är störthärligt…
%    \end{SongVerse}
%\end{SongText}
\begin{SongText}[Gå på fest]
    \begin{SongInfo}
        Mel: Du ska få min gamla cykel
    \end{SongInfo}
    \begin{SongVerse}
        När som livet kännes mörkt och grått och trist\\*%
        GÅ PÅ FEST\\*%
        När humöret allt emellan du har mist,\\*%
        GÅ PÅ FEST\\*%
        När du tror att allting grånar,\\*%
        Ska du se att himlen blånar, sommar'ns blommor dig
        förvånar,\\*%
        GÅ PÅ FEST!\\*%
        (GÅ PÅ FEST!)
    \end{SongVerse}
    \begin{SongVerse}
        Är du vissen, klen och trött och matt och svag\\*%
        GÅ PÅ FEST\\*%
        Skulle hjärtat bara slå vartannat slag,\\*%
        GÅ PÅ FEST\\*%
        Strunt i doktorn och recepter,\\*%
        Medicin och farmacepter,\\*%
        sluta upp och känna efter\\*%
        GÅ PÅ FEST!\\*%
        (GÅ PÅ FEST!)
    \end{SongVerse}
\end{SongText}
\begin{SongText}[Rosita]
    \begin{SongInfo}
        Mel: Fritiof och Carmencita
    \end{SongInfo}
    \begin{SongVerse}
        Gin tonicen, en liten grogg förutan cola,\\*%
        Den hittar du intill dig när du önskar skåla.\\*%
        Nästan på kansken, på en bricka hos tanten\\*%
        Som i baren har blandat och skapat denna dryck.\\*%
        Dit kom jag gående en kväll, mest av en nyck\\*%
        För jag ville skoja till det.
    \end{SongVerse}
    \begin{SongVerse}
        Två tusen åtta hundra spänn\\*%
        Kostade groggarna och sen gick jag hem.\\*%
        Där på diskbänken i Delfi,\\*%
        Den som aldrig gjordes ren,\\*%
        Stod en bedårande Rosita.
    \end{SongVerse}
    \begin{SongVerse}
        En som om åtta sekel jämt,\\*%
        Kommer att omnämnas som 1:a klassens skämt.\\*%
        Men vad brydde jag väl mig\\*%
        Där jag stod barskrapad och go’,\\*%
        För jag ville skoja till det.
    \end{SongVerse}
\end{SongText}
\begin{SongText}[Vikingen]
    \begin{SongInfo}
        Mel: När ölen din är kall
    \end{SongInfo}
    \begin{SongVerse}
        En viking älskar livets vann\\*%
        Hurra, hurra!\\*%
        Den hastigt i mitt svalg försvann\\*%
        Hurra, hurra!\\*%
        Till kalv, till oxe, till fisk, till fläsk\\*%
        När alla kärringar vill ha läsk\\*%
        ja, då vill alla vikingar ha en bäsk.
    \end{SongVerse}
    \begin{SongVerse}
        När vi har druckit bäsken slut\\*%
        Tragik, tragik\\*%
        Då bäres varje viking ut\\*%
        som lik, sej lik\\*%
        Och sén när vi vaknar vi sjunger en bit,\\*%
        sén korkar vi upp Skånes Aquavit
    \end{SongVerse}
    \begin{SongVerse}
        Skål för alla vikingar som kom hit!
    \end{SongVerse}
\end{SongText}
\begin{SongText}[Gräv ur tundran]
    \begin{SongInfo}
        Mel: Katjusha
    \end{SongInfo}
    \begin{SongVerse}
        Gräv ur tundran två dussin potäter,\\*%
        låt dem jäsa uti fjorton dar.\\*%
        Modersmjölken för ryssar och sovjeter\\*%
        brännes i babushkas samovar.\\*%
        Modersmjölken för ryssar och sovjeter\\*%
        brännes i babushkas samovar.
    \end{SongVerse}
    \begin{SongVerse}
        Kyl sen drycken i Sibiriens tjäle,\\*%
        tappa upp i immiga små glas.\\*%
        Höj sen glasen för fosterlandets välgång\\*%
        sjung ”Nastarovnja!” med en mäktig bas.\\*%
        Höj sen glasen för fosterlandets välgång\\*%
        sjung ”Nastarovnja!” [drick ur]\\*%
        Låt glasen gå i kras!
    \end{SongVerse}
\end{SongText}
\begin{SongText}[Fyllerian]
    \begin{SongInfo}
        Mel: Flottarkärlek
    \end{SongInfo}
    \begin{SongVerse}
        Jag var full en gång för längesen\\*%
        på knäna kröp jag hem\\*%
        varje dike var för mig ett vilohem.\\*%
        I varje skåp, i varje vrå\\*%
        så hade jag en liten vän\\*%
        ifrån skåne upp till 96\%.
    \end{SongVerse}
    \begin{SongVerse}
        Jag var full en gång för längesen\\*%
        på knäna kröp jag hem\\*%
        och i sällskap hade jag en elefant\\*%
        elefanten spruta vatten\\*%
        och jag trodde det var öl\\*%
        se’n så börja dom å’kalla mig för knöl
    \end{SongVerse}
\end{SongText}
\begin{SongText}[Till Jägermeister]
    \begin{SongInfo}
        Mel: Yesterday
    \end{SongInfo}
    \begin{SongVerse}
        Jägerdropp.\\*%
        Å min mage vill ha Jägerdropp.\\*%
        Den värme ut i fingertopp.\\*%
        Så ge mig några Jägerdropp.
    \end{SongVerse}
    \begin{SongVerse}
        Morgondag.\\*%
        Vem kan bryr sig om en morgondag?\\*%
        i kväll så är det bara du och jag.\\*%
        En Jägerdropp och du och jag.
    \end{SongVerse}
    \begin{SongVerse}
        Var-för kommer dag\\*%
        efter natt? Devete fan.\\*%
        Ång-est och albyl,\\*%
        ge mig kyld\\*%
        en Jägerdro-ro-ro-ro.
    \end{SongVerse}
    \begin{SongVerse}
        Jägerdropp.\\*%
        Å min mage vill ha Jägerdropp.\\*%
        den ger värme ut i fingertopp\\*%
        Så ge mig några Jägerdropp.
    \end{SongVerse}
\end{SongText}
\begin{SongText}[Undulaten]
    \begin{SongInfo}
        Mel: Med en enkel tulipan
    \end{SongInfo}
    \begin{SongVerse}
        Jag är en liten undulat\\*%
        Som får så dåligt med mat\\*%
        För dom jag bor hos, för dom jag bor hos,\\*%
        Dom är så snåla.\\*%
        Dom ger mig fisk varenda dag,\\*%
        Det vill jag inte ha -\\*%
        Jag vill ha brännvin, jag vill ha brännvin\\*%
        Med Coca Cola.
    \end{SongVerse}
    \begin{SongVerse}
        Jag är en stor och farlig varg\\*%
        Som är så helvetes arg.\\*%
        Jag bor i skogen, jag bor i skogen\\*%
        Bland träd och plantor.\\*%
        Jag äter gräs varenda dag,\\*%
        Det vill jag inte ha –\\*%
        Jag vill ha småbarn, jag vill ha småbarn\\*%
        Och gamla tanter.
    \end{SongVerse}
\end{SongText}
\begin{SongText}[Härjarvisan]
    \begin{SongInfo}
        Mel: Gärdebylåten\\*%
        Ur Lundsaspexet "Djangis Khan" 1954
    \end{SongInfo}
    \begin{SongVerse}
        Liksom våra fäder vikingarna i Norden\\*%
        drar vi riket runt och super oss under borden.\\*%
        Brännvinet har blivit ett elexir\\*%
        för kropp såväl som själ.\\*%
        Känner du dig liten och ynklig på jorden,\\*%
        växer du med supen och blir stor uti orden,\\*%
        slår dig för ditt håriga bröst och\\*%
        blir en man från hår till häl.
    \end{SongVerse}
    \begin{SongVerse}
        Ref:\\*%
        Ja nu ska vi ut härja\\*%
        supa, slåss och svärja\\*%
        bränna röda stugor, slå små barn och säga fula ord. (KTH!)\\*%
        Med blod ska vi stäppen färga\\*%
        nu äntligen lär jag kunna\\*%
        dra nån riktig nytta utav min Hermodskurs i mord.
    \end{SongVerse}
    \begin{SongVerse}
        Hurra nu ska man äntligen få röra på benen\\*%
        hela stammen jublar och det spritter i grenen.\\*%
        Tänk att än en gång få spränga fram på Brunte i galopp.\\*%
        Din doft o käre Brunte är trots brist i hygienen\\*%
        för en vild mongol minst lika ljuv som syrenen,\\*%
        tänk att på din rygg få rida runt i stan och spela topp!
    \end{SongVerse}
    \begin{SongVerse}
        Ref…
    \end{SongVerse}
    \begin{SongVerse}
        Ja, mordbränder är klämmiga, ta fram fotogenen\\*%
        och eftersläckningen tillhör just de fenomenen\\*%
        inom brandmansyrket som jag tycker det är nån nytta med.\\*%
        Jag målar för mitt inre upp den härliga scenen:\\*%
        Blodrött mitt i brandgult, ens prins Eugen en\\*%
        lika mustig vy kan måla, ens om han målade med sked.
    \end{SongVerse}
\end{SongText}
%\begin{SongText}[Fiskebåt]
%    \begin{SongInfo}
%        Mel: Svinnsta skär alt. Turistens klagan
%    \end{SongInfo}
%    \begin{SongVerse}
%        Raj raj raj raj raj raj fiskebåt\\*%
%        raj raj raj fiskebåt\\*%
%        raj raj raj raj raj raj fiskebåt
%    \end{SongVerse}
%    \begin{SongVerse}
%        raj raj raj fiskebåt
%    \end{SongVerse}
%    \begin{SongVerse}
%        Raj raj... hangarfartyg,
%    \end{SongVerse}
%    \begin{SongVerse}
%        Raj raj... atomubåt,
%    \end{SongVerse}
%    \begin{SongVerse}
%        Raj raj... mobila bar.
%    \end{SongVerse}
%\end{SongText}
\begin{SongText}[Var nöjd med (allt som baren ger)]
    \begin{SongInfo}
        Mel: Var nöjd med (allt som livet ger)
    \end{SongInfo}
    \begin{SongVerse}
        Var nöjd med allt, som baren ger\\*%
        Och allting som du kring dig ser\\*%
        Glöm bort bekymmer, sorger och besvär\\*%
        Var glad och nöjd för vet du vad\\*%
        En läsk den gör ju ingen glad\\*%
        Var nöjd med drickan som serveras här
    \end{SongVerse}
    \begin{SongVerse}
        Varthän jag en strövar, varthän jag än går\\*%
        Finns burkar och flaskor kring mina spår\\*%
        Jag älskar skottar och deras bon\\*%
        För whisky är ju min passion\\*%
        Och vill du med bira törsten släcka\\*%
        Så ska du en tia till bartendern räcka\\*%
        (Talas): dricka bira?\\*%
        (Talas): - javisst, kittlar dödsskönt i kistan!
    \end{SongVerse}
    \begin{SongVerse}
        Var nöjd med…
    \end{SongVerse}
    \begin{SongVerse}
        Om frukter dig lockar, oliv eller bär\\*%
        Se till att du plockar dem utan besvär\\*%
        Om du vill ha drinkar av bästa klass\\*%
        Så använd din höger- och vänstertass\\*%
        Men klorna, de ska du dra in\\*%
        Så fort du ska ta dig ett glas med gin
    \end{SongVerse}
    \begin{SongVerse}
        Var nöjd med livet som vi lever här\\*%
        Med dryck och bär
    \end{SongVerse}
\end{SongText}
%\begin{SongText}[Lille Olle]
%    \begin{SongInfo}
%        Mel: Katjusha\
%    \end{SongInfo}
%    \begin{SongVerse}
%        Lille Olle skulle gå på disco\\*%
%        men han hade inte någon sprit.\\*%
%        Lille Olle skaffa lite hembränt\\*%
%        Lille Olle gick då på en nit.
%    \end{SongVerse}
%    \begin{SongVerse}
%        Lille Olle skulle börja festa\\*%
%        spriten blandade han ut med Mer.\\*%
%        Lille Olle drack upp hela bålen\\*%
%        Lille Olle ser nu inte mer.
%    \end{SongVerse}
%    \begin{SongVerse}
%        Lille Olle skaffa sig en Ledhund\\*%
%        Den var ful och oxå ganska trind.\\*%
%        Olles ledhund drack upp femton flaskor\\*%
%        Olles ledhund är nu oxå blind.
%    \end{SongVerse}
%    \begin{SongVerse}
%        Lille Olle började med droger.\\*%
%        blanda ut sin LSD med juice.\\*%
%        Lille Olles hjärna står i lågor\\*%
%        Lille Olle dog av överdos.
%    \end{SongVerse}
%    \begin{SongVerse}
%        Lille Olle sitter nu i himlen\\*%
%        Festa kan man även göra där.\\*%
%        Lille Olle skaffade en ölback,\\*%
%        Capsar nu med Gud och Sankte Per
%    \end{SongVerse}
%\end{SongText}
\begin{SongText}[Ju mer vi är tillsammans]
    \begin{SongInfo}
        Text \& mel: Inger Jacobsen och Thore Skogman
    \end{SongInfo}
    \begin{SongVerse}
        Ju mer vi är tillsammans, tillsammans, tillsammans\\*%
        Ju mer vi är tillsammans, ju gladare vi bli.\\*%
        För mina vänner är dina vänner och\\*%
        dina vänner är mina vänner.\\*%
        Ju mer vi tillsammans, ju gladare vi bli.
    \end{SongVerse}
\end{SongText}
\begin{SongText}[Fadder din är skit]
    \begin{SongInfo}
        Text: Sonic (Albin Stridsman) Dat20\\*%
        Mel: När ölen din är kall
    \end{SongInfo}
    \begin{SongVerse}
        När nollan din är värdelös fy fan fy fan \\*%
        Nar fadder lämnar först ja då då fan fy fan
    \end{SongVerse}
    \begin{SongVerse}
        När nollan inte låten har o faddrar inte heller kan\\*%
        Ja sjung fy fan för fadder din är skit
    \end{SongVerse}
    \begin{SongVerse}
        När nollan sjunger fel ja då fy fan fy fan \\*%
        När närhet inte finnes då fy fan fy fan
    \end{SongVerse}
    \begin{SongVerse}
        När fadder inte heller kan o utomstående sången har \\*%
        ja sjung fy fan för fadder din är skit
    \end{SongVerse}
    \begin{SongVerse}
        När nollan din han spyr ja då fy fan fy fan\\*%
        När kvällen drar till efterfest hu ra hu ra
    \end{SongVerse}
    \begin{SongVerse}
        Om nollan den han hatten har och reglerna han ej kan där till \\*%
        Sjung fy fan för nollan din dum
    \end{SongVerse}
\end{SongText}
%\begin{SongText}[Bonka Bonka]
%    \begin{SongInfo}
%        Text: MÄXX (Adrian Swande) Dat 22\\*%
%        Mel: Sudda sudda 
%    \end{SongInfo}
%    \begin{SongVerse}
%        Det var en gång en nØlla som var en nykerist.\\*%
%        Den ville inte dricka, den var ju ganska trist.\\*%
%        En dag sa nØllans fadder: "Kvickt kom här och hör på,\\*%
%        vi vill att du ska bonka, med tio-komma-två".
%    \end{SongVerse}
%    \begin{SongVerse}
%        Läskburken han då greppa', envist och trögt javisst,\\*%
%        så vidtog han sin nyktra blick så torr och sund och trist.
%    \end{SongVerse}
%    \begin{SongVerse}
%        Men då sa Phösar'n:
%    \end{SongVerse}
%    \begin{SongVerse}
%        Bonka, bonka, bonka, bonka bort din nyktra blick.\\*%
%        Bonka, bonka, bonka, bonka bort ditt nyktra skick.\\*%
%        NØllan den skall dricka och va' glad.
%    \end{SongVerse}
%    \begin{SongVerse}
%        NØllan den skall spy upp nästa dag.\\*%
%        NØllan är för vi den ned skall supa.\\*%
%        Bonka, bonka bort din nyktra blick.
%    \end{SongVerse}
%    \begin{SongVerse}
%        Är du en liten nØlla som är en nykterist?\\*%
%        Är du en liten nØlla som äro ack så trist?\\*%
%        Om jag nu då din fadder vill bjuda på en dryck,\\*%
%        är du en sån som nekar och saknar vett och hyfs?
%    \end{SongVerse}
%    \begin{SongVerse}
%        Nej icke är du sådan min lille lille "vän",\\*%
%        så tag nu bonken i din mun och lyd och sug och svälj.
%    \end{SongVerse}
%    \begin{SongVerse}
%        Och då säger Phösar'n:
%    \end{SongVerse}
%    \begin{SongVerse}
%        Bonka...
%    \end{SongVerse}
%    \begin{SongVerse}
%        Varje nØllan måste vi ned supa.\\*%
%    Bonka, bonka bort din nyktra blick.
%    \end{SongVerse}
%\end{SongText}
\begin{SongText}[Jesus lever]
    \begin{SongInfo}
        Mel: Sånt är livet
    \end{SongInfo}
    \begin{SongVerse}
        Jesus lever, han bor i skövde\\*%
        Han kör en volvo, och han är gift\\*%
        Han har en villa, med rododendron\\*%
        Han sparar pengar, och jobbar skift
    \end{SongVerse}
    \begin{SongVerse}
        Redan på lekis, var han märklig\\*%
        Han ville inte, leka krig\\*%
        Men när hans kompis, Knut blev skjuten\\*%
        så lät han Jesus, uppväcka sig
    \end{SongVerse}
    \begin{SongVerse}
        Jesus lever, han bor i skövde...
    \end{SongVerse}
    \begin{SongVerse}
        Han gick i skolan, som alla andra\\*%
        an var rätt duktig, på gymnastik\\*%
        å vilken kille, han gick på vatten\\*%
        en gång så gick han, till Reykjavik
    \end{SongVerse}
    \begin{SongVerse}
        Jesus lever, han bor i skövde...
    \end{SongVerse}
    \begin{SongVerse}
        I sina tånår, så var han poppis\\*%
        och han blev bjuden, på varje fest\\*%
        Å vilken kille, han fick ju vatten \\*%
        att bli till rusdryck, utan jäst
    \end{SongVerse}
    \begin{SongVerse}
        $\:\|$ Jesus lever, han bor i skövde... $\:\|$
    \end{SongVerse}
\end{SongText}
\begin{SongVerse}[Pessemistkonsulten]
    \begin{SongInfo}
        Text \& Mel: Östen med Resten
    \end{SongInfo}
    \begin{SongVerse}
        Man avlas hit till jämmerdalen\\*%
        I slutet av augusti föddes jag\\*%
        Det var det året just i\\*%
        Slutet av augusti\\*%
        Det regnade varenda dag
    \end{SongVerse}
    \begin{SongVerse}
        Sen växte man väl upp och blev äldre\\*%
        Ja, man har ju inget annat val\\*%
        Och ända sen den första da'n\\*%
        Har man av slentrian\\*%
        Släntrat kring här i vår jämmerdal
    \end{SongVerse}
    \begin{SongVerse}
        (Ref)\\*%
        Så det är lika bra att sluta drömma\\*%
        Det går åt helvete i alla fall\\*%
        För om man drömmer om Paris\\*%
        Hamnar man på något vis\\*%
        Likt förbannat i Hudiksvall
    \end{SongVerse}
    \begin{SongVerse}
        I sin ungdom hade man väl fantasier\\*%
        Det var väl ungdomssynder man begick\\*%
        Man trodde på jämlikhet och solidaritet\\*%
        Men nu har man fått en bättre överblick
    \end{SongVerse}
    \begin{SongVerse}
        Så nu ger man väl fan i politiken\\*%
        Ja, låt folk tycka vad de vill\\*%
        Inte är det någon mening\\*%
        Att man slåss i nån förening\\*%
        Nä, det tjänar ändå ingenting till
    \end{SongVerse}
    \begin{SongVerse}
        Ref
    \end{SongVerse}
    \begin{SongVerse}
        Men så småningom så börjar saven stiga\\*%
        Och man söker sig en livskamrat\\*%
        Nån som uppfyller ens normer\\*%
        Med Dolly Partons former\\*%
        Och Tore Wretmans kunnande i mat
    \end{SongVerse}
    \begin{SongVerse}
        Men inte fasen blev det som man tänkte\\*%
        På när man blev sjutton-arton för\\*%
        Nu i efterhand så vet man\\*%
        Hon ser ut som Tore Wretman\\*%
        Och lagar mat som Dolly Parton gör
    \end{SongVerse}
    \begin{SongVerse}
        Ref
    \end{SongVerse}
    \begin{SongVerse}
        Så nu står man mellan vaggan och graven\\*%
        Medelålders och med övervikt\\*%
        Ja, det är klart man har planera'\\*%
        Att börja motionera\\*%
        Fast givetvis på lite längre sikt
    \end{SongVerse}
    \begin{SongVerse}
        Och man drömmer om en ny och fin Mercedes\\*%
        Och ett sommarhus i Portugal\\*%
        Men man får en gammal Škoda\\*%
        Med en trasig växellåda\\*%
        Och en friggebod i bästa fall
    \end{SongVerse}
    \begin{SongVerse}
        Ref
    \end{SongVerse}
    \begin{SongVerse}
        Och ni som sitter här ibland publiken\\*%
        Ni tror ni har det trevligt, men\\*%
        Ni ska vara måttligt glada\\*%
        Ni har säkert vattenskada\\*%
        I kåken när ni kommer hem igen
    \end{SongVerse}
    \begin{SongVerse}
        Och ni grabbar som sitter här och hoppas\\*%
        Och drömmer för öppna spjäll\\*%
        Det har ni ingenting för\\*%
        Att ni sitter här och gör\\*%
        För det blir ju ändå ingenting ikväll
    \end{SongVerse}
    \begin{SongVerse}
        $\:\|$Ref...$\:\|$\\*%
        I bästa Fall
    \end{SongVerse}
\end{SongVerse}
\newpage