\fakesection{Bordsvisor}
\fancypagestyle{Bordsvisor}{
    \fancyhead{} % clear all header fields
    \fancyhead[LE,RO]{\textbf{Bordsvisor}}
}
\pagestyle{Bordsvisor}
\begin{SongText}[En liten blå förgätmigej]
    \begin{SongInfo}
        Sjungs som tack sång till personalen efter sittning.\\*%
        Detta inleds med att alla gäster ställer sig med höger knä i backen och vänstra benet blottat och sjunga "Hallå personalen" tills man säkerställt att alla finns på plats\\*%
        Text \& Mel: Ulla Billquist
    \end{SongInfo}
    \begin{SongVerse}
        Hur gärna ville jag ej vara\\*%
        en liten blå förgätmigej,\\*%
        en liten blå förgätmigej.\\*%
        Då skulle jag för dig förklara,\\*%
        hur innerligt jag älskar dig.\\*%
        (och dig, och dig...)
    \end{SongVerse}
\end{SongText}
\begin{SongText}[Jag har aldrig vart på snusen]
    \begin{SongInfo}
        Mel: O, hur saligt att få vandra
    \end{SongInfo}
    \begin{SongVerse}
        Jag har aldrig vart på snusen,\\*%
        aldrig rökat en cigarr,\\*%
        Mina dygder äro tusen,\\*%
        inga syndiga laster har jag.\\*%
        Jag har aldrig sett nått naket,\\*%
        inte ens ett litet nyfött barn.\\*%
        Mina blickar går mot taket\\*%
        därmed undgår jag frestarens garn.\\*%
    \end{SongVerse}
    \begin{SongVerse}
        Bacchus spelar på gitarren,\\*%
        Satan spelar på sitt handklaver.\\*%
        Alla djävlar dansa tango,\\*%
        säg, vad kan man önska sig mer?\\*%
        Jo, att alla bäckar vore brännvin,\\*%
        hela svartån full av bayerskt öl,\\*%
        konjak i var rännsten\\*%
        och punsch i varendaste pöl.\\*%
    \end{SongVerse}
\end{SongText}
\begin{SongText}[Fader Abraham]
    \begin{SongVerse}
        Fader Abraham,\\*%
        Fader Abraham\\*%
        fyra söner hade Abraham\\*%
        och dom åt och drack\\*%
        och dom drack och åt\\*%
        och dom ropade de så här:
    \end{SongVerse}
    \begin{SongVerse}
        Höger arm!\\*%
        Vänster arm!\\*%
        Höger fot!\\*%
        Vänster fot!\\*%
        Huvudet!\\*%
        Tungan!\\*%
        SKÅL!
    \end{SongVerse}
\end{SongText}
\begin{SongText}[Brännvin är gott]
    \begin{SongInfo}
        Mel: Här kommer Karl-Alfred boy
    \end{SongInfo}
    \begin{SongVerse}
        Brännvin är jävligt gott!\\*%
        Brännvin är jävligt gott!\\*%
        Men, slår man i golvet\\*%
        så där mellan tolv-ett\\*%
        då slår man sig jävligt hårt!
    \end{SongVerse}
\end{SongText}
\begin{SongText}[Studielånet]
    \begin{SongInfo}
        Mel: Hej tometegubbar
    \end{SongInfo}
    \begin{SongVerse}
        $\|\:$Mitt lilla lån det räcker inte,\\*%
        det går till öl och till brännvin! $\:\|$\\*%
        Till öl och brännvin går det åt\\*%
        och till små flickor emellanåt.\\*%
        Mitt lilla lån det räcker inte,\\*%
        det går till öl och till brännvin!
    \end{SongVerse}
\end{SongText}
\begin{SongText}[Kalmarevisan]
    \begin{SongVerse}
        $\|\:$ För uti Kalmare stad\\*%
        ja där finns det ingen kvast $\:\|$\\*%
        förrän lördagen.\\*%
        \textbf{Hej dick}\\*%
        Hej dack\\*%
        \textbf{Jag slog i}\\*%
        och vi drack\\*%
        \textbf{Hej dickom dickom dack}\\*%
        hej dickom dickom dack.\\*%
        För uti Kalmare stad\\*%
        ja där finns det ingen kvast\\*%
        förrän lördagen.
    \end{SongVerse}
    \begin{SongVerse}
        $\|\:$ \textbf{När som bonden kommer hem}\\*%
        kommer bondekvinnan ut $\:\|$\\*%
        och är stor i sin trut\\*%
        \textbf{Hej dick . . .}
    \end{SongVerse}
    \begin{SongVerse}
        $\|\:$ \textbf{Var är pengarna du fått ?}\\*%
        Jo, dom har jag supit opp ! $\:\|$\\*%
        Uppå Kalmare slott.\\*%
        \textbf{Hej dick . . .}
    \end{SongVerse}
    \begin{SongVerse}
        $\|\:$ \textbf{Jag skall mäla dig an}\\*%
        för vår kronbefallningsman $\:\|$\\*%
        Och du skall få skam\\*%
        \textbf{Hej dick . . .}
    \end{SongVerse}
    \begin{SongVerse}
        $\|\:$ \textbf{Kronbefallningsmannen vår}\\*%
        satt på krogen i går $\:\|$\\*%
        Och var full som ett får.\\*%
        \textbf{Hej dick . . .}\\*%
    \end{SongVerse}
    \begin{SongVerse}
        $\|\:$ \textbf{Va’ sa’ bonnen ha te’ mat?}\\*%
        Sura sillar och potat $\:\|$\\*%
        Det blir sillsallat.\\*%
        \textbf{Hej dick . . .}\\*%
    \end{SongVerse}
    \begin{SongVerse}
        $\|\:$ \textbf{Säg var är din labbrapport?}\\*%
        Ja den har jag supit bort $\:\|$\\*%
        För den var för kort.\\*%
        \textbf{Hej dick . . .}
    \end{SongVerse}
\end{SongText}
%\begin{SongText}[Knall och fall]
%    \begin{SongInfo}
%        [EDITORS NOTE: Hittar varken melodin eller vart låter komm ifrån...]\\*%
%        Text:\\*%
%        Mel: Mylord
%    \end{SongInfo}
%    \begin{SongVerse}
%        Jag är en teknolog\\*%
%        Som härom kvällen låg\\*%
%        I egna spyor som jag hade kastat opp.\\*%
%        Min bästa partyfrack\\*%
%        Som hade gjort mig black\\*%
%        Vid nästa Valborg blir den balens stora
%        flopp.\\*%
%    \end{SongVerse}
%    \begin{SongVerse}
%        Nu har jag Overall\\*%
%        Som mycket mera tål.\\*%
%        Jag dricker öl och spiller ner från topp
%        till tå.\\*%
%        Men man blir korpulent,\\*%
%        Trots att man är student.\\*%
%        Fast mycket bättre kan man faktiskt inte
%        må.\\*%
%    \end{SongVerse}
%    \begin{SongVerse}
%        Ut i min nakenhet\\*%
%        Jag inte längre vet.\\*%
%        Jag missar tentor och har inga studielån.\\*%
%        Vill ju bli ingengör,\\*%
%        Men vet ej hur man gör.\\*%
%        Nu har jag givit upp och satsar nog på rån?!\\*%
%        -VILKET FÅN!\\*%
%    \end{SongVerse}
%\end{SongText}
\begin{SongText}[Lambo]
    \begin{SongInfo}
        Låten används som en hedersbetydelse.\\*%
        Medan församlingen sjunger den första versen tömmer den utpekade sitt glas, om glaset inte av någon anledning är tömt när versen är slut sjungs sången "gråa hår" i interimet.\\*%
        I sista versen kan ett nytt hyllningsobjeckt utses och sången börjar om igen!
    \end{SongInfo}
    \begin{SongVerse}
        \textbf{ALLA:} För nu glaset till din mun!\\*%
        Tjo-fa-de-rittan lambo!\\*%
        Och drick ur, din fylle-hund! (Nu dricks enheten)\\*%
        Tjo-fa-de-rittan lambo!\\*%
        Se, hur dropparna i glaset \\*%
        rinner ner i fylle-aset.\\*%
        Lambo-Hej! Lambo-Hej!\\*%
        Tjo-fa-de-rittan lambo!
    \end{SongVerse}
    \begin{SongVerse}
        (Är personen inte klar ännu börja sjung "Gråa hår")
    \end{SongVerse}
    \begin{SongVerse}
        \textbf{SOLO:} Jag nu glaset druckit har,\\*%
        \textbf{ALLA:} Tjo-fa-de-rittan lambo!\\*%
        \textbf{SOLO:} Ej en droppe finnes kvar,\\*%
        \textbf{ALLA:} Tjo-fa-de-rittan lambo!\\*%
        \textbf{SOLO:} Som bevis jag nu skall vända,\\*%
        \textbf{SOLO:} glaset på dess rätta ända.\\*%
        \textbf{ALLA:} Lambo-Hej! Lambo-Hej! Tjo-fa-de-rittan lambo!
    \end{SongVerse}
    \begin{SongVerse}
        Ja, han kunde konsten,\\*%
        han var en riktig fylle-fylle hund.\\*%
        Låt oss gå till nästa man\\*%
        och se vad han förmår.
    \end{SongVerse}
\end{SongText}
\begin{SongText}[Gråa hår]
    \begin{SongVerse}
        $\|\:$ Nu har vi väntat länge nog, länge nog, länge nog $\:\|$\\%
        Nu börjar vi få gråa hår ...\\%
        Nu börjar vi få ATP ...\\%
        Nu får vi ringa Fonus snart...\\%
        Nu sitter vi på gravens rand...\\%
        Nu börjar det att lukta lik...\\%
        Nu bäres liken ut på bår...\\%
        Nu krypa vi i kistan in…\\%
        Nu börjar de att skyffla jord…\\%
        Nu växer mossan på vår grav...\\%
        Nu har vi blivit till skelett…\\%
        Nu knackar vi på himlens port…\\%
        Nu sitter vi hos sankte Per…\\%
        Nu åker vi i Himlen in…\\%
        Nu får vi höra harpmusik…\\%
        Då flyr vi ner i Helvetet…\\%
        Nu capsar vi med Satan själv, Satan själv,\\*%
        Satan själv.\\*%
        Nu capsar vi med Satan själv, i Helvetet.\\%
        Nu har vi inga verser kvar...
    \end{SongVerse}
\end{SongText}
\begin{SongText}[Hyfsvisa]
    \begin{SongVerse}
        Kors i all sin dar,\\*%
        har du brännvin kvar?\\*%
        Är du sparsam eller snål?\\*%
        SKÅL!\\*%
        \emph{Sparsamma sjunger: "Snål!"}
    \end{SongVerse}
\end{SongText}
\begin{SongText}[Rönnerdahl]
    \begin{SongInfo}
        Mel: Sjösala vals
    \end{SongInfo}
    \begin{SongVerse}
        Rönnerdahl han skuttar med ett skratt ur sin säng\\*%
        fastnar i ett lakan, slår näsan i sänggaveln.\\*%
        Rullar ned på golvet i en våghalsig sväng,\\*%
        slutar sen att skratta när hans stortå får däng.\\*%
        Virrig i sin hjärna han reser sig på knä,\\*%
        och se så mången stjärna fast morgon de nu é,\\*%
        och se så många blåmärken som redan slagit ut på benen,\\*%
        Blåa och vackra i morgonens svaga ljus.
    \end{SongVerse}
    \begin{SongVerse}
        Rönnerdahl han vinglar upppå osäkra ben,\\*%
        Och den vita skjortan den slafsar kring vaderna,\\*%
        Packad som en alika i majsolen sken\\*%
        Skrålar han för ekorren som gungar på gren.\\*%
        ”Titta” ropar ungarna, ”pappa han är full!”\\*%
        han raglar runt i stugan där han faller omkull,\\*%
        och se så många burkar han redan har slängt ut på ängen.\\*%
        Löwenbrau, Heineken, Faxe och Norrlandsguld.
    \end{SongVerse}
\end{SongText}
\begin{SongText}[Det var länge sen]
    \begin{SongInfo}
        Mel: Det va Längesen man plocka några blommor (Östen Warnerbring)
    \end{SongInfo}
    \begin{SongVerse}
        Det var länge sen jag plocka’ några blommor.\\*%
        Det var länge sen jag tog några poäng.\\*%
        Det var länge sen jag handla’ på systemet.\\*%
        Det var länge sen jag fick en tjej i säng.
    \end{SongVerse}
    \begin{SongVerse}
        Men å andra sidan bränner jag ju hemma,\\*%
        och klarar kärleken alldeles för mig själv.\\*%
        Det var länge sen jag plocka’ några tentor\\*%
        men å andra sidan går de om igen.
    \end{SongVerse}
\end{SongText}
\begin{SongText}[Sittningsvisa]
    \begin{SongInfo}
        Mel: Raska Fötter
    \end{SongInfo}
    \begin{SongVerse}
        Raska pojkar dricka fort,fort, fort.\\*%
        Hela flaskan gick som smort, smort, smort.\\*%
        Många backar bär vi in.\\*%
        Många gubbar bär vi ut.\\*%
        Det är bara roligt.
    \end{SongVerse}
    \begin{SongVerse}
        Alla bara ropar öl, öl, öl.\\*%
        Snälla söta inget söl, söl, söl.\\*%
        Pelle får en öl så stor.\\*%
        Punsch får lille, lille bror.\\*%
        Stina får en Zingo.
    \end{SongVerse}
    \begin{SongVerse}
        Snart är goda ölen slut, slut, slut.\\*%
        Fylleaset bäres ut, ut, ut.\\*%
        Men till nästa gång igen,\\*%
        Kommer han, vår gamle vän,\\*%
        ty det har han lovat.
    \end{SongVerse}
\end{SongText}
\begin{SongText}[I koma]
    \begin{SongInfo}
        Mel: Tre Pepparkaksgubbar
    \end{SongInfo}
    \begin{SongVerse}
        I koma, i koma\\*%
        Vi faller efter hand.\\*%
        Så röda, så röda\\*%
        blir ögonen ibland.\\*%
        I glasen, i glasen\\*%
        vi finner ljuvlig saft\\*%
        och ta mig tusan får vi inte\\*%
        både liv och kraft.
    \end{SongVerse}
\end{SongText}
\begin{SongText}[Vem kan raggla]
    \begin{SongInfo}
        Mel: Vem kan segla
    \end{SongInfo}
    \begin{SongVerse}
        Vem kan raggla för utan vin,\\*%
        vem är nykter om våren,\\*%
        vem kan skilja på kron och gin,\\*%
        utan att smaka på 'ren.\\*%
    \end{SongVerse}
    \begin{SongVerse}
        Jag kan raggla för utan vin,\\*%
        å visst var jag nykter den våren,\\*%
        men jag kan ej skilja på kron och gin,\\*%
        efter den elfte tåren.\\*%
    \end{SongVerse}
    \begin{SongVerse}
        Jag berusar mig varje kväll,\\*%
        jag kan klara en sjuttis,\\*%
        men ej gå innan festens slut,\\*%
        utan att fälla tårar.\\*%
    \end{SongVerse}
    \begin{SongVerse}
        Jag är full nästan året om,\\*%
        jag kan ej längre tänka,\\*%
        men jag mår trots allt jävligt bra,\\*%
        utom när jag är bakfull.\\*%
    \end{SongVerse}
    \begin{SongVerse}
        Jag kan snart inte längre se,\\*%
        jag har snart ingen lever,\\*%
        men inte fan deppar jag för det,\\*%
        höj nu glasen och skåla!\\*%
    \end{SongVerse}
\end{SongText}
\begin{SongText}[Skål för våra vänner]
    \begin{SongVerse}
        Vi skålar för våra vänner\\*%
        och dom som vi känner\\*%
        och de som vi inte känner\\*%
        dom sätter vi på!!!\\*%
        Hejhå! Hejhå! Dom sätter vi på!\\*%
        %(EDITORS NOTE: Bytte ut "dom skiter vi i !!!" till "dom sätter vi på!")
    \end{SongVerse}
\end{SongText}
\begin{SongText}[The basic song]
    \begin{SongInfo}
        Varje rad fungerar som en rad kod i basic.
        Varje gång låter starts om brukar takten öka\\*%
        Mel: Mors lilla olle
    \end{SongInfo}
    \begin{SongVerse}
        10 LET oss nu fatta i våra glas\\*%
        20 INPUT en klunk i utav det som där has\\*%
        30 IF du fått nog THEN till 50 min vän\\*%
        40 ELSE GOTO-baka till 10 igen\\*%
        50 END
    \end{SongVerse}
\end{SongText}
\begin{SongText}[Temperaturen]
    \begin{SongVerse}
        Temperaturen är hög uti kroppen\\*%
        närmare 40 än 37...,5.\\*%
        Ja, så ska det vara när ångan är oppe.\\*%
        Och så är fallet uti detta nu!
    \end{SongVerse}
    \begin{SongVerse}
        Vi rulla, vi rulla....
    \end{SongVerse}
    \begin{SongVerse}
        Livet är kort som en barnunges tröja,\\*%
        ingenting kan vi väl gör åt det.\\*%
        Men med vår sång kan bekymren vi röja,\\*%
        bort ifrån vardagens gråaktighet.
    \end{SongVerse}
    \begin{SongVerse}
        Vi rulla, vi rulla...
    \end{SongVerse}
\end{SongText}
\begin{SongText}[Jävlaranammas Sittningsvisa]
    \begin{SongInfo}
        text: Jävlaranamma
    \end{SongInfo}
    \begin{SongVerse}
        Höj nu glasen glada bröder\\*%
        vi ska kröka så vi blöder\\*%
        vi ska dricka sånna ruskiga volymer
    \end{SongVerse}
    \begin{SongVerse}
        Ut ur Bacci sköna källa\\*%
        våra skallar skola smälla\\*%
        mot betongen när vi bryter fram som Ymer
    \end{SongVerse}
    \begin{SongVerse}
        Häll, drick ur nu, gutår\\*%
        svep och häv så fort det går\\*%
        Halsa blint ters och qvint\\*%
        levern blir som en korint
    \end{SongVerse}
    \begin{SongVerse}
        Höj nu glasen glada bröder\\*%
        vi ska kröka så vi blöder
    \end{SongVerse}
    \begin{SongVerse}
        Sociala späkningar i fekala kräkningar
   
    \end{SongVerse}
\end{SongText}
%\begin{SongText}[Fredmans sång nr 21 (kort)]
%    \begin{SongInfo}
%        Text: Carl Michael Bellman
%    \end{SongInfo}
%    \begin{SongVerse}
%        Så lunka vi så småningom\\*%
%        från Bacchi buller och tumult,\\*%
%        när döden ropar; Granne kom,\\*%
%        ditt timglas är nu fullt.\\*%
%        Du gubbe fäll din krycka ner,\\*%
%        och du yngling, lyd min lag,\\*%
%        den skönsta nymf som mot dig ler\\*%
%        inunder armen tag.
%    \end{SongVerse}
%    \begin{SongVerse}
%        Ref:\\*%
%        Tycker du att graven är för djup,\\*%
%        nå välan, så tag dig då en sup,\\*%
%        tag dig sen dito en, dito två, dito tre,\\*%
%        så dör du nöjdare.
%    \end{SongVerse}
%    \begin{SongVerse}
%        Säg är du nöjd, min granne säg,\\*%
%        så prisa världen nu till slut;\\*%
%        om vi ha en och samma väg,\\*%
%        så följoms åt; drick ut.\\*%
%        Men först med vinet rött och vitt\\*%
%        för vår värdinna bugom oss,\\*%
%        och halkom sen i graven fritt,\\*%
%        vid aftonstjärnans bloss.
%    \end{SongVerse}
%    \begin{SongVerse}
%        Ref...
%    \end{SongVerse}
%\end{SongText}
\newpage