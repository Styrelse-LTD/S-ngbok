\fakesection{Övriga visor}
\fancypagestyle{Övriga visor}{
    \fancyhead{} % clear all header fields
    \fancyhead[LE,RO]{\textbf{Övriga visor}}
}
\pagestyle{Övriga visor}
\begin{SongText}[Gäss te’ dej]
    \begin{SongInfo}
        Text: Hans Alfredson och Tage Danielsson\\*%
        Mel: Yesterday
    \end{SongInfo}
    \begin{SongVerse}
        Gäss te dej.\\*%
        Jag vill giva nå’ra gäss te’ dej.\\*%
        Om du bara säger yes te’ mej\\*%
        så ger jag nå’ra gäss te’ dej.\\*%
        Ester, dej -\\*%
        gillar jag; om blott du fäster dej\\*%
        vid en så’n som vill ge gäss te’ dej,\\*%
        så kommer jag med präst te’ dej.
    \end{SongVerse}
    \begin{SongVerse}
        Pärlor får du ej av en fattiglapp som mej.\\*%
        Ändå vill jag ge som en ringa gest te’ dej\\*%
        - Yeah! Yeah! Yeah! -
    \end{SongVerse}
    \begin{SongVerse}
        gäss te’ dej,\\*%
        de’ ä’ min delikatess te’ dej.\\*%
        Den som kunde ge en häst te’ dej,\\*%
        men jag har bara gäss te’ dej.\\*%
        Jag har bara gäss te’ dej.
    \end{SongVerse}
\end{SongText}
\begin{SongText}[Svordomsvisan]
    \begin{SongInfo}
        Text: Bosse Carlgren\\*%
        Mel: Zuckerman’s Famous Pig eller Magnus \& Brasse
    \end{SongInfo}
    \begin{SongVerse}
        Din satan, satan\\*%
        Du din satans helvetes jävla skit\\*%
        Din jävla skit
    \end{SongVerse}
    \begin{SongVerse}
        Bonnlurk, läbbiga skurk, ynkliga parasit\\*%
        Pottsork, snuskiga stork\\*%
        Din ruttna rot e’ full av kork
    \end{SongVerse}
    \begin{SongVerse}
        Avskumm, spattig o krumm\\*%
        Du e’ jävla dum\\*%
        Din usla gam, din slemmiga torsk\\*%
        Förnicklade pappskalle, skunk
    \end{SongVerse}
    \begin{SongVerse}
        Förbana mig, lägg ägg slibbiga drägg\\*%
        Skitstövel å bandit\\*%
        Slashas, ditt vidriga as\\*%
        Piss o pest o senapsgas
    \end{SongVerse}
    \begin{SongVerse}
        Sopprot, helidiot, fan va du bär emot\\*%
        Din sabla bock, ditt feta arsle våga dig aldrig mer hit\\*%
        Attans skitstropp
    \end{SongVerse}
    \begin{SongVerse}
        Hörru din sa-aa-aa. sa-a-atans\\*%
        Helvete helvele helvete helvetes, helvetes\\*%
        Helvetes jävla skit
    \end{SongVerse}
\end{SongText}
%\begin{SongText}[Integralvisan]
%    \begin{SongInfo}
%        Mel: Med en enkel tulipan
%    \end{SongInfo}
%    \begin{SongVerse}
%        En liten enkel integral\\*%
%        i ett vektoranalystal\\*%
%        ni har besväret, ni har besväret\\*%
%        att derivera.\\*%
%        Men tar man Stokes sats däruppå\\*%
%        så blir det så enkelt så\\*%
%        att integralen, att integralen\\*%
%        evaluera.\\*%
%        Och rotationen, den integreras\\*%
%        som över ytan av en boll.\\*%
%        Koordinaterna transformeras\\*%
%        så integralen blir bara noll.\\*%
%        En liten enkel integral\\*%
%        i ett vektoranalystal\\*%
%        kan va så jävligt\\*%
%        att man ej hinner med något mera.
%    \end{SongVerse}
%\end{SongText}
%\begin{SongText}[Laborationsvisan]
%    \begin{SongInfo}
%        (EDITOR NOTE: hittar varken melodi eller referat till text någonstnas)\\*%
%        (EDITOR NOTE: Hittar att "Teknis" refererar till KTH)
%    \end{SongInfo}
%    \begin{SongVerse}
%        Till Teknis kom ett gäng med livade grabbar,\\*%
%        som skulle göra en rad elektriska tabbar,\\*%
%        och vi försåg oss med flaskor i alla fickor\\*%
%        och gav oss ut för att leta mulliga flickor
%    \end{SongVerse}
%    \begin{SongVerse}
%        Fy fan för systemet!
%    \end{SongVerse}
%    \begin{SongVerse}
%        Vi drog i växelström genom stadens alla gator,\\*%
%        men kunde blott hitta några enstaka skator,\\*%
%        som kring sin axel lagt på en värmande lindning\\*%
%        och uti plattfoten hade god jordförbindning.
%    \end{SongVerse}
%    \begin{SongVerse}
%        Fy fan för systemet!
%    \end{SongVerse}
%    \begin{SongVerse}
%        Så kom en flicka, som var elektromagnetisk,\\*%
%        och jag drogs dit, fastän jag är ganska atletisk.\\*%
%        Vi drogs ihop ty vi hade motsatta poler,\\*%
%        och snart låg kraftlinjerna som strängar på fioler.
%    \end{SongVerse}
%    \begin{SongVerse}
%        Fy fan för systemet!
%    \end{SongVerse}
%    \begin{SongVerse}
%        Det blev en kittlande spänning mellan oss båda,\\*%
%        men nå’n kontakt kunde jag dock ej få att råda,\\*%
%        förrän jag lossat på hennes starka armering\\*%
%        och med försiktighet ta’t bort all isolering.
%    \end{SongVerse}
%    \begin{SongVerse}
%        Fy fan!\\*%
%        Så blev det kortslutning, så det small med detsamma,\\*%
%        och proppen gick, och jag sade: ”Jävlar anamma!”\\*%
%        Jag hade gjort en av mina svåraste tabbar,\\*%
%        ja, det är farligt att göra sådana labbar.\\*%
%        Fyyyyy!
%    \end{SongVerse}
%\end{SongText}
%\begin{SongText}[Varför Chalmers]
%    \begin{SongInfo}
%        (EDITOR NOTE: Hittar att "Teknis" refererar till KTH)\\*%
%        Mel: Den svenska flottisten
%    \end{SongInfo}
%    \begin{SongVerse}
%        Varför Chalmers\\*%
%        när det finns Teknis,\\*%
%        det kan jag inte förstå.
%    \end{SongVerse}
%    \begin{SongVerse}
%        Chalmerister hit och dit,\\*%
%        med de är ju bara skit.\\*%
%        Hej och hå, MdH!!
%    \end{SongVerse}
%\end{SongText}
%\begin{SongText}[Herrarna i hagen (alternativ)]
%    \begin{SongInfo}
%        Mel: I fjol så gick jag med herrarna i hagen
%    \end{SongInfo}
%    \begin{SongVerse}
%        I fjol så gick hon men herrarna i hagen\\*%
%        aj, aj, med herrarna i hagen\\*%
%        aj, med herrarna i hagen
%    \end{SongVerse}
%    \begin{SongVerse}
%        Men nu så går hon hemma med ungfan i magen\\*%
%        aj, aj,…
%    \end{SongVerse}
%    \begin{SongVerse}
%        I fjol var hon sexton och alla ville ha’na\\*%
%        aj, aj…
%    \end{SongVerse}
%    \begin{SongVerse}
%        Men nu är hon sjutton och ingen vill ta’na\\*%
%        aj, aj…
%    \end{SongVerse}
%    \begin{SongVerse}
%        Pojkar tycker om tjejer med ballonger\\*%
%        aj, aj…
%    \end{SongVerse}
%    \begin{SongVerse}
%        Det är bättre än rena kalsonger\\*%
%        aj, aj…
%    \end{SongVerse}
%    \begin{SongVerse}
%        Tjejer gillar pojkar som har stora kanoner\\*%
%        aj, aj…
%    \end{SongVerse}
%    \begin{SongVerse}
%        Helst ska det då vara grevar och baroner\\*%
%        aj, aj…
%    \end{SongVerse}
%    \begin{SongVerse}
%        I Finlands djupa skogar där får man supa gratis\\*%
%        aj, aj…
%    \end{SongVerse}
%\end{SongText}
%\begin{SongText}[Vem säger nej]
%    \begin{SongInfo}
%        Mel: I sommarens soliga dagar
%    \end{SongInfo}
%    \begin{SongVerse}
%        När vi följer känslornas lagar\\*%
%        då tar vi den vi behagar\\*%
%        och vi vill ej ha tjocka magar\\*%
%        det lockar oss ej nu, ännu, ännu
%    \end{SongVerse}
%    \begin{SongVerse}
%        Att vara tjej det är vår grej\\*%
%        att ej behöva resa sig\\*%
%        för att stå på, och lyckan nå\\*%
%        det går som bäst när man är två,\\*%
%        så kanske tar vi någon med oss\\*%
%        när vi går hem i kväll, javäl, javäl.
%    \end{SongVerse}
%    \begin{SongVerse}
%        När man är av nån intresserad\\*%
%        så får man ej vara generad\\*%
%        och vänta sig bli offererad\\*%
%        som tjejer gjorde förr, varför, varför?
%    \end{SongVerse}
%    \begin{SongVerse}
%        Jo därför att, nå’n ensam natt\\*%
%        den vackre prinsen plötsligt satt\\*%
%        där vid ens förr, så var det förr\\*%
%        men nu som människor vi bör\\*%
%        ha modet att ta den vi önskar\\*%
%        när lusten faller på, en då, och då.
%    \end{SongVerse}
%    \begin{SongVerse}
%        Att vara tjej…
%    \end{SongVerse}
%\end{SongText}
\begin{SongText}[Kosmonauten Nikolajeffs kvalfulla rymdfärd]
    \begin{SongInfo}
        Mel: Sovjetunionens nationalsång
    \end{SongInfo}
    \begin{SongVerse}
        Mitt namn är Nikolajeff, kosmonaut från sovjet.\\*%
        Jag flyger runt jorden i min rymdraket.\\*%
        Men jag har råkat ut för en olycka så stor:\\*%
        Jag glömde gå på muggen innan jag for!
    \end{SongVerse}
    \begin{SongVerse}
        Ref:\\*%
        Jag längtar ner - till min hemplanet,\\*%
        till fru och barn - hemmet i sovjet,\\*%
        men mest till ett hus med hjärta på dörr’n.\\*%
        Jag längtar ner - till min hemplanet,\\*%
        Jag vill ej vara instängd i raket,\\*%
        där det ej finns nå’t hus med hjärta på dörr’n.
    \end{SongVerse}
    \begin{SongVerse}
        Min kapsel är fylld utav fina instrument,\\*%
        som mäter kosmisk strålning och sånt som ej är känt.\\*%
        Här finns nästan allt jag behöver av komfort,\\*%
        men en viktig detalj, den har dom glömt bort!
    \end{SongVerse}
    \begin{SongVerse}
        Ref...
    \end{SongVerse}
    \begin{SongVerse}
        Jag lider de allra hemskaste kval.\\*%
        Jag känner mig vissen, sjuklig och skral.\\*%
        Det bubblar i magen, nu tränger det på.\\*%
        Jag undrar, hur fasen detta skall gå.
    \end{SongVerse}
    \begin{SongVerse}
        Ref...
    \end{SongVerse}
    \begin{SongVerse}
        Jag ska vara uppe i 64 varv.\\*%
        Det har Chrustschov sagt, men jag tycker det är larv.\\*%
        Det tar flera da’r, innan jag kommer ner.\\*%
        Nej, aj, vad det värker! Jag står ej ut mer!
    \end{SongVerse}
    \begin{SongVerse}
        Ref...
    \end{SongVerse}
\end{SongText}
\begin{SongText}[Minnet]
    \begin{SongInfo}
        Mel: Memory (Cats Broadway Musical /1983)
    \end{SongInfo}
    \begin{SongVerse}
        Minne, jag har tappat mitt minne.\\*%
        Är jag svensk eller finne?\\*%
        Kommer inte ihåg.\\*%
        Inne, är jag ut’ eller inne?\\*%
        Jag har luckor i minne’,\\*%
        sån’ där små alkohol.\\*%
        Men besinn’ er,\\*%
        man tätar med det brännvin man får,\\*%
        fastän minne’ och helan går.\\*%
        Minne? Muisti hävis, mutt’ minne?\\*%
        Juhlista selvisimme\\*%
        muistikatkoja on.\\*%
        Minne, lähtisin vaikka minne,\\*%
        kunhan selvittäisimme\\*%
        mitä tapahtunut on.\\*%
        Mutta tiedän\\*%
        mä keinon mikä auttaapi tuo:\\*%
        Ota ryyppy, ja muistis juo!
    \end{SongVerse}
\end{SongText}
\begin{SongText}[Balladen om den onyktre]
    \begin{SongInfo}
        Mel: När månen vandrar på fästet blå
    \end{SongInfo}
    \begin{SongVerse}
        När jag är fuller då är jag glad,\\*%
        fan vet om jag ej är vacker.\\*%
        Jag vandrar kring i vår lilla stad,\\*%
        ibland lyxhus och baracker.\\*%
        Jag sjunger ljuvligt en serenad,\\*%
        det gör jag bara när jag är glad\\*%
        och full och vacker, och full och vacker.
    \end{SongVerse}
    \begin{SongVerse}
        När jag är fuller då är jag stark,\\*%
        fan vet om jag ej är modig.\\*%
        Då kan jag slå vem som helst i mark,\\*%
        så han blir trasig och blodig.\\*%
        Jag välter träden i våran park,\\*%
        det gör jag bara när jag är stark\\*%
        och full och modig, och full och modig.
    \end{SongVerse}
    \begin{SongVerse}
        När jag är fuller då är jag rik,\\*%
        fan vet om jag ej är snille.\\*%
        Och dör jag blir jag ett vackert lik,\\*%
        begravs med gravöl och gille.\\*%
        I himlen möts jag av hornmusik,\\*%
        det gör man bara när man är rik\\*%
        och är ett snille, och är ett snille.
    \end{SongVerse}
    \begin{SongVerse}
        Men när jag vaknar upp nästa dag,\\*%
        uppå ett enkelrum med galler.\\*%
        Då känner jag mig så rysligt svag,\\*%
        och hatar bråk och kravaller.\\*%
        Min mage krånglar och är ur lag,\\*%
        nog fan så vet jag att jag idag\\*%
        är bakom galler, är bakom galler.
    \end{SongVerse}
\end{SongText}
\begin{SongText}[Fjompe tompe törstig]
    \begin{SongInfo}
        Mel: Imse vimse spindel
    \end{SongInfo}
    \begin{SongVerse}
        Fjompa tompe törstig\\*%
        Sitter på en bar\\*%
        Vickar lätt på foten\\*%
        Och sen i golvet far\\*%
        Strax nån fyller\\*%
        På hans glas och sen\\*%
        Fjompa tompe törstig\\*%
        Klättrar upp igen
    \end{SongVerse}
\end{SongText}
\begin{SongText}[Schottis på Valhall]
    \begin{SongInfo}
        Text: Ulf Peder Olrog
    \end{SongInfo}
    \begin{SongVerse}
        Ref:\\*%
        Opp och hoppa, Tor, slå på trumman, bror.\\*%
        Det är dans uppå Valhall i natt.\\*%
        Uti Frejas sal står vår asabal.\\*%
        Opp och hoppa, fast Odin har spatt.\\*%
        Slå i mera mjöd. Det får bli min död.\\*%
        Nej, se där är ju Idun, min skatt.\\*%
        Min valkyria kom hit till vår midvinterrit.\\*%
        Opp och hoppa på Valhall i natt.
    \end{SongVerse}
    \begin{SongVerse}
        Höder han hade hiskelig hicka,\\*%
        Balder den bota med ingefärsdricka.\\*%
        Vred vart väl Ving-Tor, vakna och vråla.\\*%
        Brage bråka och Skade hon skrek:
    \end{SongVerse}
    \begin{SongVerse}
        Ref..
    \end{SongVerse}
    \begin{SongVerse}
        Heimdal i hornet blåste och brumma.\\*%
        Loke han låg där och lekte och lulla.\\*%
        Gudarna gorma, röto och rulla.\\*%
        Allfader Odin kvidde och kvad:
    \end{SongVerse}
    \begin{SongVerse}
        Ref..
    \end{SongVerse}
\end{SongText}
%\begin{SongText}[Inlandsbanan]
%    \begin{SongInfo}
%        (EDITOR NOTE: Hittar inget om melodin, bara från Norrlands Nations spexensamble under "gamla godingar")\\*%
%        Mel: Amanda Lundström
%    \end{SongInfo}
%    \begin{SongVerse}
%        När vi byggde inlandsbana\\*%
%        Bomfadderi fadder inlandsbanan\\*%
%        Då var alla flickor glada\\*%
%        Bomfadderi fadderullanlej\\*%
%        Skaffa sig en fästeman\\*%
%        Bomfadderi fadderullanlej\\*%
%        Och pippade det gjorde man \\*%
%        Bomfadderi fadderullanlej
%    \end{SongVerse}
%    \begin{SongVerse}
%        Mor, oh mor jag är en hora\\*%
%        Bomfadderi fadderrullan hora\\*%
%        Å jag räknas till dem stora\\*%
%        Bomfadderi fadderullanlej\\*%
%        Sjutton år och gängad mus\\*%
%        Bomfadderi fadderullanlej\\*%
%        När andra flickor kör med ljus\\*%
%        Bomfadderi fadderullanlej
%    \end{SongVerse}
%    \begin{SongVerse}
%        Barn, oh barn du skall ej gråta\\*%
%        Bomfadderi fadderullan gråta\\*%
%        Än så länge är karlar kåta\\*%
%        Bomfadderi fadderullanlej\\*%
%        Pippar du så vaggar jag\\*%
%        Bomfadderi fadderullanlej\\*%
%        Och pippade det gjorde jag\\*%
%        Bomfadderi fadderullanlej
%    \end{SongVerse}
%\end{SongText}
\begin{SongText}[Så länge rösten är mild]
    \begin{SongInfo}
        Mel: Så länge skutan kan gå
    \end{SongInfo}
    \begin{SongVerse}
        Så länge rösten är mild,\\*%
        så länge ingen är vild,\\*%
        så länge spegeln på väggen ger halvskaplig bild.\\*%
        Så länge alla kan stå,\\*%
        så länge alla kan gå,\\*%
        så länge alla kan tralla så fyller vi på.\\*%
        För vem har sagt att just du kom med storken,\\*%
        för att bli glad av att lukta på korken.\\*%
        Nej, till kvinns och till mans,\\*%
        vi höjer bägar'n med glans,\\*%
        och låter ölen gå ner i en yrande dans!
    \end{SongVerse}
\end{SongText}
%\begin{SongText}[Var nöjd (livet ger)]
%    \begin{SongInfo}
%        Text: Disney's Djungelboken
%    \end{SongInfo}
%    \begin{SongVerse}
%        Ref:\\*%
%        Var nöjd med allt som livet ger\\*%
%        och allting som du kring dig ser\\*%
%        glöm bort bekymmer sorger och besvär.\\*%
%        Var glad och nöjd, för vet du vad?\\*%
%        En björntjänst gör ju ingen glad.\\*%
%        Var nöjd med livet som vi lever här.
%    \end{SongVerse}
%    \begin{SongVerse}
%        Varthän jag än strövar, varthän jag än går\\*%
%        står ljung och snår kring mina spår.\\*%
%        Jag älskar bin och deras bon,\\*%
%        för honung är ju min passion\\*%
%        och vill du av myror ha munnen full\\*%
%        så ta en titt under sten och mull.\\*%
%        Kanske smaka på dom!\\*%
%        Äta myror?\\*%
%        - Ha, ha det är världens käk.\\*%
%        Kittlar dödsskönt i kistan.\\*%
%        Var nöjd med allt du ser\\*%
%        och allt som livet ger.
%    \end{SongVerse}
%    \begin{SongVerse}
%        Ref...
%    \end{SongVerse}
%    \begin{SongVerse}
%        Om frukter dig lockar, banan eller bär.\\*%
%        Se till att du plockar dom utan besvär.\\*%
%        Vill du plocka frukt av bästa klass\\*%
%        så använd din höger och vänster tass,\\*%
%        men klorna dom skall du dra in\\*%
%        så fort du ska ta dig en fin apelsin.\\*%
%        Hoppas att du har förstått?\\*%
%        O, ja! Tack Baloo!
%    \end{SongVerse}
%    \begin{SongVerse}
%        Ref...
%    \end{SongVerse}
%\end{SongText}
\begin{SongText}[Halta Lottas krog]
    \begin{SongInfo}
        Mel: från filmen "En duva satt på en gren och funderade på tillvaron"
    \end{SongInfo}
    \begin{SongVerse}
        \textit{Dur:}\\*%
        Bröder, viljen I gå med oss\\*%
        Bröder, viljen I gå med oss\\*%
        Bröder, viljen I gå med oss,\\*%
        uppå Halta Lottas krog i Göteborg.
    \end{SongVerse}
    \begin{SongVerse}
        Ja, visst vilja vi gå med er...
    \end{SongVerse}
    \begin{SongVerse}
        Femton öre kostar supen...
    \end{SongVerse}
    \begin{SongVerse}
        Vi som inga pengar hava...
    \end{SongVerse}
    \begin{SongVerse}
        Varmed skola vi betala...
    \end{SongVerse}
    \begin{SongVerse}
        Med kyssar skola vi betala...
    \end{SongVerse}
    \begin{SongVerse}
        Vem skall dessa kyssar hava...
    \end{SongVerse}
    \begin{SongVerse}
        Det ska Halt Lottas syster...
    \end{SongVerse}
    \begin{SongVerse}
        \textit{Moll:}  \\*%
        Halta Lotta är nu döder...
    \end{SongVerse}
    \begin{SongVerse}
        Feldt har övertagit krogen...\\*%
        Femtio kronor kostar supen...
    \end{SongVerse}
\end{SongText}
\begin{SongText}[Tingelingeling]
    \begin{SongVerse}
        Tingelingeling nu tåget går
        ut i vida världen,
        den som femtio öre har,
        får följa med på färden
        se NN, se NN
        nu löser han biljetten
        nu kliver han på
        och så ska tåget gå
        Tingelingeling...(in absurdum)\\*%
        \emph{Vid NN sätter man in ett lämpligt egen namn,
            t.ex. Osquar, Osqulda, Ada, Konrad, Ture, m.fl. }
    \end{SongVerse}
\end{SongText}
\begin{SongText}[Sailing]
    \begin{SongInfo}
        Mel: Sailing by Rod Stewart
    \end{SongInfo}
    \begin{multicols}{2}
        \begin{SongVerse}
            %((EDITORS NOTE: Figure out how to transfer this shite))
            \textbf{(1)Alla:}\\*%
            We are sailing\\*%
            We are sailing\\*%
            Thro' the water\\*%
            'Cross the sea\\*%
            We are sailing\\*%
            Lugna waters\\*%
            To be ute\\*%
            To be free\\*%
        \end{SongVerse}
        \begin{SongVerse}
            \textbf{(2)Han:}\\*%
            I am länsing\\*%
            I am glänsig\\*%
            Forsning framåt\\*%
            Mera vind\\*%
            Elva meter\\*%
            Fulla segel\\*%
            That's the life\\*%
            My dear wife\\*%
        \end{SongVerse}
        \begin{SongVerse}
            \textbf{(3)Hon:}\\*%
            I am freezing\\*%
            It is gunging\\*%
            Båten laying\\*%
            Oh my God\\*%
            Children crying\\*%
            Starta motorn\\*%
            Oh where are we\\*%
            I'll go home\\*%
        \end{SongVerse}
        \begin{SongVerse}
            \textbf{(4)Han:}\\*%
            Can't you here me\\*%
            Tyst du stör mej\\*%
            I'm kapp-sailing\\*%
            All the time\\*%
            Med en Maxi\\*%
            To be winner\\*%
            If I don't\\*%
            I will die\\*%
        \end{SongVerse}
        \begin{SongVerse}
            \textbf{(5)Hon:}\\*%
            Can't you here me\\*%
            Can't you here me\\*%
            Reva segel\\*%
            Sun is down\\*%
            Båt is läcking\\*%
            Can't hear häcking\\*%
            Jag tar bussen\\*%
            Into town\\*%
        \end{SongVerse}
        \begin{SongVerse}
            \textbf{(6)Alla:}\\*%
            We are sailing\\*%
            We are sailing\\*%
            Home again\\*%
            Across the sea\\*%
            We are flyting\\*%
            Forever trying\\*%
            To be happy\\*%
            To be free\\*%
        \end{SongVerse}
    \end{multicols}
\end{SongText}
\begin{SongText}[What shall we do with the drunken sailor?]
    \begin{SongVerse}
        What shall we do with the drunken sailor?\\*%
        What shall we do with the drunken sailor?\\*%
        What shall we do with the drunken sailor,\\*%
        early in the morning?
    \end{SongVerse}
    \begin{SongVerse}
        Hooray and up she rises.\\*%
        Hooray and up she rises.\\*%
        Hooray and up she rises\\*%
        early in the morning.
    \end{SongVerse}
    \begin{SongVerse}
        Put him in the longboat till he's sober...
    \end{SongVerse}
    \begin{SongVerse}
        Pull out the plugand wet him all over...
    \end{SongVerse}
    \begin{SongVerse}
        Give him a hair of the dog that bit him...
    \end{SongVerse}
    \begin{SongVerse}
        Put him in the scuppers with a hose-pipe on him...
    \end{SongVerse}
    \begin{SongVerse}
        Heave him by the leg in a runnin' bowline...
    \end{SongVerse}
    \begin{SongVerse}
        Shave his head with a rusty razor...
    \end{SongVerse}
    \begin{SongVerse}
        Put him in the bed with the captain's daughter...
    \end{SongVerse}
    \begin{SongVerse}
        Vad ska vi göra med KTH:arn?\\*%
        Vad ska vi göra med KTH:arn?\\*%
        Vad ska vi göra med KTH:arn?\\*%
        Early in the morning
    \end{SongVerse}
    \begin{SongVerse}
        DRA HAN I PISSERÄNNAN!\\*%
        DRA HAN I PISSERÄNNAN!\\*%
        DRA HAN I PISSERÄNNAN!\\*%
        Early in the morning.
    \end{SongVerse}
\end{SongText}
\begin{SongText}[Uti min mage]
    \begin{SongInfo}
        Mel: Uti vår hage
    \end{SongInfo}
    \begin{SongVerse}
        Uti min mage där växa begär.\\*%
        Kom hjärtans kär.\\*%
        Där råder en hunger som ropar så här:\\*%
        Kom kryddsill och kall potatis,\\*%
        kom brännvin och quantum satis,\\*%
        kom allt som kan drickas,\\*%
        kom hjärtans fröjd!
    \end{SongVerse}
    \begin{SongVerse}
        Uti min mage där växa begär.\\*%
        Kom hjärtans kär.\\*%
        Vill du mig något så har jag det där,\\*%
        kom Skåne och Aqua Vitæ,\\*%
        kom OP och allt vad sprit ä’,\\*%
        kom ljuva genever,\\*%
        kom Överste!
    \end{SongVerse}
    \begin{SongVerse}
        Uti min mage en längtan mig tär.\\*%
        Kom hjärtans kär.\\*%
        Där råder en hunger som ropar så här:\\*%
        Kom famnande lena armar,\\*%
        kom läppar och sköna barmar,\\*%
        kom fagraste qvinnor,\\*%
        kom hjärtans fröjd!
    \end{SongVerse}
\end{SongText}
%\begin{SongText}[Göken]
%    \begin{SongInfo}
%        Mel:Räven raskar
%    \end{SongInfo}
%    \begin{SongVerse}
%        (herrarna)\\*%
%        Snapsen kallas också göken\\*%
%        snapsen kallas också göken\\*%
%        så får jag lov, så får jag lov\\*%
%        att byta byxor med fröken?
%    \end{SongVerse}
%    \begin{SongVerse}
%        (damerna)\\*%
%        Nej, det går inte alls min herre\\*%
%        nej, det går inte alls min herre\\*%
%        ty jag har, ty jag har\\*%
%        inga byxor dessvärre.
%    \end{SongVerse}
%\end{SongText}
\begin{SongText}[Räven v2]
    \begin{SongInfo}
        Mel: Dvärgarnas visa
    \end{SongInfo}
    \begin{SongVerse}
        Jag fångade en räv igår, men räven slank ur näven.\\*%
        Men lika glad är jag för det men gladast är nog räven.\\*%
        Raj-raj…
    \end{SongVerse}
    \begin{SongVerse}
        Jag fångade en sup igår, men sup slank ur näven.\\*%
        Men lika glad är jag för det men gladast är nog levern.\\*%
        Raj-raj…
    \end{SongVerse}
    \begin{SongVerse}
        Jag fångade en präst igår, men prästen damp i gatan.\\*%
        Men lika glad är jag för det men gladast är nog satan.\\*%
        Raj-raj…
    \end{SongVerse}
    \begin{SongVerse}
        Jag fångade en tjej idag, men tjejen slank ur sängen.\\*%
        Men lika glad är jag för det, vi fortsatte på ängen.\\*%
        Raj-raj…
    \end{SongVerse}
    \begin{SongVerse}
        Jag fångade en biff idag, men den var seg som kola.\\*%
        Men lika glad är jag för det, för oj vad kul att skåla.\\*%
        Raj-raj…
    \end{SongVerse}
\end{SongText}
\begin{SongText}[Teknologens testamente]
    \begin{SongInfo}
        Mel: She'll be coming 'round the mountain
    \end{SongInfo}
    \begin{SongVerse}
        $\|\:$ Du ska få min mekanikbok när jag dör. $\:\|$\\*%
        För i paradisets hagar\\*%
        gäller inte Newtons lagar.\\*%
        Du ska få min mekanikbok när jag dör.
    \end{SongVerse}
    \begin{SongVerse}
        Ref:\\*%
        $\|\:$ Hon sa aj, aj, yippe yippe aj. $\:\|$\\*%
        För i paradisets hagar\\*%
        gäller inte Newtons lagar.\\*%
        Du ska få min mekanikbok när jag dör.
    \end{SongVerse}
    \begin{SongVerse}
        $\|\:$ Du ska få min Simpsons regel när jag dör. $\:\|$\\*%
        För bland ängar och keruber\\*%
        äro alla rum som kuber.\\*%
        Du ska få min Simpsons regel när jag dör.
    \end{SongVerse}
    \begin{SongVerse}
        Ref..
    \end{SongVerse}
    \begin{SongVerse}
        $\|\:$ Du ska få mitt vinkelprisma när jag dör. $\:\|$\\*%
        För där nere i det heta\\*%
        äro alla vinklar räta.\\*%
        Du ska få mitt vinkelprisma när jag dör.
    \end{SongVerse}
    \begin{SongVerse}
        Ref..
    \end{SongVerse}
    \begin{SongVerse}
        $\|\:$ Du ska få vår norska dator när jag dör. $\:\|$\\*%
        För där nere bland allt svavel\\*%
        finns det ändå nog med dravel.\\*%
        Du ska få vår norska dator när jag dör.
    \end{SongVerse}
    \begin{SongVerse}
        Ref..
    \end{SongVerse}
\end{SongText}
\begin{SongText}[Liten Jumbo]
    \begin{SongInfo}
        Mel: Katjusha
    \end{SongInfo}
    \begin{SongVerse}
        Liten Jumbo flyger över sovjet\\*%
        MiG-23 går upp och skjuter ner.\\*%
        $\|\:$ Liten Jumbo störtar ner i havet,\\*%
        liten Jumbo flyger aldrig mer $\:\|$
    \end{SongVerse}
    \begin{SongVerse}
        Liten ubåt simmar in i Sverige\\*%
        Liten ubåt stöter på ett grund\\*%
        $\|\:$ Liten ubåt flyter upp till ytan,\\*%
        liten ubåt simmar aldrig mer $\:\|$
    \end{SongVerse}
    \begin{SongVerse}
        Liten Ivar springer över stäppen\\*%
        K-pist smattrar, Ivar faller ner.\\*%
        $\|\:$ Blod och slamsor spridda över stäppen,\\*%
        liten Ivar springer aldrig mer $\:\|$
    \end{SongVerse}
    \begin{SongVerse}
        Litet kraftverk smäller uti Sovjet\\*%
        Ryssar dör som flugor runt om kring.\\*%
        $\|\:$ Hela saken tystas ner av staten,\\*%
        litet kraftverk finns ej längre mer. $\:\|$
    \end{SongVerse}
    \begin{SongVerse}
        Olof Palme kommer ut från bion\\*%
        Krister skjuter, Olof faller ner.\\*%
        $\|\:$ Olof Palme faller ner i gatan,\\*%
        Olof Palme finns ej längre mer. $\:\|$
    \end{SongVerse}
    \begin{SongVerse}
        Litet flygplan flyger över Stockholm.\\*%
        Vindpust kommer, planet faller ner.\\*%
        $\|\:$ Plåt och bränsle sprides över staden,\\*%
        litet flygplan flyger aldrig mer. $\:\|$
    \end{SongVerse}
\end{SongText}
\begin{SongText}[Tentalåt]
    \begin{SongInfo}
        Text: Göran Fägersten\\*%
        Mel: Man ska’ ha husvagn
    \end{SongInfo}
    \begin{SongVerse}
        Jag har provat nästan allt som finns att välja på.\\*%
        Plugga, fuska, mygla, chansa, fiffla eller gå.\\*%
        Jag har tenterat på de allra konstigaste sätt,\\*%
        men äntligen jag funnit hur man ska tentera rätt.
    \end{SongVerse}
    \begin{SongVerse}
        Ref:
    \end{SongVerse}
    \begin{SongVerse}
        Man ska va stenfull,\\*%
        då är tentamen redan klar.\\*%
        Man ska var stenfull,\\*%
        då finns det bara lätta tal.\\*%
        Man ska va stenfull,\\*%
        så finns inga svåra svar.\\*%
        Man ska vara stenfull,\\*%
        uti tentans sal.
    \end{SongVerse}
    \begin{SongVerse}
        I många år så gick vi runt och undra vad vi behövde,\\*%
        vi ringde runt till Umeå och högskolan i Skövde.\\*%
        Och dessa sade oss vad vi behövde ha,\\*%
        ett riktigt jävla långtidsrus och detta helst var dag.
    \end{SongVerse}
    \begin{SongVerse}
        Ref...
    \end{SongVerse}
    \begin{SongVerse}
        Fem minuter debet fem minuter kredit\\*%
        fem minuter supa och fem minuter ledigt.\\*%
        Fem minuter capsa fem minuter lunch,\\*%
        sen så är det bara att fylla på med punsch.
    \end{SongVerse}
    \begin{SongVerse}
        Ref...
    \end{SongVerse}
\end{SongText}
%\begin{SongText}[Our family]
%    \begin{SongInfo}
%        Mel: My Bonny is over the ocean
%    \end{SongInfo}
%    \begin{SongVerse}
%        My father makes counterfit money,\\*%
%        my mother brews synthetic gin.\\*%
%        My sister sells kisses to sailors\\*%
%        by Jove how the money rolls in!\\*%
%        $\|\:$ By Jove, by Jove, by Jove how the money\\*%
%        rolls in, rolls in. $\:\|$
%    \end{SongVerse}
%    \begin{SongVerse}
%        My brother’s a slum missionary,\\*%
%        saving young maidens from sin.\\*%
%        He’ll save you a blonde for a shilling,\\*%
%        by Jove how the money rolls in!\\*%
%        $\|\:$ By Jove... $\:\|$
%    \end{SongVerse}
%    \begin{SongVerse}
%        My aunt keeps a girl’s seminary,\\*%
%        teaching young girls to begin.\\*%
%        She doesn’t say where they’re to finish,\\*%
%        by Jove how the money rolls in!\\*%
%        $\|\:$ By Jove... $\:\|$
%    \end{SongVerse}
%    \begin{SongVerse}
%        My brother’s a Harley Street surgeon.\\*%
%        His instruments are long and thin.\\*%
%        He only does one operation,\\*%
%        by Jove how the money rolls in!\\*%
%        $\|\:$ By Jove... $\:\|$
%    \end{SongVerse}
%\end{SongText}
\begin{SongText}[Storstaden Mora]
    \begin{SongInfo}
        Mel: När jag var en ung caballiero
    \end{SongInfo}
    \begin{SongVerse}
        Jag åkte till storstaden Mora,\\*%
        där träffa jag stans största – flicka.\\*%
        En flicka...
    \end{SongVerse}
    \begin{SongVerse}
        Ref:\\*%
        ...för kärlek och solsken och sång,\\*%
        för solsken och kärlek och sång. Pling plong!
    \end{SongVerse}
    \begin{SongVerse}
        Jag sade att jag kom från Smögen,\\*%
        Där kallas jag allmänt för – Börje.\\*%
        Börje...
    \end{SongVerse}
    \begin{SongVerse}
        Ref...
    \end{SongVerse}
    \begin{SongVerse}
        Hon sade att hon hette Ulla,\\*%
        Hon frågade om vi skulle – fika.\\*%
        Fika...
    \end{SongVerse}
    \begin{SongVerse}
        Ref...
    \end{SongVerse}
    \begin{SongVerse}
        Hon fråga om jag ville titta\\*%
        på hennes förtjusande – Fiat.\\*%
        En Fiat...
    \end{SongVerse}
    \begin{SongVerse}
        Ref...
    \end{SongVerse}
    \begin{SongVerse}
        Hon bjöd mig på kaffe och tårta,\\*%
        och vi blev så helvetes – mätta\\*%
        Så mätta...
    \end{SongVerse}
    \begin{SongVerse}
        Ref...
    \end{SongVerse}
    \begin{SongVerse}
        Jag spillde ut kafet på duken,\\*%
        och genast så reste sig – Ulla.\\*%
        Ulla...
    \end{SongVerse}
    \begin{SongVerse}
        Ref...
    \end{SongVerse}
    \begin{SongVerse}
        Hon gick och hämtade trasan från vasken,\\*%
        för att torka bort kaffet från – duken.\\*%
        Duken...
    \end{SongVerse}
    \begin{SongVerse}
        Ref...
    \end{SongVerse}
    \begin{SongVerse}
        Sakta hemåt jag lunka,\\*%
        Jag stannade i en trappuppgång för att – röka.\\*%
        Röka...
    \end{SongVerse}
    \begin{SongVerse}
        Ref...
    \end{SongVerse}
\end{SongText}
\begin{SongText}[Avslutningssången]
    \begin{SongInfo}
        Mel: Den blomstertid nu kommer
    \end{SongInfo}
    \begin{SongVerse}
        Den studietid nu varit\\*%
        med tentor och projekt\\*%
        Det varit ganska marigt\\*%
        och några har den knäckt.\\*%
        Ni andra har här samlats\\*%
        och nästan allt är klart.\\*%
        Hur kan nu detta kännas,\\*%
        om inte underbart.
    \end{SongVerse}
    \begin{SongVerse}
        Nu kommer härlig sommar\\*%
        för oss som än är kvar\\*%
        För er som nu skall knega\\*%
        glöm ej bort hur det var\\*%
        med onsdagar på puben\\*%
        och greven som dessert\\*%
        Ni säkert alltid ångrar\\*%
        att inte stanna här
    \end{SongVerse}
\end{SongText}
\begin{SongText}[Rysk Bompa]
    \begin{SongInfo}
        Text: De Lyckliga Kompisarna
    \end{SongInfo}
    \begin{SongVerse}
        En kväll förvandlades bompan till Kreml\\*%
        (Bompan till Kreml, bompan till Kreml)\\*%
        Och in kom Boris med skjorta och hjälm\\*%
        (Skjorta och hjälm och ingenting mer)\\*%
        Jag skall bjuda laget runt\\*%
        För ni super jävligt tunt
    \end{SongVerse}
    \begin{SongVerse}
        Hörru, har du supit något sen sist?\\*%
        Hur många hjärnceller har du mist?\\*%
        Har du däckat eller spytt\\*%
        Eller på östfronten inte flytt?\\*%
        Här är du och här är jag\\*%
        Borden gungar i Moskva\\*%
        Hörru Boris, du vet precis\\*%
        Hur slipstenen skall dra
    \end{SongVerse}
    \begin{SongVerse}
        Vi gick till Slussen och kollade på fyrverkeri\\*%
        (Fyrverkeri, fyrverkera)\\*%
        I förrgår var det Frankrike, idag är det jag\\*%
        (Fyrverkeri, hurra)\\*%
        Han tröck på knappen, skådespelet tog fart\\*%
        Jag nödgades fråga min gamla kamrat
    \end{SongVerse}
    \begin{SongVerse}
        Hörru, har du supit något sen sist?\\*%
        Hur många hjärnceller har du mist?\\*%
        Har du däckat eller spytt\\*%
        Eller på östfronten inte flytt?\\*%
        Här är du och här är jag\\*%
        Borden gungar i Moskva\\*%
        Hörru Boris, du vet precis\\*%
        Hur slipstenen skall dras
    \end{SongVerse}
    \begin{SongVerse}
        Hörru Boris, du vet precis\\*%
        Hur slipstenen skall dras\\*%
    \end{SongVerse}
\end{SongText}
\begin{SongText}[Plåtniklas sång]
    \begin{SongInfo}
        Text:Thomas Funck
    \end{SongInfo}
    \begin{SongVerse}
        Ja, då ska vi se vad man kan göra\\*%
        Och sen ska vi se vad man kan se\\*%
        Plåtniklas är en riktig snitsare\\*%
    \end{SongVerse}
    \begin{SongVerse}
        Hmmm, nu ska vi se vad man kan hitta\\*%
        Och sen ska vi se vad man kan se\\*%
        Man tager vad man haver\\*%
        Och det sätter man ihop\\*%
        Och vad kan detta bliva som man rappam-pam-pam-paa!\\*%
    \end{SongVerse}
    \begin{SongVerse}
        (Pratstund)\\*%
    \end{SongVerse}
    \begin{SongVerse}
        Först ska vi ha ett förstoringsglas\\*%
        Och då kan vi ta botten på en vas\\*%
        Som gått i kras\\*%
        Ja, bra!\\*%
    \end{SongVerse}
    \begin{SongVerse}
        Och här tar vi en spegel och sätter på sne\\*%
        Och denna aparaten får sitta här bredvid\\*%
        Wow!\\*%
    \end{SongVerse}
    \begin{SongVerse}
        Och nu är det här glaset det enda som är kvar\\*%
        Det sätter vi i röret det blir ett okular\\*%
    \end{SongVerse}
    \begin{SongVerse}
        Och då har vi sett vad man kan göra\\*%
        Så nu ska vi se vad man kan se\\*%
        Plåtniklas gjorde just en kikare!\\*%
    \end{SongVerse}
\end{SongText}
\newpage