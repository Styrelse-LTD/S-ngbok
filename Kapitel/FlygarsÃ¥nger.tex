\fakesection{Flygarsånger}
\fancypagestyle{Flygarsånger}{
    \fancyhead{} % clear all header fields
    \fancyhead[LE,RO]{\textbf{Flygarsånger}}
}
\pagestyle{Flygarsånger}
\begin{SongText}[Gasflygmaskinen]
    \begin{SongInfo}
        Mel: I sommarens soliga dagars
    \end{SongInfo}
    \begin{SongVerse}
        En gasflygmaskin det uppfann jag\\*%
        och upp genom luften försvann jag\\*%
        och flyga i tre minuter han jag\\*%
        sen hände allt det här, och lite till
    \end{SongVerse}
    \begin{SongVerse}
        Först kom ett brak\\*%
        sen kom ett vin\\*%
        sen kom hela flygmaskin\\*%
        sen kom en krans\\*%
        från fabror Franz\\*%
        och sen kom stadens ambulans\\*%
        och sen kom det tre journalister\\*%
        från Stockholms morgonblad,\\*%
        och sist kom jag.
    \end{SongVerse}
\end{SongText}
\begin{SongText}[Flygarsupen]
    \begin{SongInfo}
        Mel: Och flickan hon går i ringen\\*%
        Visan har sitt ursprung i det svenska militärflygets barndom och är förmodligen skriven av bl.a flyglegenden Nils Söderberg
    \end{SongInfo}
    \begin{SongVerse}
        Vi flygare vi tar supar\\*%
        med fart och med kläm.\\*%
        Vi tar dom i våra strupar\\*%
        en tre fyra fem.\\*%
        Håhå jaja, ja jävlar i de'.\\*%
        De går ner i magen i Mach 93!
    \end{SongVerse}
    \begin{SongVerse}
        Vi flygare vi tar supar\\*%
        på vårt lilla sätt.\\*%
        Vi störtar dem ned i djupen\\*%
        uti vårt porträtt.\\*%
        Håhå jaja, vad vilet är kort!\\*%
        Den nubben skall säkert få fler i eskort!
    \end{SongVerse}
\end{SongText}
\begin{SongText}[Flygarhalvan]
    \begin{SongInfo}
        Mel: Och flickan hon går i ringen
    \end{SongInfo}
    \begin{SongVerse}
        Och halvan den bliver trötter\\*%
        att på bordet ståmot himMelen den blå.\\*%
        Den skevar åt barbord\\*%
        och svänger en stund.\\*%
        Gör sedan en looping\\*%
        och landar i mun.
    \end{SongVerse}
\end{SongText}

\begin{SongText}[Sten]
    \begin{SongInfo}
        Mel: Vem kan segla förutan vind\\*%
        Sångledaren skall stå som capten Morgan och efterlikna STEN\\*%
        Visan tillkom under N0llningen 1998 och är ett av de vinnande bidragen på övningsgasquen!
    \end{SongInfo}
    \begin{SongVerse}
        Vem kan flyga för utan plan\\*%
        kan flyga utan roder.\\*%
        Vem kan krascha med Pipern sin\\*%
        utan att fälla tårar
    \end{SongVerse}
    \begin{SongVerse}
        STEN kan flyga för utan plan\\*%
        kan styra utan roder.\\*%
        Men ej krascha med Pipern sin\\*%
        utan att fälla tårar.
    \end{SongVerse}
\end{SongText}

\begin{SongText}[Jas-Hymn]
    \begin{SongInfo}
        Mel: Nu grönskar det\\*%
        Från Stora sångartäfvlan, KTH 1992
    \end{SongInfo}
    \begin{SongVerse}
        För vi har världens bästa plan,\\*%
        med mängder av bomber och speed\\*%
        Kom med, kom med på två machs färd,\\*%
        i våran bistra tid\\*%
        Vart plan är byggt som utav stål,\\*% 
        till bredden fyllt med el(ektronik)\\*%
        Det väldigt mycket G-kraft tål\\*%
        Nej, ingen är JASen lik
    \end{SongVerse}
    \begin{SongVerse}
        Den flyger högt, den flyger lågt\\*%
        Den flyger snabbt och fort\\*%
        Den flyger rätt, den flyger lätt\\*%
        Den flyger helt enkelt som smort\\*%
        Den landar kanske ej så bra\\*%
        Men vad sjuttsingen gör väl det?\\*%
        Den kan ju flyga ifrån en ryss,\\*%
        och svänga i sju-åtta G
    \end{SongVerse}
    \begin{SongVerse}
        Sverige har en gammal tradition\\*%
        Att bygga flygplan av klass\\*%
        Vi \textit{Tunn}vis plan \textit{Lans}erat har\\*%
        Så s\textit{Jas}, \textit{Drak} iväg och förs\textit{Vigg}\\*%
        Vi är så bra, så hejdans bäst\\*%
        Ingen kan våra flygplan slå\\*%
        När vi kommer svepande över er\\*%
        Med full last och EBK (EBK!)
    \end{SongVerse}
\end{SongText}

\begin{SongText}[JASen]
    \begin{SongInfo}
        mel: Måsen\\*%
        Sjunges som svar till JAS-Hymnen
    \end{SongInfo}
    \begin{SongVerse}
        Och Jasen styrde mot västerbron\\*%
        men styrsystemet var trasigt.\\*%
        Piloten ut sköt sig med kanon\\*%
        för planet vingla så knasigt.\\*%
        Han vill uppåt, han ville mer,\\*%
        men planet svarte jag vill ju ner.\\*%
        Mot alla Jon på Västerbron
    \end{SongVerse}
\end{SongText}

\begin{SongText}[Stigfinarvisan]
    \begin{SongInfo}
        Mel: Var nöjd med\\*%
        Visan tillkom under N0llningen 1996 och är ett av de vinnande bidragen på övningsgasquen!
    \end{SongInfo}
    \begin{SongVerse}
        ref.\\*%
        Var nöjd med allt som flyget ger\\*%
        och allting du i luften ser.\\*%
        Glöm bort bekymmer krascher och besvär.\\*%
        Var glad och nöjd för vet du vad\\*%
        en JAS-krash gör ju ingen glad.\\*%
        Var nöjd med flyget som vi läser här.
    \end{SongVerse}
    \begin{SongVerse}
        Varthän jag än flyger,\\*%
        varthän det än bär,\\*%
        så håller jag ölen i handen så här.\\*%
        Jag älskar JAS i formation\\*%
        och Viggen är ju min passion.\\*%
        Och vill du ha bränsle få munnen full\\*%
        så ta en titt under vingen hull.\\*%
        (Talas av Sångledaren): Flyga Lansen? Går det?\\*%
        Var nöjd med allt du ser och allt som flyget ger,\\*%
        (Talas av Sångledaren): Men när?\\*%
        allt flyget ger.
    \end{SongVerse}
    \begin{SongVerse}
        (ref)
    \end{SongVerse}
    \begin{SongVerse}
        Om G-kraft dig lockar\\*%
        en loop eller roll,\\*%
        se till att du plockar dem utan kontroll.\\*%
        Vill du ha fart av bästa klass,\\*%
        så använda höger och vänster klaff.\\*%
        Och landningsställe det ska ut\\*%
        när du ksa ned och ta en sup.\\*%
        (Talas av Sångledaren): Hoppas du har förstått?\\*%
        (Talas): Javisst Sten!\\*%
        Var nöjd med allt du ser\\*%
        och allt som flyget ger\\*%
        allt flyget ger, allt flyget ger!
    \end{SongVerse}
\end{SongText}
\begin{SongText}[Molnen]
    \begin{SongInfo}
        Mel: Nu har vi jul här i vårt hus\\*%
        Visan tillkom under N0llningen 1998 och är ett av de vinnande bidragen på övningsgasquen!
    \end{SongInfo}
    \begin{SongVerse}
        Nu tar vi fart , flyget så klart\\*%
        Vädret är toppen, Hopp tralala!\\*%
        Tornet sa nu, startbana sju, startbana sju.\\*%
    \end{SongVerse}
    \begin{SongVerse}
        $\|\:$Vi flyger så högt vi kan i luften.$\:\|$
    \end{SongVerse}
    \begin{SongVerse}
        Bara högre, ännu högre, ännu högre i molnen!\\*%
        ser ingenting, moln runtomkring.\\*%
        Motorn har stannat, Hoop tralala!\\*%
        Planet i stall, det  här var ju ball, det här var ju ball.
    \end{SongVerse}
    \begin{SongVerse}
        $\|\:$Vi flyger så högt vi kan i luften.$\:\|$
    \end{SongVerse}
    \begin{SongVerse}
        Vi störtar så klart, hoppa  fart.\\*%
        Planet det kraschat, Hopp tralala!\\*%
        Ingen skärm ut, nu är det slut, nu är det slut.
    \end{SongVerse}
\end{SongText}
\begin{SongText}[Vi vill flyga]
    \begin{SongInfo}
        Mel: We will rock you\\*%
        text: Rosén S. Pierrau - T-89\\*%
        Ackompanjeras kontinuerligt med två dunkar i bordet samt en handklapp: dunk-dunk klapp dunk-dunk klapp
    \end{SongInfo}
    \begin{SongVerse}
        $\|\:$ Nu när vi har kommit hit ikväll,\\*%
        ska vi ge er andra en jätte smäll!\\*%
        Här på denna fest,är vi bäst\\*%
        för det är ju flyga vi gillar mest!
    \end{SongVerse}
    \begin{SongVerse}
        Vi vill, vi vill FLYGA!\\*%
        Vi vill, vi vill FLYGA!$\:\|$
    \end{SongVerse}
    \begin{SongInfo}
        Visan upprepas tills man uppnått önskad effekt, dvs. visat vilka som dominerar festen.\\*%
        KTH:s Flygsektion
    \end{SongInfo}
\end{SongText}
%\begin{SongText}[T-teknologen]
%    \begin{SongInfo}
%        Mel: O helga natt
%    \end{SongInfo}
%    \begin{SongVerse}
%        O, T-teknolog, som skapades för världen,\\*%
%        då hönan Agda till borggården steg ned.\\*%
%        Skonar oss från utmattning och skjuvning,\\*%
%        för er han hållfens smärta led.\\*%  
%        Men hoppets blåa färg han visar vägen\\*%
%        och ljuset skimrar över land och hav.\\*%
%        Folk, fall nu neder,\\*%
%        och hälsa glatt hans godhet.\\*%
%        O, T-teknolog, du hjälpande kamrat.\\*%
%        O, T-teknolog, du frälsning ge oss skall
%    \end{SongVerse}
%    \begin{SongVerse}
%        O, T-ingenjör, som frälsar från ondo\\*%
%        var gordisk knut lösr med sitt kunskapsvärd,\\*%
%        bygger skepp för handelsfar runt jorden,\\*%
%        han sätter tågen i rullning i vår värld.\\*%
%        Från hiMelen dyker Drakar, Viggar, Gripar.\\*%
%        Med FEM-metoder tyglas datorns kraft.\\*%
%        Folk, faller nu neder,\\*%
%        och prisa högt hans vishet.\\*%
%        O, T-ingenjör, du älskade geni.\\*%
%        O, T-ingenjör, du frälsning åt oss ger.
%    \end{SongVerse}
%\end{SongText}
%\begin{SongText}[Rymdraketsvalsen]
%    \begin{SongInfo}
%        text: P.RaMel
%    \end{SongInfo}
%    \begin{SongVerse}
%        Bort mot den blåa evighet\\*%
%        genom stratosfären,\\*%
%        svävar med ljusets hastighet\\*%
%        en liten vit raket.
%    \end{SongVerse}
%    \begin{SongVerse}
%        Trygg i dess noskon vid sin ratt,\\*%
%        sitter resenären.\\*%
%        Det är en lite apekatt\\*%
%        som viftar glatt:\\*%
%    \end{SongVerse}
%    \begin{SongVerse}
%        Jorden, hallå! Hallå! Hallå!\\*%
%        Upp i det blå, ja blå, ja blå,\\*%
%        bort mot en fjärran hamn\\*%
%        styr jag nu\\*%
%        uti Vetenskapens namn.
%    \end{SongVerse}
%    \begin{SongVerse}
%        Jorden, farväl! Farväl! Farväl!\\*%
%        Om jag far fel, far fel, far fel,\\*%
%        hälsa från himMelen\\*%
%        till djungeln\\*%
%        och till min lilla hjärtevän!
%    \end{SongVerse}
%\end{SongText}
\begin{SongText}[Lyft ditt välförsedda glas]
    \begin{SongInfo}
        Mel: Ding, dong merrily on high
    \end{SongInfo}
    \begin{SongVerse}
        Lyft ditt välförsedda glas,\\*%
        det är en ljuvlig börda.\\*%
        Nu har flygarna kalas,\\*%
        vi segern snar skall skörda!\\*%
        $\|\:$ Ding, dinge-dinge-ding, dinge-dinge-ding, dinge-dinge-ding, dong-dong.
        I morgon är det lördag.$\:\|$
    \end{SongVerse}
    \begin{SongVerse}
        Lyft nu glaset till din mun\\*%
        se, döden på dig väntar!\\*%
        Nu har flygarna kalas.\\*%
        Hör, liemannen flämtar!\\*%
        $\|\:$ Ding, dinge-dinge-ding, dinge-dinge-ding, dinge-dinge-ding, dong-dong.
        Begravningsklockor klämtar.$\:\|$
    \end{SongVerse}
\end{SongText}
\begin{SongText}[Startbana Sju]
    \begin{SongInfo}
        (Fylleflyarnas Paradsång)\\*%
        Mel: Temperaturen är hög uti kroppen
    \end{SongInfo}
    \begin{SongVerse}
        Nervositeten är hög uti kroppen\\*%
        Vi lyften med planer från startbana sju, eller fem\\*%
        Så ska det vara när planet är oppe\\*%
        Och så är fallet uti detta nu
    \end{SongVerse}
    \begin{SongVerse}
        Vi lyfter..., just nu
    \end{SongVerse}
    \begin{SongVerse}
        Adrenalinet det tigen i knoppen\\*%
        Kicken den lockar mig att prova på, lite bus\\*%
        Vippa med stjärten och stalla med vingen\\*%
        Känner mig lyckligt av vinflaskans rus
    \end{SongVerse}
    \begin{SongVerse}
        Vi rollar..., vi kommer ej ur!
    \end{SongVerse}
    \begin{SongVerse}
        Nu känner jag att jag har tappat kontrollen\\*%
        Störtar mot marken i sjuhundra knyck, eller mer\\*%
        Spriten den står mig högt upp i bollen\\*%
        Svartnar för ögat - jag ser inget mer
    \end{SongVerse}
    \begin{SongVerse}
        Vi störtar...,"KNÄSKÅL"
    \end{SongVerse}
    \begin{SongInfo}
        Visan tillkom under N0llningen 1999 och är det vinnande bidraget på övningsgasquen!
    \end{SongInfo}
\end{SongText}
\begin{SongText}[Här kommer flygaressen]
    \begin{SongInfo}
        med: Här kommer Pippi Långstrump
    \end{SongInfo}
    \begin{SongVerse}
        Vi kan bygga flygplan, lätta snabba flygplan\\*%
        Flyger över Mach 3 och pressat fett med G
    \end{SongVerse}
    \begin{SongVerse}
        För här kommer flygaressen\\*%
        tjolahopp tjolahej tjolahoppsan-sa,\\*%Sha la la Sha la la Sha la la la la la\\*%
        Här kommer flygaressen\\*%
        här kommer faktiskt vi
    \end{SongVerse}
    \begin{SongVerse}
        Har ni sett våra kärra, vår snabba grymma kärra\\*%
        Har ni sett vårt Gripen, ja det heter faktiskt så
    \end{SongVerse}
    \begin{SongVerse}
        För här kommer flygaressen......
    \end{SongVerse}
    \begin{SongVerse}
        Har ni sett dess motor, världens bästa motor\\*%
        Starkast utav alla, ja den e' helt enkelt BÄST
    \end{SongVerse}
    \begin{SongVerse}
        För här kommer flygaressen......
    \end{SongVerse}
    \begin{SongVerse}
        Vi flyger genom luften, ljudlöst genom luften\\*%
        Tittar ner på alla och skrattar åt er så
    \end{SongVerse}
    \begin{SongVerse}
        För här kommer flygaressen......
    \end{SongVerse}
    \begin{SongVerse}
        Efter vi har landat, kommit ner å landat\\*%
        Ska vi gå och Gasqua, för de gör vi faktiskt bäst
    \end{SongVerse}
    \begin{SongVerse}
        För här kommer flygaressen......
    \end{SongVerse}
    \begin{SongVerse}
        Nu så ska vi Festa, dricka av det bästa\\*%
        Vi kan hålla låda, låda hela natten lång
    \end{SongVerse}
    \begin{SongVerse}
        För här kommer flygaressen......
    \end{SongVerse}
    \begin{SongInfo}
        Visan tillkom undr N0llningen 2000 och är det vinnande bidragen på övningsgasquen!
    \end{SongInfo}
\end{SongText}
\begin{SongText}[En solig sommardag]
    \begin{SongInfo}
        Mel: En kulen natt\\*%
        text: Rasmus Ahl, Flyg-01
    \end{SongInfo}
    \begin{SongVerse}
        En solig sommardag\\*%
        Mitt plan jag spaka.\\*%
        Å efter en eller 74 roll\\*%
        Det börja knaka.\\*%
        Jag tappade delar v olika slag\\*%
        Då tänkte jag att det bästa nog var\\*%
        Att ta en sup för att sedan hoppa ut\\*%
        Nu é livet slut.
    \end{SongVerse}
    \begin{SongVerse}
        Så upp till gud jag far\\*%
        Å på porten plingar.\\*%
        När Perra slutligen släpper mig in\\*%
        Så får jag vingar.\\*%
        Då ska jag t mig ett stadigt järn\\*%
        å mot ett moln jag ett ordentligt spjärn.\\*%
        Å flyga tills jag inte orkar mer.\\*%
        Då far jag ner.
    \end{SongVerse}
    \begin{SongVerse}
        Här nere i helvetets varma vrå\\*%
        Finns det inga kvinner.\\*%
        Å flyga går inte alls nå'n bra\\*%
        Då mina vingar brinner.\\*%
        Inte ens en lättöl går det att få,\\*%
        Eller lite T-Röd att festa på.\\*%
        Så nu när hopper runnit ut,\\*%
        Så är sången slut.
    \end{SongVerse}
\end{SongText}
\begin{SongText}[Bamsevisan]
    \begin{SongInfo}
        Mel: Känd
    \end{SongInfo}
    \begin{SongVerse}
        Bamse, Bamse starkast utav alla\\*%
        Och är bra att ha när man vill slåss.\\*%
        Fienden han skiter i sin bralla\\*%
        Och har inga Pampers på förstås.
    \end{SongVerse}
    \begin{SongVerse}
        Och kommer några flygplan hit\\*%
        Så går dom sönder bit för bit,\\*%
        Och G-dräkten blit full av skit\\*%
        Av Bamses dynamit.
    \end{SongVerse}
    \begin{SongVerse}
        Nos och fena målades så balla,\\*%
        Bamses smällar enkelt plockar loss,\\*%
        Flygplanen sen ner från himlen falla\\*%
        Genom luften likt ett tomebloss, HURRA!
    \end{SongVerse}
    \begin{SongVerse}
        Skjuta, döda, dräpa, slakta\\*%
        Mörda, hacka, flå och slita\\*%
        Sönder i små stycken syno-\\*%
        nymerna tog slut
    \end{SongVerse}
    \begin{SongVerse}
        Bamse, Bamse smäller högst av alla\\*%
        Luftvärnet dom vill ha fler förstås!
    \end{SongVerse}
\end{SongText}
\begin{SongText}[MACH-visan]
    \begin{SongInfo}
        Mel: Cottonfields\\*%
        Text: Ida Tångring
    \end{SongInfo}
    \begin{SongVerse}
        Det ska vara pojkar som kan flyga,\\*%
        Som kan bjuda på en tur\\*%
        Upp i himlen, högt upp i det blå\\*%
        Stanna kvar uti hangaren\\*%
        Du ska se till att han kan ta dig\\*%
        Utan vingar lika högt ändå!
    \end{SongVerse}
    \begin{SongVerse}
        För när det en gång blivit natt\\*%
        Ska det va grabb från MACH\\*%
        En med G-kraft utan bara F-N\\*%
        Du har tur om du lyckas få en\\*%
        För där under overallen\\*%
        Gömmer han ett riktigt flygaress
    \end{SongVerse}
    \begin{SongInfo}
        Det vinnande bidraget under N0llningen 2002
    \end{SongInfo}
\end{SongText}
\begin{SongText}[Flygarvisan]
    \begin{SongInfo}
        Mel: raskar över isen
    \end{SongInfo}
    \begin{SongVerse}
        $\|\:$Viggen flyger ifrån miggen$\:\|$\\*%
        Så vår vi lov, så får vi lov\\*%
        Att sjunga flygarens visa
    \end{SongVerse}
    \begin{SongVerse}
        Så här gör flygaren var han står.\\*%
        och var han sitter och var han går\\*%
        Så får vi lov, så vår vi lov\\*%
        Att sjunga flygarens visa.
    \end{SongVerse}
    \begin{SongVerse}
        $\|\:$JASen skjuter sönder miggen$\:\|$\\*%
        så får vi log, så får vi log\\*%
        Att sjunga flygarens visa
    \end{SongVerse}
    \begin{SongVerse}
        Så här gör...
    \end{SongVerse}
    \begin{SongInfo}
        Tillkom under N0llningen 2002
    \end{SongInfo}
\end{SongText}
\begin{SongText}[BumbiN0llan]
    \begin{SongInfo}
        Mel: bumbibjörnarna
    \end{SongInfo}
    \begin{SongVerse}
        Usla och kassa, gasquar en massa\\*%
        Våra propellrar dom snurrar som fan\\*%
        Parta och festa, det klarar dom flesta\\*%
        Men tidigt på morgonen super vi än\\*%
        Hipp hurra, för här kommer flygarn0llan\\*%
        Susar frammåt under stjärnorna\\*%
        Manfred får följa med\\*%
        Manfrida från följa med!
    \end{SongVerse}
    \begin{SongInfo}
        Det vinnande bidraget under n0llnongen 2003
    \end{SongInfo}
\end{SongText}
\begin{SongText}[FLygturen]
    \begin{SongInfo}
        Mel: Rövarvisan från kamomilla stad\\*%
        text: Mari Gyström och Jessic Johansson
    \end{SongInfo}
    \begin{SongVerse}
        Nu far vi ut på flygarfest,\\*%
        Ja vi ska ut å flyga.\\*%
        Vi kapar första bästa plan,\\*%
        Som nån av oss kan flyga.\\*%
        Snart är det dags för snabb take-off,\\*%
        Glöm ej spänn fast din ö'öl-back.\\*%
        Nu drar vi ivåg med vårt charterplan.\\*%
        Både Manfred, Mandrida å övrigt pack.
    \end{SongVerse}
    \begin{SongVerse}
        Vi taxar ut från gatan snart,\\*%
        Å det är dags att starta.\\*%
        Det lyser längs vårt landningsstråk,\\*%
        Å e bara till å gasa.\\*%
        Vi lättar, lyften, flyger nu,\\*%
        Ingen chans att någon blir lack.\\*%
        Det pyser, det skvätter vi öppnar en öl,\\*%
        Både Manfred, Manfrida å övrigt pack.
    \end{SongVerse}
    \begin{SongVerse}
        Vi åger med en väldans fart,\\*%
        Flyger med en MIG 22,\\*%
        Här hinkas det en massa öl,\\*%
        Så vi får svårt att stå.\\*%
        Rollar, tippar, gir gör vi,\\*%
        Vi flyger snart i 15 MACH\\*%
        Men annars så gör vi så lite vi kan,\\*%
        Både Manfred, Manfrida och övrigt pack.
    \end{SongVerse}
    \begin{SongVerse}
        Nu när vi druckit upp vår öl,\\*%
        Så är det dags att landa.\\*%
        Vi sätter planet snyggt å fint,\\*%
        Så ingen behöver sanda.\\*%
        Vi klarar oss från krasch idag,\\*%
        För alla tillhör ju MACH.\\*%
        Så höj era glas å skåla med oss,\\*%
        Båda MAnfred, MAnfrida å övrigt pack.
    \end{SongVerse}
\end{SongText}
\begin{SongText}[Tunnans hymn]
    \begin{SongInfo}
        Mel: Jagg vill vara din, Margareta
    \end{SongInfo}
    \begin{SongVerse}
        Jag vill ha ett plan, ska du veta\\*%
        Av sällan skådat slag, den tjocka feta\\*%
        Stor och grå som få\\*%
        Vad vore det väl då\\*%
        Jo, en tunna i luften.
    \end{SongVerse}
    \begin{SongVerse}
        Tunnan flyger fram som en svala\\*%
        När alla andra plan mot marken dala\\*%
        Det kan aldrig ske\\*%
        Att tunnan faller ner\\*%
        Den bli kvar i luften.
    \end{SongVerse}
    \begin{SongVerse}
        Jag vill ha ett plan, ska du veta\\*%
        Av sällan skådat slag, den tjocka feta\\*%
        Stor å grå som få\\*%
        Vad vore det väl då\\*%
        Jo, en tunna i luften.
    \end{SongVerse}
    \begin{SongInfo}
        Visan tillkom under N0llningen 2004 och är det vinnande bidragen på övningsgasquen!
    \end{SongInfo}
\end{SongText}
%\begin{SongText}[MACHs sommarvisa]
%    \begin{SongInfo}
%        Mel: Idas sommarvisa
%    \end{SongInfo}
%    \begin{SongVerse}
%        Det finns en grupp på jorden\\*%
%        Som aldrig nånsin ler\\*%
%        De kallas sig för straffspark\\*%
%        och ner på N0llan ser.
%    \end{SongVerse}
%    \begin{SongVerse}
%        De kallas sig för Manfred Air Co Holics\\*%
%        Och de ska lära n0llan hur man dricker öl\\*%
%    \end{SongVerse}
%    \begin{SongVerse}
%        Å lilla n0llan sjunger\\*%
%        Men det låter bara skit\\*%
%        Så därför vill vi sjunga\\*%
%        Att Manfred är elite.
%    \end{SongVerse}
%    \begin{SongInfo}
%        Visan tillkom under n0llningen 2005 och är det vinnande bidragen på övningsgasquen!
%    \end{SongInfo}
%\end{SongText}
\begin{SongText}[Ser du n0llan]
    \begin{SongInfo}
        Mel: Ser du stjärnan
    \end{SongInfo}
    \begin{SongVerse}
        Ser du JASen i det blå\\*%
        Du vill flygplan bygga då\\*%
        ' Söker in på bästa skolan\\*%
        M...D...H...
    \end{SongVerse}
    \begin{SongVerse}
        Straffspark ser var du än gå'\\*%
        Manfred han bär färgen grå\\*%
        Om du ser en n0llan grå\\*%
        Straff...då...få...
    \end{SongVerse}
    \begin{SongVerse}
        M.A.C.H. på kåren festar glatt\\*%
        Super nästan varje natt\\*%
        Alla sinnen strejkar och du\\*%
        Fal...ler...platt!
    \end{SongVerse}
\end{SongText}
\begin{SongText}[You've lost that loving feeling]
    \begin{SongInfo}
        text och Mel: The Righteous Brothers
    \end{SongInfo}
    \begin{SongVerse}
        You never close your eyes anymore when I kiss your lips\\*%
        And there's no tenderness like before in your fingertips\\*%
        You're trying hard not to show it\\*%
        But baby, baby I know it
    \end{SongVerse}
    \begin{SongVerse}
        You lost that lovin' feelin'\\*%
        Whoa, that lovin' feelin'\\*%
        You lost that lovin' feelin'\\*%
        Now it's gone, gone, gone, whoa-oh
    \end{SongVerse}
    \begin{SongVerse}
        Now there's no welcome look in your eyes when I reach for you\\*%
        And now you're starting to criticize little things I do\\*%
        It makes me just feel like crying\\*%
        'Cause baby, something beautiful's dyin'
    \end{SongVerse}
    \begin{SongVerse}
        You lost that lovin' feelin'\\*%
        Whoa, that lovin' feelin'\\*%
        You lost that lovin' feelin'\\*%
        Now it's gone, gone, gone, whoa-oh
    \end{SongVerse}
    \begin{SongVerse}
        Baby, baby, I'd get down on my knees for you\\*%
        If you would only love me like you used to do, yeah\\*%
        We had a love, a love, a love you don't find everyday\\*%
        So don't, don't, don't, don't let it slip away
    \end{SongVerse}
    \begin{SongVerse}
        Baby, baby, baby, baby\\*%
        I beg you please, please, please, please\\*%
        I need your love, need your love\\*%
        I need your love, I need your love\\*%
        So bring it on back, so bring it on back\\*%
        Bring it on back, bring it on back
    \end{SongVerse}
    \begin{SongVerse}
        Bring back that lovin' feelin'\\*%
        Whoa, that lovin' feelin'\\*%
        Bring back that lovin' feelin'\\*%
        'Cause it's gone, gone, gone\\*%
        And I can't go on, whoa-oh
    \end{SongVerse}
    \begin{SongVerse}
        Bring back that lovin' feelin'\\*%
        Whoa, that lovin' feelin'\\*%
        Bring back that lovin' feelin'\\*%
        'Cause it's gone, gone, gone
    \end{SongVerse}
\end{SongText}
%\begin{SongText}[Våran draken]
%    \begin{SongVerse}
%        Våran draken\\*%
%        Den visar baken\\*%
%        För den ryska Yaken\\*%
%        När Draken flyger Alla andra smyger\\*%
%        och även när Draken skjuter\\*%
%        Alla andra tjuter\\*%
%        För Draken är ju bäst!
%    \end{SongVerse}
%    \begin{SongVerse}
%        När Draken flyger\\*%
%        Även straffspark de lyder för\\*%
%        Draken är ju bäst\\*%
%        Nu är det fest!
%    \end{SongVerse}
%    \begin{SongVerse}
%        När Yaken oss vill kränka\\*%
%        Den en missil oss skänka\\*%
%        Våran Draken den visar baken\\*%
%        För den ryska Yaken
%    \end{SongVerse}
%    \begin{SongVerse}
%        Där Draken flyger\\*%
%        Även straffspark de lyder\\*%
%        För Draken den är ju bäst!
%    \end{SongVerse}
%    \begin{SongVerse}
%        Nu är det fest!
%    \end{SongVerse}
%    \begin{SongInfo}
%        Visan tillkom under n0llningen 2008 och är det vinnande bidraget till övningsgasquen!
%    \end{SongInfo}
%\end{SongText}
\begin{SongText}[Sångsparks Hyllningen]
    \begin{SongInfo}
        Mel: Svampbob Fyrkant
        Text: Willy "Bonka" Öst
    \end{SongInfo}
    \begin{SongVerse}
        Är ni med, n0llan? (Sångledaren)\\*%
        Aj aj, Manfred!\\*%
        Jag hör er inte! (Sångledaren)\\*%
        Aj aj Manfred!
    \end{SongVerse}
    \begin{SongVerse}
        \textbf{Ååååååååååååååååååååååååååååå...} (Sångledaren)
    \end{SongVerse}
    \begin{SongVerse}
        Vem flygen JAS högt uypp i det blå? (Sångledaren)\\*%
        Manfred Sångspark!\\*%
        Ful och hejdlös, han super som få! (Sångledaren)\\*%
        Manfred Sångspark!\\*%
        Jag varnar dig n0llan, visst finns det en risk! (Sångledaren)\\*%
        Manfred Sångspark!\\*%
        Han får dig att dricka precis som en fisk! (Sångledaren)\\*%
        Manfred Sångspark!
    \end{SongVerse}
    \begin{SongVerse}
        \textbf{Redo?!} (Sångledaren)
    \end{SongVerse}
    \begin{SongVerse}
        Manfred Sångspark!\\*%
        Manfred Sångspark!\\*%
        Manfred Sångspark!\\*%
        Manfred Sångspark!\\*%
    \end{SongVerse}
\end{SongText}
\begin{SongText}[Eloise]
    \begin{SongInfo}
        Text och Mel: Arvingarna
    \end{SongInfo}
    \begin{SongVerse}
        Samlar mina tankar i ensamhet innan jag går ut\\*%
        Trodde jag var smart när jag gjorde slut\\*%
        Ah, ah, ah!!\\*%
        Längtar efter dig och nu står jag här utanför din dörr\\*%
        När jag ringer på öppnar du då?\\*%
        eller vill du att jag ska gå?\\*%
        (Så jag ber dig: Kom ut till mig)
    \end{SongVerse}
    \begin{SongVerse}
        Är vi mer än bara vänner\\*%
        Så visa vad du känner\\*%
        Och sen får känslorna bestämma\\*%
        Eloise!\\*%
        Även vindarna kan vända\\*%
        FÖr mig är du den enda\\*%
        Och kärleken den är värd ett högre pris\\*%
        Eloise!
    \end{SongVerse}
\end{SongText}
\newpage