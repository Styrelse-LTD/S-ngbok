\fakesection{Flygarsånger}

\fancypagestyle{Flygarsånger}{
\fancyhead{} % clear all header fields
\fancyhead[LE,RO]{\textbf{Flygarsånger}}
}
\pagestyle{Flygarsånger}


\begin{SongText}[Ode till blygbensinet]
    \begin{SongInfo}
        Mel: Helan går\\*%
        Denna visa räddades ur papperskorgen av sedemera örlogskapten KG Lewenhaupt i nådens år 1964.
    \end{SongInfo}
    \begin{SongVerse}
        Flygbensin är drycken för vart fyllesvin\\*%
        Flygbensin är rått som terpentin\\*%
        Det piggar upp sin trötta kropp\\*%
        och hetta upp ditt blodomlopp\\*%
        Flygbensin ...\\*%
        är Manfreds medicin
    \end{SongVerse}
\end{SongText}
\begin{SongText}[Gasflygmaskinen]
    \begin{SongInfo}
        mel: I sommarens soliga dagars
    \end{SongInfo}
    \begin{SongVerse}
        En gasflygmaskin det uppfann jag\\*%
        och upp genom luften försvann jag\\*%
        och flyga i tre minuter han jag\\*%
        sen hände allt det här, och lite till
    \end{SongVerse}
    \begin{SongVerse}
        Först kom ett brak\\*%
        sen kom ett vin\\*%
        sen kom hela flygmaskin\\*%
        sen kom en krans\\*%
        från fabror Franz\\*%
        och sen kom stadens ambulans\\*%
        och sen kom det tre journalister\\*%
        från Stockholms morgonblad,\\*%
        och sist kom jag.
    \end{SongVerse}
\end{SongText}
\begin{SongText}[Flygarsupen]
    \begin{SongInfo}
        mel: Och flickan hon går i ringen\\*%
        Visan har sitt ursprung i det svenska militärflygets barndom och är förmodligen skriven av bl.a flyglegenden Nils Söderberg
    \end{SongInfo}
    \begin{SongVerse}
        Vi flygare vi tar supar\\*%
        med fart och med kläm.\\*%
        Vi tar dom i våra strupar\\*%
        en tre fyra fem.\\*%
        Håhå jaja, ja jävlar i de'.\\*%
        De går ner i magen i Mach 93!
    \end{SongVerse}
    \begin{SongVerse}
        Vi flygare vi tar supar\\*%
        på vårt lilla sätt.\\*%
        Vi störtar dem ned i djupen\\*%
        uti vårt porträtt.\\*%
        Håhå jaja, vad vilet är kort!\\*%
        Den nubben skall säkert få fler i eskort!
    \end{SongVerse}
\end{SongText}
\begin{SongText}[Flygarhalvan]
    \begin{SongInfo}
        mel: Och flickan hon går i ringen
    \end{SongInfo}
    \begin{SongVerse}
        Och halvan den bliver trötter\\*%
        att på bordet ståmot himmelen den blå.\\*%
        Den skevar åt barbord\\*%
        och svänger en stund.\\*%
        Gör sedan en looping\\*%
        och landar i mun.
    \end{SongVerse}
\end{SongText}

\begin{SongText}[Sten]
    \begin{SongInfo}
        mel: Vem kan segla förutan vind\\*%
        Sångledaren skall stå som capten Morgan och efterlikna STEN\\*%
        Visan tillkom under N0llningen 1998 och är ett av de vinnande bidragen på övningsgasquen!
    \end{SongInfo}
    \begin{SongVerse}
        Vem kan flyga för utan plan\\*%
        kan flyga utan roder.\\*%
        Vem kan krascha med Pipern sin\\*%
        utan att fälla tårar
    \end{SongVerse}
    \begin{SongVerse}
        STEN kan flyga för utan plan\\*%
        kan styra utan roder.\\*%
        Men ej krascha med Pipern sin\\*%
        utan att fälla tårar.
    \end{SongVerse}
\end{SongText}

\begin{SongText}[Jas-Hymn]
    \begin{SongInfo}
        Mel: Nu grönskar det\\*%
        Från Stora sångartäfvlan, KTH 1992
    \end{SongInfo}
    \begin{SongVerse}
        För vi har världens bästa plan,\\*%
        med mängder av bomber och speed\\*%
        Kom med, kom med på två machs färd,\\*%
        i våran bistra tid\\*%
        Vart plan är byggt som utav stål,\\*% 
        till bredden fyllt med elektronik\\*%
        Det väldigt mycket G-kraft tål\\*%
        Nej, ingen är JASen lik
    \end{SongVerse}
    \begin{SongVerse}
        Den flyger högt, den flyger lågt\\*%
        Den flyger snabbt och fort\\*%
        Den flyger rätt, den flyger lätt\\*%
        Den flyger helt enkelt som smort\\*%
        Den landar kanske ej så bra\\*%
        Men vad sjuttsingen gör väl det?\\*%
        Den kan ju flyga ifrån en ryss,\\*%
        och svänga i sju-åtta G
    \end{SongVerse}
    \begin{SongVerse}
        Sverige har en gammal tradition\\*%
        Att bygga flygplan av klass\\*%
        Vi tonvis plan lanserat har\\*%
        Så schas, dra iväg och försvinn\\*%
        Vi är så bra, så hejdans bäst\\*%
        Ingen kan våra flygplan slå\\*%
        När vi kommer svepande över er\\*%
        Med full last och EBK
    \end{SongVerse}
\end{SongText}
\begin{SongText}[Stigfinarvisan]
    \begin{SongInfo}
        mel: Var nöjd med\\*%
        Visan tillkom under N0llningen 1996 och är ett av de vinnande bidragen på övningsgasquen!
    \end{SongInfo}
    \begin{SongVerse}
        ref.\\*%
        Var nöjd med allt som flyget ger\\*%
        och allting du i luften ser.\\*%
        Glöm bort bekymmer krascher och besvär.\\*%
        Var glad och nöjd för vet du vad\\*%
        en JAS-krash gör ju ingen glad.\\*%
        Var nöjd med flyget som vi läser här.
    \end{SongVerse}
    \begin{SongVerse}
        Varthän jag än flyger,\\*%
        varthän det än bär,\\*%
        så håller jag ölen i handen så här.\\*%
        Jag älskar JAS i formation\\*%
        och Viggen är ju min passion.\\*%
        Och vill du ha bränsle få munnen full\\*%
        så ta en titt under vingen hull.\\*%
        Flyga Lansen? Går det? (Sångledaren)\\*%
        Var nöjd med allt du ser och allt som flyget ger,\\*%
        men när. (Sångledaren)\\*%
        allt flyget ger.
    \end{SongVerse}
    \begin{SongVerse}
        (ref)
    \end{SongVerse}
    \begin{SongVerse}
        Om G-kraft dig lockar\\*%
        en loop eller roll,\\*%
        se till att du plockar dem utan kontroll.\\*%
        Vill du ha fart av bästa klass,\\*%
        så använda höger och vänster klaff.\\*%
        Och landningsställe det ska ut\\*%
        när du ksa ned och ta en sup.\\*%
        Hoppas du har förstått? (Sångledaren)\\*%
        Javisst Sten!\\*%
        Var nöjd med allt du ser\\*%
        och allt som flyget ger\\*%
        allt flyget ger, allt flyget ger!
    \end{SongVerse}
\end{SongText}
\begin{SongText}[Molnen]
    \begin{SongInfo}
        mel: Julpolska\\*%
        Visan tillkom under N0llningen 1998 och är ett av de vinnande bidragen på övningsgasquen!
    \end{SongInfo}
    \begin{SongVerse}
        Nu tar vi fart , flyget så klart\\*%
        Vädret är toppen, Hopp tralala!\\*%
        Tornet sa nu, startbana sju, startbana sju.\\*%
    \end{SongVerse}
    \begin{SongVerse}
        $\|\:$Vi flyger så högt vi kan i luften.$\:\|$
    \end{SongVerse}
    \begin{SongVerse}
        Bara högre, ännu högre, ännu högre i molnen!\\*%
        ser ingenting, moln runtomkring.\\*%
        Motorn har stannat, Hoop tralala!\\*%
        Planet i stall, det  här var ju ball, det här var ju ball.
    \end{SongVerse}
    \begin{SongVerse}
        $\|\:$Vi flyger så högt vi kan i luften.$\:\|$
    \end{SongVerse}
    \begin{SongVerse}
        Vi störtar så klart, hoppa  fart.\\*%
        Planet det kraschat, Hopp tralala!\\*%
        Ingen skärm ut, nu är det slut, nu är det slut.
    \end{SongVerse}
\end{SongText}
\begin{SongText}[Vi vill flyga]
    \begin{SongInfo}
        mel: We will rock you\\*%
        text: Rosén S. Pierrau - T-89\\*%
        Ackompanjeras kontinuerligt med två dunkar i bordet samt en handklapp: dunk-dunk klapp dunk-dunk klapp
    \end{SongInfo}
    \begin{SongVerse}
        $\|\:$ Nu när vi har kommit hit ikväll,\\*%
        ska vi ge er andra en jätte smäll!\\*%
        Här på denna fest,är vi bäst\\*%
        för det är ju flyga vi gillar mest! 
    \end{SongVerse}
    \begin{SongVerse}
        Vi vill, vi vill FLYGA!\\*%
        Vi vill, vi vill FLYGA!$\:\|$
    \end{SongVerse}
    \begin{SongInfo}
        Visan upprepas tills man uppnått önskad effekt, dvs. visat vilka som dominerar festen.\\*%
        KTH:s Flygsektion
    \end{SongInfo}
\end{SongText}
\begin{SongText}[]
    \begin{SongInfo}
        mel: O helga natt
    \end{SongInfo}
    \begin{SongVerse}
        
    \end{SongVerse}
\end{SongText}


\newpage