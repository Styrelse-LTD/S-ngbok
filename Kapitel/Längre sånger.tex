\fakesection{Längre sånger}
\fancypagestyle{Längre sånger}{
\fancyhead{} % clear all header fields
\fancyhead[LE,RO]{\textbf{Längre sånger}}
}
\pagestyle{Längre sånger}
\begin{SongText}[The Scotsman's Kilt]
\begin{SongVerse}
Well, a Scotsman clad in kilt left a bar one evening fair,\\*%
and one could tell by how he walked that he'd drunk more than his shared.\\*%
He fumbled 'round until he could no longer keep his feet,\\*%
and he stumbled off into the grass to sleep beside the street.
\end{SongVerse}
\begin{SongVerse}
Ring-ding diddle-iddle-i-dee-o, Ring di diddle-i-o
and he stumbled off into the grass to sleep beside the street.
\end{SongVerse}
\begin{SongVerse}
About that time two young and lovely girls just happened by,\\*%
One says to the other with a twinkle in her eye:\\*%
"See a sleeping Scotsman, so strong and handsome built,\\*%
I wonder if it's true what they don't wear beneath the kilt!"
\end{SongVerse}
\begin{SongVerse}
Ring...\\*%
I wonder if it's true what they don't wear beneath the kilt!"
\end{SongVerse}
\begin{SongVerse}
They crept up on that sleeping Scotsman quiet as could be;\\*%
lifted up his kilt about an inch so they could see.\\*%
And there, behold, for them to view beneath his Scottish skirt,\\*%
was nothin' more than God had graced him with upon his birth. \\*%
Ring...\\*%
was nothing' more than God had graced him with upon his birth.
\end{SongVerse}
\begin{SongVerse}
They marvelled for a moment, then one said: "We must be gone.\\*%
Let's leave a present for our friend before we move along."\\*%
As a gift they left a blue silk ribbon tied into a bow,\\*%
around the bonnie star the Scotsman's kilt did lift and show.
\end{SongVerse}
\begin{SongVerse}
Ring...\\*%
around the bonnie star the Scotsman's kilt did lift and show. 
\end{SongVerse}
\begin{SongVerse}
Now the Scotsman woke to nature's call, and stumbled towards the trees.\\*%
Behind the bush he lifts his kilt, and gawks at what he sees.\\*%
And in a startled voice he says, to what's before his eyes,\\*%
"Oh, lad I don't know where ye been, but I see ye won first prize!"
\end{SongVerse}
\begin{SongVerse}
Ring...\\*%
"Oh, lad I don't know where ye been, but I see ye won first prize!"
\end{SongVerse}
\begin{SongVerse}
Our Scottish friend still dressed in kilt continued down the street\\*%
He hadn't gone ten yards or more when a girl he chanced to meet\\*%
She said I heard what's under there. Tell me is it so?\\*%
He said just slip your hand up miss if you'd really like to know
\end{SongVerse}
\begin{SongVerse}
Ring-ding diddle-iddle-i-dee-o, Ring di diddle-i-o\\*%
He said just slip your hand up miss if you would like to know
\end{SongVerse}
\begin{SongVerse}
She put her hand right up his kilt and much to her surprise\\*%
The Scotsman smiled and a very strange look came into his eyes\\*%
She cried why sir that’s gruesome and then she heard him roar\\*%
If you put your hand up once again, you'll find it's grew some more.
\end{SongVerse}
\begin{SongVerse}
Ring-ding diddle-iddle-i-dee-o, Ring di diddle-i-o\\*%
If you put your hand up once again, you'll find it's grew some more.
\end{SongVerse}
\end{SongText}
\begin{SongText}[Chandelier's shop]
\begin{SongVerse}
As I went into the Chandelier Shop, some candles for to buy\\*%
I went inside the Chandelier Shop, but no one could I spy\\*%
Now I was disappointed so some angry words I said\\*%
When I heard the sound of \emph{(clap, clap, clap)} right above me head\\*%
When I heard the sound of \emph{(clap, clap, clap)} right above me head
\end{SongVerse}
\begin{SongVerse}
Now I was quick and I was slick, so up the stairs I fled\\*%
And very surprised was I to find the Chandelier's wife in bed\\*%
And with her was another man of quiete considerable size\\*%
And they were having a \emph{(clap, clap, clap)} right before me eyes\\*%
And they were having a \emph{(clap, clap, clap)} right before me eyes
\end{SongVerse}
\begin{SongVerse}
Now when the fun was over and done, the lady raised her head\\*%
And very surprised was she to find me standing by her bed\\*%
"If you will be discreet me boy, if you will be so kind\\*%
Why don't you come up for some \emph{(clap, clap, clap)} whenever you feel inclined\\*%
Why don't you come up for some \emph{(clap, clap, clap)} whenever you feel inclined 
\end{SongVerse}
\begin{SongVerse}
So many the night and many the day when the Chandelier wasn't home\\*%
To get myself some candles to the Chandelier Shop I'd roam\\*%
But never the ones she gives to him, she'd give me instead\\*%
A little bit more of the \emph{(clap, clap, clap)} to light me way to bed\\*%
Just a little bit more of the \emph{(clap, clap, clap)} to light me way to bed 
\end{SongVerse}
\begin{SongVerse}
So come all you married men, take heed whenever you go to town\\*%
And if you leave your wife alone be sure and tie her down\\*%
And if you'll be so kind to her just lie her down on the floor\\*%
And give her so much of the \emph{(clap, clap, clap)} she doesn't want anymore\\*%
And give her so much of the \emph{(clap, clap, clap)} she doesn't want anymore 
\end{SongVerse}
\end{SongText}
\begin{SongText}[Tullarmästarn]
\begin{SongInfo}
    Text: Galenskaparna och after shave
\end{SongInfo}
\begin{SongVerse}
Tullarmästarn står vid Stenas terminal, kolla och kolla\\*%
Resenärer väller in i tusental, kolla och kolla\\*%
Skriker order till sin tullarpersonal, koll-la-la-la-la-la,\\*%
öppna varje bag och alla skidfodral\\*%
visa vad ni har i hatt och strumpor\\*%
Har ni tagit med er illegal ranson\\*%
det är bäst att göra kroppsvisitation\\*%
Av med kläderna och tar er inte ton\\*%
Stopp stopp stopp stopp stopp\\*%
Säg har ni snus och vin i påsen?\\*%
Kolla mannen där och kolla väskan där\\*%
det är visst nåt som kluckar\\*%
Kön tar aldrig slut, nej ingen slipper ut\\*%
och salen fylls av suckar\\*%
Öppna väskan pojkar, öppna väskan pojkar,\\*%
öppna väskan pojkar\\*%
(Ska vi se..)\\*%
Så ska dansen gå!
\end{SongVerse}
\begin{SongVerse}
Öl och brännvin, gin och tobak nu, vi ska ta\\*%
öl och brännvin gin och tobak nu, vi ska ta\\*%
öl och brännvin gin och tobak nu, vi ska ta\\*%
se på hunden han verkar rabiessmittad.\\*%
Ögonen med röntgenblick som ser allt du har,\\*%
ögonen med röntgenblick som ser allt du har,\\*%
roa sig och sätta fast nån\\*%
kinderna rodnar liksom röda rosor på äng\\*%
varje gång en väska måste öppnas.
\end{SongVerse}
\begin{SongVerse}
Tullarmästarn står vid Stenas terminal, kolla och kolla\\*%
Alla kollas utan minsta marginal kolla och kolla\\*%
Kommer man till Sverige har man inget val, koll-la-la-la-la-la\\*%
Känn igenom både mun och tarmkanal,\\*%
kika gärna in i skor och stövlar\\*%
Kolla turken där vad har han i sin féz?\\*%
Tullarmästarn skriker han blir alldeles hes\\*%
Här är stället där man inga chanser ges\\*%
Stopp stopp stopp stopp stopp\\*%
Säg har ni snus och vin i påsen?\\*%
Muddra en arab och skruva ned en Saab,\\*%
och kläm på varje finne.\\*%
Se, där går en man, han kom från Köpenhamn,\\*%
vi tar honom här inne\\*%
Öppna väskan pojkar, öppna väskan pojkar,\\*%
öppna väskan pojkar, öppna väskan pojkar\\*%
(Nå så, släpp min taske!)\\*%
Sen får dansken gå!
\end{SongVerse}
\end{SongText}
\begin{SongText}[The rattlin bog ]
\begin{SongInfo}
    Text: Irish Descendants
\end{SongInfo}
\begin{SongVerse}
(EDITOR NOTE: Om möjligt effektivisera texten)\\*%
%\emph{Sjunges så att man lägger till en rad nerifrån och upp av
%nedanstående del innan varje refräng. ex: vid 4... så börjar man
%på rad 4: och sjunger neråt Se ”1:” i låten.. (Låten börjar med
%den första refrängen)}
\emph{Sjunges så att man fortsätter versen en rad i listan över vart 
du sjöng versen innan. ex: vid 4... sjungs rad 4,3,2,1,0,ref. 
(låten börjar med den den första refrängen)}
\begin{itemize}
    \item E: rash on tick
    \item D: tick on the louse
    \item C: louse on the hair
    \item B: hair on the worm
    \item A: worm on the feather
    \item 9: feather on the bird
    \item 8: bird in the egg
    \item 7: egg in the bird
    \item 6: bird in the nest
    \item 5: nest on the limb
    \item 4: limb on the branch-
    \item 3: branch on the tree,
    \item 2: tree in the hole-
    \item 1: hole in the bog -
    \item 0: in the bog down in the valley-o
    \item Ref…
\end{itemize}
Ref:
ho-ho the rattlin bog, the bog down in the valley-o
rare bog the rattling bog, the bog down in the valley-o
in the bog there was a hole, a rare hole, a rattlin hole.
\end{SongVerse}
\begin{SongVerse}
1:
hole in the bog in the bog down in the valey-o\\*%
Ref... 
\end{SongVerse}
\begin{SongVerse}
in the hole the was a tree, a rare tree, a rattlin tree.\\*%
2...
\end{SongVerse}
\begin{SongVerse}
on the tree there was a branch a rare branch a rattling branch.\\*%
3...
\end{SongVerse}
\begin{SongVerse}
on the bracnh there was a limb a rare limb, a rattlin limb.\\*%
4...
\end{SongVerse}
\begin{SongVerse}
on the limb there was a nest a rare nest a rattlin nest.\\*%
5...
\end{SongVerse}
\begin{SongVerse}
in the nest there was a bird a rare bird a rattlin bird.\\*%
6...
\end{SongVerse}
\begin{SongVerse}
in the bird there was a egg a rare egg a rattlin egg.\\*%
7...
\end{SongVerse}
\begin{SongVerse}
in the egg there was a bird a rare bird a rattlin bird.\\*%
8...
\end{SongVerse}
\begin{SongVerse}
on the bird there was a feather a rare feather a rattlin feather.\\*%
9...
\end{SongVerse}
\begin{SongVerse}
on the feather there was a worm a rare worm a rattlin worm\\*%
A...
\end{SongVerse}
\begin{SongVerse}
on the worm there was a hair a rare hair a rattlin hair. \\*%
B...
\end{SongVerse}
\begin{SongVerse}
om the hair there was a louse a rare louse a rattlin louse\\*%
C...
\end{SongVerse}
\begin{SongVerse}
on the louse there was a tick a rare tick a rattlin tick\\*%
D...
\end{SongVerse}
\begin{SongVerse}
on the tick there was a rash a rare rash a rattlin rash.\\*%
E...
\end{SongVerse}
\begin{SongVerse}
Ref...
\end{SongVerse}
\end{SongText}
\begin{SongText}[Under en filt i Madrid]
\begin{SongInfo}
    Text: Galenskaparna och after shave 
\end{SongInfo}
\begin{SongVerse}
Under en filt i Madrid\\*%
ligger en flicka på glid\\*%
tittar på mannen bredvid\\*%
under en filt i Madrid.
\end{SongVerse}
\begin{SongVerse}
Bakom ett berg i Geneve\\*%
där får en moder ett brev\\*%
från hennes dotter på glid\\*%
under en filt i Madrid. 
\end{SongVerse}
\begin{SongVerse}
Framför en stolpe i Bonn\\*%
sitter det nu inte nån\\*%
endast en tom La Garonne\\*%
framför en stolpe i Bonn.
\end{SongVerse}
\begin{SongVerse}
Men där i vindarnas drev\\*%
fladdrar ett brev från Geneve\\*%
postat nån gång i Bretagne\\*%
doftar av billig champagne.
\end{SongVerse}
\begin{SongVerse}
På en bordell i Borås\\*%
smörjer en herre sitt krås\\*%
bakom ett skjul i Tasjkent\\*%
där står ett fönster på glänt.
\end{SongVerse}
\begin{SongVerse}
Någon har kastat ett skal\\*%
genom en jak i Nepal\\*%
ingenting är som det skall\\*%
solen är blott en marschall.
\end{SongVerse}
\begin{SongVerse}
Själv är jag blott en kostym\\*%
mamma är bara parfym\\*%
pappa förspiller sin tid\\*%
under en filt i Madrid.
\end{SongVerse}
\begin{SongVerse}
Under ett lakan i Prag\\*%
där ligger en kvinna och jag\\*%
sängen är full av resår\\*%
sången jag sjunger är svår.
\end{SongVerse}
\begin{SongVerse}
Omöjlig att komma ur\\*%
jag vet då fan inte hur\\*%
orden får snart inte rum\\*%
jag får väl sjunga mig stum.
\end{SongVerse}
\begin{SongVerse}
Tonerna trängs i min gom\\*%
sätt mig på tåget till Rom\\*%
låt mig få sluta min tid\\*%
under en filt i Madrid! 
\end{SongVerse}
\end{SongText}
%\begin{SongText}[Balladen om Fredrik Åkare och fröken Cecilia Lind]
%\begin{SongInfo}
%    Text: Cornelis Wreeswijk
%\end{SongInfo}
%\begin{SongVerse}
%Från Öckerö brygga hörs dragspel och bas\\*%
%och fullmånen lyser som var den av glas.\\*%
%Där dansar Fredrik Åkare kind emot kind\\*%
%med lilla fröken Cecilia Lind.
%\end{SongVerse}
%\begin{SongVerse}
%Hon dansar och blundar så nära intill.\\*%
%Hon följer i dansen precis vart han vill.\\*%
%Han för och hon följer lätt som en vind,\\*%
%men säg varför rodnar Cecilia Lind?
%\end{SongVerse}
%\begin{SongVerse}
%Säg var det för det Fredrik Åkare sa?\\*%
%"Du doftar så gott och du dansar så bra\\*%
%din midja är smal och barmen är trind.\\*%
%Vad du är vacker Cecilia Lind"
%\end{SongVerse}
%\begin{SongVerse}
%Men dansen tog slut och vart skulle dom gå?\\*%
%De bodde så nära varandra ändå\\*%
%Till slut kom dom fram till Cecilias grind\\*%
%"Nu vill jag bli kysst" sa Cecilia Lind 
%\end{SongVerse}
%\begin{SongVerse}
%Vet hut, Fredrik Åkare skäms gamla karln!\\*%
%Cecilia Lind är ju bara ett barn\\*%
%Ren som en blomma, skygg som en hind\\*%
%"Jag fyller snart sjutton", sa Cecilia Lind
%\end{SongVerse}
%\begin{SongVerse}
%Och stjärnorna vandra och timmarna fly\\*%
%och Fredrik är gammal, men månen är ny\\*%
%Ja Fredrik är gammal men kärleken blind\\*%
%"Åh kyss mig igen", sa Cecilia Lind.
%\end{SongVerse}
%\end{SongText}
\begin{SongText}[Brev från kolonien]
\begin{SongInfo}
    Text: Cornelis Wreeswijk
\end{SongInfo}
\begin{SongVerse}
Hejsan morsan hejsan stabben.\\*%
Här e’ brev från älskingsgrabben.\\*%
Vi har kul på kollonien,\\*%
vi bor tjugoåtta gangstergrabbar i en...
\end{SongVerse}
\begin{SongVerse}
...stor barack med massa sängar.\\*%
Kan ni skicka mera pengar?\\*%
För det vore en god gärning,\\*%
jag har spelat bort varenda dugg på tärning.
\end{SongVerse}
\begin{SongVerse}
Här e’ roligt vill jag lova,\\*%
fastän lite svårt att sova.\\*%
Killen som har sängen över mej\\*%
han vaknar inte, han, när han behöver, nej!
\end{SongVerse}
\begin{SongVerse}
Jag har tappat två framtänder,\\*%
för jag skulle gå på händer,\\*%
när vi lattjade charader,\\*%
så när morsan nu får se mej får hon spader.
\end{SongVerse}
\begin{SongVerse}
Uti skogen finns baciller,\\*%
men min kompis han har piller\\*%
som han köpt av en ful typ\\*%
och om man äter dom blir man en jättekul typ.
\end{SongVerse}
\begin{SongVerse}
Jag är inte rädd för spöken\\*%
och min kompis, han har kröken\\*%
som han gjort utav potatis\\*%
och den säljer han i baracken nästan gratis.
\end{SongVerse}
\begin{SongVerse}
Våran fröken är försvunnen\\*%
hon har dränkt sig uti brunnen\\*%
för en morgon blev hon galen\\*%
när vi släppte ut en huggorm i matsalen
\end{SongVerse}
\begin{SongVerse}
Föreståndarn han har farit,\\*%
han blir aldrig vad han varit\\*%
för polisen kom och tog hand om\\*%
honom för en vecka sedan när vi lekte skogsbrand
\end{SongVerse}
\begin{SongVerse}
Uti skogen finns det rådjur,\\*%
i baracken finns det smådjur\\*%
och min bäste kompis Tage,\\*%
han har en liten fickkniv inuti sin mage
\end{SongVerse}
\begin{SongVerse}
Honom skall dom operera.\\*%
Ja, nu vet jag inge’ mera.\\*%
Kram och kyss och hjärtligt tack sen\\*%
men nu ska vi ut och bränna grannbaracken.
\end{SongVerse}
\end{SongText}
%\begin{SongText}[Kungens man]
%\begin{SongInfo}
%    Text: Björn Afzelius
%\end{SongInfo}
%\begin{SongVerse}
%Maria går på vägen som leder in till byn,\\*%
%hon sjunger och hon skrattar åt lärkorna i skyn.\\*%
%Hon är på väg till torget för att sälja lite bröd,\\*%
%och solen stiger, varm och stor, och färgar himlen röd.
%\end{SongVerse}
%\begin{SongVerse}
%Då möter hon en herre på en häst med yvig man.\\*%
%Han säger: 'Jag är kungens man, så jag tar vad jag vill ha.\\*%
%Och du är allt för vacker för att inte ha nå'n man,\\*%
%följ med mej in i skogen skall jag visa vad jag kan!'
%\end{SongVerse}
%\begin{SongVerse}
%Hon tvingas ned i gräset, och han tar på hennes kropp.\\*%
%Hon slingrar sej och ber honom, för Guds skull, hålla opp.\\*%
%Men riddar'n bara skrattar, berusad av sin glöd.\\*%
%Så hon tar hans kniv och stöter till, och riddaren är död
%\end{SongVerse}
%\begin{SongVerse}
%Dom fängslade Maria, hon stenades för dråp.\\*%
%Men minnet efter riddar'n blev firat varje år.\\*%
%Ja, herrarna blir hjältar, men folket de blir dömt.\\*%
%Och vi, som vet hur allt går till, får veta att vi drömt!
%\end{SongVerse}
%\end{SongText}
%\begin{SongText}[Mitt lilla rugbylag och jag]
%\begin{SongInfo}
%    Text: Galenskaparna och after shave
%\end{SongInfo}
%\begin{SongVerse}
%Jag har suttit på min kammare och tänkt på dig idag\\*%
%jag har tänkt så fina tankar och nu har jag ett förslag:\\*%
%livet är en rugbymatch, så fullt av hugg och slag,\\*%
%vill du spela i mitt rugbylag?
%\end{SongVerse}
%\begin{SongVerse}
%$\|\:$ Mitt lilla rugbylag och du\\*%
%mitt lilla rugbylag och jag\\*%
%mitt lilla rugby-rugby-lag $\:\|$
%\end{SongVerse}
%\begin{SongVerse}
%Tacklingarna klarar du,\\*%
%tacklingarna klarar jag,\\*%
%om du bara spela vill\\*%
%i mitt rugbylag
%\end{SongVerse}
%\begin{SongVerse}
%$\|\:$ Mitt lilla rugbylag och du\\*%
%mitt lilla rugbylag och jag\\*%
%mitt lilla rugby-rugby-lag $\:\|$
%\end{SongVerse}
%\end{SongText}
%\begin{SongText}[Fredmans epistel nr 2 ]
%\begin{SongInfo}
%    Text: Carl Michael Bellman\\*%
%    Till fader Berg, rörande posten
%\end{SongInfo}
%\begin{SongVerse}
%Nå, skrufva fiolen,\\*%
%hej spelman, skynda dig!\\*%
%Kära syster, hej!\\*%
%Svara inte nej,\\*%
%svara ja, så blir vi glada.\\*%
%Sätt dig du på stolen\\*%
%och stryk din silfversträng!\\*%
%Röda stråken släng\\*%
%och med armen sväng;\\*%
%gör ej fiolen skada!\\*%
%Du svettas, stor sak,\\*%
%i brännvin skall du bada,\\*%
%ty under detta tak är\\*%
%Bacchi lada.\\*%
%- - Ganska riktigt!\\*%
%Ditt kall är viktigt\\*%
%båd’ för öra, syn och smak.
%\end{SongVerse}
%\begin{SongVerse}
%Bland nymfernas skara\\*%
%är du omistlig man;\\*%
%du båd’ vill och kan\\*%
%mer än någon ann\\*%
%de unga hjärtan binda,\\*%
%och kärlekens snara\\*%
%på dina strängar står\\*%
%Varje ton du slår,\\*%
%du ett hjärta får\\*%
%att konstigt sammanlinda.\\*%
%Just på en minut\\*%
%små ögon blifva blinda,\\*%
%och flickorna till slut\\*%
%de blir så trinda.\\*%
%- - Hur du bullrar!\\*%
%Men nymfen kullrar,\\*%
%och skrattar med din trut. 
%\end{SongVerse}
%\begin{SongVerse}
%Jag älskar de sköna\\*%
%men vinet ändå mer;\\*%
%jag på båda ser\\*%
%och åt båda ler\\*%
%men skiljer ändå båda.\\*%
%En nymf i det gröna\\*%
%och vin i gröna glas:\\*%
%Lika gott kalas,\\*%
%båda om mig dras.\\*%
%Ge stråken mera kåda:\\*%
%Konfonium tag där\\*%
%uti min gröna låda;\\*%
%och vinet står ju här.\\*%
%Jag är i våda.\\*%
%- - Supa, dricka\\*%
%och ha sin flicka\\*%
%är vad Sankte Fredman lär. 
%\end{SongVerse}
%\end{SongText}
\begin{SongText}[Fredmans epistel nr 5]
\begin{SongInfo}
    Text: Carl Michael Bellman\\*%
    Till the trogne bröder på Terra Nova i Gaffelgränd
\end{SongInfo}
\begin{SongVerse}
Käre bröder, så låtom oss supa i frid\\*%
i denna här världens ondsko och strid\\*%
lätt oss streta,\\*%
arbeta\\*%
stampa,\\*%
trampa\\*%
druvor pressa, ty än är det tid.\\*%
The ölepheser ä stridbare män\\*%
the gutårinter hovera jamän\\*%
blifva besatta\\*%
och skratta\\*%
och supa\\*%
och stupa\\*%
emellan buteljerna sen. 
\end{SongVerse}
\begin{SongVerse}
Slå i stopenom, slå locket ihop\\*%
fly allan förargelse, töm edra stop\\*%
I all kättja\\*%
och flättja.\\*%
Stampa\\*%
trampa
\end{SongVerse}
\begin{SongVerse}
nu Bacchi pressar vi klingande rop!\\*%
Brännvinsapostlar uppstiga var dag,\\*%
stöta basuner, förkunna vår lag,\\*%
rusta och rasa\\*%
och flasa\\*%
och mumla\\*%
och tumla som hjältar i blodiga slag.\\*%
Gå då ödmjukt till flaska inbördes gutår!\\*%
Gå baklänges bort där som nykterhet rår;\\*%
Var nu listig\\*%
och dristig\\*%
stampa\\*%
trampa\\*%
i Bacchi vingård stå kvar där du står.\\*%
Drick min Theophile, strupen är djup\\*%
si, i Damasco där ligger en slup,\\*%
fuller med flaskor,\\*%
damasker\\*%
kantiner\\*%
och viner\\*%
Åh kära, ro hit med en sup! 
\end{SongVerse}
\end{SongText}
%\begin{SongText}[Fredmans epistel nr 9 ]
%\begin{SongInfo}
%    Text: Carl Michael Bellman\\*%    
%    Till gumman på Thermopolium Boreale och hennes jungfrur
%\end{SongInfo}
%\begin{SongVerse}
%Käraste bröder, systrar och vänner,\\*%
%si, fader Berg, han skrufvar och spänner\\*%
%strängarna på fiolen,\\*%
%och stråken han tar i hand.\\*%
%Ögat är borta, näsan är klufven:\\*%
%si, hur han står och spottar på skrufven,\\*%
%ölkannan står på stolen.\\*%
%Nu knäpper han lite grand,\\*%
%V:cello - - - grinar mot solen,\\*%
%V:cello - - - pinar fiolen,\\*%
%V:cello - - - han sig förvillar,\\*%
%drillar ibland.\\*%
%Käraste bröder, dansa på tå,\\*%
%handskar i hand och hattarna på!\\*%
%Si på jungfru Lona\\*%
%röda band i skona,\\*%
%nya strumpor, himmelsblå!\\*%
%Si, Jergen Puckel fläktar med hatten,\\*%
%pipan i mun, och brännvin som vatten\\*%
%dricker han gör fukter\\*%
%med hufvud och hand och fot.
%\end{SongVerse}
%\begin{SongVerse}
%Gullguler rock med styfva dycrenger;\\*%
%tätt uti nacken hårpiskan hänger,\\*%
%ryggen i hundra bukter\\*%
%och kindbenen stå som klot;\\*%
%V:cello - - - gapar på noten,\\*%
%V:cello - - - skrapar med foten,\\*%
%V:cello - - - pipan han stoppar,\\*%
%hoppar emot.\\*%
%Käraste systrar, alltid honnett:\\*%
%bröderna dansa, jämt menuett,\\*%
%hela natten fulla.\\*%
%Rak i lifvet Ulla...\\*%
%Ge nu hand håll takten rätt!
%\end{SongVerse}
%\begin{SongVerse}
%Si, hvem är det i nattrock, så nätter,\\*%
%med gula byxor, hvita stöfletter,\\*%
%som dansar där med Lotta,\\*%
%den där, som har röd peruk?\\*%
%Ta mig sju tusan, se, två i flocken\\*%
%sydde manschetter, snören på rocken...\\*%
%Drick fader Berg, och spotta!\\*%
%Tvi... svagdricka gör mig sjuk.\\*%
%V:cello - - - Kruset skall brinna,\\*%
%V:cello - - - huset skall brinna:\\*%
%V:cello - - - ingen skall klämta.\\*%
%Flämta, min buk!\\*%
%Käresta systrar, tagen i ring,\\*%
%dansa och fläkta, tumla och spring!\\*%
%Var nu blind och döfver!\\*%
%Spelman nu ger öfver,\\*%
%raglar med fiolen kring.
%\end{SongVerse}
%\begin{SongVerse}
%Hej, mina flickor, lyfta på kjolen,\\*%
%dansa och skratta... hör basfiolen!\\*%
%Ge fader Berg konfonium\\*%
%och hoglands med gröna blan!\\*%
%Hör, fader Berg, såg du, hvad hon heter,\\*%
%hon där vid skänken vinögd och feter?\\*%
%“Gumman är Thermopolium!“\\*%
%Hon är det ja, ta mig f-n!\\*%
%V:cello - - - Trumpen och blinder!\\*%
%V:cello - - - Gumpen är trinder...\\*%
%V:cello - - - Hals-fräs min gumma!\\*%
%Brumma, dulcian!\\*%
%Käraste bröder, här är behag:\\*%
%här är musik och flickor hvar dag,\\*%
%här är Bacchus buden,\\*%
%här är kärleksguden,\\*%
%här är allting här är jag. 
%\end{SongVerse}
%\end{SongText}
\begin{SongText}[Fredmans sång nr 21 (lång)]
\begin{SongInfo}
    Text: Carl Michael Bellman
\end{SongInfo}
\begin{SongVerse}
Så lunka vi så småningom
från Bacchi buller och tumult,
när döden ropar; Granne kom,
ditt timglas är nu fullt.
Du gubbe fäll din krycka ner,
och du yngling, lyd min lag,
den skönsta nymf som mot dig ler
inunder armen tag. 
\end{SongVerse}
\begin{SongVerse}
Ref:
Tycker du att graven är för djup,
nå välan, så tag dig då en sup,
tag dig sen dito en, dito två, dito tre,
så dör du nöjdare.
\end{SongVerse}
\begin{SongVerse}
Du vid din remmare och préss,
rödbrusig och med hatt på sned,
snart skrider fram din likprocess
i några svarta led.
Och du som pratar där så stort,
med band och stjärnor på din rock,
re’n snickarn kistan färdig gjort,
och hyvlar på dess lock. 
\end{SongVerse}
\begin{SongVerse}
Ref...
\end{SongVerse}
\begin{SongVerse}
Men du som med en trumpen min,
bland riglar, galler, järn och lås,
dig vilar på ditt penningskrin,
inom din stängda bås.
Och du som svartsjuk slår i kras
buteljer, speglar och pokal;
bjud nu god natt, drick ut ditt glas,
och hälsa din rival.
\end{SongVerse}
\begin{SongVerse}
Ref...
\end{SongVerse}
\begin{SongVerse}
Och du som under titlars klang
din tiggarstav förgyllt vart år,
som knappast har, med all din rang,
en skilling till din bår.
Och du som ilsken, feg och lat,
fördömer vaggan som dig välvt,
och ändå dagligt är plakat
till glasets sista hälft. 
\end{SongVerse}
\begin{SongVerse}
Ref...
\end{SongVerse}
\begin{SongVerse}
Du som vid Martis fältbasun
i blodig skjorta sträckt ditt steg,
och du som tumlar i paulun,
i Chloris armar feg.
Och du som med din gyllne bok
vid templets genljud reser dig,
som rister huvud lärd och klok,
och för mot avgrund krig,
\end{SongVerse}
\begin{SongVerse}
Ref...
\end{SongVerse}
\begin{SongVerse}
Men du som med en ärlig min
plär dina vänner häda jämt,
och dem förtalar vid ditt vin,
och det liksom på skämt.
Och du som ej försvarar dem,
fastän ur deras flaskor du,
du väl kan slicka dina fem,
vad svarar du väl nu?
\end{SongVerse}
\begin{SongVerse}
Ref...
\end{SongVerse}
\begin{SongVerse}
Men du som till din återfärd,
ifrån det du till bordet gick,
ej klingat för din raska värd,
fastän han ropar: ”Drick”.
Driv sådan gäst från mat och vin!
Kör honom med sitt anhang ut,
och sen med en ovänlig min,
ryck remmarn ur hans trut.
\end{SongVerse}
\begin{SongVerse}
Ref...
\end{SongVerse}
\begin{SongVerse}
Säg är du nöjd, min granne säg,
så prisa världen nu till slut;
om vi ha en och samma väg,
så följoms åt; drick ut.
Men först med vinet rött och vitt
för vår värdinna bugom oss,
och halkom sen i graven fritt,
vid aftonstjärnans bloss.
\end{SongVerse}
\begin{SongVerse}
Ref... 
\end{SongVerse}
\end{SongText}
\begin{SongText}[Fredmans sång nr 35 (Gubben Noak)]
\begin{SongInfo}
    Text: Carl Michael Bellman
\end{SongInfo}
\begin{SongVerse}
||: Gubben Noak :||\\*%
var en hederman.\\*%
när han gick ur arken\\*%
planterade han på marken\\*%
||: mycket vin, ja :||\\*%
detta gjorde han.
\end{SongVerse}
\begin{SongVerse}
||: Noak rodde :||\\*%
ur sin gamla ark,\\*%
köpte sig butäljer,\\*%
sådana man säljer\\*%
||: för att dricka :||\\*%
på vår nya park.
\end{SongVerse}
\begin{SongVerse}
||: Han väl visste :||\\*%
att en mänska var\\*%
törstig av naturen\\*%
som de andra djuren\\*%
||: därför han ock :||\\*%
vin planterat har.
\end{SongVerse}
\begin{SongVerse}
||: Gumman Noak :||\\*%
var en hedeersfru.\\*%
Hon gav man sin dricka;\\*%
fick, jag sådan flicka\\*%
||: gifte jag mig :||\\*%
just på stunden nu.
\end{SongVerse}
\begin{SongVerse}
||: Aldrig sad’ hon :||\\*%
Kära far nå nå,\\*%
sätt ifrån dig kruset.\\*%
Nej, det ena ruset\\*%
||: på det andra :||\\*%
lät hon gubben få.
\end{SongVerse}
\begin{SongVerse}
||: Gubben Noak :||\\*%
brukte egna hår,\\*%
pipskägg, hakan trinder,\\*%
rosenröda kinder,\\*%
||: drack ibotten :||\\*%
Hurra och gutår!
\end{SongVerse}
\begin{SongVerse}
Då var lustigt, då var lustigt\\*%
På vår gröna jord;\\*%
Man fick väl till bästa,\\*%
Ingen torstig nästa\\*%
Satt och blängde, satt och blängde\\*%
Vid ett dukat bord.
\end{SongVerse}
\begin{SongVerse}
Inga skålar, inga skålar\\*%
Gjorde då besvär,\\*%
Då var ej den läran:\\*%
Jag skall ha den äran.\\*%
Nej i botten, nej i botten\\*%
Drack man ur så här. 
\end{SongVerse}
\end{SongText}
%\begin{SongText}[Fredmans epistel nr 48]
%\begin{SongInfo}
%    Text: Carl Michael Bellman
%\end{SongInfo}
%\begin{SongVerse}
%Solen glimmar blank och trind,\\*%
%vattnet likt en spegel!\\*%
%Småningom uppblåser vind,\\*%
%i de fallna segel;\\*%
%vimpeln sträcks, och med en år\\*%
%Olle på en höbåt står:\\*%
%Kerstin ur kajutan går,\\*%
%skjuter lås och regel.
%\end{SongVerse}
%\begin{SongVerse}
%Stålet gnistrar, pipan tänds,\\*%
%Olle klår sitt öra,\\*%
%rodret vrides, skutan vänds,\\*%
%gubben har att göra.\\*%
%Under skarpa ögonbryn\\*%
%grinar han mot sol’n i skyn;\\*%
%Kerstin, gubbens hjärtegryn,\\*%
%skall nu seglen föra.
%\end{SongVerse}
%\begin{SongVerse}
%Seglen fladdra, skutan går,\\*%
%Jerker tar sin lyra;\\*%
%lyran brummar, böljan slår,\\*%
%allt med våld och yra.\\*%
%Skutan knarkar, bräcklig, gles,\\*%
%vimpelns fläkt i toppen ses.\\*%
%Tuppen gol så sträf och hes\\*%
%nu slog klockan fyra. 
%\end{SongVerse}
%\end{SongText}
%\begin{SongText}[Fredmans sång nr 64 ]
%\begin{SongInfo}
%    Mel: Carl Michael Bellman
%\end{SongInfo}
%\begin{SongVerse}
%Fjäril vingad syns på Haga
%mellan dimmors frost och dun.
%Sig sitt gröna skjul tilltaga
%och i blomma sin paulun.
%Minsta kräk i kärr och syra,
%nyss av solens värme väckt.
%Till en ny högtidlig yra
%eldars vid zephirens fläckt.
%\end{SongVerse}
%\begin{SongVerse}
%Haga, i ditt sköte röjes
%gräsets brodd och gula plan.
%Stolt i dina rännar höjes
%gungande den hvita svan.
%Längst ur skogens glesa kamrar
%höras täta återskall
%än från den graniten hamrar,
%än från yx i björk och tall.
%\end{SongVerse}
%\begin{SongVerse}
%Se, Brunnsvikens små najader
%höja sina gyllne horn,
%och de frustande kaskader
%sprutas över Solna torn.
%Under skygd av välvda stammar
%på den väg man städad ser,
%fålen yvs och hjulet dammar.
%Bonden milt åt Haga ler. 
%\end{SongVerse}
%\end{SongText}
\begin{SongText}[Sånt är livet]
\begin{SongInfo}
    Text: Anita Lindblom
\end{SongInfo}
\begin{SongVerse}
Sånt är livet! Sånt är livet!\\*%
Så mycken falskhet bor det här.\\*%
Den man förlorar, vinner en annan,\\*%
så håll i vännen, som du har kär!
\end{SongVerse}
\begin{SongVerse}
Hon (han) kom om våren, som en vårvind.\\*%
Min kärlek fick hon (han) och allt hon (han) tog.\\*%
Men så kom hösten och den kärlek\\*%
hon (han) svor var evig bara dog.
\end{SongVerse}
\begin{SongVerse}
Ref:\\*%
||: Ja, sånt är livet! Sånt är livet!\\*%
Så mycken falskhet bor det här.\\*%
Den man förlorar, vinner en annan,\\*%
så håll i vännen som du har kär! :||
\end{SongVerse}
\begin{SongVerse}
Hon (han) fick en annan. Jag har sett dem.\\*%
Hon (han) verkar lycklig, och han (hon) är ung.\\*%
Det jag har lärt mig, är just detta:\\*%
När hjärtat svider - sjung, blott sjung !
\end{SongVerse}
\begin{SongVerse}
Ref... (För sånt är livet!)
\end{SongVerse}
\begin{SongVerse}
Vårt liv är fattigt utan kärlek.\\*%
Jag fick en annan som har mig kär.\\*%
Vars (hans) gamla kärlek har fått korgen.\\*%
Han (hon) undrar säkert vem jag är !
\end{SongVerse}
\begin{SongVerse}
Ref... (Men sånt är livet!)
\end{SongVerse}
\end{SongText}
%\begin{SongText}[Glöm inte bort]
%\begin{SongInfo}
%    Text:  Anders Engbergs orkester
%\end{SongInfo}
%\begin{SongVerse}
%Kärlek och löften som kommer och går\\*%
%Något som någon har glömt\\*%
%Det kan va’ långt till att säga förlåt\\*%
%Kan kännas så avigt och dumt
%\end{SongVerse}
%\begin{SongVerse}
%Ordet man tänkt kan va’ svårt att få fram\\*%
%Fastnar i halsen ibland\\*%
%I dina ögon finns redan ett svar\\*%
%Det som ska sägas i moll
%\end{SongVerse}
%\begin{SongVerse}
%Glöm inte bort att det finns rosor\\*%
%Glöm inte bort att ta dom röda\\*%
%Glöm inte bort att dom har törner\\*%
%Och kan va’ känsliga och spröda
%\end{SongVerse}
%\begin{SongVerse}
%Kärlek kan lagas där något gått fel\\*%
%En blomma kan göra den hel\\*%
%Dörrar kan öppnas, en mur rivas ner\\*%
%Sträck fram dom och människor ler
%\end{SongVerse}
%\begin{SongVerse}
%Glöm inte bort att det finns rosor\\*%
%Glöm inte bort att ta dom röda\\*%
%Glöm inte bort att dom har törner\\*%
%Men dom får hjärtan börja glöda
%\end{SongVerse}
%\begin{SongVerse}
%Ta telefonen, ring upp, prata ut\\*%
%Skicka ett brev, få det sagt\\*%
%Trösklarna kan bli för höga till slut\\*%
%När människor söker kontakt
%\end{SongVerse}
%\begin{SongVerse}
%Glöm inte bort att det finns rosor\\*%
%Glöm inte bort att ta dom röda\\*%
%Glöm inte bort att dom har törner\\*%
%Och kan va’ känsliga och spröda
%\end{SongVerse}
%\begin{SongVerse}
%Kärlek kan lagas där något gått fel\\*%
%En blomma kan göra den hel\\*%
%Dörrar kan öppnas, en mur rivas ner\\*%
%Sträck fram dom och människor ler
%\end{SongVerse}
%\begin{SongVerse}
%Glöm inte bort att det finns rosor\\*%
%Glöm inte bort att ta dom röda\\*%
%Glöm inte bort att dom har törner\\*%
%Men dom får hjärtan börja glöda
%\end{SongVerse}
%\begin{SongVerse}
%||: Glöm inte bort :||
%\end{SongVerse}
%\begin{SongVerse}
%Glöm inte bort att det finns rosor\\*%
%Glöm inte bort att ta dom röda\\*%
%Glöm inte bort att dom har törner\\*%
%Men dom får hjärtan börja glöda
%\end{SongVerse}
%\end{SongText}
%\begin{SongText}[Jag ger dig min morgon]
%\begin{SongInfo}
%    Text: Fred Åkerström
%\end{SongInfo}
%\begin{SongVerse}
%Åter igen gryr dagen vid din bleka skuldra\\*%
%Genom frostigt glas syns solen som en huldra\\*%
%Ditt hår, det flyter över hela kudden\\*%
%Om du var vaken skulle jag ge dig allt det där jag aldrig ger dig\\*%
%Men du, jag ger dig min morgon, jag ger dig min dag
%\end{SongVerse}
%\begin{SongVerse}
%Vår gardin den böljar svagt där solen strömmar\\*%
%Långt bakom ditt öga svinner nattens drömmar\\*%
%Du drömmer om nåt fint, jag ser dig småle\\*%
%Om du var vaken... 
%\end{SongVerse}
%\begin{SongVerse}
%Utanför vårt fönster hör vi markens sånger\\*%
%Som ett rastlöst barn om våren dagen kommer\\*%
%Lyssna till den sång som jorden sjunger\\*%
%Om du var vaken...
%\end{SongVerse}
%\begin{SongVerse}
%Likt en sländas spröda vinge ögat skälver\\*%
%Solens smälta i ditt hår kring pannan välver\\*%
%Du, jag tror vi flyr rakt in i solen\\*%
%Om du var vaken.
%\end{SongVerse}
%\end{SongText}
\begin{SongText}[Balladen om Theobald Thor]
\begin{SongInfo}
    Text: Christian Engström\\*%
    Mel: Ball of Kirriemuir
\end{SongInfo}
\begin{SongVerse}
En man som hette Theobald Thor\\*%
Han var en skicklig tamburmajor\\*%
Succén han gjorde var alltid stor\\*%
För han snurra och svängde sin kuk
\end{SongVerse}
\begin{SongVerse}
Det var en stor kuk (Hur stor!?)\\*%
lång, kraftig och tung\\*%
Från dess topp till dess rot\\*%
Var den tre, fyra fot\\*%
Med en medelstor ryggsäck till pung\\*%
(pung, pung, pung, pungeli,  pung, pung)
\end{SongVerse}
\begin{SongVerse}
En dag gick Thobald ut en stund\\*%
Att gå för sig själv i en lummig lund\\*%
Han mötte en söt liten dam med sin hund\\*%
Som fick se honom svänga sin kuk
\end{SongVerse}
\begin{SongVerse}
Det var en stor…
\end{SongVerse}
\begin{SongVerse}
Och Theobald prova ett trick han lärt\\*%
Han släppte sin lem i en kraftig snärt\\*%
I huvet på hunden som avled tvärt\\*%
när han snurra och svängde sin kuk
\end{SongVerse}
\begin{SongVerse}
Det var en stor…
\end{SongVerse}
\begin{SongVerse}
Men damen hon blev helt bestört\\*%
Hon svor och skrek nåt oerhört\\*%
Så det var ingen lyckad flört\\*%
Att snurra och svänga sin kuk
\end{SongVerse}
\begin{SongVerse}
Fast det var en stor…
\end{SongVerse}
\begin{SongVerse}
Till följd av damens arga gnäll\\*%
Han anhölls redan samma kväll\\*%
Och sattes i en ensamcell\\*%
Att snurra och svänga sin kuk
\end{SongVerse}
\begin{SongVerse}
Det var en stor…
\end{SongVerse}
\begin{SongVerse}
När målet kom i rätten opp\\*%
Sa åklagren: "det får bli stopp\\*%
Man får ej vifta med sin snopp\\*%
Och snurra och svänga sin kuk"
\end{SongVerse}
\begin{SongVerse}
Fast det var en stor…
\end{SongVerse}
\begin{SongVerse}
Men domarn han var tolerant\\*%
Han sa: själv gör jag likadant\\*%
Jag tycker det är intressant\\*%
Att snurra och svänga min kuk
\end{SongVerse}
\begin{SongVerse}
Jag har en stor…
\end{SongVerse}
\begin{SongVerse}
Så Theobald han släpptes fri\\*%
Och liksom domarn tycker vi\\*%
att tjejer de ska skita i\\*%
om vi snurrar och svänger vår kuk
\end{SongVerse}
\begin{SongVerse}
Vi har en stor kuk (Hur stor!?)\\*%
lång, kraftig och tung\\*%
Från dess topp till dess rot\\*%
Var den SJU ÅTTA fot\\*%
Med en JÄTTEstor ryggsäck till pung!
\end{SongVerse}
\end{SongText}
\begin{SongText}[Balladen om Signhild Svahn]
\begin{SongInfo}
    Text: Orbis Primus LiU\\*%
    Mel: Ball of Kirriemuir
\end{SongInfo}
\begin{SongVerse}
En dam som hette Signhild Svahn,\\*%
Hon var en bra skicklig kleptoman.\\*%
Kollegan var hennes könsorgan,\\*%
För hon stoppade allt i sin mus. 
\end{SongVerse}
\begin{SongVerse}
Och det var en stor mus, (Hur stor!?)\\*%
djup, saftig, med spänst.\\*%
Utan början och slut, du kan ej hitta ut\\*%
Ur ett medelstort hav utav mens.
\end{SongVerse}
\begin{SongVerse}
(mens, mens, mens, mensili, mens, mens)
\end{SongVerse}
\begin{SongVerse}
En handväska, hon hade ej,\\*%
hon var inte den typ av tjej,\\*%
med förvaring för varande grej,\\*%
hon stoppade allt i sin mus. För
\end{SongVerse}
\begin{SongVerse}
För det var en stor…
\end{SongVerse}
\begin{SongVerse}
En dag gick Signhild ut på stan.\\*%
Hon mötte då en mytoman.\\*%
Hon tog hans kans katt, var helt spontan,\\*%
och stoppa´ den upp i sin mus,\\*%
Och det var en stor…
\end{SongVerse}
\begin{SongVerse}
Och karln förstod knappt vad som skett.\\*%
Han skrev och gorma utan vett.\\*%
Och då blev Signhilds angreppssätt,\\*%
att stoppa´ han upp i sin mus.
\end{SongVerse}
\begin{SongVerse}
För det var en stor…
\end{SongVerse}
\begin{SongVerse}
Signhild gick sen nonchalant,\\*%
vidare och stal briljant.\\*%
Men ägaren var observant,\\*%
när hon stoppa den upp i sin mus.
\end{SongVerse}
\begin{SongVerse}
För det var en stor...
\end{SongVerse}
\begin{SongVerse}
Signhild hon blev anmäld för,\\*%
att stjäla saker man ej bör\\*%
och kväva karlar så dom dör\\*%
av at stoppa dem upp i sin mus
\end{SongVerse}
\begin{SongVerse}
Fast det var en stor…
\end{SongVerse}
\begin{SongVerse}
Men domarn satte sig på tvär.\\*%
"Jag själv gör just precis sådär.\\*%
På tålamodet männen tär\\*%
och då stoppas dom upp i min mus
\end{SongVerse}
\begin{SongVerse}
För jag har en stor…"
\end{SongVerse}
\begin{SongVerse}
Så Signhild lagrar än däri\\*%
Och liksom henne tycker vi.\\*%
att killar de ska skita i\\*%
vad man stoppar upp i sin mus.
\end{SongVerse}
\begin{SongVerse}
Vi har en stor mus (Hur stor!?)\\*%
djup, saftig och spänst.\\*%
Utan start, utan slut,\\*%
Du kan ej hitta ut\\*%
Ur ett JÄTTEstort hav utav mens.
\end{SongVerse}
\end{SongText}
\begin{SongText}[Naken]
\begin{SongInfo}
    Text: 250 kg kärlek
\end{SongInfo}
\begin{SongVerse}
Jag åkte skridskor under Västerbron i tron \\*%
att isen var tjock men det var den inte\\*%
så jag plumsade ner i en vak\\*%
när jag låg där och skrek då kom en man\\*%
jag ropa "hjälp mej upp" men det gjorde inte han\\*%
han klädde av sig naken och hoppa ner i vaken \\*%
och sa:
\end{SongVerse}
\begin{SongVerse}
Ref:\\*%
Oh oh vad det är skönt å va naken\\*%
Svänga med snabeln och vicka på baken\\*%
Oh oh vad det är skönt å va naken\\*%
Svänga med snabeln och vicka på baken
\end{SongVerse}
\begin{SongVerse}
En sommarkväll hade vårat gäng \\*%
fest i stadens simbassäng \\*%
alla var glada, nakna och fulla\\*%
en del var faktiskt jättefulla\\*%
men när vi tömde bassängen och fyllde på med isen \\*%
för att kyla bärsen då kom polisen\\*%
å dom haffa mej.... dom sa:\\*%
"dej håller vi kvar får vi höra ditt försvar"
\end{SongVerse}
\begin{SongVerse}
Jag sa:\\*%
Ref... 
\end{SongVerse}
\begin{SongVerse}
Jag åkte till Åland och handlade sprit \\*%
men då åkte jag dit i tullen\\*%
De trodde visst jag var terrorist \\*%
och letade långt upp i tarmen \\*%
ett finger gick ju bra, men inte hela armen\\*%
å dom hitta lite grann, så dom leta lite mer\\*%
å så fråga dom varför jag står här å ler\\*%
jag svarar:
\end{SongVerse}
\begin{SongVerse}
Ref...
\end{SongVerse}
\begin{SongVerse}
Ååh, ååh, ååh, ååh
\end{SongVerse}
\begin{SongVerse}
Ref... 
\end{SongVerse}
\end{SongText}
%\begin{SongText}[Flickan i Havanna]
%\begin{SongInfo}
%    Text:Evert Taube
%\end{SongInfo}
%\begin{SongVerse}
%Flickan i Havanna,\\*%
%hon har inga pengar kvar,\\*%
%sitter i ett fönster,\\*%
%vinkar åt en karl.\\*%
%Kom, du glade sjömatros!\\*%
%Du skall få min röda ros.\\*%
%Jag är vacker! Du är ung!\\*%
%Sjung, av hjärtat, sjung!
%\end{SongVerse}
%\begin{SongVerse}
%Flickan i Havanna,\\*%
%stänger dörr’n av cederträ.\\*%
%Sjömannen är inne,\\*%
%flickan på hans knä.\\*%
%Vill du bli mitt hjärtas kung?\\*%
%Har du pengar i din pung?\\*%
%Jag är vacker! Du är ung!\\*%
%Sjung, av hjärtat, sjung!
%\end{SongVerse}
%\begin{SongVerse}
%Flickan i Havanna,\\*%
%hörer då en sjömansröst:\\*%
%“Pengar har jag inga,\\*%
%men en sak till tröst.“\\*%
%Och ut ur sin jacka blå\\*%
%tager han det hon skall få.
%\end{SongVerse}
%\begin{SongVerse}
%Du är vacker, du är ung!\\*%
%Sjung, av hjärtat, sjung!
%\end{SongVerse}
%\begin{SongVerse}
%Flickan i Havanna,\\*%
%skådar då med tjusad blick\\*%
%ringen med rubiner\\*%
%som hon genast fick.\\*%
%Ringen kostar femton pund.\\*%
%Stanna du - en liten stund!\\*%
%Jag är vacker! Du är ung!\\*%
%Sjung, av hjärtat, sjung!
%\end{SongVerse}
%\begin{SongVerse}
%Flickan i Havanna,\\*%
%hon har inga pengar kvar,\\*%
%sitter i ett fönster,\\*%
%vinkar åt karl.\\*%
%Handen prydes av en ring\\*%
%och kring barmen crêpe de chin.\\*%
%Jag är vacker! Jag är ung!\\*%
%Sjung, av hjärtat, sjung!
%\end{SongVerse}
%\end{SongText}
%\begin{SongText}[Fritiof och Carmencita]
%\begin{SongInfo}
%    Text: Evert Taube (1890-1976)
%\end{SongInfo}
%\begin{SongVerse}
%Samborombon, en liten by förutan gata,\\*%
%den ligger inte långt från Rio de la Plata.\\*%
%Nästan i kanten av den blåa Atlanten\\*%
%och med Pampas bakom sig\\*%
%många hundra gröna mil.\\*%
%Dit kom jag ridande en afton i april,\\*%
%för jag ville dansa tango. 
%\end{SongVerse}
%\begin{SongVerse}
%Dragspel, fiol och mandolin,\\*%
%hördes från krogen och i salen steg jag in.\\*%
%Där på bänken i mantilj\\*%
%och med en ros invid sin barm,\\*%
%satt den bedårande lilla Carmencita.\\*%
%Mamman, värdinnan satt i vrån,\\*%
%hon tog mitt ridspö, min pistol och min 
%manton.\\*%
%Jag bjöd upp och Carmencita sa:\\*%
%- Si graçias señor,\\*%
%Vámos á bailár - este tango!
%\end{SongVerse}
%\begin{SongVerse}
%- Carmencita, lilla vän,\\*%
%håller du utav mig än?\\*%
%Får jag tala med din pappa och din mamma,\\*%
%jag vill gifta mig med dig, Carmencita!\\*%
%- Nej, Don Fritiof Andersson,\\*%
%kom ej till Samborombon, 
%\end{SongVerse}
%\begin{SongVerse}
%om ni hyser andra planer när det gäller mig,\\*%
%än att dansa tango!
%\end{SongVerse}
%\begin{SongVerse}
%- Ack, Carmencita, gör mig inte så besviken,\\*%
%jag tänkte skaffa mig ett jobb här i 
%butiken,\\*%
%sköta mig noga, bara spara och knoga,\\*%
%inte spela och dricka, men bara älska dig.\\*%
%Säg, Carmencita, det är ändå blott med mig,\\*%
%säg, som du vill dansa tango? 
%\end{SongVerse}
%\begin{SongVerse}
%Nej, Fritiof, Ni förstår musik,\\*%
%men jag tror inte Ni kan stå i en butik\\*%
%och förresten sa min pappa just idag,\\*%
%att han visste\\*%
%vem som snart skulle fria till hans dotter.\\*%
%En som har tjugotusen kor\\*%
%och en estancia som är förfärligt stor.\\*%
%Han har prisbelönta tjurar,\\*%
%han har oxar, får och svin,\\*%
%och han dansar underbar tango.
%\end{SongVerse}
%\begin{SongVerse}
%- Carmencita, lilla vän,\\*%
%akta dig för rika män!\\*%
%Lyckan den bor ej i oxar eller kor,\\*%
%och den kan heller inte köpas för pengar.\\*%
%Men min kärlek gör dig rik,\\*%
%skaffa mig ett ett jobb i er butik!\\*%
%Och när vi blir gifta söta ungar ska du få,\\*%
%som kan dansa tango. 
%\end{SongVerse}
%\end{SongText}
\begin{SongText}[Balladen om den kaxiga myran]
\begin{SongInfo}
    Text: Stefan Demert
\end{SongInfo}
\begin{SongVerse}
Jag uppstämma vill min lyra,\\*%
fast det blott är en gitarr,\\*%
och berätta om en myra,\\*%
som gick ut att leta barr.\\*%
Han gick ut i morgondiset,\\*%
sen han druckit sin choklad\\*%
och försvann i lingonriset\\*%
både mätt och nöjd och glad,\\*%
både mätt och nöjd och glad.
\end{SongVerse}
\begin{SongVerse}
Det var långan väg att vandra\\*%
det var lång till närmsta tall.\\*%
Han kom bort ifrån dom andra\\*%
men var glad i alla fall.\\*%
Femti meter ifrån stacken\\*%
just när solnedgången kom,\\*%
hitta' han ett barr på marken\\*%
som han tyckte mycket om,\\*%
som han tyckte mycket om.
\end{SongVerse}
\begin{SongVerse}
För att lyfta fick han stånka,\\*%
han fick spänna varje lem,\\*%
men så började han kånka\\*%
på det fina barret hem.\\*%
När han gått i fyra timmar\\*%
kom han till en ölbutelj,\\*%
han såg allting som i dimma\\*%
bröstet hävdes som en bälg,\\*%
bröstet hävdes som en bälg.
\end{SongVerse}
\begin{SongVerse}
Den låg kvar sen förra lördan.\\*%
- Jag skall släcka törsten min,\\*%
tänkte han och lade bördan\\*%
utanför och klättra' in.\\*%
Han drack upp den sista droppen\\*%
som fanns kvar i den butelj.\\*%
Och sedan slog han sig för kroppen\\*%
och skrek ut: - Jag är en älg!\\*%
och skrek ut: - Jag är en älg!
\end{SongVerse}
\begin{SongVerse}
- Ej ett barr jag drar till tjället,\\*%
nu så ska jag tamejfan\\*%
lämna skogen och i stället\\*%
vända upp och ner på stan.\\*%
Men han kom aldrig till staden,\\*%
något spärrade han stig,\\*%
en koloss där låg bland bladen\\*%
och vår myra hejdar sig,\\*%
och vår myra hejdar sig.
\end{SongVerse}
\begin{SongVerse}
Den var hiskelig att skåda,\\*%
den var stor och den var grå,\\*%
och vår myra skrek: - Anåda,\\*%
om du hindrar mig att gå!\\*%
Han for ilsken på kolossen\\*%
som låg utsträckt i hans väg.\\*%
Men vår myra kom ej loss sen,\\*%
han satt fast som i en deg,\\*%
han satt fast som i en deg. 
\end{SongVerse}
\begin{SongVerse}
Sorgligt slutar denna sången.\\*%
Myran stretade och drog,\\*%
men kolossen höll'en fången\\*%
tills han svalt ihjäl och dog.\\*%
Undvik alkoholens yra:\\*%
Du blir stursk, men kroppen loj,\\*%
och om Du är född till en myra\\*%
- brottas aldrig med ett TOY,\\*%
- brottas aldrig med ett TOY. 
\end{SongVerse}
\end{SongText}
%\begin{SongText}[Sjösala vals]
%\begin{SongInfo}
%    Text: Evert Taube
%\end{SongInfo}
%\begin{SongVerse}
%Rönnerdahl han skuttar med ett skratt ur sin säng\\*%
%Solen står på Orrberget. Sunnanvind brusar.\\*%
%Rönnerdahl han valsar över Sjösala äng.\\*%
%Hör min vackra visa, kom sjung min refräng!\\*%
%Tärnan har fått ungar och dyker i min vik,\\*%
%ur alla gröna dungar hörs finkarnas musik.\\*%
%Och se, så många blommor som redan slagit ut på ängen!\\*%
%Gullviva, mandelblom, kattfot och blå viol.
%\end{SongVerse}
%\begin{SongVerse}
%Rönnerdahl han virvlar sina lurviga ben\\*%
%under vita skjortan som viftar kring vaderna.\\*%
%Lycklig som en lärka uti majsolens sken,\\*%
%sjunger han för ekorrn, som gungar på gren!\\*%
%Kurre, kurre, kurre nu dansar Rönnedahl.\\*%
%Koko! Och göken ropar uti hans gröna dal.\\*%
%Och se, så många blommor som redan slagit ut på ängen!\\*%
%Gullviva, mandelblom, kattfot och blå viol. 
%\end{SongVerse}
%\begin{SongVerse}
%Rönnerdahl han binder utav blommor en krans,\\*%
%binder den kring håret, det gråa och rufsiga,\\*%
%valsar in i stugan och har lutan till hands,\\*%
%väcker frun och barnen med drill och kadans.\\*%
%Titta! ropar ungarna, Pappa är en brud,\\*%
%med blomsterkrans i håret och nattskjortan till skrud!\\*%
%Och se, så många blommor som redan slagit ut på ängen!\\*%
%Gullviva, mandelblom, kattfot och blå viol.
%\end{SongVerse}
%\begin{SongVerse}
%Rönnerdahl är gammal men han valsar ändå,\\*%
%Rönnerdahl har sorger och ont om sekiner.\\*%
%Sällan får han rasta - han får slita för två.\\*%
%Hur han klarar skivan, kan ingen förstå,\\*%
%ingen, utom tärnan i viken - hon som dök\\*%
%och ekorren och finken och vårens första gök.\\*%
%Och blommorna, de blommor som redan slagit ut på ängen,\\*%
%Gullviva, mandelblom, kattfot och blå viol. 
%\end{SongVerse}
%\end{SongText}
\begin{SongText}[Borås, Borås]
\begin{SongInfo}
    Text: Galenskaparna och Aftershave\\*%
    Mel: New york, New york
\end{SongInfo}
\begin{SongVerse}
Sprid nyheterna, jag kommer idag\\*%
för jag vill bli en del av dig\\*%
Borås, Borås.
\end{SongVerse}
\begin{SongVerse}
I postorderskor och Algotskostym\\*%
gör jag entré i stora stan\\*%
Borås, Borås.
\end{SongVerse}
\begin{SongVerse}
Ja, jag vill vakna upp i stan som aldrig sover\\*%
Där blir jag kung för en kväll\\*%
kvällen är min
\end{SongVerse}
\begin{SongVerse}
På Broadways estrad var jag nummer ett\\*%
men det är här som jag vill slå\\*%
Borås, Borås.
\end{SongVerse}
\begin{SongVerse}
För om jag lyckas här\\*%
finns inga mer besvär\\*%
då har jag allt\\*%
Borås, Borås.
\end{SongVerse}
\begin{SongVerse}
Borås, Borås. 
\end{SongVerse}
\begin{SongVerse}
Ja, jag vill vakna upp i stan som aldrig sover\\*%
Där blir jag kung för en kväll\\*%
kvällen är min\\*%
min är den kväll när jag gör vad jag vill.
\end{SongVerse}
\begin{SongVerse}
På Broadways estrad var jag nummer ett\\*%
men det är här jag vill slå\\*%
Borås, Borås.
\end{SongVerse}
\begin{SongVerse}
För om jag lyckas här\\*%
finns inga mer besvär\\*%
då har jag allt\\*%
Borås, Borås. 
\end{SongVerse}
\end{SongText}
%\begin{SongText}[Traktorhjulen]
%\begin{SongInfo}
%    Text: Smaklösa
%\end{SongInfo}
%\begin{SongVerse}
%Traktorhjulen går runt, runt, runt\\*%
%runt, runt, runt\\*%
%runt, runt, runt\\*%
%Traktorhjulen går runt, runt, runt\\*%
%runt omkring på åkern
%\end{SongVerse}
%\begin{SongVerse}
%Betupptagarn går runt, runt, runt\\*%
%runt, runt, runt\\*%
%runt, runt, runt\\*%
%Betupptagarn går runt, runt, runt\\*%
%och tar upp alla betor
%\end{SongVerse}
%\begin{SongVerse}
%Traktorhjulen går runt, runt, runt\\*%
%runt, runt, runt\\*%
%runt, runt, runt\\*%
%Traktorhjulen går runt, runt, runt\\*%
%till sockerbruket i Roma
%\end{SongVerse}
%\begin{SongVerse}
%Betarna går runt, runt, runt\\*%
%runt, runt, runt\\*%
%runt, runt, runt\\*%
%Betorna går runt, runt, runt\\*%
%och blir till sockerbitar
%\end{SongVerse}
%\begin{SongVerse}
%En del tar en bit, en del tar två\\*%
%en del tar tre bitar till kaffet\\*%
%En del tar en bit, en del tar två\\*%
%en del tar tre bitar till kaffet
%\end{SongVerse}
%\begin{SongVerse}
%men var kommer alla bitar ifrån, bitar ifrån, bitar ifrån\\*%
%var kommer alla bitar ifrån\\*%
%från sockerbruket i Roma
%\end{SongVerse}
%\begin{SongVerse}
%men om traktorhjulen skull sönder, skull sönder, skull sönder\\*%
%om traktoren skull sönder, då skulle inte traktorhjulen gå runt
%\end{SongVerse}
%\begin{SongVerse}
%men för det mesta går traktorhjulen gå runt, runt, runt\\*%
%runt, runt, runt\\*%
%runt, runt, runt\\*%
%Traktorhjulen går runt, runt, runt\\*%
%över hela vårt avlånga land
%\end{SongVerse}
%\begin{SongVerse}
%Ända från ystad till haparanda\\*%
%där går traktorhjulen fram och tillbaka och boljar över åkern\\*%
%precis som bonden har följt årstiderna växlingar genom årtusen\\*%
%där går traktorhjulen runt, runt, runt\\*%
%runt omkring, runt, runt, runt\\*%
%där går traktorhjulen runt, runt, runt, runt, runt, runt, runt
%\end{SongVerse}
%\end{SongText}
\begin{SongText}[Gällivarevisan]
\begin{SongInfo}
    Text: Helmer Andersson\\*%
    Mel:  Stadsbudsvisan\\*%
    LTU's Ultima Thule anger att visan ska sjungas "med kraftig finsk brytning" så texten reflekterar därefter
\end{SongInfo}
\begin{SongVerse}
Dänkte på lörda skulle fara\\*%
in te Gällivara på Karakatorg\\*%
dom pjuda trikka prännvinsflaska\\*%
gott som satans paska, voj voj.\\*%
De vara rolika liven,\\*%
para lagsmål å niven,\\*%
åka pålisstationen,\\*%
vara jävlika fasonen.\\*%
Ligga inne halva natten,\\*%
leva limpa å vatten,\\*%
komma ut morronröken,\\*%
säja knappast sen ajöken.
\end{SongVerse}
\begin{SongVerse}
Men dänkte perkele anamma,\\*%
kanse vara samma åka Visskafoss,\\*%
där sänner ja en gammal likka\\*%
som jag prukar likka, förståss.\\*%
De prukar pli ganska sällan,\\*%
fara hälsa på fjällan,\\*%
kanse få någe mellan,\\*%
bara inte fassna fällan.\\*%
De vara klart man riskera,\\*%
ingenting reflektera,\\*%
pliva kanske nå’t mera,\\*%
måste gå å operera.
\end{SongVerse}
\begin{SongVerse}
Men nu jak luta erotiken,\\*%
öppna pritfabriken, sälja akvavit.\\*%
Då komma pålismästarn nära,\\*%
fråka om jak pära priten.\\*%
De vara jäklika token,\\*%
pliva hånkad av snoken,\\*%
åka Luleå-kroken,\\*%
längta baka uti skoken.\\*%
Sitta inne halva åren,\\*%
bliva krå uti håren,\\*%
knappast röra på låren,\\*%
komma bakas först på våren.\\*%
Liikavaara-Frasse, hembrännare
\end{SongVerse}
\end{SongText}
%\begin{SongText}[Öppna landskap]
%\begin{SongInfo}
%    Text: Ulf Lundell
%\end{SongInfo}
%\begin{SongVerse}
%Jag trivs bäst i öppna landskap\\*%
%Nära havet vill jag bo\\*%
%Några månader om året\\*%
%Så att själen kan få ro.\\*%
%Jag trivs bäst i öppna landskap\\*%
%Där vindarna får fart.\\*%
%Där lärkorna står högt i skyn\\*%
%Och sjunger underbart.\\*%
%Där bränner jag mitt brännvin själv\\*%
%Och kryddar med Johannesört\\*%
%Och dricker det med välbehag\\*%
%Till sill och hembakt vört.\\*%
%Jag trivs bäst i öppna landskap\\*%
%Nära havet vill jag bo. 
%\end{SongVerse}
%\begin{SongVerse}
%Jag trivs bäst i fred och frihet\\*%
%För både kropp och själ.\\*%
%Ingen kommer i min närhet\\*%
%Som stänger in och stjäl.\\*%
%Jag trivs bäst när dagen bräcker\\*%
%När fälten fylls av ljus.\\*%
%När tuppar gal på avstånd\\*%
%Och det är långt till närmaste hus.\\*%
%Men ändå så pass när\\*%
%Att en tyst och stilla natt\\*%
%När man sitter under stjärnorna\\*%
%Kan höra fest och skratt.\\*%
%Jag trivs bäst i fred och frihet\\*%
%För både kropp och själ. 
%\end{SongVerse}
%\begin{SongVerse}
%Jag trivs bäst när havet svallar\\*%
%Och måsarna ger skri.\\*%
%När stranden fylls av snäckskal\\*%
%Med havsmusik uti.\\*%
%När det klara och enkla\\*%
%Får råda som det vill.\\*%
%När ja är ja och nej är nej\\*%
%Och tvivlet tiger still.\\*%
%Då binder jag en krans av löv\\*%
%Och lägger den vid närmsta sten\\*%
%Där runor ristas för vår skull\\*%
%Nån gång för länge sen.\\*%
%När stranden fylls av snäckskal\\*%
%Med havsmusik uti.\\*%
%Jag trivs bäst i öppna landskap\\*%
%Nära havet vill jag bo. 
%\end{SongVerse}
%\end{SongText}
\newpage