\fancypagestyle{Hyfs och Etikett}{
\fancyhead{} % clear all header fields
\fancyhead[LE,RO]{\textbf{Hyfs och Etikett}}
}
\pagestyle{Hyfs och Etikett}

\invisiblesection{\textbf{Hyfs och Etikett}}

\textbf{Man presenterar när följande inträffar:}

\begin{itemize}
    \item en herre för en dam (en herre reser sig då alltid)
    \item en yngre person för en äldre (den yngre skall då stå upp)
    \item en person för ett sällskap (den presenterade skall stå upp)
\end{itemize}

Presentationen skall verkställas då ett par personer kunna förväntas att med varande inleda en konversation. Vid början av en bjudning presenteras gäster för varandra. Under en stor bjudning presenteras gäster för varandra eller så kan en allmän presentation förnyas genom speciell presentation.

\invisiblesubsection{\textbf{Skålande}}

Regler för skålande tillämpas av tradition endast vid drickande av vin. Tal hålls inte till brännvin. Orsaken därtill är kanske att söka i den svenska brännvinsglasets konstruktion. Detta föreställer en tratt och var även till sitt ursprung en sådan. För att hålla drycken kvar, måste man täppa pipen med ett finger. På detta sättet kunde man inte i oändlighet hålla brännvinet i beredskap utan tvingades att ganska hastigt svälja super i ett enda svep för att få fingret och handen fria. Våra brännvinsvisor äro således epigrammässigt korttade. Däremot är det icke ovanligt att tal stundom hållas efter en middag vid likör eller grogg. Men de får i så fall icke dragas ut till tröttsamma orationer.

En skål är ett tyst tal för den som tilldrickes.

En skål får icke föreslås, innan det första talet hållits. Detta tal är vanligen välkomsttalet.

Det åligger varje presidium att dricka med envar av sina honörer och att göra detta i varje omgång vin. Under presidium är värdens ålligande att dricka med varje gäst, personliga eller genom sin ställföreträdare.

Varje herre skall ägna första skålen åt sin bordsdam; undantag utgör blott förbudet att dricka med värdinnan.

Dam har icke skyldighet att proponera en skål men kan likväl, om hon så önskar dricka någon till. Värdinnan skall dricka med gästerna.

Underordnad placering bör icke dricka med placerad person, innan denne druckit med den lägre placerade. Den som fått emottaga en skål är skyldig att besvara den.

Man får icke dricka med presidiehavare, alltså icke med värd eller värdinna och vid den stora taffeln, icke lämpligen med den som presiderar vid det särskilda bordet. Däremot skola dessa personer dricka med gästerna och med varandra på det sätt, att högre presidiehavare skåla med lägre. Sålunda skall värden dricka med värdinnan.

Endast värdfolk har förslagsrätt till kollektiv skål av typen: "Allgemeines","Hyfsa glasen", "Langs hele kysten", "Overall" osv.

En gammal svensk regel, vilken börjat försvinna, är att man icke får dricka ensam, enär detta skulle innebära att man föreslog sig själv en skål.

När det är dags för själva skålandet höjes glaset, med öppen hand (dvs insidan av handen mot den tilldruckne), ungefär till tredje knappen uppifrån på skjortan för herrar och mellan brösten och hakan för damerna. Samtidigt som detta utföres mötes de skålandes blickar.

Detta med att man tittar varandra i ögonen kommer från vikingatiden, då det inte var lämpligt att släppa sin motståndare med blicken, då risken var stor at man fick sota för sin oförsiktighet med ett svärt mot bröstet. Att sträcka sig över eller framför en brodsgranne är i bordsskicket en dösdsynd.

\invisiblesubsection{\textbf{Tal och samtal}}

Varje tillfälle har sina speciella tal. Gemensamt för alla tillfällen är att talen skola behandla:
\begin{enumerate}
    \item Första talet: Välkomst och antydan om orsaken till att man samlats.
    \item De följande talen: Sammankomstens ändamål t.ex. hyllningar osv.
    \item Sluttalet: Ett tack.
\end{enumerate}

Det första talet hålles regelmässigt av den som kallat till samling, värden eller värdinnan.

De följande talen hålles av dem som känna sig självmant drivna härtill. Det stora cirkusnumret i en bankett med damer är \textit{"talet till kvinnan"}. Det bör överlåtas åt någon skön olympier med vältalighet, om sådan finnes tillgänglig, men kan också med fördel anförtros åt en äldre kavaljer av den oförbränneliga sorten, med rika fonder av insinuant malis och pålitlig takt. Varje minsta fläkt av frivolitet verkar obehagligt, även om åhörarna inte visar det.

Sluttalet är nästan undantagslöst att man, när måltiden är till ända tackar för maten. Denna skyldighet åligger regelbundet den herre som har sin plats vid värdinnans vänstra sida eller den person som vilken festligheten anordnats.
\filbreak
\invisiblesubsection{Bal}
Bal betecknar en högtidlig tillställning av uteslutande dansant natur, där förfriskningar komma i andra hand. Vad beträffar bruk och vanor vid våra dagars baltillställningar gälla med utomordentlig stränghet alla allmänna etikettsregler. Balen kan nästan sägas vara en uppvisning av belevenhet. I densamma ingår alla sällskapslivets huvudmoment såsom salongkonversation, uppträdande, måltid osv.

\textit{"Det är ansett såsom dåligt att snyta sig i bordsduken..."} - \textbf{Ur en bok om etikett från 1400-talet}
\filbreak
\invisiblesubsection{Sittning}

En sittning, sits eller sitz som de även kan kallas är en tillställning av middags typ. I överblick finns det tre olika sorter av dessa som har både skillnader och likheter. Överskridande så kommer mat, Toastmasters(Värd/värdinna) och dryck finnas. Sånger och Spex i olika slag är att förvänta.

Innan en sittning har påbörjats så anses det vara ohövligt att sitta ner vid bordet. Innan detta brukar Sveriges nationalsång sjungas sen inledande tal av Toastmasters där sittningsregler gås igenom. 

Ofta bes gästerna sätta sig ner gruppvis från mest tid studerande till minst tid studerande men detta kan variera.

Glöm inte att sjunga En liten blå Förgätmigej som ett tack till personalen efter middagen.

\invisiblesubsubsection{Finsittning}

En finsittning är som namnet bedrar en finare tillställning. Kavajkostym är minst förväntat men klädkåd bör alltid medföras med inbjudan. 

Mat blir serverad och dryck finns på plats. 

Dessa sittningar äro i oftast fall för jubeleum, högtider och andra dyliga händelser.

\invisiblesubsubsection{Standardsittning}

Standardsittning eller helt enkelt sittning är den vanligaste av sorten. Mat förväntas serveras, dryck finnes på plats men klädkåden är slappare. Overall är ett lämpligt plagg, antigen uppcabbad eller nercabbad med något prydligare plagg upptill

\invisiblesubsubsection{Fulsittning}

En fulsittning kallas så inte för att klädkåden måste följa där efter men få förväntingar finns. Huvudsakligen är det att mat och dryck medtages själv och ingen större koordination än önskat umgänge är vanligt.



\newpage