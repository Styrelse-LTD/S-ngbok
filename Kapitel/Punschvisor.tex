\fakesection{Punschvisor}
\fancypagestyle{Punschvisor}{
    \fancyhead{} % clear all header fields
    \fancyhead[LE,RO]{\textbf{Punschvisor}}
}
\pagestyle{Punschvisor}
%\begin{SongText}[Änglapunschen]
%    \begin{SongInfo}
%        Mel: Änglamark
%    \end{SongInfo}
%    \begin{SongVerse}
%        Kalla den gudagåva eller himlanektar, vad du vill.\\*%
%        Punschen den gyllne, de gamle oss skänkte.\\*%
%        Vet att så länge som punschen nånsin funnits till\\*%
%        glädjen den höjde och sorgerna dränkte.\\*%
%        Blunda och dröm om en blommande sommarnatt\\*%
%        svala bersåer där punschen står immig.\\*%
%        Eller en höstdag när Nordan har lekt tafatt\\*%
%        varm punsch som ångar och ärtsoppa simmig.
%    \end{SongVerse}
%    \begin{SongVerse}
%        Punschen den älskas ju av alla och envar.\\*%
%        Låt festen börja, låt punschen flöda!\\*%
%        Skål alla vänner som har nåt' i glasen kvar,\\*%
%        hedra nu minnet av gamle kung Oscars da'r!\\*%
%        Kalla den gudagåva eller himlanektar om du vill.\\*%
%        Punschen den gyllene, som får oss att drömma.\\*%
%        Fukta din strupe, låt inte flaskan stå still,\\*%
%        skåla för punschen och glasen vi tömma.
%    \end{SongVerse}\end{SongText}
\begin{SongText}[Giv oss vår punsch]
    \begin{SongInfo}
        Mel: God save the Queen
    \end{SongInfo}
    \begin{SongVerse}
        Ge oss vår punsch idag\\*%
        den som är sval och bra\\*%
        eller väl värmd.
    \end{SongVerse}
    \begin{SongVerse}
        Glasen vi tömma här\\*%
        det blir en glad affär\\*%
        för vi ska ha mera punsch.
    \end{SongVerse}
    \begin{SongVerse}
        Vi är här sen lunch.
    \end{SongVerse}\end{SongText}
\begin{SongText}[Ärtor och punsch]
    \begin{SongInfo}
        Mel: Fritiof och Carmencita
    \end{SongInfo}
    \begin{SongVerse}
        Ärtor och punsch, en liten rätt med traditioner\\*%
        den smakar bra och väcker många sensationer,\\*%
        blekgul till färgen, smaker går intill märgen,\\*%
        med en senap som blandas enligt gammal tradition\\*%
        så ska den njutas denna svenska folkpassion,\\*%
        en torsdagskväll i varje månad.
    \end{SongVerse}
    \begin{SongVerse}
        Skål nu vänner uti denna mäss,\\*%
        ärter och punsch ska vi njuta utan stress,\\*%
        inga sura miner vill vi se i afton,\\*%
        när doften utav ärter och punsch sprids i salongen\\*%
        Tänk på Grönstedt, Cederlund och Flagg,\\*%
        åh - vilken doft från denna arraksgula tagg,\\*%
        hela natten ska vi njuta denna underbara brygd,\\*%
        skål för lilla ärten och punschen.
    \end{SongVerse}\end{SongText}
\begin{SongText}[Vädjan till punschen]
    \begin{SongInfo}
        Mel: Sov du lilla videung
    \end{SongInfo}
    \begin{SongVerse}
        Kom nu lilla punschen min.\\*%
        följ nu efter supen.\\*%
        Snart skall du åka in\\*%
        ner igenom strupen,\\*%
        till mitt stora magpalats,\\*%
        där det finns så mycket plats.\\*%
        Kom nu lilla punschen.\\*%
        Följ nu efter supen.
    \end{SongVerse}\end{SongText}
%\begin{SongText}[Krya punschvisan]
%    \begin{SongInfo}
%        Mel: Putti putti
%    \end{SongInfo}
%    \begin{SongVerse}
%        Putti putti kanna full med punsch\\*%
%        kanna full med punsch\\*%
%        avslutar denna baluns.\\*%
%        För till middag och till lunch\\*%
%        måste man ha punsch.\\*%
%        Vill man bli punschig, måste man bums ha punsch\\*%
%        Vi vill ha PUUUNSCH
%    \end{SongVerse}\end{SongText}
\begin{SongText}[Min älskling]
    \begin{SongInfo}
        Mel: Min älskling du är som en ros
    \end{SongInfo}
    \begin{SongVerse}
        Min älskling du är som en punsch,\\*%
        en nyupphälld och kall.\\*%
        Ack, som en Cederlunds så ljuv,\\*%
        min älskling, jag är knall.
    \end{SongVerse}
    \begin{SongVerse}
        Så underbar blir du av punschen,\\*%
        och ser så vacker ut.\\*%
        Att älska dig det skall jag än\\*%
        när punschen tagit slut.
    \end{SongVerse}
    \begin{SongVerse}
        När hela flaskan på står där tom\\*%
        och torkan i min gom\\*%
        härjar så brännande och from,\\*%
        då fäller jag min dom.
    \end{SongVerse}
    \begin{SongVerse}
        Min älskling du är som en punsch,\\*%
        en flaska Cederlundsch.\\*%
        Ack, jag vill bara älska dig,\\*%
        min älskling och min punsch!
    \end{SongVerse}\end{SongText}
\begin{SongText}[Punsch-Lunch Visa]
    \begin{SongInfo}
        Mel: Jag vill ha blommig falukorv
    \end{SongInfo}
    \begin{SongVerse}
        Jag vill ha kall och simmig punsch till lunch, mamma.\\*%
        N'åt annat vill jag inte ha.\\*%
        Öl får man för ofta, vin smakar som kofta,\\*%
        starksprit det är riktigt läbbigt.\\*%
        Nej, jag vill ha kall och simmig punsch\\*%
        till lunch, mamma\\*%
        ett stort glas härlig punsch till lunch
    \end{SongVerse}\end{SongText}
\begin{SongText}[Punschpolkett]
    \begin{SongInfo}
        Mel: Nudistpolkav
    \end{SongInfo}
    \begin{SongVerse}
        Inte dricker jag sprit, nej det passar ej hit,\\*%
        till min kaffetår ikväll,\\*%
        men av punschen blir jag snäll,\\*%
        ja se punschen går idag som igår.\\*%
        Går till ärtor och glass,\\*%
        är en dryck utav klass,\\*%
        smakar mjuk men ändå vass,\\*%
        å vad det är underbart\\*%
        att få tömma punschpokalen.
    \end{SongVerse}
    \begin{SongVerse}
        Hoppsansa, se upp i svängarna,\\*%
        hoppsansa, bland gobelängarna,\\*%
        punschen tillhör överdängarna,\\*%
        dricka punsch det gör ditt sinne rent.
    \end{SongVerse}
    \begin{SongVerse}
        Ria faderia faderallan la,\\*%
        hej hå hoppsansa!
    \end{SongVerse}\end{SongText}
\begin{SongText}[Födelsedagspunsch]
    \begin{SongInfo}
        Mel: Med en enkel tulipan
    \end{SongInfo}
    \begin{SongVerse}
        Nu med en ny och stadig krök\\*%
        med armen gör vi försök\\*%
        att lyfta koppen, att lyfta koppen\\*%
        som står och väntar.
    \end{SongVerse}
    \begin{SongVerse}
        Håll blicken fäst vid koppens rand\\*%
        och darra inte på hand.\\*%
        Nu allesammans, nu allesammans\\*%
        på munnen gläntar.
    \end{SongVerse}
    \begin{SongVerse}
        En liten punschtår så här placerad\\*%
        i ena handen sig bättre gör,\\*%
        än tio liter uppå Systemet\\*%
        och inga pengar att köpa för.\\*%
        Spill inga droppar på ditt bord\\*%
        och spill ej mer några ord.\\*%
        Nu tar vi punschen, nu tar vi punschen\\*%
        som står och väntar.
    \end{SongVerse}\end{SongText}
\begin{SongText}[Studiemedelsrundo]
    \begin{SongInfo}
        Mel: Ösa sand
    \end{SongInfo}
    \begin{SongVerse}
        Vi dricker punsch till lunch,\\*%
        när vi har fått avin.\\*%
        Vi lunchar hela dagen\\*%
        tills kassan går i sin.
    \end{SongVerse}
    \begin{SongVerse}
        För mycket punsch till lunch,\\*%
        nu mår jag som ett svin.\\*%
        Helkass hela dagen\\*%
        och munnen en latrin.
    \end{SongVerse}
    \begin{SongVerse}
        Vi dricker punsch till lunch.\\*%
        Det är en medicin\\*%
        som hjälper mot det mesta\\*%
        och ger oss glad min.
    \end{SongVerse}
    \begin{SongVerse}
        Vi dricker punsch ger stuns\\*%
        åt tentamensamnestin.\\*%
        Vi dricker för att trösta\\*%
        och fira som små svin.
    \end{SongVerse}\end{SongText}
\begin{SongText}[Punsch punsch]
    \begin{SongInfo}
        Mel: Ritch ratch
    \end{SongInfo}
    \begin{SongVerse}
        Punsch, punsch fillibom bom bom\\*%
        fillibom bom bom fillibom bom bom.\\*%
        Punsch, punsch fillibom bom bom\\*%
        fillibom bom bom fillibom.\\*%
        Vi har ju både Cederlunds och\\*%
        Carlshamns flagg, Grönstedts blå\\*%
        och lilla Caloric.\\*%
        $\|$ Det duger ej med sodavatten,\\*%
        sodavatten, sodavatten.\\*%
        Det duger ej med sodavatten,\\*%
        ge mig lite punsch! $\|$
    \end{SongVerse}\end{SongText}
%\begin{SongText}[Tvekan inför punschen]
%    \begin{SongInfo}
%        Mel: Rosa på bal
%    \end{SongInfo}
%    \begin{SongVerse}
%        Jag borde nog inte dricka mer\\*%
%        varken öl eller vin eller brännevin.\\*%
%        En kaffetår blott, men såvitt jag ser\\*%
%        står det punsch brevi’n.
%    \end{SongVerse}
%    \begin{SongVerse}
%        Tänk vilken underbar färg och odör\\*%
%        -nej, det blir bara man bråkar och stör.\\*%
%        Men för att inte min kväll skall bli trist,\\*%
%        krävs både slughet och list!
%    \end{SongVerse}
%    \begin{SongVerse}
%        Så jag tror nog att jag tar en\\*%
%        punsch, kanske två, kanske tre!\\*%
%        Se’n blir det groggar i baren,\\*%
%        jag klarar säkert av det!
%    \end{SongVerse}
%    \begin{SongVerse}
%        Trots att jag egentligen är rätt knall,\\*%
%        piggar nog punsch upp mig i alla fall.\\*%
%        Jag kvicknar till med en punsch i min bål.\\*%
%        Nu har jag bestämt mig, SKÅL!
%    \end{SongVerse}\end{SongText}
\begin{SongText}[Djungelpunsch]
    \begin{SongInfo}
        Mel: Var nöjd med allt som livet ger
    \end{SongInfo}
    \begin{SongVerse}
        Jag gillar alla tiders punsch.\\*%
        Punsch till frukost, punsch till lunch,\\*%
        en punsch till förrätt, varmrätt och dessert.\\*%
        Jag gillar punsch för vet du vad,\\*%
        rent kaffe gör ju ingen glad.\\*%
        Så punsch för fulla muggar vill jag ha.
    \end{SongVerse}
    \begin{SongVerse}
        Med konjak du lockar.\\*%
        Den bästa Renault.\\*%
        Förlåt om jag chockar\\*%
        och tar punsch ändå.\\*%
        Och bjuder du på fin likör\\*%
        får du ursäkta om det stör.\\*%
        Jag väljer hellre Grönstedts Blå,\\*%
        en Cederlunds eller Flaggpunsch å\\*%
        - kanske har du ren Platin?
    \end{SongVerse}
    \begin{SongVerse}
        Jag gillar punsch.\\*%
        Ger du mig punsch så är jag din.\\*%
        För evigt din.
    \end{SongVerse}\end{SongText}
\begin{SongText}[Glad såsom få]
    \begin{SongInfo}
        Mel: Vårsång (glad såsom fågeln)
    \end{SongInfo}
    \begin{SongVerse}
        Glad såsom få efter glassdesserten\\*%
        väntar vi tåligt på kaffe och punsch.\\*%
        Längtan nu dallrar i rumsatmosfären,\\*%
        frostog är flaskan vi lagt in vid lunch.
    \end{SongVerse}
    \begin{SongVerse}
        Stämningen högt upp vid takåsen dallrar,\\*%
        samtalet flyter likt vårsolen fram.\\*%
        Vitsarna haglar och skratten de skallrar,\\*%
        Här har vi glädjen med stoj och med glam.
    \end{SongVerse}
    \begin{SongVerse}
        Se, hur den simmiga punschen vi slå,\\*%
        droppande gå, i glasen små.\\*%
        Sös såsom honung och mild i smaken,\\*%
        guldgul är färgen och kylan bra den.\\*%
        Den för oss i trance, den ger ögonen glans,\\*%
        vi behöver den lite mans!
    \end{SongVerse}\end{SongText}
\begin{SongText}[Punschens lov]
    \begin{SongInfo}
        Mel: Rövarvisan (Camomilla stad)
    \end{SongInfo}
    \begin{SongVerse}
        Ja, punschen är och och punschen var\\*%
        och punschen skall förbliva.\\*%
        En lidelse vi alla har\\*%
        som inget kan fördriva.\\*%
        Ja punschen tinar opp, såväl\\*%
        som svalkar både kropp och själ.\\*%
        Den botar begären och lindrar besvären.\\*%
        Ja, punschen den gör både gott och väl!
    \end{SongVerse}\end{SongText}
%\begin{SongText}[Valfri punschvisa]
%    \begin{SongInfo}
%        Text: ur Chalmersspexet Gutenberg, 1991\\*%
%        Mel: Fredmans epistel nr2: Nå skruva fiolen
%    \end{SongInfo}
%    \begin{SongVerse}
%        Men vad är väl detta?\\*%
%        Har Svakar nu fått fnatt?\\*%
%        Har han gått och tatt\\*%
%        upp ut ur sin skatt?\\*%
%        En okorkad kall punschflaska.
%    \end{SongVerse}
%    \begin{SongVerse}
%        När man dricker punsch,\\*%
%        då ska man ta sig ton.\\*%
%        Full som en kanon, det är tradition\\*%
%        Och sångerna ska rimma
%    \end{SongVerse}
%    \begin{SongVerse}
%        Av minnen i vår sal\\*%
%        Om glasen det imma\\*%
%        Gul, ljus och sval\\*%
%        Punschen stimma\\*%
%        Små droppar i kroppar\\*%
%        Frostnupen ner i djupen\\*%
%        Möts mitt i strupen av en visa lätt banal.
%    \end{SongVerse}\end{SongText}
%\begin{SongText}[Robespierres punschvisa]
%    \begin{SongInfo}
%        Text: Henrik Dahlgren \& Mikael Josefsson (Ur Handelsspexet Robespierre 1995)\\*%
%        Mel: Hyllning till Sverige
%    \end{SongInfo}
%    \begin{SongVerse}
%        Många söker tillfredsställelse,\\*%
%        eller någonting som skänker ro.\\*%
%        När jag själv såg min förälskelse,\\*%
%        stod det klart, den var ljus å go’.
%    \end{SongVerse}
%    \begin{SongVerse}
%        Redan som ett litet knyte\\*%
%        fick jag smaka min första pralin.\\*%
%        Ett facilt och läckert byte -\\*%
%        jag blev fast för en söt blondin.
%    \end{SongVerse}
%    \begin{SongVerse}
%        Min första kärlek hette Cederlund;\\*%
%        så bedårande och kall o rik.\\*%
%        Jag minns vår första herdestund -\\*%
%        din gestalt var som ljuv musik.
%    \end{SongVerse}
%    \begin{SongVerse}
%        Ack, så underbart att få trycka\\*%
%        mina läppar mot din hals.\\*%
%        En sådan obeskrivbar lycka.\\*%
%        Vi försvann i eldig vals.
%    \end{SongVerse}
%    \begin{SongVerse}
%        Det blev fler och fler romanser\\*%
%        för varje år som flöt förbi.\\*%
%        Jag är fylld av de essenser\\*%
%        som giver styrka och energi.
%    \end{SongVerse}\end{SongText}
\begin{SongText}[Marco Polos punschvisa]
    \begin{SongInfo}
        Text: Jonas Lundberg (Ur Handelsspexet Marco Polo 1986)\\*%
        Mel: Fredmans epistel nr 2 (Nå, skrufva fiolen)
    \end{SongInfo}
    \begin{SongVerse}
        Ge mig buteljen,\\*%
        hej Marco, skynda dig.\\*%
        Gula droppar rinn\\*%
        ner i strupen min,\\*%
        ger tron på liv tillbaka.
    \end{SongVerse}
    \begin{SongVerse}
        Javisst, käre broder,\\*%
        här får du blott en klunk.\\*%
        Vi får spara nå’t\\*%
        fram till nästa låt.\\*%
        Vi spelar flera kvällar.
    \end{SongVerse}
    \begin{SongVerse}
        Du skojar med mig.\\*%
        I Carlshamns vill jag bada.\\*%
        Men det får du ej,\\*%
        då tar spriten skada.
    \end{SongVerse}
    \begin{SongVerse}
        Zum zum zum...
    \end{SongVerse}
    \begin{SongVerse}
        Ju mera jag dricker,\\*%
        desto nyktrare blir jag.\\*%
        Jag kan dricka en\\*%
        hela sjuttiofem.\\*%
        Jag faller ej i dvala.
    \end{SongVerse}
    \begin{SongVerse}
        Det där var ju konstigt,\\*%
        det tror jag inte på.\\*%
        Så du säger att,\\*%
        att du ej får fnatt.\\*%
        Så kan det inte vara.
    \end{SongVerse}
    \begin{SongVerse}
        Jag har en idé.\\*%
        Jag tror jag kan förklara.\\*%
        Jag undrar vad det é,\\*%
        men han skämtar nog bara
    \end{SongVerse}
    \begin{SongVerse}
        Zum zum zum... -låt höra!
    \end{SongVerse}
    \begin{SongVerse}
        För oss snillen är\\*%
        punschpromillen\\*%
        ej så stark.
    \end{SongVerse}\end{SongText}
%\begin{SongText}[Punschen kommer (kall)]
%    \begin{SongInfo}
%        Mel: Änkevalsen ur Glada Änkan
%    \end{SongInfo}
%    \begin{SongVerse}
%        punschen kommer,\\*%
%        ljuv och sval.\\*%
%        Glasen imma,\\*%
%        röster stimma\\*%
%        i vår sal.\\*%
%        Skål för glada minnen!\\*%
%        Skål för varje vår!\\*%
%        Inga sorger finnes mer\\*%
%        när punsch vi får.
%    \end{SongVerse}\end{SongText}
%\begin{SongText}[Punschen kommer (varm)]
%    \begin{SongInfo}
%        Mel: Änkevalsen ur Glada Änkan
%    \end{SongInfo}
%    \begin{SongVerse}
%        Punschen kommer,\\*%
%        punschen kommer,\\*%
%        god och varm.\\*%
%        Vettet svinner,\\*%
%        droppen rinner,\\*%
%        ner i tarm.\\*%
%        Skål för glada minnen!\\*%
%        Skål för varje vår!\\*%
%        Inga sorger finnes mer\\*%
%        när punsch vi får.
%    \end{SongVerse}\end{SongText}
\begin{SongText}[Punschen rinner genom strupen]
    \begin{SongInfo}
        Mel: Midnatt råder
    \end{SongInfo}
    \begin{SongVerse}
        Punschen, punschen rinner ner i strupen,\\*%
        ner i djupen.\\*%
        Blandas, konfronteras där med supen,\\*%
        där med supen.\\*%
        Gula droppar stärker våra kroppar,\\*%
        PUNSCH, PUNSCH, PUNSCH!
    \end{SongVerse}\end{SongText}
\begin{SongText}[Nu har vi punch]
    \begin{SongInfo}
        Mel: Nu har vi ljus här i vårt hus
    \end{SongInfo}
    \begin{SongVerse}
        Nu har vi punsch här på vårt bord,\\*%
        punschen är kommen, räddad är vår kväll.\\*%
        Kaffet är varmt, punschen är kall,\\*%
        sjung i vår trall:\\*%
        Punschen rinner ner i lilla magen,\\*%
        skapar där de kända välbehagen,\\*%
        som vi gillar, som vi gillar,\\*%
        ja, som vi gillar till tusen.
    \end{SongVerse}\end{SongText}
\begin{SongText}[Nya punschvisan]
    \begin{SongInfo}
        Text: Ur Chalmersspexet Sven Duva 1965\\*%
        Mel: Mjölnarens dotter (Svensk folkvisa)
    \end{SongInfo}
    \begin{SongVerse}
        Vad ger dig din sisu, vad håller dig varm?\\*%
        Skrororompompej!\\*%
        Jo, punschen som upptas i mage och tarm.\\*%
        Försvinnande god!\\*%
        Försvinnande god!\\*%
        Får upp humöret och promillen, hej!
    \end{SongVerse}
    \begin{SongVerse}
        Du blir som en ny män’ska om du tar en punsch.\\*%
        Skrororompompej!\\*%
        Och den nya män’skan vill också ha en punsch.\\*%
        Försvinnande god!\\*%
        Försvinnande god!\\*%
        Får upp humöret och promillen, hej!
    \end{SongVerse}\end{SongText}
%\begin{SongText}[Jag gillar punschen]
%    \begin{SongInfo}
%        Mel: Te Deum Laudamus
%    \end{SongInfo}
%    \begin{SongVerse}
%        Det var en gång jag tänkte\\*%
%        att punschen övergiva,\\*%
%        men det får aldrig ske\\*%
%        så länge jag får leva\\*%
%        och när jag en gång dör\\*%
%        skall ristas på min grav;\\*%
%        Här vilar en som svenska punschen\\*%
%        gillat har:
%    \end{SongVerse}
%    \begin{SongVerse}
%        Jag gillar, jag gillar punschen,\\*%
%        jag gillar den som punschen skapat har.\\*%
%        Jag gillar, jag gillar punschen,\\*%
%        jag gillar punschen och dess far.
%    \end{SongVerse}
%\end{SongText}
\newpage