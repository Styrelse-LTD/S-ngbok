
\fakesection{Punschvisor}

\fancypagestyle{Punschvisor}{
\fancyhead{} % clear all header fields
\fancyhead[LE,RO]{\textbf{Punschvisor}}
}
\pagestyle{Punschvisor}



\subsection{\textbf{Änglapunschen}}

Mel: Änglamark\bigskip


Kalla den gudagåva eller himlanektar, vad du vill.

Punschen den gyllne, de gamle oss skänkte.

Vet att så länge som punschen nånsin funnits till

glädjen den höjde och sorgerna dränkte.

Blunda och dröm om en blommande sommarnatt

svala bersåer där punschen står immig.

Eller en höstdag när Nordan har lekt tafatt

varm punsch som ångar och ärtsoppa simmig.\bigskip

Punschen den älskas ju av alla och envar.

Låt festen börja, låt punschen flöda!

Skål alla vänner som har nåt' i glasen kvar,

hedra nu minnet av gamle kung Oscars da'r!

Kalla den gudagåva eller himlanektar om du vill.

Punschen den gyllene, som får oss att drömma.

Fukta din strupe, låt inte flaskan stå still,

skåla för punschen och glasen vi tömma.

\subsection{\textbf{Giv oss vår punsch}}

Mel: God save the Queen\bigskip

Ge oss vår punsch idag

den som är sval och bra

eller väl värmd.

Glasen vi tömma här

det blir en glad affär

för vi ska ha mera punsch.

Vi är här sen lunch.

\subsection{\textbf{Ärtor och punsch}}

Mel: Fritiof och Carmencita\bigskip

Ärtor och punsch, en liten rätt med traditioner

den smakar bra och väcker många sensationer,

blekgul till färgen, smaker går intill märgen,

med en senap som blandas enligt gammal tradition

så ska den njutas denna svenska folkpassion,

en torsdagskväll i varje månad.\bigskip

Skål nu vänner uti denna mäss,

ärter och punsch ska vi njuta utan stress,

inga sura miner vill vi se i afton,

när doften utav ärter och punsch sprids i salongen

Tänk på Grönstedt, Cederlund och Flagg,

åh - vilken doft från denna arraksgula tagg,

hela natten ska vi njuta denna underbara brygd,

skål för lilla ärten och punschen.

\subsection{\textbf{Vädjan till punschen}}

Mel: Sov du lilla videung\bigskip

Kom nu lilla punschen min.

följ nu efter supen.

Snart skall du åka in

ner igenom strupen,

till mitt stora magpalats,

där det finns så mycket plats.

Kom nu lilla punschen.

Följ nu efter supen.

\subsection{\textbf{Krya punschvisan}}

Mel: Putti putti\bigskip


Putti putti kanna full med punsch

kanna full med punsch

avslutar denna baluns.

För till middag och till lunch

måste man ha punsch.

Vill man bli punschig, måste man bums ha punsch

Vi vill ha PUUUNSCH

\subsection{\textbf{Min älskling}}

Mel: Min älskling du är som en ros\bigskip

Min älskling du är som en punsch,

en nyupphälld och kall.

Ack, som en Cederlunds så ljuv,

min älskling, jag är knall.\bigskip

Så underbar blir du av punschen,

och ser så vacker ut.

Att älska dig det skall jag än

när punschen tagit slut.\bigskip

När hela flaskan på står där tom

och torkan i min gom

härjar så brännande och from,

då fäller jag min dom.\bigskip

Min älskling du är som en punsch,

en flaska Cederlundsch.

Ack, jag vill bara älska dig,

min älskling och min punsch!\bigskip

\subsection{\textbf{Punsch-Lunch Visa}}

Mel: Jag vill ha blommig falukorv\bigskip


Jag vill ha kall och simmig punsch till lunch, mamma.

N'åt annat vill jag inte ha.

Öl får man för ofta, vin smakar som kofta,

starksprit det är riktigt läbbigt.

Nej, jag vill ha kall och simmig punsch

till lunch, mamma

ett stort glas härlig punsch till lunch

\subsection{\textbf{Punschpolkett}}

Mel: Nudistpolkav\bigskip

Inte dricker jag sprit, nej det passar ej hit,

till min kaffetår ikväll,

men av punschen blir jag snäll,

ja se punschen går idag som igår.

Går till ärtor och glass,

är en dryck utav klass,

smakar mjuk men ändå vass,

å vad det är underbart

att få tömma punschpokalen.\bigskip

Hoppsansa, se upp i svängarna,

hoppsansa, bland gobelängarna,

punschen tillhör överdängarna,

dricka punsch det gör ditt sinne rent.\bigskip

Ria faderia faderallan la,

hej hå hoppsansa!

\subsection{\textbf{Födelsedagspunsch}}

Mel: Med en enkel tulipan\bigskip


Nu med en ny och stadig krök

med armen gör vi försök

att lyfta koppen, att lyfta koppen

som står och väntar.

Håll blicken fäst vid koppens rand

och darra inte på hand.

Nu allesammans, nu allesammans

på munnen gläntar.

En liten punschtår så här placerad

i ena handen sig bättre gör,

än tio liter uppå Systemet

och inga pengar att köpa för.

Spill inga droppar på ditt bord

och spill ej mer några ord.

Nu tar vi punschen, nu tar vi punschen

som står och väntar.


\subsection{\textbf{Studiemedelsrundo}}

Mel: Ösa sand\bigskip


Vi dricker punsch till lunch,

när vi har fått avin.

Vi lunchar hela dagen

tills kassan går i sin.\bigskip

För mycket punsch till lunch,

nu mår jag som ett svin.

Helkass hela dagen

och munnen en latrin.

\subsection{\textbf{Studiemedelsrundo v2}}

Mel: Ösa sand\bigskip

Vi dricker punsch till lunch,

när vi har fått avin.

Vi lunchar hela dagen

tills kassan gått i sin.\bigskip

Vi dricker punsch till lunch.

Det är en medicin

som hjälper mot det mesta

och ger oss glad min.\bigskip

Vi dricker punsch ger stuns

åt tentamensamnestin.

Vi dricker för att trösta

och fira som små svin.\bigskip

\subsection{\textbf{Punsch punsch}}

Mel: Ritch ratch\bigskip

Punsch, punsch fillibom bom bom

fillibom bom bom fillibom bom bom.

Punsch, punsch fillibom bom bom

fillibom bom bom fillibom.

Vi har ju både Cederlunds och

Carlshamns flagg, Grönstedts blå

och lilla Caloric.

||: Det duger ej med sodavatten,

sodavatten, sodavatten.

Det duger ej med sodavatten,

ge mig lite punsch! :||

\subsection{\textbf{Tvekan inför punschen}}

Mel: Rosa på bal\bigskip

Jag borde nog inte dricka mer

varken öl eller vin eller brännevin.

En kaffetår blott, men såvitt jag ser

står det punsch brevi’n.\bigskip

Tänk vilken underbar färg och odör

-nej, det blir bara man bråkar och stör.

Men för att inte min kväll skall bli trist,

krävs både slughet och list!\bigskip

Så jag tror nog att jag tar en

punsch, kanske två, kanske tre!

Se’n blir det groggar i baren,

jag klarar säkert av det!\bigskip

Trots att jag egentligen är rätt knall,

piggar nog punsch upp mig i alla fall.

Jag kvicknar till med en punsch i min bål.

Nu har jag bestämt mig, SKÅL!\bigskip

\subsection{\textbf{Djungelpunsch}}

Mel: Var nöjd med allt som livet ger\bigskip

Jag gillar alla tiders punsch.

Punsch till frukost, punsch till lunch,

en punsch till förrätt, varmrätt och dessert.

Jag gillar punsch för vet du vad,

rent kaffe gör ju ingen glad.

Så punsch för fulla muggar vill jag ha.\bigskip

Med konjak du lockar.

Den bästa Renault.

Förlåt om jag chockar

och tar punsch ändå.

Och bjuder du på fin likör

får du ursäkta om det stör.

Jag väljer hellre Grönstedts Blå,

en Cederlunds eller Flaggpunsch å

- kanske har du ren Platin?\bigskip

Jag gillar punsch.

Ger du mig punsch så är jag din.

För evigt din.

\subsection{\textbf{Glad såsom få}}

Mel: Vårsång (glad såsom fågeln)\bigskip

Glad såsom få efter glassdesserten

väntar vi tåligt på kaffe och punsch.

Längtan nu dallrar i rumsatmosfären,

frostog är flaskan vi lagt in vid lunch.

Stämningen högt upp vid takåsen dallrar,

samtalet flyter likt vårsolen fram.

Vitsarna haglar och skratten de skallrar,

Här har vi glädjen med stoj och med glam.

Se, hur den simmiga punschen vi slå,

droppande gå, i glasen små.

Sös såsom honung och mild i smaken,

guldgul är färgen och kylan bra den.

Den för oss i trance, den ger ögonen glans,

vi behöver den lite mans.

\subsection{\textbf{Punschens lov}}

Mel: Rövarvisan (Camomilla stad)\bigskip

Ja, punschen är och och punschen var

och punschen skall förbliva.

En lidelse vi alla har

som inget kan fördriva.

Ja punschen tinar opp, såväl

som svalkar både kropp och själ.

Den botar begären och lindrar besvären.

Ja, punschen den gör både gott och väl!

\subsection{\textbf{Valfri punschvisa}}

Text: ur Chalmersspexet Gutenberg, 1991

Mel: Fredmans epistel nr2: Nå skruva fiolen\bigskip


Men vad är väl detta?

Har Svakar nu fått fnatt?

Har han gått och tatt

upp ut ur sin skatt?

En okorkad kall punschflaska.\bigskip

När man dricker punsch,

då ska man ta sig ton.

Full som en kanon, det är tradition

Och sångerna ska rimma\bigskip

Av minnen i vår sal

Om glasen det imma

Gul, ljus och sval

Punschen stimma

Små droppar i kroppar

Frostnupen ner i djupen

Möts mitt i strupen av en visa lätt banal.  

\subsection{\textbf{Robespierres punschvisa}}

Text: Henrik Dahlgren \& Mikael Josefsson (Ur Handelsspexet
Robespierre 1995)

Mel: Hyllning till Sverige\bigskip

Många söker tillfredsställelse,

eller någonting som skänker ro.

När jag själv såg min förälskelse,

stod det klart, den var ljus å go’.\bigskip

Redan som ett litet knyte

fick jag smaka min första pralin.

Ett facilt och läckert byte -

jag blev fast för en söt blondin.\bigskip

Min första kärlek hette Cederlund;

så bedårande och kall o rik.

Jag minns vår första herdestund -

din gestalt var som ljuv musik.\bigskip

Ack, så underbart att få trycka

mina läppar mot din hals.

En sådan obeskrivbar lycka.

Vi försvann i eldig vals.\bigskip

Det blev fler och fler romanser

för varje år som flöt förbi.

Jag är fylld av de essenser

som giver styrka och energi.\bigskip

\subsection{\textbf{Marco Polos punschvisa}}

Text: Jonas Lundberg (Ur Handelsspexet Marco Polo 1986)

Mel: Fredmans epistel nr 2 (Nå, skrufva fiolen)\bigskip

Ge mig buteljen,

hej Marco, skynda dig.

Gula droppar rinn

ner i strupen min,

ger tron på liv tillbaka.\bigskip

Javisst, käre broder,

här får du blott en klunk.

Vi får spara nå’t

fram till nästa låt.

Vi spelar flera kvällar.\bigskip

Du skojar med mig.

I Carlshamns vill jag bada.

Men det får du ej,

då tar spriten skada.\bigskip

Zum zum zum...\bigskip

Ju mera jag dricker,

desto nyktrare blir jag.

Jag kan dricka en

hela sjuttiofem.

Jag faller ej i dvala.\bigskip

Det där var ju konstigt,

det tror jag inte på.

Så du säger att,

att du ej får fnatt.

Så kan det inte vara.\bigskip

Jag har en idé.

Jag tror jag kan förklara.

Jag undrar vad det é,

men han skämtar nog bara\bigskip

Zum zum zum... -låt höra!\bigskip

För oss snillen är

punschpromillen

ej så stark.\bigskip


\subsection{\textbf{Punschen kommer (kall)}}

Mel: Änkevalsen ur Glada Änkan\bigskip

punschen kommer,

ljuv och sval.

Glasen imma,

röster stimma

i vår sal.

Skål för glada minnen!

Skål för varje vår!

Inga sorger finnes mer

när punsch vi får.

\subsection{\textbf{Punschen kommer (varm)}}

Mel: Änkevalsen ur Glada Änkan\bigskip

Punschen kommer,

punschen kommer,

god och varm.

Vettet svinner,

droppen rinner,

ner i tarm.

Skål för glada minnen!

Skål för varje vår!

Inga sorger finnes mer

när punsch vi får.

\subsection{\textbf{Punschen rinner genom strupen}}

Mel: Midnatt råder\bigskip


Punschen, punschen rinner ner i strupen,

ner i djupen.

Blandas, konfronteras där med supen,

där med supen.

Gula droppar stärker våra kroppar,

PUNSCH, PUNSCH, PUNSCH!


\subsection{\textbf{Nu har vi punch}}

Mel: Nu har vi ljus\bigskip


Nu har vi punsch här på vårt bord,

punschen är kommen, räddad är vår kväll.

Kaffet är varmt, punschen är kall,

sjung i vår trall:

Punschen rinner ner i lilla magen,

skapar där de kända välbehagen,

som vi gillar, som vi gillar,

ja, som vi gillar till tusen.

\subsection{\textbf{Nya punschvisan}}

Text: Ur Chalmersspexet Sven Duva 1965

Mel: Mjölnarens dotter (Svensk folkvisa)\bigskip


Vad ger dig din sisu, vad håller dig varm?

Skrororompompej!

Jo, punschen som upptas i mage och tarm.

Försvinnande god!

Försvinnande god!

Får upp humöret och promillen, hej!\bigskip

Du blir som en ny män’ska om du tar en punsch.

Skrororompompej!

Och den nya män’skan vill också ha en punsch.

Försvinnande god!

Försvinnande god!

Får upp humöret och promillen, hej!

\subsection{\textbf{Jag gillar punschen}}

Mel: Te Deum Laudamus\bigskip


Det var en gång jag tänkte

att punschen övergiva,

men det får aldrig ske

så länge jag får leva

och när jag en gång dör

skall ristas på min grav;

Här vilar en som svenska punschen

gillat har:\bigskip

Jag gillar, jag gillar punschen,

jag gillar den som punschen skapat har.

Jag gillar, jag gillar punschen,

jag gillar punschen och dess far.

\subsection{\textbf{Låtnamn}}

Text: 

Mel: \bigskip

\newpage