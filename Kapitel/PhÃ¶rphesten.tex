
\fakesection{Phörphesten}

\fancypagestyle{Phörphesten}{
\fancyhead{} % clear all header fields
\fancyhead[LE,RO]{\textbf{Phörphesten}}
}
\pagestyle{Phörphesten}


\begin{SongText}[En enkel liten sång]
    \begin{SongInfo}
        Mel: Med en enkel tulipan
    \end{SongInfo}
    \begin{SongVerse}
        Vi tar en enkel liten sång\\*%
        Men den får ej bli för lång\\*%
        Vi har så bråttom, vi har så bråttom\\*%
        Att fatta glaset.\\*%
        Vi alla titlar lägger bort\\*%
        Så att det inte blir torrt\\*%
        Ty uti kväll ska vi ha det trevligt\\*%
        Uppå kalaset
    \end{SongVerse}
    \begin{SongVerse}
        Det glada gänget har träffats åter\\*%
        Och glädjen högt uti taket står.\\*%
        Vår kära värd nu sin röst upplåter\\*%
        Vi undra nu vad åt honom går.\\*%
        Jo, med en enkel liten fras\\*%
        Han höjer härmed sitt glas.
    \end{SongVerse}
    \begin{SongVerse}
        Han säger: Hejsan.\\*%
        Vi tackar: Hoppsan.\\*%
        Så tar vi Huttsan. Hej Hej!
    \end{SongVerse}
\end{SongText}

\begin{SongText}[Hej allesammans]
    \begin{SongInfo}
        Mel: Hej tomtegubbar
    \end{SongInfo}
    \begin{SongVerse}
        Hej allesammans är ni redo\\*%
        För nu börjar kalaset\\*%
        Hej allesammans är det säkert\\*%
        Att ni fått nått i glaset?\\*%
        Fått lite vin, och lite mat\\*%
        Det är precis så som vi vill ha’t.\\*%
        Höj allesammans högra armen\\*%
        Och låt oss öppna kalaset
    \end{SongVerse}
\end{SongText}

\begin{SongText}[Phesten kan börja]
    \begin{SongInfo}
        Mel: Vårvindar friska
    \end{SongInfo}
    \begin{SongVerse}
        Phesten kan börja, ingen får sörja\\*%
        här finns det både brännvin och mat\\*%
        Helan ska tömmas, sorgerna glömmas\\*%
        Ingen får vara dålig, kamrat.
    \end{SongVerse}
    \begin{SongVerse}
        Klappa mitt hjärta, fröjdas min själ\\*%
        Nubben serveras genast nåväl.\\*%
        Nu tar vi supen, öppna på strupen\\*%
        Gästernas välkomstskål
    \end{SongVerse}
    \begin{SongVerse}
        Phesten kan börjas, kråset skall smörjas\\*%
        Glädjen ska vara gäst här idag\\*%
        Glasen förvara dropparna rara,\\*%
        Dyrare blir de dag för dag..
    \end{SongVerse}
    \begin{SongVerse}
        Pärlan på bordet lockande står.\\*%
        Skratta och sjung för nu är det vår.\\*%
        Känn hur det våras! Låt dig bedåras!\\*%
        Skål allihop! Gu’tår. 
    \end{SongVerse}
    \begin{SongVerse}
        Vem sade ordet SKÅL här vid bordet\\*%
        viskande for det sällskapet kring.\\*%
        Fattom kristallen, nubben är kall den\\*%
        stiger åt skallen, kling klingeling!
    \end{SongVerse}
    \begin{SongVerse}
        Käraste vänner, välkomna hit!\\*%
        Hoppas ni har en "bon appetit"\\*%
        Nu lilla nubben, tager vi stubben\\*%
        SKÅL lilla nubben, kling klingeling. 
    \end{SongVerse}
    \begin{SongVerse}
        Hutten den lilla smakar ej illa\\*%
        När man den tager i trevligt lag.\\*%
        Känslor så ömma inom oss strömma\\*%
        Sorger vi glömma, Kling!
    \end{SongVerse}
    \begin{SongVerse}
        Vänligt de blicka bordets små bloss\\*%
        En trevlig kväll de tillönska oss.\\*%
        Flickor och svenner livsglädje känner,\\*%
        Skål kära vänner, Skål! 
    \end{SongVerse}
\end{SongText}

\begin{SongText}[Välkomstvisa]
    \begin{SongInfo}
        Jänta å jag
    \end{SongInfo}
    \begin{SongVerse}
        Välkommen hit, ta er en bit,\\*%
        njut era drycker, ät med god aptit.\\*%
        Skoj ska vi ha, leka vi ska,\\*%
        trivas sådär förtroligt.
    \end{SongVerse}
    \begin{SongVerse}
        Humöret toppen på alla ska nå,\\*%
        skratta och sjunga det kan vi få,\\*%
        med minne av kvällen hemåt vi gå.\\*%
        I kväll ska vi ha det roligt.
    \end{SongVerse}
\end{SongText}

\begin{SongText}[Öppnings skål]
    \begin{SongInfo}
        Mel: Vintern rasar
    \end{SongInfo}
    \begin{SongVerse}
        Gommen längtar efter vin att svalka\\*%
        Tungan som ska njuta av dess smak\\*%
        Innan ned i strupen det ska halka\\*%
        Ska vi ha en sång att öppna med.
    \end{SongVerse}
    \begin{SongVerse}
        Skål på dig och mig och skål på er alla\\*%
        Och hjärtligen välkomna till vårt hus\\*%
        Tar nå’nting här slut på mer vi kalla\\*%
        Ät och drick go’vänner utan krus
    \end{SongVerse}
\end{SongText}
\newpage