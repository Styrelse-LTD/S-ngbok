\fakesection{Julvisor}
\fancypagestyle{Julvisor}{
    \fancyhead{} % clear all header fields
    \fancyhead[LE,RO]{\textbf{Julvisor}}
}
\pagestyle{Julvisor}
%\begin{SongText}[Sankta Lucia]
%    \begin{SongVerse}
%        Natten går tunga fjät runt gård och stuva.\\*%
%        Kring jord som sol'n förlät, skuggorna ruva.\\*%
%        Då i vårt mörka hus, stiger med tända ljus\\*%
%        Sankta Lucia, Sankta Lucia.\\*%
%        Då i vårt mörka hus, stiger med tända ljus\\*%
%        Sankta Lucia, Sankta Lucia.
%    \end{SongVerse}
%    \begin{SongVerse}
%        Natten var stor och stum. Nu, hör, det svingar\\*%
%        i alla tysta rum sus som av vingar.\\*%
%        Se, på vår tröskel står, vitklädd med ljus i hår\\*%
%        Sankta Lucia, Sankta Lucia.\\*%
%        Se, på vår tröskel står, vitklädd med ljus i hår\\*%
%        Sankta Lucia, Sankta Lucia.
%    \end{SongVerse}
%    \begin{SongVerse}
%        Mörkret skall flykta snart ur jordens dalar.\\*%
%        Så hon ett underbart ord till oss talar.\\*%
%        Dagen skall åter ny, stiga ur rosig sky.\\*%
%        Sankta Lucia, Sankta Lucia.\\*%
%        Dagen skall åter ny, stiga ur rosig sky.\\*%
%        Sankta Lucia, Sankta Lucia.
%    \end{SongVerse}
%    \begin{SongVerse}
%    \end{SongVerse}
%\end{SongText}
%\begin{SongText}[Lusse lelle]
%    \begin{SongVerse}
%        Lusse lelle, Lusse lelle,\\*%
%        elva nätter före jul.\\*%
%        Lusse lelle, Lusse lelle,\\*%
%        elva nätter före jul.\\*%
%        Nu äro vi hitkomna,\\*%
%        så näst före jul.\\*%
%        Nu äro vi hitkomna,\\*%
%        så näst före jul.
%    \end{SongVerse}
%\end{SongText}
%\begin{SongText}[Staffansvisan]
%    \begin{SongVerse}
%        Staffan var en stalledräng. Vi tackom nu så gärna.\\*%
%        Han vattna' sina fålar fem. Alltför den ljusa stjärna.\\*%
%        Ingen dager synes än. Stjärnorna på himmelen de blänka.
%    \end{SongVerse}
%    \begin{SongVerse}
%        Hastigt lägges sadeln på, Vi tackom nu så gärna.\\*%
%        innan solen månd' uppgå. Alltför den ljusa stjärna...
%    \end{SongVerse}
%    \begin{SongVerse}
%        Bästa fålen apelgrå, Vi tackom nu så gärna.\\*%
%        den rider Staffan själv uppå. Alltför den ljusa stjärna...
%    \end{SongVerse}
%    \begin{SongVerse}
%        Innan någon vaknat har, Vi tackom nu så gärna.\\*%
%        framme han vid skogen var. Alltför den ljusa stjärna...
%    \end{SongVerse}
%    \begin{SongVerse}
%        I den fula ulven spår, Vi tackom nu så gärna.\\*%
%        raskt och oförskräckt han går. Alltför den ljusa stjärna...
%    \end{SongVerse}
%    \begin{SongVerse}
%        Gamle björnen i sitt bo, Vi tackom nu så gärna.\\*%
%        ej får vara uti ro. Alltför den ljusa stjärna...
%    \end{SongVerse}
%    \begin{SongVerse}
%        Nu är eld uti var spis, Vi tackom nu så gärna.\\*%
%        julegröt och julegris. Alltför den ljusa stjärna...
%    \end{SongVerse}
%    \begin{SongVerse}
%        Nu är fröjd uti vart hus, Vi tackom nu så gärna.\\*%
%        julegröt och juleljus, Alltför den ljusa stjärna...
%    \end{SongVerse}
%    \begin{SongVerse}
%    \end{SongVerse}
%\end{SongText}
%\begin{SongText}[Stilla Natt]
%    \begin{SongVerse}
%        Stilla natt, heliga natt,\\*%
%        allt är frid, stjärnan blid\\*%
%        skiner på barnet i stallets strå,\\*%
%        och de vakande fromma två.\\*%
%        Kristus till jorden är kommen,\\*%
%        oss är en frälsare född.
%    \end{SongVerse}
%    \begin{SongVerse}
%        Stora stund, heliga stund,\\*%
%        änglars här går sin rund\\*%
%        kring de vaktande herdars hjord.\\*%
%        Rymden ljuder av glädjens ord:\\*%
%        Kristus till jorden är kommen,\\*%
%        eder är frälsaren född.
%    \end{SongVerse}
%    \begin{SongVerse}
%        Stilla natt, heliga natt,\\*%
%        mörkret flyr, dagen gryr.\\*%
%        Räddningstimman för världen slår,\\*%
%        nu begynner vårt jubelår.\\*%
%        Kristus till jorden är kommen,\\*%
%        oss är en frälsare född.
%    \end{SongVerse}
%    \begin{SongVerse}
%    \end{SongVerse}
%\end{SongText}
%\begin{SongText}[Tre Pepparkaksgubbar]
%    \begin{SongVerse}
%        Vi komma, vi komma från Pepparkakeland,\\*%
%        och vägen vi vandrat tillsamman hand i hand.\\*%
%        Så bruna, så bruna vi äro alla tre,\\*%
%        korinter till ögon och hattarna på sne’.
%    \end{SongVerse}
%    \begin{SongVerse}
%        Tre gubbar, tre gubbar från pepparkakeland,\\*%
%        till julen, till julen vi komma hand i hand.\\*%
%        Men tomten och bocken vi lämnat vid vår spis,\\*%
%        de ville inte resa från vår pepparkakegris.
%    \end{SongVerse}
%    \begin{SongVerse}
%    \end{SongVerse}
%\end{SongText}
%\begin{SongText}[Goder morgon eller Goder afton]
%    \begin{SongVerse}
%        Goder morgon, goder morgon\\*%
%        båd’ herre och fru\\*%
%        vi önskar eder alla\\*%
%        en fröjdefull jul.
%    \end{SongVerse}
%    \begin{SongVerse}
%        Goder morgon, goder morgon\\*%
%        välkommen var gäst!\\*%
%        vi önskar eder alla\\*%
%        en fröjdefull fest.
%    \end{SongVerse}
%    \begin{SongVerse}
%        Goder afton, goder afton\\*%
%        båd’ herre och fru\\*%
%        vi önskar eder alla\\*%
%        en fröjdefull jul.
%    \end{SongVerse}
%    \begin{SongVerse}
%        Goder afton, goder afton\\*%
%        välkommen var gäst!\\*%
%        vi önskar eder alla\\*%
%        en fröjdefull fest.
%    \end{SongVerse}
%    \begin{SongVerse}
%    \end{SongVerse}
%\end{SongText}
%\begin{SongText}[En sockerbagare]
%    \begin{SongVerse}
%        En sockerbagare här bor i staden,\\*%
%        han bakar kakor mest hela dagen.\\*%
%        Han bakar stora, han bakar små,\\*%
%        han bakar några med socker på.
%    \end{SongVerse}
%    \begin{SongVerse}
%        Och i hans fönster hänger julgranssaker\\*%
%        och hästar, grisar och pepparkakor.\\*%
%        Och är du snäller så kan du få,\\*%
%        men är du stygger så får du gå.
%    \end{SongVerse}
%    \begin{SongVerse}
%    \end{SongVerse}
%\end{SongText}
%\begin{SongText}[God morgon mitt herrskap ]
%    \begin{SongVerse}
%        God morgon mitt herrskap! Här kommer Lussebrud;\\*%
%        hon kommer till Er med stor ära.\\*%
%        Lucian sig påminner att få roa Er en stund,\\*%
%        om hon Er vid hälsan finner i denna morgonstund.\\*%
%        Stånder upp nu, mitt herrskap, och ät och må väl!\\*%
%        Det fröjdar Lucian av hjärtat!\\*%
%        Hon önskar Eder alla en fröjdefull jul.\\*%
%        Från olyckor alle bevare Eder Gud!\\*%
%        Goder afton, goder afton\\*%
%        Båd herre och fru.\\*%
%        Vi önskar eder alla\\*%
%        En fröjdefull jul
%    \end{SongVerse}
%\end{SongText}
%\begin{SongText}[Tomtarnas Julnatt]
%    \begin{SongVerse}
%        Midnatt råder, tyst det är i husen, tyst i husen.\\*%
%        Alla sova, släckta äro ljusen, äro ljusen.
%    \end{SongVerse}
%    \begin{SongVerse}
%        Ref:\\*%
%        Tipp, tapp, tipp, tapp,\\*%
%        tippe-tippe-tipp, tapp.\\*%
%        Tipp, tipp, tapp.
%    \end{SongVerse}
%    \begin{SongVerse}
%        Se, då krypa tomtar upp ur vrårna, upp ur vrårna.\\*%
%        Lyssna, speja, trippa fram på tårna, fram på tårna.\\*%
%        Ref...
%    \end{SongVerse}
%    \begin{SongVerse}
%        Snälla folket låtit maten rara, maten rara\\*%
%        stå på bordet åt en tomteskara, tomteskara.\\*%
%        Ref...
%    \end{SongVerse}
%    \begin{SongVerse}
%        Hur de mysa, hoppa upp bland faten, upp bland faten,\\*%
%        tissla, tassla: "God är julematen, julematen".\\*%
%        Ref...
%    \end{SongVerse}
%    \begin{SongVerse}
%        Gröt och skinka, lilla äppelbiten, äppelbiten,\\*%
%        tänk, så rart det smakar Nisse liten, Nisse liten!\\*%
%        Ref...
%    \end{SongVerse}
%    \begin{SongVerse}
%        Nu till lekar! Glada skrattet klingar, skrattet klingar,\\*%
%        runt om granen skaran muntert svingar, muntert svingar.\\*%
%        Ref...
%    \end{SongVerse}
%    \begin{SongVerse}
%        Natten lider. Snart de tomtar snälla, tomtar snälla,\\*%
%        kvickt och näpet allt i ordning ställa, ordning ställa.\\*%
%        Ref...
%    \end{SongVerse}
%    \begin{SongVerse}
%        Sedan åter in i tysta vrårna, tysta vrårna\\*%
%        tomteskaran tassar nätt på tårna, nätt på tårna.\\*%
%        Ref...
%    \end{SongVerse}
%    \begin{SongVerse}
%    \end{SongVerse}
%\end{SongText}
%\begin{SongText}[Nu är det jul igen]
%    \begin{SongVerse}
%        Nu är det jul igen\\*%
%        och nu är det jul igen\\*%
%        och julen varar väl till påska.\\*%
%        Nu är det jul igen\\*%
%        och nu är det jul igen\\*%
%        och julen varar väl till påska.\\*%
%        Det var inte sant\\*%
%        och det var inte sant,\\*%
%        för däremellan kommer fasta.\\*%
%        Det var inte sant\\*%
%        och det var inte sant,\\*%
%        för däremellan kommer fasta.
%    \end{SongVerse}
%\end{SongText}
%\begin{SongText}[Rudolf med röda mulen]
%    \begin{SongVerse}
%        Rudolf med röda mulen\\*%
%        hette en helt vanlig ren\\*%
%        som blivit kall om mulen,\\*%
%        därav kom dess röda sken.\\*%
%        Rudolf fick alltid höra\\*%
%        "Se, han har sitt dimmljus på",\\*%
%        att han blev led åt detta,\\*%
%        är en sak man kan förstå.
%    \end{SongVerse}
%    \begin{SongVerse}
%        Men en mörk julaftonskväll\\*%
%        Tomtefar han sa:\\*%
%        "Vill du inte Rudolf säg,\\*%
%        med din mule lysa mig."\\*%
%        Allt se'n den dagen renen\\*%
%        tomtens egen släde drar.\\*%
%        Rudolf med röda mulen\\*%
%        lyser väg för tomtefar.
%    \end{SongVerse}
%\end{SongText}
%\begin{SongText}[Hej Tomtegubbar]
%    \begin{SongVerse}
%        Hej tomtegubbar slå i glasen\\*%
%        och låt oss lustiga vara.\\*%
%        Hej tomtegubbar slå i glasen\\*%
%        och låt oss lustiga vara.\\*%
%        En liten tid vi leva här\\*%
%        med mycken möda och stort besvär.\\*%
%        Hej tomtegubbar slå i glasen\\*%
%        och låt oss lustiga vara.
%    \end{SongVerse}
%\end{SongText}
\begin{SongText}[Rudolf med becquerellen]
    \begin{SongVerse}
        Mel: Rudolf med röda mulen
    \end{SongVerse}
    \begin{SongVerse}
        Rudolf med becquerellen, var en ganska vanlig ren;\\*%
        upplevde Tjernobyl smällen - därav kom hans gröna sken.\\*%
        Rudolfs alla små kamrater avskyr att gå utomhus\\*%
        och se den nukleära renen lysa som ett varningsljus.
    \end{SongVerse}
    \begin{SongVerse}
        Natten är rätt konstig sen\\*%
        vår Rudolf blivit grön.\\*%
        "Ge dig av!" är allas bön.\\*%
        "Du smälter ju bort snön!"
    \end{SongVerse}
    \begin{SongVerse}
        Alltsen Tjernobyl vandrar Rudolf för sig själv, allén,\\*%
        lyser för sig själv och blir nog underlig och schizofren.
    \end{SongVerse}
\end{SongText}
\begin{SongText}[Vem kan jula förutan gran? ]
    \begin{SongVerse}
        Mel: Vem kan segla förutan vind?
    \end{SongVerse}
    \begin{SongVerse}
        Vem kan jula förutan gran?\\*%
        Vad är jul utan klappar?\\*%
        Vem kan pröjsa för julefröjd\\*%
        utan att fälla tårar?
    \end{SongVerse}
    \begin{SongVerse}
        Jag kan jula förutan gran.\\*%
        Jag kan avstå från klappar,\\*%
        men ej pröjsa för julefröjd\\*%
        utan att fälla tårar.
    \end{SongVerse}
\end{SongText}
\begin{SongText}[Tomtevisan]
    \begin{SongVerse}
        Mel: Kalmarvisan
    \end{SongVerse}
    \begin{SongVerse}
        $\|\:$ För uti tomteland,\\*%
        ja där är det alltid snö $\:\|$\\*%
        och det blir aldrig tö.\\*%
        \textbf{Hej dick}\\*%
        Hej dack\\*%
        \textbf{Jag slog i}\\*%
        Och vi drack\\*%
        \textbf{Hej dickom dickom dack }\\*%
        Hej dickom dickom dack\\*%
        För uti tomteland ja\\*%
        där är det alltid snö\\*%
        och det blir aldrig tö
    \end{SongVerse}
    \begin{SongVerse}
        $\|\:$ När så tomten kommer hem\\*%
        kommer tomtegumman ut $\:\|$\\*%
        och är stor i sin trut.\\*%
        \textbf{Hej dick...}\\*%
    \end{SongVerse}
    \begin{SongVerse}
        $\|\:$ Var är klapparna du fått?\\*%
        Jo, dem har jag supit bort $\:\|$\\*%
        Upp på Nordpolens slott.\\*%
        \textbf{Hej dick...}\\*%
    \end{SongVerse}
    \begin{SongVerse}
        $\|\:$ Rudolf, renen vår\\*%
        satt på Kåren igår $\|\:$\\*%
        och var full som ett ….älg\\*%
        \textbf{Hej dick...}\\*%
    \end{SongVerse}
    \begin{SongVerse}
    \end{SongVerse}
\end{SongText}
\begin{SongText}[Nu höjas tusen fyllda glas]
    \begin{SongVerse}
        Mel: Nu tändas tusen juleljus
    \end{SongVerse}
    \begin{SongVerse}
        Nu höjas tusen fyllda glas,\\*%
        kring bordets glada rund,\\*%
        och tusen, tusen, skålar ock,\\*%
        vi tömmer per sekund.
    \end{SongVerse}
    \begin{SongVerse}
        Och från butelj och fat i kväll,\\*%
        oss öl och brännvin bärs,\\*%
        tills hel- och halvan alla fått,\\*%
        och mången och en ters.
    \end{SongVerse}
\end{SongText}
\begin{SongText}[Töntarnas julenatt]
    \begin{SongVerse}
        Mel: Midnatt råder
    \end{SongVerse}
    \begin{SongVerse}
        Julfrid råder, ute skiner ljusen, skiner ljusen.\\*%
        Ingen sover, det är larm till tusen, larm till tusen.\\*%
        Tipp tapp, tipp tapp, tippe tippe tipslapp,\\*%
        tipp, tipp, tapp.
    \end{SongVerse}
    \begin{SongVerse}
        Se då krälar tomtar fram på gatan, fram på gatan.\\*%
        Tigga, sälja, supa, leva satan, leva satan.\\*%
        Tipp tapp...
    \end{SongVerse}
    \begin{SongVerse}
        Tovigt skägg som stinker utav spriten, utav spriten,\\*%
        väcker avsky hos den som är liten, som är liten.\\*%
        Tipp tapp...
    \end{SongVerse}
    \begin{SongVerse}
        Synen utav tomtar som vomera, som vomera,\\*%
        gör att ingen tror på tomten mera, tomten mera.\\*%
        Tripp trapp, tipp tapp, sopetipp och julklapp,\\*%
        tipp, topp, tupp.
    \end{SongVerse}
    \begin{SongVerse}
    \end{SongVerse}
\end{SongText}
%\begin{SongText}[Gläns över sjö och strand]
%    \begin{SongVerse}
%        Gläns över sjö och strand,\\*%
%        Stjärna ur fjärran,\\*%
%        Du som i Österland\\*%
%        Tändes av Herran.\\*%
%        Stjärnan från Betlehem\\*%
%        Leder ej bort, men hem.\\*%
%        Barnen och herdarna\\*%
%        Följa dig gärna,\\*%
%        Strålande stjärna,\\*%
%        Strålande stjärna.
%    \end{SongVerse}
%    \begin{SongVerse}
%        Natt över Judaland,\\*%
%        Natt över Sion.\\*%
%        Borta vid västerrand\\*%
%        Slocknar Orion.\\*%
%        Herden, som sover trött,\\*%
%        Barnet, som slumrar sött,\\*%
%        Vakna vid underbar\\*%
%        Korus av röster,\\*%
%        Skåda en härligt klar\\*%
%        Stjärna i öster;
%    \end{SongVerse}
%    \begin{SongVerse}
%        Gånga från lamm och hem,\\*%
%        Sökande Eden.\\*%
%        Stjärnan från Betlehem\\*%
%        Visar dem leden\\*%
%        Fram genom hindrande\\*%
%        Jordiska fängsel\\*%
%        Hän till det glindrande\\*%
%        Lustgårdens stängsel.\\*%
%        Hän till det glindrande\\*%
%        Lustgårdens stängsel.
%    \end{SongVerse}
%    \begin{SongVerse}
%        Armar där sträckas dem,\\*%
%        Läppar där viska,\\*%
%        Viska och räckas dem,\\*%
%        Ljuva och friska:\\*%
%        Stjärnan från Betlehem\\*%
%        Leder ej bort, men hem.\\*%
%        Barnen och herdarna\\*%
%        Följa dig gärna,\\*%
%        Strålande stjärna,\\*%
%        Strålande stjärna.
%    \end{SongVerse}
%\end{SongText}
\begin{SongText}[Hej tomtegubbar v2 ]
    \begin{SongVerse}
        Hej, tomtegubbar, vrid på gasen\\*%
        och tag er sjäva av daga.\\*%
        Rödsprit vi hällt i alla glasen\\*%
        och fjäril’n vinglar på Haga.\\*%
        Av aquavit, man får kolik,\\*%
        och alla tomtar skola-stå-lik.\\*%
        Hej, tomtegubbar, vrid på gasen\\*%
        och tag er själva av daga.
    \end{SongVerse}
\end{SongText}
\begin{SongText}[Hej tomtegubbar v3]
    \begin{SongVerse}
        Hej, tomtegubbar, lyft på luvan\\*%
        och borsta dammet ur skägget.\\*%
        Nu skall vi ta och bota snuvan\\*%
        med extra påtårstillägget.\\*%
        En liten smutt, uppå en hutt,\\*%
        den kan på livsanden sätta sprutt.\\*%
        Hej, tomtegubbar, öppna slussen,\\*%
        för nu går hutten med bussen.
    \end{SongVerse}
\end{SongText}
%\begin{SongText}[Kväde till lucian Benke]
%    \begin{SongVerse}
%        Mel: Staffan var en stalledräng.
%    \end{SongVerse}
%    \begin{SongVerse}
%        Benke var en stalledräng\\*%
%        som hade en liten hjärna.\\*%
%        Med sina IQ 75\\*%
%        var han ingen stjärna.\\*%
%        Handelshögen blev hans hem.\\*%
%        Han behövde aldrig mera tänka.
%    \end{SongVerse}
%    \begin{SongVerse}
%        Vacker är han som en Gud\\*%
%        och har en liten hjärna.\\*%
%        I sin konstifika skrud\\*%
%        är han ingen tärna.\\*%
%        Vår lucia då han blev\\*%
%        Glansen från hans leende den blänka.
%    \end{SongVerse}
%    \begin{SongVerse}
%        Med hans kropp gick upp ett ljus,\\*%
%        Benke blev en stjärna.\\*%
%        Femton kilo julesnus\\*%
%        finns det i hans hjärna.\\*%
%        Handelshögen blev hans hem.\\*%
%        Han behövde aldrig mera tänka.
%    \end{SongVerse}
%    \begin{SongVerse}
%        Benke är en ljusets fe,\\*%
%        som har snus i hjärnan.\\*%
%        Men ingen bryr sig väl om det,\\*%
%        här uti Tavernan.\\*%
%        Handelshögen är hans hem.\\*%
%        Han behöver aldrig mera tänka.
%    \end{SongVerse}
%    \begin{SongVerse}
%    \end{SongVerse}
%\end{SongText}
\newpage