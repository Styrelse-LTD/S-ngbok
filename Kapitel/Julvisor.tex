
\fakesection{Julvisor}

\fancypagestyle{Julvisor}{
\fancyhead{} % clear all header fields
\fancyhead[LE,RO]{\textbf{Julvisor}}
}
\pagestyle{Julvisor}



\subsection{\textbf{Sankta Lucia}}

Natten går tunga fjät runt gård och stuva.

Kring jord som sol'n förlät, skuggorna ruva.

Då i vårt mörka hus, stiger med tända ljus

Sankta Lucia, Sankta Lucia.

Då i vårt mörka hus, stiger med tända ljus

Sankta Lucia, Sankta Lucia.\bigskip

Natten var stor och stum. Nu, hör, det svingar

i alla tysta rum sus som av vingar.

Se, på vår tröskel står, vitklädd med ljus i hår

Sankta Lucia, Sankta Lucia.

Se, på vår tröskel står, vitklädd med ljus i hår

Sankta Lucia, Sankta Lucia.\bigskip

Mörkret skall flykta snart ur jordens dalar.

Så hon ett underbart ord till oss talar.

Dagen skall åter ny, stiga ur rosig sky.

Sankta Lucia, Sankta Lucia.

Dagen skall åter ny, stiga ur rosig sky.

Sankta Lucia, Sankta Lucia. \bigskip

\subsection{\textbf{Lusse lelle}}

Lusse lelle, Lusse lelle,

elva nätter före jul.

Lusse lelle, Lusse lelle,

elva nätter före jul.

Nu äro vi hitkomna,

så näst före jul.

Nu äro vi hitkomna,

så näst före jul.

\subsection{\textbf{Staffansvisan}}

Staffan var en stalledräng. Vi tackom nu så gärna.

Han vattna' sina fålar fem. Alltför den ljusa stjärna.

Ingen dager synes än. Stjärnorna på himmelen de blänka.\bigskip

Hastigt lägges sadeln på, Vi tackom nu så gärna.

innan solen månd' uppgå. Alltför den ljusa stjärna.

Ingen dager synes än. Stjärnorna på himmelen de blänka. \bigskip

Bästa fålen apelgrå, Vi tackom nu så gärna.

den rider Staffan själv uppå. Alltför den ljusa stjärna.

Ingen dager synes än. Stjärnorna på himmelen de blänka.\bigskip

Innan någon vaknat har, Vi tackom nu så gärna.

framme han vid skogen var. Alltför den ljusa stjärna.

Ingen dager synes än. Stjärnorna på himmelen de blänka.\bigskip

I den fula ulven spår, Vi tackom nu så gärna.

raskt och oförskräckt han går. Alltför den ljusa stjärna.

Ingen dager synes än. Stjärnorna på himmelen de blänka.\bigskip

Gamle björnen i sitt bo, Vi tackom nu så gärna.

ej får vara uti ro. Alltför den ljusa stjärna.

Ingen dager synes än. Stjärnorna på himmelen de blänka.\bigskip

Nu är eld uti var spis, Vi tackom nu så gärna.

julegröt och julegris. Alltför den ljusa stjärna.

Ingen dager synes än. Stjärnorna på himmelen de blänka.\bigskip

Nu är fröjd uti vart hus, Vi tackom nu så gärna.

julegröt och juleljus, Alltför den ljusa stjärna.

Ingen dager synes än. Stjärnorna på himmelen de blänka\bigskip

\subsection{\textbf{Stilla Natt}}

Stilla natt, heliga natt,

allt är frid, stjärnan blid

skiner på barnet i stallets strå,

och de vakande fromma två.

Kristus till jorden är kommen,

oss är en frälsare född.\bigskip

Stora stund, heliga stund,

änglars här går sin rund

kring de vaktande herdars hjord.

Rymden ljuder av glädjens ord:

Kristus till jorden är kommen,

eder är frälsaren född. \bigskip

Stilla natt, heliga natt,

mörkret flyr, dagen gryr.

Räddningstimman för världen slår,

nu begynner vårt jubelår.

Kristus till jorden är kommen,

oss är en frälsare född.\bigskip

\subsection{\textbf{Tre Pepparkaksgubbar}}

Vi komma, vi komma från Pepparkakeland,

och vägen vi vandrat tillsamman hand i hand.

Så bruna, så bruna vi äro alla tre,

korinter till ögon och hattarna på sne’.\bigskip

Tre gubbar, tre gubbar från pepparkakeland,

till julen, till julen vi komma hand i hand.

Men tomten och bocken vi lämnat vid vår spis,

de ville inte resa från vår pepparkakegris.\bigskip

\subsection{\textbf{Goder morgon eller Goder afton}}

Goder morgon, goder morgon

båd’ herre och fru

vi önskar eder alla

en fröjdefull jul.\bigskip

Goder morgon, goder morgon

välkommen var gäst!

vi önskar eder alla

en fröjdefull fest.\bigskip

Goder afton, goder afton

båd’ herre och fru

vi önskar eder alla

en fröjdefull jul. \bigskip

Goder afton, goder afton

välkommen var gäst!

vi önskar eder alla

en fröjdefull fest.\bigskip

\subsection{\textbf{En sockerbagare}}

En sockerbagare här bor i staden,

han bakar kakor mest hela dagen.

Han bakar stora, han bakar små,

han bakar några med socker på. \bigskip

Och i hans fönster hänger julgranssaker

och hästar, grisar och pepparkakor.

Och är du snäller så kan du få,

men är du stygger så får du gå. \bigskip


\subsection{\textbf{God morgon mitt herrskap }}

God morgon mitt herrskap! Här kommer Lussebrud;

hon kommer till Er med stor ära.

Lucian sig påminner att få roa Er en stund,

om hon Er vid hälsan finner i denna morgonstund.

Stånder upp nu, mitt herrskap, och ät och må väl!

Det fröjdar Lucian av hjärtat!

Hon önskar Eder alla en fröjdefull jul.

Från olyckor alle bevare Eder Gud!

Goder afton, goder afton

Båd herre och fru.

Vi önskar eder alla

En fröjdefull jul 

\subsection{\textbf{Tomtarnas Julnatt}}

Midnatt råder, tyst det är i husen, tyst i husen.

Alla sova, släckta äro ljusen, äro ljusen. \bigskip

Ref:

Tipp, tapp, tipp, tapp,

tippe-tippe-tipp, tapp.

Tipp, tipp, tapp. \bigskip

Se, då krypa tomtar upp ur vrårna, upp ur vrårna.

Lyssna, speja, trippa fram på tårna, fram på tårna.

Ref... \bigskip

Snälla folket låtit maten rara, maten rara

stå på bordet åt en tomteskara, tomteskara.

Ref... \bigskip

Hur de mysa, hoppa upp bland faten, upp bland faten,

tissla, tassla: "God är julematen, julematen".

Ref... \bigskip

Gröt och skinka, lilla äppelbiten, äppelbiten,

tänk, så rart det smakar Nisse liten, Nisse liten!

Ref... \bigskip

Nu till lekar! Glada skrattet klingar, skrattet klingar,

runt om granen skaran muntert svingar, muntert svingar.

Ref...\bigskip

Natten lider. Snart de tomtar snälla, tomtar snälla,

kvickt och näpet allt i ordning ställa, ordning ställa.

Ref... \bigskip

Sedan åter in i tysta vrårna, tysta vrårna

tomteskaran tassar nätt på tårna, nätt på tårna.

Ref... \bigskip

\subsection{\textbf{Nu är det jul igen}}

Nu är det jul igen

och nu är det jul igen

och julen varar väl till påska.

Nu är det jul igen

och nu är det jul igen

och julen varar väl till påska.

Det var inte sant

och det var inte sant,

för däremellan kommer fasta.

Det var inte sant

och det var inte sant,

för däremellan kommer fasta.

\subsection{\textbf{Rudolf med röda mulen}}

Rudolf med röda mulen

hette en helt vanlig ren

som blivit kall om mulen,

därav kom dess röda sken.

Rudolf fick alltid höra

"Se, han har sitt dimmljus på",

att han blev led åt detta,

är en sak man kan förstå. \bigskip

Men en mörk julaftonskväll

Tomtefar han sa:

"Vill du inte Rudolf säg,

med din mule lysa mig."

Allt se'n den dagen renen

tomtens egen släde drar.

Rudolf med röda mulen

lyser väg för tomtefar. 

\subsection{\textbf{Hej Tomtegubbar}}

Hej tomtegubbar slå i glasen

och låt oss lustiga vara.

Hej tomtegubbar slå i glasen

och låt oss lustiga vara.

En liten tid vi leva här

med mycken möda och stort besvär.

Hej tomtegubbar slå i glasen

och låt oss lustiga vara. 

\subsection{\textbf{Rudolf med becquerellen}}

Mel: Rudolf med röda mulen\bigskip

Rudolf med becquerellen, var en ganska vanlig ren;

upplevde Tjernobyl smällen - därav kom hans gröna sken.

Rudolfs alla små kamrater avskyr att gå utomhus

och se den nukleära renen lysa som ett varningsljus.\bigskip

Natten är rätt konstig sen

vår Rudolf blivit grön.

"Ge dig av!" är allas bön.

"Du smälter ju bort snön!"\bigskip

Alltsen Tjernobyl vandrar Rudolf för sig själv, allén,

lyser för sig själv och blir nog underlig och schizofren. 

\subsection{\textbf{Vem kan jula förutan gran? }}

Mel: Vem kan segla förutan vind?\bigskip

Vem kan jula förutan gran?

Vad är jul utan klappar?

Vem kan pröjsa för julefröjd

utan att fälla tårar? \bigskip

Jag kan jula förutan gran.

Jag kan avstå från klappar,

men ej pröjsa för julefröjd

utan att fälla tårar. 

\subsection{\textbf{Tomtevisan}}

Mel: Kalmarvisan\bigskip


$\|\:$ För uti tomteland,

ja där är det alltid snö $\:\|$

och det blir aldrig tö.

Hej dick

Hej dack

Jag slog i

Och vi drack

Hej dickom dickom dack

Hej dickom dickom dack

För uti tomteland ja

där är det alltid snö

och det blir aldrig tö\bigskip

$\|\:$ När så tomten kommer hem

kommer tomtegumman ut $\:\|$

och är stor i sin trut.

Hej dick...\bigskip

$\|\:$ Var är klapparna du fått?

Jo, dem har jag supit bort $\:\|$

Upp på Nordpolens slott.

Hej dick...\bigskip

$\|\:$ Rudolf, renen vår

satt på Kåren igår $\|\:$

och var full som ett ….älg

Hej dick...\bigskip

\subsection{\textbf{Nu höjas tusen fyllda glas}}

Mel: Nu tändas tusen juleljus\bigskip

Nu höjas tusen fyllda glas,

kring bordets glada rund,

och tusen, tusen, skålar ock,

vi tömmer per sekund.\bigskip

Och från butelj och fat i kväll,

oss öl och brännvin bärs,

tills hel- och halvan alla fått,

och mången och en ters. 

\subsection{\textbf{Töntarnas julenatt}}

Mel: Midnatt råder\bigskip

Julfrid råder, ute skiner ljusen, skiner ljusen.

Ingen sover, det är larm till tusen, larm till tusen.

Tipp tapp, tipp tapp, tippe tippe tipslapp,

tipp, tipp, tapp.\bigskip

Se då krälar tomtar fram på gatan, fram på gatan.

Tigga, sälja, supa, leva satan, leva satan.

Tipp tapp...\bigskip

Tovigt skägg som stinker utav spriten, utav spriten,

väcker avsky hos den som är liten, som är liten.

Tipp tapp...\bigskip

Synen utav tomtar som vomera, som vomera,

gör att ingen tror på tomten mera, tomten mera.

Tripp trapp, tipp tapp, sopetipp och julklapp,

tipp, topp, tupp.\bigskip

\subsection{\textbf{Gläns över sjö och strand}}

Gläns över sjö och strand,

Stjärna ur fjärran,

Du som i Österland

Tändes av Herran.

Stjärnan från Betlehem

Leder ej bort, men hem.

Barnen och herdarna

Följa dig gärna,

Strålande stjärna,

Strålande stjärna. \bigskip

Natt över Judaland,

Natt över Sion.

Borta vid västerrand

Slocknar Orion.

Herden, som sover trött,

Barnet, som slumrar sött,

Vakna vid underbar

Korus av röster,

Skåda en härligt klar

Stjärna i öster;\bigskip

Gånga från lamm och hem,

Sökande Eden.

Stjärnan från Betlehem

Visar dem leden

Fram genom hindrande

Jordiska fängsel

Hän till det glindrande

Lustgårdens stängsel.

Hän till det glindrande

Lustgårdens stängsel.\bigskip

Armar där sträckas dem,

Läppar där viska,

Viska och räckas dem,

Ljuva och friska:

Stjärnan från Betlehem

Leder ej bort, men hem.

Barnen och herdarna

Följa dig gärna,

Strålande stjärna,

Strålande stjärna. 

\subsection{\textbf{Hej tomtegubbar v2 }}


Hej, tomtegubbar, vrid på gasen

och tag er sjäva av daga.

Rödsprit vi hällt i alla glasen

och fjäril’n vinglar på Haga.

Av aquavit, man får kolik,

och alla tomtar skola-stå-lik.

Hej, tomtegubbar, vrid på gasen

och tag er själva av daga.

\subsection{\textbf{Hej tomtegubbar v3}}


Hej, tomtegubbar, lyft på luvan

och borsta dammet ur skägget.

Nu skall vi ta och bota snuvan

med extra påtårstillägget.

En liten smutt, uppå en hutt,

den kan på livsanden sätta sprutt.

Hej, tomtegubbar, öppna slussen,

för nu går hutten med bussen.

\subsection{\textbf{Kväde till lucian Benke}}

Mel: Staffan var en stalledräng. \bigskip


Benke var en stalledräng

som hade en liten hjärna.

Med sina IQ 75

var han ingen stjärna.

Handelshögen blev hans hem.

Han behövde aldrig mera tänka. \bigskip

Vacker är han som en Gud

och har en liten hjärna.

I sin konstifika skrud

är han ingen tärna.

Vår lucia då han blev

Glansen från hans leende den blänka. \bigskip

Med hans kropp gick upp ett ljus,

Benke blev en stjärna.

Femton kilo julesnus

finns det i hans hjärna.

Handelshögen blev hans hem.

Han behövde aldrig mera tänka. \bigskip

Benke är en ljusets fe,

som har snus i hjärnan.

Men ingen bryr sig väl om det,

här uti Tavernan.

Handelshögen är hans hem.

Han behöver aldrig mera tänka.\bigskip

\subsection{\textbf{Låtnamn}}

Text: 

Mel: \bigskip


\newpage