\fakesection{Sommarvisor}
\fancypagestyle{Sommarvisor}{
    \fancyhead{} % clear all header fields
    \fancyhead[LE,RO]{\textbf{Sommarvisor}}
}
\pagestyle{Sommarvisor}
%\begin{SongText}[Änglamark]
%    \begin{SongInfo}
%        Text:Evert Taube
%    \end{SongInfo}
%    \begin{SongVerse}
%        Kalla den änglamarken eller himlajorden om du vill,\\*%
%        jorden vi ärvde och lunden den gröna.\\*%
%        Vildrosor och blåsippor och lindblommor och kamomill\\*%
%        låt dem få leva, de är ju så sköna.
%    \end{SongVerse}
%    \begin{SongVerse}
%        Låt barnen dansa som änglar kring lönn och alm,\\*%
%        leka tittut mellan blommande grenar.\\*%
%        Låt fåglar leva och sjunga för oss sin psalm,\\*%
%        låt fiskar simma kring bryggor och stenar.\\*%
%        Sluta att utrota skogarnas alla djur!\\*%
%        Låt örnen flyga, låt rådjuren löpa!\\*%
%        Låt sista älven som brusar i vår natur\\*%
%        brusa alltjämt mellan fjällar och gran och fur!
%    \end{SongVerse}
%    \begin{SongVerse}
%        Kalla den änglamarken eller himlajorden om du vill,\\*%
%        jorden vi ärvde och lunden den gröna.\\*%
%        Vildrosor och blåsippor och lindblommor och kamomill\\*%
%        låt dem få leva, de är ju så sköna.
%    \end{SongVerse}
%\end{SongText}
\begin{SongText}[Den blomstertid nu kommer]
    \begin{SongVerse}
        Den blomstertid nu kommer\\*%
        med lust och fägring stor.\\*%
        Du nalkas ljuva sommar\\*%
        Då gräs och grödor gror.\\*%
        Med blid och livlig värma\\*%
        till allt som varit dött;\\*%
        sig solens strålar närma\\*%
        och allt blir återfött. \\*%
        De fagra blomsterängarna\\*%
        Och åkerns ädla säd.\\*%
        De rika örtesängar\\*%
        Och lundens gröna träd.\\*%
        De skola oss påminna,\\*%
        Guds godhets rikedom;\\*%
        att vi den nåd besinna\\*%
        som räcker året om.
    \end{SongVerse}
\end{SongText}
\begin{SongText}[Summan kummer]
    \begin{SongInfo}
        text: När-Revyn\\*%
        Sjunges med fördel på Absolut Gotland.
    \end{SongInfo}
    \begin{SongVerse}
        Ref:\\*%
        Summan kummer me sol\\*%
        kummit yvar hällmark u stein,\\*%
        yvar förfädas bain, yvar martall, yvar ain.\\*%
        Yvar maurar u mack leiksum yvar pinnsvein u rack\\*%
        Skeinar soli igen pa ladingen.
    \end{SongVerse}
    \begin{SongVerse}
        Summar de är när soli kummar yvar Austagarn,\\*%
        summar de är när brimsar brummar\\*%
        u myggar bleir me barn.\\*%
        Summar de är när lycke bjaudar sorgi upp till dans.\\*%
        Visst skudd de vare trist skudd de var um\\*%
        summan inte fanns.
    \end{SongVerse}
    \begin{SongVerse}
        Ref...
    \end{SongVerse}
    \begin{SongVerse}
        Vackat kan vare mosse pa en gammel sprucken stain.\\*%
        Vackat kan var u hald si bei fast bärgningi är klain.\\*%
        Vackat kan vare sånt som kalles fäult nån annanstans.\\*%
        Visst skudd de vare trist skudd de va um\\*%
        Gotland inte fanns.
    \end{SongVerse}
\end{SongText}
\begin{SongText}[Nu grönskar det]
    \begin{SongInfo}
        mel:Ur Bondekantaten (J S Bach)
    \end{SongInfo}
    \begin{SongVerse}
        Nu grönskar det i dalens famn\\*%
        nu doftar äng och lid.\\*%
        Kom med, kom med på vandringsfärd\\*%
        i vårens glada tid!\\*%
        Var dag är som en gyllne skål\\*%
        till bredden fylld med vin.\\*%
        Så drick, min vän, drick sol och\\*%
        doft, ty dagen, den är din!\\*%
        Långt bort från stadens gråa hus\\*%
        vi glatt vår kosa styr\\*%
        och följer vägens vita band\\*%
        mot ljusa äventyr.\\*%
        Med öppna ögon låt oss se\\*%
        på livets rikedom,\\*%
        som gror och sjuder överallt\\*%
        där våren går i blom!
    \end{SongVerse}
\end{SongText}
\begin{SongText}[Längtan till landet]
    \begin{SongInfo}
        text: Herman Sätherberg
    \end{SongInfo}
    \begin{SongVerse}
        Vintern rasat ut bland våra fjällar,\\*%
        drivans blommor smälta bort och dö,\\*%
        himlen ler i vårens ljusa kvällar,\\*%
        solen kysser liv i skog och sjö.
    \end{SongVerse}
    \begin{SongVerse}
        Snart är sommarn här i purpurvågor,\\*%
        guldbelagda, azurskiftande\\*%
        ligga ängarne i dagens lågor,\\*%
        och i lunden dansa källarne.
    \end{SongVerse}
    \begin{SongVerse}
        Ja, jag kommer! Hälsen, glada vindar,\\*%
        ut till landet, ut till fåglarne,\\*%
        att jag älskar dem, till björk och lindar,\\*%
        sjö och berg, jag vill dem återse.
    \end{SongVerse}
    \begin{SongVerse}
        Se dem än som i min barndoms stunder,\\*%
        följa bäckens dans till klarnad sjö,\\*%
        trastens sång i furuskogens lunder,\\*%
        vattenfågelns lek kring fjärd och ö.
    \end{SongVerse}
\end{SongText}
\newpage