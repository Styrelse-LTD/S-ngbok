\fakesection{Snapsvisor}
\fancypagestyle{Snapsvisor}{
    \fancyhead{} % clear all header fields
    \fancyhead[LE,RO]{\textbf{Snapsvisor}}
}
\pagestyle{Snapsvisor}
\subsection{\textbf{Om snapsen}}
Om man nu dricker snaps så är det fullständigt uteslutet att peta i
sig några droppar utan att sjunga en sång därtill.
Snapsarnas historia är lång och bristfälligt dokumenterad. Genom
muntlig tradition har dock en del kunskaper bevarats till
eftervärlden.\\%
I den grå forntiden var snapsarnas antal i princip oändligt. Under
senare delen av 1800-talet krympte antalet till maximalt 17. Dessa
känner vi fortfarande namnen på:
\begin{multicols}{2}[]
    \begin{enumerate}
        \item Helan
        \item Halvan
        \item Tersen
        \item Quarten
        \item Quinten
        \item Sexten
        \item Septen
        \item Rivan
        \item Rafflan
        \item Rännan
        \item Smuttan
        \item Smuttans unge
        \item Lilla Manasse
        \item Lilla Manasses brorsa
        \item Femton dropppar
        \item Kreaturens återuppståndelse
        \item Ett evigt liv
    \end{enumerate}
\end{multicols}
Alla sjutton dricks endast undantagsvis och då av personer som
inte har alltför stora ambitioner för resten av aftonen.\\%
I modern tid är det vanligt att man låter sig sjunga några fler visor
genom att inte tömma snapsglaset varje gång. Den något ryska
seden är dock att helan till varje pris ska tömmas i botten.\\%
Traditionen är att alltid börja sjunga Helan!
\input{Songs/Snapsvisor/Fetasmåryskor.tex}
\input{Songs/Snapsvisor/Helangår.tex}
\begin{SongText}[Helan var bra]
    \begin{SongInfo}
        Mel: Å j’änta å ja
    \end{SongInfo}
    \begin{SongVerse}
        Helan var bra, nu ska vi ta\\*%
        halvan i detta rycket.\\*%
        En går väl an, men två är minsann\\*%
        inte ett dugg för mycket.
    \end{SongVerse}
    \begin{SongVerse}
        Ensam är stark, två river mer\\*%
        så skynda er nu att slänga den ner\\*%
        kanske ni kan få mer om en stund,\\*%
        ja skål på er allihopa.
    \end{SongVerse}
\end{SongText}

%\begin{SongText}[Hur länge skall på borden]
%    \begin{SongInfo}
%        Mel: Jag minns den ljuva tiden
%    \end{SongInfo}
%    \begin{SongVerse}
%        Hur länge ska på borden\\*%
%        den lilla halvan stå.\\*%
%        Ska snart ej höras orden\\*%
%        nu halvan går låt gå.\\*%
%        Det ärvda vikingasinne\\*%
%        till supen trår igen\\*%
%        och helans trogna minne\\*%
%        i halvan går igen.
%    \end{SongVerse}
%\end{SongText}

\begin{SongText}[Hej feskegubbar!]
    \begin{SongInfo}
        Mel: Hej tomtegubbar
    \end{SongInfo}
    \begin{SongVerse}
        Hej feskegubbar, dra i näten\\*%
        och hala hem några dunkar!\\*%
        Hej feskegubbar, ej förgäten\\*%
        att taga duktiga klunkar!\\*%
        Nu Helan går, vi mer förmår.\\*%
        Vänd botten upp och sen gutår!\\*%
        Hej feskegubbar, ej förgäten\\*%
        att taga duktiga klunkar!
    \end{SongVerse}
\end{SongText}

\begin{SongText}[Humlorna]
    \begin{SongInfo}
        Sjungs om drinken "Geting" dricks\\*%
        Mel: Här kommer Karl-Alfred boy
    \end{SongInfo}
    \begin{SongVerse}
        Vi äro små humlor vi, bzz-bzz!\\*%
        vi äro små humlor vi, bzz-bzz!\\*%
        Vi äro små humlor som tar oss en geting,\\*%
        vi äro små humlor vi, bzz-bzz!
    \end{SongVerse}
\end{SongText}

\begin{SongText}[Denna thaft]
    \begin{SongInfo}
        Mel: Helan
    \end{SongInfo}
    \begin{SongVerse}
        Denna thaft\\*%
        är den bätha thaft thythemet haft.\\*%
        Denna thaft\\*%
        är den bätha thaft dom haft.\\*%
        Och den thom inte har nån kraft,\\*%
        han dricka thkall av denna thaft.\\*%
        Denna thaft, till landth, till sjöth, till havth.
    \end{SongVerse}
\end{SongText}

\begin{SongText}[Nu tar vi den]
    \begin{SongInfo}
        Mel: O Tannenbaum
    \end{SongInfo}
    \begin{SongVerse}
        $\:\|$ Nu tar vi den$\:\|$
    \end{SongVerse}
\end{SongText}

\begin{SongText}[Elarens snapsvisa]
    \begin{SongInfo}
        Mel: Så lunka vi så småningom
    \end{SongInfo}
    \begin{SongVerse}
        Go’ vänner gör som broder Seth,\\*%
        låt nubben sista resan ta..\\*%
        Ty minns att varje liten skvätt\\*%
        gör elar’n god och gla’.
    \end{SongVerse}
    \begin{SongVerse}
        $\:\|$ Upp på kårhuset är det sed\\*%
        att låta helan snabbt gå ned,\\*%
        ta sig se’n dito en,\\*%
        dito två, dito tre.\\*%
        Det gör var elare! $\:\|$
    \end{SongVerse}
\end{SongText}

\input{Songs/Snapsvisor/Törstenrasar.tex}
\begin{SongText}[Finsk snapsvisa (kort)]
    \begin{SongVerse}
        NU!!
    \end{SongVerse}
\end{SongText}

\input{Songs/Snapsvisor/Finsksnapsvisalång.tex}
\input{Songs/Snapsvisor/Snöret.tex}
\begin{SongText}[Öppna bäsken]
    \begin{SongInfo}
        Mel: Öppna landskap
    \end{SongInfo}
    \begin{SongVerse}
        Jag trivs bäst bland öppna flaskor,\\*%
        känna doften av en Bäsk.\\*%
        Som jag lagrat några månader,\\*%
        då den blir som allra bäst.\\*%
        För jag bränner ju mitt brännvin själv,\\*%
        och kryddar det med malörtsblom.\\*%
        Mitt sinne lyfts av dryckens krav,\\*%
        där den glimmar i sitt glas.
    \end{SongVerse}
\end{SongText}

\begin{SongText}[Skogskokarvisa]
    \begin{SongInfo}
        Mel: Mors lilla Olle
    \end{SongInfo}
    \begin{SongVerse}
        Mors lilla Olle i skogen gick\\*%
        leta’ bland furorna, fast han ej fick.\\*%
        Sågade sju och en halv\\*%
        och drog hem,\\*%
        kokade O.P. och cognac av dem.\\*%
        Mor nu fick syn på’n\\*%
        gav till ett vrål:\\*%
        Kokar du skogen till ren alkohol?\\*%
        - Klart, sade Olle, och allt som jag fällt\\*%
        är till vår fest här på gården beställt.
    \end{SongVerse}
\end{SongText}

\begin{SongText}[Ack’va visa]
    \begin{SongInfo}
        Mel: Vi går över daggstänkta berg
    \end{SongInfo}
    \begin{SongVerse}
        Vi går över ån efter sprit, fallera,\\*%
        men efter vatten går vi ej en bit, fallera.\\*%
        Ja drick kära broder fast näsan är röder\\*%
        ty tids nog så blir den ack’ va vit, fallera.
    \end{SongVerse}
\end{SongText}

\begin{SongText}[Ångbåten]
    \begin{SongInfo}
        Mel: Jazzgossen
    \end{SongInfo}
    \begin{SongVerse}
        Och så kommer det en ångbåt\\*%
        som säger tuut- tuut- tuut,\\*%
        och så kommer det en ubåt\\*%
        som säger...\\*%
        (Varpå snapsen sveps , och gurglas innan den sväljs)
    \end{SongVerse}
\end{SongText}

\begin{SongText}[Ode till flygbensinet]
    \begin{SongInfo}
        Mel: Helan går\\*%
        Denna visa räddades ur papperskorgen av sedemera örlogskapten KG Lewenhaupt i nådens år 1964.
    \end{SongInfo}
    \begin{SongVerse}
        Flygbensin är drycken för vart fyllesvin\\*%
        Flygbensin är rått som terpentin\\*%
        Det piggar upp din trötta kropp\\*%
        och hettar upp ditt blodomlopp\\*%
        Flygbensin ...\\*%
        är Mannfreds medicin.
    \end{SongVerse}
\end{SongText}

\begin{SongText}[Tallen]
    \begin{SongInfo}
        Mel: Längtan till landet
    \end{SongInfo}
    \begin{SongVerse}
        Fordom odlade man vindruvsranka\\*%
        av vars saft man gjorde ädelt vin.\\*%
        Nu man pressar saften av en planka,\\*%
        doftande av äkta terpentin.\\*%
        Höj nu bägaren, o broder, syster\\*%
        och låt svenska skogen rinna kall\\*%
        ner i strupen och, om du är dyster,\\*%
        låt oss dricka upp en liten tall.
    \end{SongVerse}
\end{SongText}

\begin{SongText}[En liten fyllhund]
    \begin{SongInfo}
        Mel: Mors lilla Olle
    \end{SongInfo}
    \begin{SongVerse}
        En liten fyllhund på krogen satt,\\*%
        rosor på kinden - men blicken var matt,\\*%
        Läpparna små, liksom näsan var blå.\\*%
        "Ack, om jag kunde så skulle jag gå."
    \end{SongVerse}
\end{SongText}

\begin{SongText}[Nu tar vi rus]
    \begin{SongInfo}
        Mel: Nu är det ljus, här i vårt hus
    \end{SongInfo}
    \begin{SongVerse}
        Nu tar vi rus här i vårt hus\\*%
        drick tills du faller och fall ner igen.\\*%
        Baren är din,\\*%
        tag dig en gin,\\*%
        spy och försvinn.\\*%
        Mamma, pappa, alla fatta glasen,\\*%
        ställa er som jag till fylleasen.\\*%
        Låt nu spriten\\*%
        och akvaviten\\*%
        få mera sällskap\\*%
        - ge hit en!
    \end{SongVerse}
\end{SongText}

\input{Songs/Snapsvisor/Merabrännvin.tex}
\begin{SongText}[Ser du stjärnan i det blå]
    \begin{SongInfo}
        Mel: Ser du stjärnan i det blå
    \end{SongInfo}
    \begin{SongVerse}
        $\:\|$ Ser du stjärnan i det blå?\\*%
        Ta en sup så ser du två.\\*%
        Tar du sedan något mer,\\*%
        så ser du fler. $\:\|$
    \end{SongVerse}
\end{SongText}

\begin{SongText}[Byssan lull]
    \begin{SongInfo}
        Mel: Byssan lull
    \end{SongInfo}
    \begin{SongVerse}
        $\:\|$ Byssan lull utav brännvin blir man full,\\*%
        och slipsen man doppar i smöret. $\:\|$\\*%
        Ja, näsan den blir röd,\\*%
        och ögonen får glöd,\\*%
        och tusan så bra blir humöret.
    \end{SongVerse}
\end{SongText}

\input{Songs/Snapsvisor/Måsen.tex}
\begin{SongText}[Förtsa snapsen heter göken]
    \begin{SongInfo}
        Mel: Räven raskar över isen
    \end{SongInfo}
    \begin{SongVerse}
        Första snapsen heter göken.\\*%
        Första snapsen heter göken.\\*%
        Får jag lov, får jag lov,\\*%
        Att byta byxor med fröken?\\*%
        Andra snapsen den var värre.\\*%
        Andra snapsen den var värre.\\*%
        Får jag lov, får jag lov,\\*%
        Att byta byxor med min herre?\\*%
        Mina byxor är himmelsblå,\\*%
        Men med dina är det si och så.\\*%
        Så, får jag lov, får jag lov,\\*%
        Att byta byxor med göken?
    \end{SongVerse}
\end{SongText}

\begin{SongText}[Helangorakatt]
    \begin{SongInfo}
        Mel: Vi gå över daggstänkta berg
    \end{SongInfo}
    \begin{SongVerse}
        Det var en gång en helangorakatt, Fallera\\*%
        Som älskade en vanlig bonakatt, Fallera\\*%
        Och följden blev en jamare\\*%
        Och den var inte tamare\\*%
        Den kallas för HALVANGÅR…a-katt, Fallera.
    \end{SongVerse}
\end{SongText}

\begin{SongText}[Solen]
    \begin{SongInfo}
        Mel: Champtown races (Trad USA)
    \end{SongInfo}
    \begin{SongVerse}
        Solen den går upp och ner, doda, doda\\*%
        Jag ska aldrig supa mer, hej didoda dej\\*%
        Hej didoda dej, hej didoda dej\\*%
        Jag ska aldrig supa mer, hej didoda dej
    \end{SongVerse}
    \begin{SongVerse}
        Men detta det var inte sant, doda, doda\\*%
        Imorgon gör jag likadant, hej didoda dej\\*%
        Hej didoda dej, hej didoda dej\\*%
        Imorgon gör jag likadant, hej didoda dej
    \end{SongVerse}
    \begin{SongVerse}
        Chalmers är ett jävla skit, doda, doda\\*%
        Dom kan inte gö teknik, hej didoda dej\\*%
        Hej didoda dej, hej didoda dej\\*%
        Dom kan inte gö teknik, hej didoda dej
    \end{SongVerse}
    \begin{SongVerse}
        Detta det var ju faktiskt sant, doda, doda\\*%
        Imorgon gör dom likadant, hej didoda dej...
    \end{SongVerse}
    \begin{SongVerse}
        KTH är kunglig skam, doda, doda\\*%
        Inget mer electro-lab, hej didoda dej\\*%
        Hej didoda dej, hej didoda dej\\*%
        Inget mer electro-lab, hej didoda dej
    \end{SongVerse}
    \begin{SongVerse}
        Detta det var ju faktiskt sant, doda, doda\\*%
        Imorgon gör dom likadant, hej didoda dej...
    \end{SongVerse}
    \begin{SongInfo}
        Text: MVP3 dat21 (Sista två verserna)
    \end{SongInfo}
\end{SongText}

\begin{SongText}[Påfyllningssång]
    \begin{SongInfo}
        Mel: Mors lilla Olle
    \end{SongInfo}
    \begin{SongVerse}
        Helan så ensam i magen gick,\\*%
        Undrade varför ej sällskap han fick?\\*%
        Värden, ack säg var är halvan i kväll?\\*%
        Bed honom komma till oss är du snäll.
    \end{SongVerse}
    \begin{SongVerse}
        Värden ser glasen ger upp ett skri.\\*%
        Skynda sig sedan att fort fylla i.\\*%
        Nu har den kommit till rätta vår vän.\\*%
        Svep den nu innan den smiter igen.
    \end{SongVerse}
\end{SongText}

\begin{SongText}[Sill och Nubbe]
    \begin{SongInfo}
        Mel: Vi gå över daggstänkta berg
    \end{SongInfo}
    \begin{SongVerse}
        Till nubben så tager man sill, fallera\\*%
        Men också en ansjovis om man vill, fallera\\*%
        Men om man är oviss\\*%
        Om sillen är ansjovis\\*%
        Så tar man bara några nubbar till, fallera.
    \end{SongVerse}
\end{SongText}

\begin{SongText}[En gång i måna’n]
    \begin{SongInfo}
        Mel: Mors lilla Olle
    \end{SongInfo}
    \begin{SongVerse}
        En gång i måna’n är månen full,\\*%
        men aldrig jag sett honom ramla omkull.\\*%
        Stum av beundran hur mycket han tål,\\*%
        höjer jag glaset och utbringar skål.\\*%
        Höjen nu glasen och dricken ur.\\*%
        Nu, kära bröder, står kvarten i tur.\\*%
        Nubben, den giver oss ny energi.\\*%
        Säkert den minskar vårt livs entropi.
    \end{SongVerse}
\end{SongText}

\begin{SongText}[Byssan lull v2]
    \begin{SongInfo}
        Mel: Byssan lull
    \end{SongInfo}
    \begin{SongVerse}
        $\|\:$ Bysan lull snart så blir du så full;\\*%
        Då får du sju jamare på festen. $\:\|$\\*%
        Den första får du nu,\\*%
        Den andra får du sen,\\*%
        De sista får du smyga fram ur västen.\\*%
        $\|\:$ Byssan lull gå på sittning och bli full,\\*%
        för oss har det blivit en vana. $\:\|$\\*%
        Den första var en bäsk,\\*%
        Den andra var en bäsk,\\*%
        Den tredje var bäskast av alla.
    \end{SongVerse}
\end{SongText}

\begin{SongText}[Låt tersen gå]
    \begin{SongInfo}
        Mel: I sommarens soliga dagar
    \end{SongInfo}
    \begin{SongVerse}
        Med sång skall man hålla ett gill-e\\*%
        Så där så det ekar i natten den still-e.\\*%
        Man sjunger till supen och sill-en,\\*%
        Sätt igång – låt tersen gå –\\*%
        Hallå, hallå!\\*%
        Du som är ung, var med och sjung\\*%
        Sitt inte tyst och trög och tung.\\*%
        Man ska va’ gla’, tra la la la\\*%
        Med sång och snaps så går det så bra\\*%
        Trots rymdsatelliterna rusa.\\*%
        Låt tersen gå – låt gå –\\*%
        Hallå, hallå!
    \end{SongVerse}
\end{SongText}

\begin{SongText}[Imse vimse hutt]
    \begin{SongInfo}
        Mel: Imse vimse spindel
    \end{SongInfo}
    \begin{SongVerse}
        Imse vimse blir man\\*%
        Av en liten hutt\\*%
        Pulsen börjar öka\\*%
        Hjärtat tar ett skutt\\*%
        Knäna skälver, näsan blir blå\\*%
        -fast det är så läskigt\\*%
        vågar vi ändå.
    \end{SongVerse}
\end{SongText}

\begin{SongText}[Nubben goa]
    \begin{SongInfo}
        Mel: Gubben noa
    \end{SongInfo}
    \begin{SongVerse}
        Nubben goa, nubben goa\\*%
        Är en hedersdryck.\\*%
        När den går till magen,\\*%
        Blir man lätt i tagen.\\*%
        Nubben goa, nubben goa\\*%
        Tar man med en knyck.
    \end{SongVerse}
    \begin{SongVerse}
        Nubben goa, nubben goa\\*%
        Är en hederssup.\\*%
        Uti alko-hålet\\*%
        Töm den, om du tål’et.\\*%
        Nubben goa, nubben goa…
    \end{SongVerse}
    \begin{SongVerse}
        (När sången sjungits en gång kan det efterfrågas "Omstart fast på --")\\*%
        --engelska\\*%
        Nubben goa, nubben goa\\*%
        Is a famous drink\\*%
        Said the Prince of Wales\\*%
        Ever since det gales:\\*%
        Nubben goa, nubben goa\\*%
        Is a famous drink.
    \end{SongVerse}
    \begin{SongVerse}
        --Tyska\\*%
        Nubben goa, nubben goa\\*%
        Ist ein Ehrenschnaps.\\*%
        Wann das Glas wir lehren,\\*%
        will es sich vermehren.\\*%
        Nubben goa, nubben goa\\*%
        Ist ein Ehrenschnaps
    \end{SongVerse}
    \begin{SongVerse}
        --Latin\\*%
        Nubben goa, nubben goa\\*%
        Est in pocula.\\*%
        Nunc es den bibendum,\\*%
        hux.flux capiendum.\\*%
        Nubben goa, nubben goa\\*%
        Tempus est att ta.
    \end{SongVerse}
\end{SongText}

\begin{SongText}[Other lands...]
    \begin{SongInfo}
        Mel: Längtan till landet
    \end{SongInfo}
    \begin{SongVerse}
        Other lands have vineyards without number\\*%
        and their fruit becomes the native wine.\\*%
        But the Swedes squeeze liquor out of lumber,\\*%
        smelling fragrantly of terpentine.\\*%
        Now if you can take another measure,\\*%
        cool and clear go gurgling down the line.\\*%
        Let's enjoy the Swedish forest pleasure.\\*%
        Let's fill up and drink another pine!
    \end{SongVerse}
\end{SongText}

\begin{SongText}[När gäddorna]
    \begin{SongVerse}
        När gäddorna leker i vikar och vass\\*%
        Och solen går ner bakom Sjöbloms dass.\\*%
        Ja, då är det vår…\\*%
        (se upp där nere – nu kommer den!)\\*%
        Å helan går!\\*%
        Sjung hopp fallerallan lej.
    \end{SongVerse}
\end{SongText}

\input{Songs/Snapsvisor/Jagvartallengång.tex}
\input{Songs/Snapsvisor/Nuskaviklämmasepten.tex}
\begin{SongText}[Jag tror, jag tror]
    \begin{SongInfo}
        Mel: Jag tror, jag tror på sommaren
    \end{SongInfo}
    \begin{SongVerse}
        Jag tror, jag tror på akvavit\\*%
        jag tror, jag tror på dynamit\\*%
        den ger en kraft att sjunga ut\\*%
        och inga krämpor blir akut.\\*%
        Man glömmer vardagslivets jäkt\\*%
        och känner stundens ruseffekt.\\*%
        En snaps, en skål, en trudelutt\\*%
        och sen så tar vi våran hutt.\\*%
    \end{SongVerse}
\end{SongText}

\begin{SongText}[Motsatsen till AI]
    \begin{SongInfo}
        Text: Fredrik Gustafsson 3e plats i SM i nyskrivna snapsvisor 2023\\*%
        Mel: Rövarnas visa
    \end{SongInfo}
    \begin{SongVerse}
        En Artificiell Intelligens den kan nu svara\\*%
        på alla frågor som du ställer och det är ej bara\\*%
        snart gör den robotkirurgi\\*%
        då ska vi arbetslösa bli\\*%
        men den kan ej sjunga och snapsa som vi\\*%
        för den saknar vår Mänskliga Idioti.
    \end{SongVerse}
\end{SongText}

\newpage