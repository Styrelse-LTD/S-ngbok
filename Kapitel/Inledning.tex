\fancypagestyle{Inledning}{
  \fancyhead{} % clear all header fields
  \fancyhead[LE,RO]{\textbf{Inledning}}
}
\pagestyle{Inledning}


Förening: \hrulefill\\*%

Stad: \hrulefill\\*%


%EDITORS NOTE: Kanske lägg till datum?

\begin{multicols}{2}
  Ägare: \hrulefill%

  Datum: \hrulefill%

  Ägare: \hrulefill%

  Datum: \hrulefill%

  Ägare: \hrulefill%

  Datum: \hrulefill%

  Ägare: \hrulefill%

  Datum: \hrulefill%

  Ägare: \hrulefill%

  Datum: \hrulefill%

  Ägare: \hrulefill%

  Datum: \hrulefill%

  \hfill

  Ovvenamn: \hrulefill%

  \hfill

  Ovvenamn: \hrulefill%

  \hfill

  Ovvenamn: \hrulefill%

  \hfill

  Ovvenamn: \hrulefill%

  \hfill

  Ovvenamn: \hrulefill%

  \hfill

  Ovvenamn: \hrulefill%

  \hfill
\end{multicols}

%Mail: \underline{\hspace{3cm}}


\newpage

\textbf{Förord}

Här tackade Peter "Sidde" Sidmar, DaT02 alla sina vännner som hjälpt honom
att få ihop bonkboken och det är väl ändå rätt att jag gör det samma.

Vi börjar med alla som skrev orginalet.

Sidde, DaT02:
\begin{itemize}
  \item Sammanställning
\end{itemize}
BuRRE, DaT01:
\begin{itemize}
  \item Alla (gammla) bilder
  \item Korrekturläsning
\end{itemize}
Chrisse, DaT99:
\begin{itemize}
  \item Drinktips
  \item låtar
  \item Sponsring
  \item Tryck \& bindning
  \item Korrekturläsning
\end{itemize}
Neander, DaT98:
\begin{itemize}
  \item Bonkutskottets historia
\end{itemize}

Som Sidde noterade i orginalet läser få dessa förord så mer tid än vad jag redan
har lagt kommer inte läggas ner. Jag vill ytterligt tacka alla dessa personer:

Nelly Olsson , Utomstånde
\begin{itemize}
  \item Korrekturläsning
  \item Prototyp av Stora Lila Beta
\end{itemize}
Frys, CEs17
\begin{itemize}
  \item Tillgång av Phästmesteriets Sångbok
  \item Traditionskonsultering
  \item Capsregler
\end{itemize}
%Sin, DaT21
%\begin{itemize}
%  \item All grafik
%\end{itemize}
Cornelia Wahlman DaT22
\begin{itemize}
  \item Korrekturläsning
  \item Kläddesignskonsultering
\end{itemize}

Ett tackkapitel skulle inte vara klar utan att tacka min sittande styrelse, föreningen och alla studenter jag pluggat/pluggar/kommer plugga med.

Hoppas ni finner ypperlig nytta av boken.

\newpage

\invisiblesection{Inledning}

\invisiblesection{\textbf{Förord}}

Den bok du härvid vidmakthåller är den samlade produkten av en mängd sannolikt onyktra individer och dubiösa idéer därifrån -- vilka bokens tryckare påträffat och funnit värda att föreviga.

Historien om denna skrifts tillkomst är av det komplexa slaget, många detaljer och fakta är för evigt av tiden förlorade eller bortglömda. Detta till trots är vårt mål att giva en så korrekt skildring vi förmår, utan att lämna någon väsentlig detalj.

\textbf{Svammeltext för den uttråkade om hur Bonkutskottetstartade för länge, länge, länge sedan.}

Nu är det ett antal år sedan vi startade Bonkutskottet (BU) och efter ett par hårda år av fester och plugg så kanske minnet sviker här och där. Här kommer i alla fall historien om födseln av BU. Hösten 1999 satt jag ordförande för LTD med världens bästa styrelse. Vi ordnade och fixade en massa för våra medlemmar. Vi fixade sittningar, tripper till Sjöslaget, Tjällossningen och Eskilstuna Wintergames etc. På den här tiden hade vi en boom som kallades IT boomen. Alla IT företag hade gott om pengar och visste inte vart de skulle göra av med allt. Företagen slet och drog i oss för att vi skulle börja jobba just hos dem. Ibland fick till och med förslag om att man skulle sluta plugga redan i ettan och tvåan och börja jobba istället. Det ni! Som ni förstår hade vi bråda tider i styrelsen och vi förstod ganska snart att vi inte kunde lämna över denna belastning på nästkommande styrelse. Jag föreslog att vi skulle starta ett utskott som tog hand om allt som hade med fester att göra eftersom det tog upp en stor del av arbetet i styrelsen. Jo vad händer då? Det är ju så om man yttrar något så får man ju självklart ta hand om det också. Men vad gör väl det när det handlar om fest?! Det tog inte lång tid innan folk anslöt sig till utskottet. Typ en minut efter att styrelsen beslutat att ett nytt utskott skulle skapas var vi sex medlemmar och 10 minuter efter det var vi åtta. Alla vi åtta var redan en väl fungerande och sammansvetsad grupp så några formella möten hade vi inte. Vi försökte en gång med ett informellt formellt möte med lite öl hemma hos vår Bonkutskotts suspect (Bus), Johan(Johan Andersson, dat98), men det slutade med kåren istället. Jag har till och med för mig att vi beslutade att BU inte skulle ha några formella möten. Detta för att vi skulle slippa ha årsmöten, ekonomi, skriva protokoll etc. Vi äskade alla pengar vi behövde från LTD styrelsen. Allt för att vi skulle kunna fokusera på det viktigaste, festerna. En ”uppskattad” del av vår n0llning bestod i att bonka en halvliter \(>\)4 procentig alkoholhaltig vätska. Jag antar det var därför vi kom på namnet bonkutskottet. Vi bestämde att alla bonkare skulle ha var sin bonk och som minst rymde 50 cl vätska. Ett märke fixade jag tillsammans med Hagenström(Robert Hagenström, dat98), som ritade bilden och Pets(Peter W Källklint, dat98), syster som kan latin. Det syddes endast upp 100 stycken märken. 10 märken var som skulle numreras från 1 till 10, plus bonkarens egna medlemsnummer. Vi skrev dit numren för att en bonkare i framtiden skulle kunna identifiera vilken oldie som gett bort märket. Märket jag pratar om är det som fortfarande är BUs logo/märke. För att fira denna händelse med det nya märket anordnade vi en resa till fjärran östern, närmare bestämt Eskilstuna och Wintergames. På den tiden fick man pris om man kom på första eller sista plats. Som bonkare ska man (om man inte är 100 procent säker på att vinna första platsen) vinna sistaplatsen i de tävlingar man deltar. Vi ”vann” stort i Eskilstuna. Efter Wintergames drog vi raskt till nästa fest, Tjällossningen i Örebro, där vi skulle delta i trehjulingsrace. Någon trehjuling hade vi inte med oss och vi hade så roligt i lokalen där vi bodde att vi inte ens kom upp till starten där de andra tävlade. Till sjöslaget fixade vi till en underbart fin fana. En kopia av bonkmärket limmades och syddes på fanan för hand. Mäktigt stolta drog vi till sjöss och slog klackarna i taket. Vart fanan tog vägen vet ingen. Den var synlig tillsammans med Cribbe(Christan Andersson, dat98), precis innan vi gick av båten men vi tror att fanan smet sin kos. Den kanske hade så kul på Sjöslaget att den ville åka ett par rundor till. Vad vet jag? Kanske väntar den på att vi bonkare ska komma tillbaks och ta en runda till?

/Jonas Neander, dat98, 2005-10-24

Efter en pandemi som satte alla i zoomlektioner och ändrade studentlivet på MDU(H) totalt, så började phester och n0llningen igen på riktigt. Under 2022, efter en nyinsatt styrelse som försökte för kung och fosterland att starta igång traditioner igen så hittades en fil av namnet "Bonkboken v1.001.pdf" liggandes i något skrymsle i våran google drive. Bonkboken visade sig vara en hel sångbok med alla låtar som man hört och många som glömts bort. Men vafan, det måste väll gå att göra något med den. Så vi skickade iväg Flora(Ebba Norlin, dat21) för att leta reda på alla dessa "bonkutskotts" medlemmar och kolla om dom var okej med att vi styckade deras verk. Uppenbarligen var dom det med tanke på att du läser detta. Så jag sitter nu som ordförande och lär mig LaTeX, letar sånger och seder för att nu kunna visa upp en sångbok som är både \textbf{Stor} och \textbf{Lila}.

Bonkutskottet blev med åren utbytt, bortglömd och/eller bytte namn till Phestphix som idag styr och ställer med LTD's Phester. Men namnet lever nu vidare med Bonkutskottet som en bonk-entusiast grupp.

/Ashley "MVP3" Björs, dat21, 2023-09-08\newpage

\invisiblesection{\textbf{Bonk}}

Nu kanske nu undrar vad en bonk är? Jo en bonk är ett verktyg som man kan använda för att på ett snabbt sätt förtära en stor mängd alkoholhaltig dryck t.ex. öl eller cider. Vissa delar av detta avlånga land kallas den för bong, dônk eller andra varianter istället av någon märklig anledning. Vissa delar har fel. En bonk är uppbyggd av en tratt och en slang som man sätter ihop så att man kan hälla ner drycken i slangen via tratten, ni förstår säkert. Det man sen gör är att man häller på dryck för att sedan bonka i sig den från slangänden.

\textbf{Regler för bonkning}

\begin{itemize}
  \item Bonken måste bestå av minst 50cl dryck.
  \item Drycken ska ha en procenthalt av minst 4,5\%.
  \item Man får inte "bita av" sin bonk utan man måste ta den i ett svep.
  \item Om en medlem av någon oförklarlig anledning skulle misslyckas med sin bonk måste denna ta 2st likadana bonkar inom 3 timmar.
  \item För att en bonk ska godkännas måste man hålla kvar den i magen i minst 30min.
  \item Det krävs 2st bonkutskottare som är närvarande vid bonktillfället för att officiellt godkänna den.
\end{itemize}\newpage

\invisiblesection{\textbf{Hur man bonkar}}

Att bonka är inte så svårt som det kan verka. Det är bara följa denna enkla beskrivning:

Fyll på genom att hålla änden på slangen högre än tratten (för att det inte ska rinna ut direkt) och sedan hälla ner drycken i tratten. Se till att ha en person som håller i tratten och för att underlätta ska hen ställa sig högre upp än den som ska bonka t.ex. på en bänk.

Inta rätt position. Den ska vara så nära den som håller bonken som möjligt för att drycken ska rinna så rakt ner som möjligt. Sen får man välja om man vill börja stående eller på knä. Det går fortare om man går ner på knä senare, men ger mer skum i magen också.

Nu tar man slangänden och håller den i höjd men munnen, se till att tratthållaren följer med ner med tratten för att undvika alkoholmissbruk.

Höj nu tratten så att vätskenivån hamnar precis vid slangkanten. Eventuellt skum i slangänden är tillåtet att suga i sig nu. Sätt munnen runt slangen och håll fast slangen lite lätt med tänderna. Håll också slangen ganska nära munnen med en valfri hand.

När man känner sig klar höjer man den fria handen och då lyfter tratthållaren tratten. Samtidigt som man höjer handen ska man vinkla huvudet uppåt och öppna svalget. Om man tidigare inte gick ner på knä, bör man göra det nu.

Har man gjort rätt kommer drycken att rinna rakt ner i magen. Nu är det bara att vänta tills allt runnit ner, det brukar gå på några sekunder bara. Nu kan det vid vissa tillfällen vara rätt bra att rapa lite, se bara till att det endast är luft som kommer upp.

Grattis nu ska du ha bonkat klart!!!\newpage

\invisiblesection{\textbf{Phelsökningsschema}}

\begin{adjustbox}{angle=-90}
  \begin{tabular}{|p{0.3\textheight}|p{0.3\textheight}|p{0.3\textheight}|}
    \hline
    \textbf{Symptom}                                                     & \textbf{Orsak} & \textbf{Åtgärd} \\
    \hline
    Drycken varken smakar eller tillfredställer. Skjortan känns blöt     &
    Munnen är inte öppen, eller så används glaset på fel del av ansiktet    &
    Träna hemma framför spegeln tills du behärskar tekniken                                                 \\
    \hline
    Drycken varken smakar eller tillfredställer. Öhlen är ovanligt ljus &
    Glaset är tomt                                                       &
    Hitta någon som köper dig en ny öhl                                                                     \\
    \hline
    Fötterna känns kalla och blöta                                       &
    Glaset står i fel vinkel                                             &
    Håll glasets öppna ände riktat mot taket                                                                \\
    \hline
    Baren är dimmig och ofokuserad                                       &
    Du tittar genom botten av ditt glas                                  &
    Hitta någon som köper dig en ny öhl                                                                     \\
    \hline
  \end{tabular}
\end{adjustbox}

\begin{adjustbox}{angle=-90}
  \begin{tabular}{|p{0.3\textheight}|p{0.3\textheight}|p{0.3\textheight}|}
    \hline
    \textbf{Symptom}                                                & \textbf{Orsak} & \textbf{Åtgärd}              \\
    \hline
    Baren rör på sig                                                &
    Du förs ut                                                      &
    Håller du på att flyttas till en annan bar? om inte, skrik att du blir kidnappad                                \\
    \hline
    Baren gungar                                                    &
    Du sitter i korsdrag                                            &
    Sätt dig i ett hörn, eller träd stolsryggen innanför skjortan                                                   \\
    \hline
    Väggen mitt emot dig är täckt av plattor och fluorescerande band &
    Du har fallit baklänges                                         &
    Om ingen står på din drickande arm och din öhl är full, ligg kvar. Be annars någon att säkra fast dig vid baren \\
    \hline
  \end{tabular}
\end{adjustbox}


%\begin{tabularx}{0.9\textwidth}{ 
%  | >{\centering\arraybackslash}X 
%  | >{\centering\arraybackslash}X 
%  | >{\centering\arraybackslash}X | }
%    \hline
%    \textbf{Symptom} & \textbf{Orsak} & \textbf{Åtgärd} \\
%    \hline
%    Drycken varken smakar eller tillfredställer. Skjortan känns blöt & 
%    Munnen är inte öppen, eller används glaset på fel del av ansiktet & 
%    Träna hemma framför spegeln tills du behärskar tekniken \\
%    \hline
%    Drycken varken smakar eller tillfredställer. Öhlen är ovqanligt ljus &
%    Glaset är tomt &
%    Hitta någon som köper dig en ny öhl \\
%    \hline
%    Fötterna känns kalla och blöta &
%    Glaset står i fel vinkel &
%    Håll glasets öppna ända riktat mot taket \\
%    \hline
%    Baren är dimmig och ofokuserad &
%    Du tittat genom botten av ditt glas &
%    Hitta någon som köper dig en ny öhl \\
%    \hline
%    Baren rör på sig &
%    Du förs ut &
%    Håller du på att flyttas till en annan bar? om inte, skrik att du blir kidnappad \\
%    \hline
%    Baren gungar &
%    Du sitter i korsdrag &
%    Sätt dig i ett hörn, eller träd stolsryggen innanför skjortan \\
%    \hline
%    Väggen mitt emot dig är täkt av plattor och fluorescerande band &
%    Du har fallit baklänges &
%    Om ingen står på din drickande arm och din öhl är full, ligg kvar. Be annars någon att säkra fast dig vid baren \\
%    \hline
%    
%    
%    
%\end{tabularx}
%
%\begin{tabularx}{0.9\textwidth}{ 
%  | >{\centering\arraybackslash}X 
%  | >{\centering\arraybackslash}X 
%  | >{\centering\arraybackslash}X | }
%
%    \hline
%    Ljuset dämpas. Din nästa och läppar blöder&
%    Du har fallit framåt &
%    Om ingen står på din drickande arm och din öhl är full, ligg kvar. Be annars någon att säkra fast dig vid baren \\
%    \hline
%    Du vaknar av att din säng är hård, kall och våt. Du verkar inte kunna hitta ditt sovrum, väggar eller tak &
%    Du har tillbringat natten i rännstenen &
%    Kontrollera om baren är öppen. Om inte unna dig en sovmorgon\\
%    \hline
%    Det har plötsligt blivit mycket folk &
%    Du ser dubbelt &
%    Blunda med ena ögat \\
%    \hline
%    Allt blir mörkt och tyst &
%    Baren stängs &
%    \textbf{PANIK!} \\
%    \hline
%
%    \end{tabularx}

\newpage

\newpage