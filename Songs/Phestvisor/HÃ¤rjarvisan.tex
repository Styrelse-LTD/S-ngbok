\begin{SongText}[Härjarvisan]
    \begin{SongInfo}
        Mel: Gärdebylåten\\*%
        Ur Lundsaspexet "Djangis Khan" 1954
    \end{SongInfo}
    \begin{SongVerse}
        Liksom våra fäder vikingarna i Norden\\*%
        drar vi riket runt och super oss under borden.\\*%
        Brännvinet har blivit ett elexir\\*%
        för kropp såväl som själ.\\*%
        Känner du dig liten och ynklig på jorden,\\*%
        växer du med supen och blir stor uti orden,\\*%
        slår dig för ditt håriga bröst och\\*%
        blir en man från hår till häl.
    \end{SongVerse}
    \begin{SongVerse}
        Ref:\\*%
        Ja nu ska vi ut härja\\*%
        supa, slåss och svärja\\*%
        bränna röda stugor, slå små barn och säga fula ord. (KTH!)\\*%
        Med blod ska vi stäppen färga\\*%
        nu äntligen lär jag kunna\\*%
        dra nån riktig nytta utav min Hermodskurs i mord.
    \end{SongVerse}
    \begin{SongVerse}
        Hurra nu ska man äntligen få röra på benen\\*%
        hela stammen jublar och det spritter i grenen.\\*%
        Tänk att än en gång få spränga fram på Brunte i galopp.\\*%
        Din doft o käre Brunte är trots brist i hygienen\\*%
        för en vild mongol minst lika ljuv som syrenen,\\*%
        tänk att på din rygg få rida runt i stan och spela topp!
    \end{SongVerse}
    \begin{SongVerse}
        Ref…
    \end{SongVerse}
    \begin{SongVerse}
        Ja, mordbränder är klämmiga, ta fram fotogenen\\*%
        och eftersläckningen tillhör just de fenomenen\\*%
        inom brandmansyrket som jag tycker det är nån nytta med.\\*%
        Jag målar för mitt inre upp den härliga scenen:\\*%
        Blodrött mitt i brandgult, ens prins Eugen en\\*%
        lika mustig vy kan måla, ens om han målade med sked.
    \end{SongVerse}
    \begin{SongVerse}
        Ref…
    \end{SongVerse}
\end{SongText}
