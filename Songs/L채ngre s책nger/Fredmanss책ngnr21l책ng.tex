\begin{SongText}[Fredmans sång nr 21 (lång)]
    \begin{SongInfo}
        Text: Carl Michael Bellman
    \end{SongInfo}
    \begin{SongVerse}
        Så lunka vi så småningom
        från Bacchi buller och tumult,
        när döden ropar; Granne kom,
        ditt timglas är nu fullt.
        Du gubbe fäll din krycka ner,
        och du yngling, lyd min lag,
        den skönsta nymf som mot dig ler
        inunder armen tag.
    \end{SongVerse}
    \begin{SongVerse}
        Ref:
        Tycker du att graven är för djup,
        nå välan, så tag dig då en sup,
        tag dig sen dito en, dito två, dito tre,
        så dör du nöjdare.
    \end{SongVerse}
    \begin{SongVerse}
        Du vid din remmare och préss,
        rödbrusig och med hatt på sned,
        snart skrider fram din likprocess
        i några svarta led.
        Och du som pratar där så stort,
        med band och stjärnor på din rock,
        re’n snickarn kistan färdig gjort,
        och hyvlar på dess lock.
    \end{SongVerse}
    \begin{SongVerse}
        Ref...
    \end{SongVerse}
    \begin{SongVerse}
        Men du som med en trumpen min,
        bland riglar, galler, järn och lås,
        dig vilar på ditt penningskrin,
        inom din stängda bås.
        Och du som svartsjuk slår i kras
        buteljer, speglar och pokal;
        bjud nu god natt, drick ut ditt glas,
        och hälsa din rival.
    \end{SongVerse}
    \begin{SongVerse}
        Ref...
    \end{SongVerse}
    \begin{SongVerse}
        Och du som under titlars klang
        din tiggarstav förgyllt vart år,
        som knappast har, med all din rang,
        en skilling till din bår.
        Och du som ilsken, feg och lat,
        fördömer vaggan som dig välvt,
        och ändå dagligt är plakat
        till glasets sista hälft.
    \end{SongVerse}
    \begin{SongVerse}
        Ref...
    \end{SongVerse}
    \begin{SongVerse}
        Du som vid Martis fältbasun
        i blodig skjorta sträckt ditt steg,
        och du som tumlar i paulun,
        i Chloris armar feg.
        Och du som med din gyllne bok
        vid templets genljud reser dig,
        som rister huvud lärd och klok,
        och för mot avgrund krig,
    \end{SongVerse}
    \begin{SongVerse}
        Ref...
    \end{SongVerse}
    \begin{SongVerse}
        Men du som med en ärlig min
        plär dina vänner häda jämt,
        och dem förtalar vid ditt vin,
        och det liksom på skämt.
        Och du som ej försvarar dem,
        fastän ur deras flaskor du,
        du väl kan slicka dina fem,
        vad svarar du väl nu?
    \end{SongVerse}
    \begin{SongVerse}
        Ref...
    \end{SongVerse}
    \begin{SongVerse}
        Men du som till din återfärd,
        ifrån det du till bordet gick,
        ej klingat för din raska värd,
        fastän han ropar: ”Drick”.
        Driv sådan gäst från mat och vin!
        Kör honom med sitt anhang ut,
        och sen med en ovänlig min,
        ryck remmarn ur hans trut.
    \end{SongVerse}
    \begin{SongVerse}
        Ref...
    \end{SongVerse}
    \begin{SongVerse}
        Säg är du nöjd, min granne säg,
        så prisa världen nu till slut;
        om vi ha en och samma väg,
        så följoms åt; drick ut.
        Men först med vinet rött och vitt
        för vår värdinna bugom oss,
        och halkom sen i graven fritt,
        vid aftonstjärnans bloss.
    \end{SongVerse}
    \begin{SongVerse}
        Ref...
    \end{SongVerse}
\end{SongText}
