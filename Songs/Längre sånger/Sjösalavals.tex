%\begin{SongText}[Sjösala vals]
%\begin{SongInfo}
%    Text: Evert Taube
%\end{SongInfo}
%\begin{SongVerse}
%Rönnerdahl han skuttar med ett skratt ur sin säng\\*%
%Solen står på Orrberget. Sunnanvind brusar.\\*%
%Rönnerdahl han valsar över Sjösala äng.\\*%
%Hör min vackra visa, kom sjung min refräng!\\*%
%Tärnan har fått ungar och dyker i min vik,\\*%
%ur alla gröna dungar hörs finkarnas musik.\\*%
%Och se, så många blommor som redan slagit ut på ängen!\\*%
%Gullviva, mandelblom, kattfot och blå viol.
%\end{SongVerse}
%\begin{SongVerse}
%Rönnerdahl han virvlar sina lurviga ben\\*%
%under vita skjortan som viftar kring vaderna.\\*%
%Lycklig som en lärka uti majsolens sken,\\*%
%sjunger han för ekorrn, som gungar på gren!\\*%
%Kurre, kurre, kurre nu dansar Rönnedahl.\\*%
%Koko! Och göken ropar uti hans gröna dal.\\*%
%Och se, så många blommor som redan slagit ut på ängen!\\*%
%Gullviva, mandelblom, kattfot och blå viol. 
%\end{SongVerse}
%\begin{SongVerse}
%Rönnerdahl han binder utav blommor en krans,\\*%
%binder den kring håret, det gråa och rufsiga,\\*%
%valsar in i stugan och har lutan till hands,\\*%
%väcker frun och barnen med drill och kadans.\\*%
%Titta! ropar ungarna, Pappa är en brud,\\*%
%med blomsterkrans i håret och nattskjortan till skrud!\\*%
%Och se, så många blommor som redan slagit ut på ängen!\\*%
%Gullviva, mandelblom, kattfot och blå viol.
%\end{SongVerse}
%\begin{SongVerse}
%Rönnerdahl är gammal men han valsar ändå,\\*%
%Rönnerdahl har sorger och ont om sekiner.\\*%
%Sällan får han rasta - han får slita för två.\\*%
%Hur han klarar skivan, kan ingen förstå,\\*%
%ingen, utom tärnan i viken - hon som dök\\*%
%och ekorren och finken och vårens första gök.\\*%
%Och blommorna, de blommor som redan slagit ut på ängen,\\*%
%Gullviva, mandelblom, kattfot och blå viol. 
%\end{SongVerse}
%\end{SongText}
