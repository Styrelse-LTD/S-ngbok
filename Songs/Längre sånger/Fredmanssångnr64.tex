%\begin{SongText}[Fredmans sång nr 64 ]
%\begin{SongInfo}
%    Mel: Carl Michael Bellman
%\end{SongInfo}
%\begin{SongVerse}
%Fjäril vingad syns på Haga
%mellan dimmors frost och dun.
%Sig sitt gröna skjul tilltaga
%och i blomma sin paulun.
%Minsta kräk i kärr och syra,
%nyss av solens värme väckt.
%Till en ny högtidlig yra
%eldars vid zephirens fläckt.
%\end{SongVerse}
%\begin{SongVerse}
%Haga, i ditt sköte röjes
%gräsets brodd och gula plan.
%Stolt i dina rännar höjes
%gungande den hvita svan.
%Längst ur skogens glesa kamrar
%höras täta återskall
%än från den graniten hamrar,
%än från yx i björk och tall.
%\end{SongVerse}
%\begin{SongVerse}
%Se, Brunnsvikens små najader
%höja sina gyllne horn,
%och de frustande kaskader
%sprutas över Solna torn.
%Under skygd av välvda stammar
%på den väg man städad ser,
%fålen yvs och hjulet dammar.
%Bonden milt åt Haga ler. 
%\end{SongVerse}
%\end{SongText}
