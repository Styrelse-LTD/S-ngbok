\begin{SongText}[Fredmans epistel nr 2 ]
    \begin{SongInfo}
        Text: Carl Michael Bellman\\*%
        Till fader Berg, rörande posten
    \end{SongInfo}
    \begin{SongVerse}
        Nå, skrufva fiolen,\\*%
        hej spelman, skynda dig!\\*%
        Kära syster, hej!\\*%
        Svara inte nej,\\*%
        svara ja, så blir vi glada.\\*%
        Sätt dig du på stolen\\*%
        och stryk din silfversträng!\\*%
        Röda stråken släng\\*%
        och med armen sväng;\\*%
        gör ej fiolen skada!\\*%
        Du svettas, stor sak,\\*%
        i brännvin skall du bada,\\*%
        ty under detta tak är\\*%
        Bacchi lada.\\*%
        - - Ganska riktigt!\\*%
        Ditt kall är viktigt\\*%
        båd’ för öra, syn och smak.
    \end{SongVerse}
    \begin{SongVerse}
        Bland nymfernas skara\\*%
        är du omistlig man;\\*%
        du båd’ vill och kan\\*%
        mer än någon ann\\*%
        de unga hjärtan binda,\\*%
        och kärlekens snara\\*%
        på dina strängar står\\*%
        Varje ton du slår,\\*%
        du ett hjärta får\\*%
        att konstigt sammanlinda.\\*%
        Just på en minut\\*%
        små ögon blifva blinda,\\*%
        och flickorna till slut\\*%
        de blir så trinda.\\*%
        - - Hur du bullrar!\\*%
        Men nymfen kullrar,\\*%
        och skrattar med din trut. 
    \end{SongVerse}
    \begin{SongVerse}
        Jag älskar de sköna\\*%
        men vinet ändå mer;\\*%
        jag på båda ser\\*%
        och åt båda ler\\*%
        men skiljer ändå båda.\\*%
        En nymf i det gröna\\*%
        och vin i gröna glas:\\*%
        Lika gott kalas,\\*%
        båda om mig dras.\\*%
        Ge stråken mera kåda:\\*%
        Konfonium tag där\\*%
        uti min gröna låda;\\*%
        och vinet står ju här.\\*%
        Jag är i våda.\\*%
        - - Supa, dricka\\*%
        och ha sin flicka\\*%
        är vad Sankte Fredman lär. 
    \end{SongVerse}
\end{SongText}