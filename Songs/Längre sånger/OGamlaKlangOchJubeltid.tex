\begin{SongText}[O gamla klang och jubeltid]
    \begin{SongInfo}
        Melodi: O alte Burschenherrlichkeit\\*
        Text: August Lind, 1921
    \end{SongInfo}
    \begin{SongInfo}
        Visan är en utav de mest studentikosa i denna bok, och sjungs ihärdigt i traditionella studentstäder såsom Uppsala och Lund än idag. 
        Vissa delar är markerade, och när du utpekas baserat på ditt program skall du resa dig upp och sjunga! 
        Till teknologer räknas de som studerar ett tekniskt program.
        I brist på riktiga Jurister \& Teologer så hävdar Perpetuum Mobiles att MDUs Statsvetare \& Beteendevetare passar in i dessa kategorier.
        Medicinare definerar vi som resterande avv HVV akademin. Resterande utbildningar klassas som Filosofer.

        Vid den fjärde versen, skall man kraftfullt banka handen på bordet vid första versalen av ordet "Kärnan". Vid den sista versen reser alla sig upp på sina stolar, och får ej sätta sig tillbaks under resten av kvällen, vilket är varför den sjungs allra sist på gasquer. Ordet ”quae” uttalas  ”kvaaj”.
    \end{SongInfo}
    \begin{SongVerse}
        O gamla klang- och jubeltid\\*
        Ditt minne skall förbliva\\*
        Och än åt livets bistra strid\\*
        Ett rosigt skimmer giva!\\*
        Snart tystnar allt vårt yra skämt\\*
        Vår sång blir stum, vårt glam förstämt\\*
        O jerum, jerum, jerum\\*
        O quae mutatio rerum!
    \end{SongVerse}

    

    \begin{longtable}{p{0.2\textwidth}p{0.8\textwidth}}
        \textit{Teknologer:}&Var äro de som kunde allt\\*
        &Blott ej sin ära svika\\*
        &Som voro män av äkta halt\\*
        &Och världens herrar lika?\\*
        &De drogo bort från vin och sång\\*
        &Till vardagslivets tråk och tvång\\*
        &O jerum, jerum, jerum\\*
        &O quae mutatio rerum!\\*
        \textit{Filosofer:}&Den ene vetenskap och vett\\*
        &In i scholares mänger\\*
    \end{longtable}
    \begin{longtable}{p{0.2\textwidth}p{0.8\textwidth}}
        \textit{Jurister:}&Den andre i sitt anlets svett\\*
        &På paragrafer vränge\\*
        \textit{Teologer:}&En plåstrar själen som är skral\\*
        \textit{Medicinare:}&En lappar hop dess trasiga fodral\\*
        \textit{Alla:}&O jerum, jerum, jerum\\*
        &O quae mutatio rerum!
        
    \end{longtable}


    \begin{SongVerse}
        Men hjärtat i en sann student\\*
        Skall ingen tid förfrysa\\*
        Den glädjeeld han där har tänt\\*
        Hans hela liv skall lysa\\*
        Det gamla skalet brustit har\\*
        Men KÄRNAN finnes frisk dock kvar\\*
        Och vad han än må mista\\*
        Det skall docka aldrig brista!
    \end{SongVerse}
    \begin{SongVerse}
        Så sluten bröder fast vår krets\\*
        Till glädjens värn och ära\\*
        Trots allt vi tryggt och väl tillfreds\\*
        Vår vänskap trohet svära\\*
        Lyft bägaren högt och klinga vän\\*
        De gamla gudar leva än\\*
        Bland skålar och pokaler\\*
        Bland skålar och pokaler!
    \end{SongVerse}
\end{SongText}


%textwidth in cm: \printinunitsof{cm}\prntlen{\textwidth}
%
%textheight in cm: \printinunitsof{cm}\prntlen{\textheight}
%
%footskip in cm: \printinunitsof{cm}\prntlen{\footskip}
%
%voffset in cm: \printinunitsof{cm}\prntlen{\voffset}
%
%hoffset in cm: \printinunitsof{cm}\prntlen{\hoffset}
%
%headsep in cm: \printinunitsof{cm}\prntlen{\headsep}
%
%oddsidemargin in cm: \printinunitsof{cm}\prntlen{\oddsidemargin}
%
%evensidemargin in cm: \printinunitsof{cm}\prntlen{\evensidemargin}
\newpage