\begin{SongText}[Fredmans sång nr 35 (Gubben Noak)]
    \begin{SongInfo}
        Text: Carl Michael Bellman
    \end{SongInfo}
    \begin{SongVerse}
        ||: Gubben Noak :||\\*%
        var en hederman.\\*%
        när han gick ur arken\\*%
        planterade han på marken\\*%
        ||: mycket vin, ja :||\\*%
        detta gjorde han.
    \end{SongVerse}
    \begin{SongVerse}
        ||: Noak rodde :||\\*%
        ur sin gamla ark,\\*%
        köpte sig butäljer,\\*%
        sådana man säljer\\*%
        ||: för att dricka :||\\*%
        på vår nya park.
    \end{SongVerse}
    \begin{SongVerse}
        ||: Han väl visste :||\\*%
        att en mänska var\\*%
        törstig av naturen\\*%
        som de andra djuren\\*%
        ||: därför han ock :||\\*%
        vin planterat har.
    \end{SongVerse}
    \begin{SongVerse}
        ||: Gumman Noak :||\\*%
        var en hedersfru.\\*%
        Hon gav man sin dricka;\\*%
        fick, jag sådan flicka\\*%
        ||: gifte jag mig :||\\*%
        just på stunden nu.
    \end{SongVerse}
    \begin{SongVerse}
        ||: Aldrig sad’ hon :||\\*%
        Kära far nå nå,\\*%
        sätt ifrån dig kruset.\\*%
        Nej, det ena ruset\\*%
        ||: på det andra :||\\*%
        lät hon gubben få.
    \end{SongVerse}
    \begin{SongVerse}
        ||: Gubben Noak :||\\*%
        brukte egna hår,\\*%
        pipskägg, hakan trinder,\\*%
        rosenröda kinder,\\*%
        ||: drack ibotten :||\\*%
        Hurra och gutår!
    \end{SongVerse}
    \begin{SongVerse}
        Då var lustigt, då var lustigt\\*%
        På vår gröna jord;\\*%
        Man fick väl till bästa,\\*%
        Ingen torstig nästa\\*%
        Satt och blängde, satt och blängde\\*%
        Vid ett dukat bord.
    \end{SongVerse}
    \begin{SongVerse}
        Inga skålar, inga skålar\\*%
        Gjorde då besvär,\\*%
        Då var ej den läran:\\*%
        Jag skall ha den äran.\\*%
        Nej i botten, nej i botten\\*%
        Drack man ur så här.
    \end{SongVerse}
\end{SongText}
