\begin{SongText}[Gällivarevisan]
    \begin{SongInfo}
        Text: Helmer Andersson\\*%
        Mel:  Stadsbudsvisan\\*%
        LTU's Ultima Thule anger att visan ska sjungas "med kraftig finsk brytning" så texten reflekterar därefter
    \end{SongInfo}
    \begin{SongVerse}
        Dänkte på lörda skulle fara\\*%
        in te Gällivara på Karakatorg\\*%
        dom pjuda trikka prännvinsflaska\\*%
        gott som satans paska, voj voj.\\*%
        De vara rolika liven,\\*%
        para lagsmål å niven,\\*%
        åka pålisstationen,\\*%
        vara jävlika fasonen.\\*%
        Ligga inne halva natten,\\*%
        leva limpa å vatten,\\*%
        komma ut morronröken,\\*%
        säja knappast sen ajöken.
    \end{SongVerse}
    \begin{SongVerse}
        Men dänkte perkele anamma,\\*%
        kanse vara samma åka Visskafoss,\\*%
        där sänner ja en gammal likka\\*%
        som jag prukar likka, förståss.\\*%
        De prukar pli ganska sällan,\\*%
        fara hälsa på fjällan,\\*%
        kanse få någe mellan,\\*%
        bara inte fassna fällan.\\*%
        De vara klart man riskera,\\*%
        ingenting reflektera,\\*%
        pliva kanske nå’t mera,\\*%
        måste gå å operera.
    \end{SongVerse}
    \begin{SongVerse}
        Men nu jak luta erotiken,\\*%
        öppna pritfabriken, sälja akvavit.\\*%
        Då komma pålismästarn nära,\\*%
        fråka om jak pära priten.\\*%
        De vara jäklika token,\\*%
        pliva hånkad av snoken,\\*%
        åka Luleå-kroken,\\*%
        längta baka uti skoken.\\*%
        Sitta inne halva åren,\\*%
        bliva krå uti håren,\\*%
        knappast röra på låren,\\*%
        komma bakas först på våren.\\*%
        Liikavaara-Frasse, hembrännare
    \end{SongVerse}
\end{SongText}
