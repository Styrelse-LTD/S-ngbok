\begin{SongText}[Fredmans sång nr 21 (kort)]
    \begin{SongInfo}
        Text: Carl Michael Bellman
    \end{SongInfo}
    \begin{SongVerse}
        Så lunka vi så småningom\\*%
        från Bacchi buller och tumult,\\*%
        när döden ropar; Granne kom,\\*%
        ditt timglas är nu fullt.\\*%
        Du gubbe fäll din krycka ner,\\*%
        och du yngling, lyd min lag,\\*%
        den skönsta nymf som mot dig ler\\*%
        inunder armen tag.
    \end{SongVerse}
    \begin{SongVerse}
        Ref:\\*%
        Tycker du att graven är för djup,\\*%
        nå välan, så tag dig då en sup,\\*%
        tag dig sen dito en, dito två, dito tre,\\*%
        så dör du nöjdare.
    \end{SongVerse}
    \begin{SongVerse}
        Säg är du nöjd, min granne säg,\\*%
        så prisa världen nu till slut;\\*%
        om vi ha en och samma väg,\\*%
        så följoms åt; drick ut.\\*%
        Men först med vinet rött och vitt\\*%
        för vår värdinna bugom oss,\\*%
        och halkom sen i graven fritt,\\*%
        vid aftonstjärnans bloss.
    \end{SongVerse}
    \begin{SongVerse}
        Ref...
    \end{SongVerse}
\end{SongText}